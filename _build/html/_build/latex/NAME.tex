%% Generated by Sphinx.
\def\sphinxdocclass{report}
\documentclass[a4paper,10pt,icelandic]{sphinxmanual}
\ifdefined\pdfpxdimen
   \let\sphinxpxdimen\pdfpxdimen\else\newdimen\sphinxpxdimen
\fi \sphinxpxdimen=.75bp\relax

\PassOptionsToPackage{warn}{textcomp}
\usepackage[utf8]{inputenc}
\ifdefined\DeclareUnicodeCharacter
% support both utf8 and utf8x syntaxes
  \ifdefined\DeclareUnicodeCharacterAsOptional
    \def\sphinxDUC#1{\DeclareUnicodeCharacter{"#1}}
  \else
    \let\sphinxDUC\DeclareUnicodeCharacter
  \fi
  \sphinxDUC{00A0}{\nobreakspace}
  \sphinxDUC{2500}{\sphinxunichar{2500}}
  \sphinxDUC{2502}{\sphinxunichar{2502}}
  \sphinxDUC{2514}{\sphinxunichar{2514}}
  \sphinxDUC{251C}{\sphinxunichar{251C}}
  \sphinxDUC{2572}{\textbackslash}
\fi
\usepackage{cmap}
\usepackage[T1]{fontenc}
\usepackage{amsmath,amssymb,amstext}
\usepackage{babel}



\usepackage{times}
\expandafter\ifx\csname T@LGR\endcsname\relax
\else
% LGR was declared as font encoding
  \substitutefont{LGR}{\rmdefault}{cmr}
  \substitutefont{LGR}{\sfdefault}{cmss}
  \substitutefont{LGR}{\ttdefault}{cmtt}
\fi
\expandafter\ifx\csname T@X2\endcsname\relax
  \expandafter\ifx\csname T@T2A\endcsname\relax
  \else
  % T2A was declared as font encoding
    \substitutefont{T2A}{\rmdefault}{cmr}
    \substitutefont{T2A}{\sfdefault}{cmss}
    \substitutefont{T2A}{\ttdefault}{cmtt}
  \fi
\else
% X2 was declared as font encoding
  \substitutefont{X2}{\rmdefault}{cmr}
  \substitutefont{X2}{\sfdefault}{cmss}
  \substitutefont{X2}{\ttdefault}{cmtt}
\fi


\usepackage[Sonny]{fncychap}
\usepackage[,numfigreset=1,mathnumfig]{sphinx}

\fvset{fontsize=\small}
\usepackage{geometry}


% Include hyperref last.
\usepackage{hyperref}
% Fix anchor placement for figures with captions.
\usepackage{hypcap}% it must be loaded after hyperref.
% Set up styles of URL: it should be placed after hyperref.
\urlstyle{same}

\usepackage{sphinxmessages}
\setcounter{tocdepth}{1}



\usepackage{amsmath}
\usepackage{amssymb}
\usepackage{hyperref}


\title{NAME Documentation}
\date{feb. 14, 2020}
\release{2018}
\author{AUTHOR}
\newcommand{\sphinxlogo}{\sphinxincludegraphics{hi_horiz_raunvisindadeild.png}\par}
\renewcommand{\releasename}{Útgáfa}
\makeindex
\begin{document}

\ifdefined\shorthandoff
  \ifnum\catcode`\=\string=\active\shorthandoff{=}\fi
  \ifnum\catcode`\"=\active\shorthandoff{"}\fi
\fi

\pagestyle{empty}
\sphinxmaketitle
\pagestyle{plain}
\sphinxtableofcontents
\pagestyle{normal}
\phantomsection\label{\detokenize{index::doc}}



\chapter{Algebra}
\label{\detokenize{Kafli01:algebra}}\label{\detokenize{Kafli01::doc}}

\section{Talnakerfi}
\label{\detokenize{Kafli01:talnakerfi}}

\bigskip\hrule\bigskip



\subsection{Náttúrulegu tölurnar  \protect\(\mathbb{N}\protect\)}
\label{\detokenize{Kafli01:natturulegu-tolurnar-mathbb-n}}
Tölurnar \(0,1,2,3, \dots\) köllum við \textit{náttúrulegu tölurnar}. \textit{Mengi} náttúrulegu talnanna er táknað með \(\mathbb{N}.\)


\subsection{Heiltölurnar \protect\(\mathbb{Z}\protect\)}
\label{\detokenize{Kafli01:heiltolurnar-mathbb-z}}
\textit{Heilu tölurnar} eru talnakerfi allra heilla talna. Það inniheldur
\begin{itemize}
\item {} 
neikvæðu heiltölurnar; \(-1,-2,-3, \dots\)

\item {} 
jákvæðu heiltölurnar; \(1,2,3, \dots\)

\item {} 
og töluna \(0\).

\end{itemize}

Mengi heiltalna er táknað með \(\mathbb{Z}\).


\subsection{Ræðu tölurnar \protect\(\mathbb{Q}\protect\)}
\label{\detokenize{Kafli01:raeu-tolurnar-mathbb-q}}
\textit{Ræðar tölur} samanstanda af öllum brotum \(\frac{p}{q}\) þar sem \(p\) og \(q\) eru heilar tölur og \(q \neq 0\). Talan \(p\) nefnist \textit{teljari} brotsins og talan \(q\) \textit{nefnari} þess.

Mengi ræðra talna er táknað með \(\mathbb{Q}\).


\subsection{Rauntölurnar \protect\(\mathbb{R}\protect\)}
\label{\detokenize{Kafli01:rauntolurnar-mathbb-r}}
Ekki eru allar tölur ræðar tölur.
Tölur sem ekki er hægt að skrifa sem brot heilla talna köllum við \textit{óræðar tölur}.
Til dæmis er talan \(\pi\) (pí) óræð.

Mengi allra ræðra talna, auk óræðra talna, nefnist \textit{rauntölurnar}. Það er táknað með \(\mathbb{R}\).

\noindent{\hspace*{\fill}\sphinxincludegraphics[width=0.750\linewidth]{{mynd-mengi}.svg}\hspace*{\fill}}

Hér höfum við mengjamynd af talnamengjunum. Sjáum til dæmis að \(\mathbb{N}\) er \textit{hlutmengi} í \(\mathbb{R}\), það er allar náttúrulegar tölur eru rauntölur.
(Í {\hyperref[\detokenize{Kafli04:s-mengi}]{\sphinxcrossref{\DUrole{std,std-ref}{kaflanum um mengi}}}} má finna skilgreiningu á {\hyperref[\detokenize{Kafli04:s-hlutmengi}]{\sphinxcrossref{\DUrole{std,std-ref}{hlutmengi}}}})


\section{Forgangsröðun aðgerða}
\label{\detokenize{Kafli01:forgangsroun-agera}}

\bigskip\hrule\bigskip


Þegar reiknað er með tölum þarf að framkvæma \textit{aðgerðir} í eftirfarandi röð:
\begin{enumerate}
\sphinxsetlistlabels{\arabic}{enumi}{enumii}{}{.}%
\item {} 
Reikningsaðgerðir innan sviga

\item {} 
Margföldunar\sphinxhyphen{} og deilingaraðgerðir

\item {} 
Samlagningar\sphinxhyphen{} og frádráttaraðgerðir

\end{enumerate}

\begin{sphinxadmonition}{tip}{Dæmi:}
Notum forgangsröðun aðgerða:

\sphinxstylestrong{1.} Reiknum \(1+2 \cdot 3\)
\begin{equation*}
\begin{split}\begin{aligned}
        1+2 \cdot 3&=1+6\\
        &=7
\end{aligned}\end{split}
\end{equation*}
\sphinxstylestrong{2.} Reiknum \((1+2)\cdot 3+4\)
\begin{equation*}
\begin{split}\begin{aligned}
        (1+2)\cdot 3+4&= 3 \cdot 3+4\\
        &=9+4 \\
        &= 13
\end{aligned}\end{split}
\end{equation*}
\sphinxstylestrong{3.} Reiknum \(((1+1) \cdot 5+3 \cdot (2-4))\cdot 2\)
\begin{equation*}
\begin{split}\begin{aligned}
        ((1+1) \cdot 5+3 \cdot (2-4))\cdot 2 &=(2 \cdot 5+3 \cdot (-2)) \cdot 2\\
        & = (10-6) \cdot 2\\
        &=4 \cdot 2\\
        &=8
\end{aligned}\end{split}
\end{equation*}\end{sphinxadmonition}


\section{Reiknireglur}
\label{\detokenize{Kafli01:reiknireglur}}
Nokkrar einfaldar reiknireglur gilda um tölur í talnakerfunum:
\begin{equation*}
\begin{split}\begin{aligned}
(a+b)+c=a+(b+c) &  \qquad \textit{ (tengiregla samlagningar)}\\
(ab)c=a(bc) & \qquad \textit{ (tengiregla margföldunar)}\\
a+b=b+a & \qquad \textit{ (víxlregla samlagningar)}\\
ab=ba & \qquad \textit{ (víxlregla margföldunar)}\\
a(b+c)=ab+ac & \qquad \textit{ (dreifiregla)}\\
1 \cdot a=a & \qquad \textit{ (1 er margföldunarhlutleysa)}\\
a+0=a & \qquad \textit{ (0 er samlagningarhlutleysa)}\\
0 \cdot a=0  & \qquad \textit{ (margföldun með núlli gefur núll)}\\
\end{aligned}\end{split}
\end{equation*}
\begin{sphinxadmonition}{warning}{Aðvörun:}
Athugum að tvær neikvæðar tölur margfaldaðar saman verða að jákvæðri tölu, til dæmis \((-3)\cdot (-4) = 3\cdot 4 =12\) .
\end{sphinxadmonition}

\begin{sphinxadmonition}{tip}{Dæmi:}
Prófum dreifiregluna {[}\(a(b+c)=ab+ac\){]} fyrir tölurnar \(a=3\), \(b=-9\) og \(c=5\).

Vinstra megin jafnaðarmerkisins stendur \(3(-9+5)=3\cdot(-4)=-12\).

Hægra megin jafnaðarmerkisins stendur \(3\cdot(-9)+3\cdot 5=-27+15=-12\).

Við höfum því sömu stærð báðum megin jafnaðarmerkis.
\end{sphinxadmonition}


\section{Frumtölur}
\label{\detokenize{Kafli01:frumtolur}}

\subsection{Skilgreining: Deilanleiki}
\label{\detokenize{Kafli01:skilgreining-deilanleiki}}
Heiltala \(a\) er sögð vera deilanleg með heiltölunni \(b\) ef til er heiltala \(x\) þannig að \(a=bx\).

\begin{sphinxadmonition}{tip}{Dæmi:}
Talan 14 er deilanleg með 2 því hana má skrifa sem \(14=2\cdot 7\) .
\end{sphinxadmonition}

\begin{sphinxadmonition}{note}{Athugasemd:}
Allar tölur \(a\) eru deilanlegar með einum og sjálfri sér því \(a= 1 \cdot a\) . Tölur geta haft fleiri deila, til dæmis er \(12\) deilanleg með \(3\) og \(4\) og talan \(15\) er deilanleg með \(3\) og \(5\).
\end{sphinxadmonition}

Sumar náttúrulegar tölur eru aðeins deilanlegar með einum og sjálfri sér.
Þær eru nefndar \textit{frumtölur}, en talan \(1\) er \sphinxstylestrong{ekki} skilgreind sem frumtala.


\subsection{Skilgreining: Frumtölur}
\label{\detokenize{Kafli01:skilgreining-frumtolur}}
Ef náttúruleg tala \(p \geq 2\) er aðeins deilanleg með einum og sjálfri sér þá segjum við að \(p\) sé \sphinxstyleemphasis{frumtala}.

\begin{sphinxadmonition}{tip}{Dæmi:}
\(7\) er frumtala, því eina jákvæða heiltölulausninn á \(7=bx\) er \(b=7\) og \(x=1\), þ.e. \(7 = 7 \cdot 1\),

en \(6\) er ekki frumtala því \(6=2 \cdot 3\).
\end{sphinxadmonition}

\begin{sphinxadmonition}{note}{Athugasemd:}
Nokkrar fyrstu frumtölurnar eru: \(2,3,5,7,11,13,17,19,23,29, \dots\)
\end{sphinxadmonition}

\noindent{\hspace*{\fill}\sphinxincludegraphics[width=1.000\linewidth]{{frumtölur}.png}\hspace*{\fill}}

Hér eru 25 fyrstu frumtölurnar merktar bláar.

\phantomsection\label{\detokenize{Kafli01:s-frumattun}}
Allar heiltölur sem eru ekki frumtölur eru margfeldi frumtalna. Af því leiðir að sérhverja heiltölu megi skrifa sem margfeldi frumtalna.


\subsection{Frumþáttun}
\label{\detokenize{Kafli01:frumattun}}
Sérhverja náttúrulega tölu \(a \geq 2\) má skrifa sem margfeldi frumtalna
\begin{equation*}
\begin{split}a=p_1 p_2 p_3 \dots p_m\end{split}
\end{equation*}
þar sem sumar frumtölur geta verið endurteknar.

\begin{sphinxadmonition}{tip}{Dæmi:}\begin{equation*}
\begin{split}7=7, \qquad 24=2 \cdot 2 \cdot 2 \cdot 3=2^3 \cdot 3, \qquad 250= 2 \cdot 5 \cdot 5 \cdot 5=2 \cdot 5^3.\end{split}
\end{equation*}\end{sphinxadmonition}

Engin skilvirk aðferð hefur verið fundin til að \textit{frumþátta} stórar tölur. Eftirfarandi aðferð til að frumþátta byggir á prófun:
\begin{enumerate}
\sphinxsetlistlabels{\arabic}{enumi}{enumii}{}{.}%
\item {} 
Athugum hvort einhver frumtala minni en talan gangi upp í hana. Hér er best að byrja á lægstu frumtölunni, 2, og svo næstu frumtölu, 3, o.s.fr.v.

\item {} 
Ef við finnum slíka frumtölu deilum við með henni og fáum út aðra tölu, og skoðum hana.

\item {} 
Athugum hvort við finnum frumtölu sem gengur upp í nýju töluna og endurtökum þá skref 1 og 2.

\item {} 
Höldum þessu áfram þar til við finnum enga frumtölu sem gengur upp í töluna, þá er talan sjálf frumtala. Frumþáttunin er svo rituð sem margfeldi allra frumtalnanna sem við fundum.

\end{enumerate}

\begin{sphinxadmonition}{note}{Athugasemd:}
Ef við leitum að frumtölu sem gengur upp í tölu þá nægir að skoða frumtölur \sphinxstylestrong{minni en eða jafnar kvaðratrót tölunnar}. Þetta getur flýtt fyrir við að frumþátta stórar tölur.
\end{sphinxadmonition}

\begin{sphinxadmonition}{tip}{Dæmi:}\begin{quote}

\sphinxstylestrong{1.} Frumþáttum töluna \(273\). Athugum að \(2\) gengur ekki upp í \(273\), en \(3\) gerir það, samkvæmt prófun. Skoðum þá næst \(273/3=91\). Með prófun sést að \(2,3\) og \(5\) ganga ekki upp í \(91\), en \(7\) gerir það.  Skoðum þá næst \(91/7=13\). En \(13\) er frumtala. Því höfum við fundið frumþáttunina. Hún er \(273=3 \cdot 7 \cdot 13\). Oft er þægilegt að setja upp frumþáttunina í tré:

\noindent{\hspace*{\fill}\sphinxincludegraphics[width=0.750\linewidth]{{frumthattun}.svg}\hspace*{\fill}}

\sphinxstylestrong{2.} Frumþáttum töluna \(101\). Notfærum okkur athugasemd hér að ofan. Við þurfum bara að prófa frumtölur minni en eða jafnar \(\sqrt{101} \approx 10.05\). Með prófun sést að \(2, 3,5\) og \(7\) ganga ekki upp í \(101\). Því er \(101\) frumtala.
\end{quote}

\begin{sphinxadmonition}{note}{Athugasemd:}
Allar tölur sem hafa þversummu sem er margfeldi af þremur eru deilanlegar með þremur.
Til dæmis má sjá í lið \sphinxstylestrong{1.} hér að ofan að þversumma 273 er \(2+7+3 = 12 =3 \cdot 4\) og því er 273 deilanleg með 3 (\(273=3\cdot 91\) ).
\end{sphinxadmonition}
\end{sphinxadmonition}

Þessi aðferð frumþáttunar byggir á því að finna frumtölu sem gengur upp í töluna, en oft getur verið þægilegra að finna samsetta tölu og frumþátta hana síðan.

\begin{sphinxadmonition}{tip}{Dæmi:}
Frumþáttum töluna 270:

\begin{figure}[H]
\centering

\noindent\sphinxincludegraphics[width=0.600\linewidth]{{frth270}.svg}
\end{figure}

Frumþáttun 270 er því:
\begin{equation*}
\begin{split}270 = 2\cdot 5\cdot 3\cdot 3\cdot 3\end{split}
\end{equation*}\end{sphinxadmonition}

\begin{sphinxadmonition}{tip}{Dæmi:}
Ef við skoðum \(36\) þá getum við fundið alla deila hennar með því að skoða allar tölur sem ganga upp í \(36\).
Hér eru allar tölurnar sem ganga upp í \(36\)
\begin{equation*}
\begin{split}{1, 2, 3, 4, 6, 9, 12, 18, 36}\end{split}
\end{equation*}\end{sphinxadmonition}


\subsection{Stærsti samdeilir og minnsta samfeldi}
\label{\detokenize{Kafli01:staersti-samdeilir-og-minnsta-samfeldi}}
\textit{Stærsti samdeilir} tveggja talna er stærsta talan sem gengur upp í báðar tölurnar.
Hann er hægt að finna með því að frumþátta báðar tölurnar og finna hvaða frumþættir eru sameiginlegir.

\begin{sphinxadmonition}{tip}{Dæmi:}
Finnum stærsta samdeili 792 og 756.

\sphinxstyleemphasis{Lausn}

Byrjum á því að frumþátta tölurnar:
\begin{equation*}
\begin{split}\begin{aligned}
792 &= 2\cdot 2\cdot 2\cdot 3\cdot 3\cdot 11 \\
&= 2^3\cdot 3^2\cdot 11  \\
& \\
756 &= 2\cdot 2\cdot 3\cdot 3\cdot 3\cdot 7 \\
&= 2^2\cdot 3^3 \cdot 7
\end{aligned}\end{split}
\end{equation*}
Sjáum því að sameiginlegir frumþættir 792 og 756 eru \(2^2\) og \(3^2\) og því er stærsti samdeilir þeirra:
\begin{equation*}
\begin{split}\text{ssd}(792,756) = 2^2 \cdot 3^2 = 36\end{split}
\end{equation*}
Það þýðir að 36 er stærsta talan sem gengur upp í bæði 792 og 756
\begin{equation*}
\begin{split}\begin{aligned}
792&=36\cdot 22 \\
756&=36\cdot 21
\end{aligned}\end{split}
\end{equation*}\end{sphinxadmonition}

\textit{Minnsta samfeldi} tveggja talna er minnsta talan sem báðar tölurnar ganga upp í.
Það er hægt að finna með því að margfalda saman frumþættina í hæsta veldinu sem þeir koma fram í.

\begin{sphinxadmonition}{tip}{Dæmi:}
Finnum minnsta samfeldi 792 og 756.

\sphinxstyleemphasis{Lausn}

Eins og áður er frumþáttun talnanna:
\begin{equation*}
\begin{split}\begin{aligned}
792 &= 2\cdot 2\cdot 2\cdot 3\cdot 3\cdot 11 \\
&= 2^3\cdot 3^2\cdot 11  \\
& \\
756 &= 2\cdot 2\cdot 3\cdot 3\cdot 3\cdot 7 \\
&= 2^2\cdot 3^3 \cdot 7
\end{aligned}\end{split}
\end{equation*}
Frumþættirnir sem fram koma eru 2, 3, 7 og 11.
Hæsta veldið á 2 er 3 (í frumþáttuninni á 792), hæsta veldið á 3 er 3 (í frumþáttuninni á 756) en 7 og 11 koma aðeins einu sinni fyrir. Því er:
\begin{equation*}
\begin{split}\text{msf}(792,756) = 2^3\cdot 3^3 \cdot 7\cdot 11 = 16632\end{split}
\end{equation*}\end{sphinxadmonition}

\begin{sphinxadmonition}{note}{Athugasemd:}
Fyrir sérhvert par talna \(a\) og \(b\) gildir að
\begin{equation*}
\begin{split}a \cdot b = \text{ssd}(a,b)\cdot \text{msf}(a,b)\end{split}
\end{equation*}\end{sphinxadmonition}


\section{Brotareikningur}
\label{\detokenize{Kafli01:brotareikningur}}
Rifjum upp skilgreiningu á ræðum tölum.


\subsection{Skilgreining: Ræðar tölur}
\label{\detokenize{Kafli01:skilgreining-raear-tolur}}
\sphinxstyleemphasis{Ræðar tölur} samanstanda af öllum brotum \(\frac{p}{q}\) þar sem \(p\) og \(q\) eru heilar tölur og \(q \neq 0\). Talan \(p\) nefnist \sphinxstyleemphasis{teljari} brotsins en talan \(q\) \sphinxstyleemphasis{nefnari} þess.

\begin{sphinxadmonition}{note}{Athugasemd:}
Allar heilar tölur eru ræðar tölur með nefnarann \(1\), til dæmis er \(3= \frac31\).
\end{sphinxadmonition}


\subsection{Fullstytt brot}
\label{\detokenize{Kafli01:fullstytt-brot}}
Ef \(a\), \(b\) og \(t\) eru heilar tölur gildir
\begin{equation*}
\begin{split}\frac{at}{bt}=\frac{a}{b}\end{split}
\end{equation*}
Þegar við styttum töluna \(t\) út tölum við um að \textit{stytta} brotið og þegar við margföldum með \(t\) fyrir ofan og neðan strik tölum við um að \textit{lengja} brotið.

\begin{sphinxadmonition}{tip}{Dæmi:}
Skoðum brotið \(\frac{9}{21}\). Við getum lengt brotið með \(2\) með því að margfalda með \(2\) fyrir ofan og neðan strik
\begin{equation*}
\begin{split}\frac{9}{21}=\frac{9 \cdot 2}{21 \cdot 2}=\frac{18}{42}\end{split}
\end{equation*}
Við getum einnig stytt brotið \(\frac{9}{21}\) með því að athuga að \(9=3 \cdot 3\) og \(21=3 \cdot 7\). Þá fæst
\begin{equation*}
\begin{split}\frac{9}{21}=\frac{3 \cdot 3}{3 \cdot 7}=\frac{3}{7}\end{split}
\end{equation*}\end{sphinxadmonition}

\begin{sphinxadmonition}{note}{Athugasemd:}
Við sjáum að sama brotið er hægt að skrifa á margan hátt. Þess vegna er góð venja að \sphinxstyleemphasis{fullstytta} brotið. Við segjum að brot sé \sphinxstyleemphasis{fullstytt} ef við getum ekki stytt það frekar. Til þess að fullstytta brot er hægt að frumþátta bæði nefnara og teljara og stytta út sameiginlega frumþætti.
\end{sphinxadmonition}

\begin{sphinxadmonition}{tip}{Dæmi:}
Fullstyttum brotið \(\frac{525}{980}\).

Frumþáttum tölurnar eins og lýst er í kaflanum á undan. Við fáum \(525=3 \cdot 5 \cdot 5 \cdot 7\) og \(980=2 \cdot 2 \cdot 5 \cdot 7 \cdot 7\).

Nú getum við ritað
\begin{equation*}
\begin{split}\frac{525}{980}=\frac{3 \cdot 5 \cdot 5 \cdot 7}{2 \cdot 2 \cdot 5 \cdot 7 \cdot 7}\end{split}
\end{equation*}
og við sjáum því að við gettum stytt út eina fimmu og eina sjöu. Eftir stendur þá brotið
\begin{equation*}
\begin{split}\frac{525}{980}=\frac{3 \cdot 5 \cdot 5 \cdot 7}{2 \cdot 2 \cdot 5 \cdot 7 \cdot 7} = \frac{3 \cdot 5}{2 \cdot 2 \cdot 7}\end{split}
\end{equation*}
sem hefur enga sameiginlega frumtölu fyrir ofan og neðan strik.
Margföldum nú tölurnar saman og fáum \(\frac{3 \cdot 5}{2 \cdot 2 \cdot 7}=\frac{15}{28}\), sem er því brotið fullstytt.
\begin{equation*}
\begin{split}\frac{525}{980}=\frac{15}{28}\end{split}
\end{equation*}\end{sphinxadmonition}

\begin{sphinxadmonition}{warning}{Aðvörun:}
Stærð almenns brots breytist ekki við lengingu eða styttingu.
\end{sphinxadmonition}


\subsection{Röðun ræðra talna}
\label{\detokenize{Kafli01:roun-raera-talna}}
Þegar fjallað er um mengi heilla talna hafa flestir mjög skýra hugmynd um hvað það þýðir að ein tala sé stærri en önnur. Við vitum til dæmis að \(3468\) er stærri en \(2497\) og við skrifum \(3468>2497\).

Í mengi ræðra talna eru hlutir ekki jafn einfaldir. Til dæmis þykir ekki augljóst hvor af tölunum
\(\frac{13}{512}\) og \(\frac{26}{1023}\) er stærri.
Ein leiðin væri einfaldlega að slá báðar tölurnar inn í vasareikni og sjá að
\begin{equation*}
\begin{split}\frac{13}{512}\approx 0,02539 \quad \text{og} \quad \frac{26}{1023}\approx 0,02542.\end{split}
\end{equation*}
Þá sést að
\begin{equation*}
\begin{split}\frac{26}{1023}>\frac{13}{512} \quad \text{því að} \quad 0,02542>0,02539.\end{split}
\end{equation*}
Við viljum hins vegar geta reiknað hlutina á blaði án vasareiknis.

Ef tvö brot hafa sama nefnara er auðvelt að skera úr um hvort þeirra er stærra.
Við vitum að brotið \(\frac{p}{s}\) er stærra en \(\frac{q}{s}\) og skrifum
\begin{equation*}
\begin{split}\frac{p}{s}>\frac{q}{s} \quad \text{ef} \quad p>q.\end{split}
\end{equation*}
Þetta getum við sagt því að brotin hafa sameiginlegan nefnara \(s\). Til dæmis er \(\frac{7}{3}\) stærra en \(\frac{6}{3}\) því bæði brotin hafa nefnarann \(3\).
Þessa staðreynd getum við nýtt okkur þegar reynt er að meta flóknari brot.

Til þess að geta lagt tvö brot saman, eða fundið mismun þeirra, þurfum við að gera þau samnefnd, það er, við þurfum  þannig að þau hafi sömu tölu í nefnara. Þetta er hægt að gera með því að lengja brotin.

Fyrir tvö brot \(\dfrac{a}{b}\) og \(\dfrac{c}{d}\) getum við til dæmis gert þau samnefnd með því að lengja það fyrra með \(d\) og það seinna með \(b\). Þá hafa þau bæði sama nefnarann, sem er \(b \cdot d\).

\begin{sphinxadmonition}{tip}{Dæmi:}
Hvort brotið er stærra \(\frac{3}{4}\) eða \(\frac{5}{12}\)?

Lengjum fyrra brotið með \(12\) og það seinna með \(4\)
\begin{equation*}
\begin{split}\begin{aligned}
        \frac{3}{4} &= \frac{3\cdot 12}{4\cdot 12}=\frac{36}{48} \\
        &\text{og} \\
        \frac{5}{12} &= \frac{5\cdot 4}{12\cdot 4}=\frac{20}{48}
\end{aligned}\end{split}
\end{equation*}
Nú hafa brotin sama jákvæða nefnara, \(36>20\) og því segjum við að \(\frac{3}{4}>\frac{5}{12}\).
\end{sphinxadmonition}


\subsection{Reiknireglur}
\label{\detokenize{Kafli01:id1}}
Í mengi ræðra talna má leggja saman, draga frá, margfalda og deila. Þessar aðgerðir eru framkvæmdar svona:
\begin{enumerate}
\sphinxsetlistlabels{\arabic}{enumi}{enumii}{}{.}%
\item {} 
\end{enumerate}
\begin{equation*}
\begin{split}\frac{p}{q}+\frac{r}{s}=\frac{ps}{qs}+\frac{qr}{qs}=\frac{ps+qr}{qs}\end{split}
\end{equation*}\begin{enumerate}
\sphinxsetlistlabels{\arabic}{enumi}{enumii}{}{.}%
\setcounter{enumi}{1}
\item {} 
\end{enumerate}
\begin{equation*}
\begin{split}\frac{p}{q}-\frac{r}{s}=\frac{ps-qr}{qs}\end{split}
\end{equation*}\begin{enumerate}
\sphinxsetlistlabels{\arabic}{enumi}{enumii}{}{.}%
\setcounter{enumi}{2}
\item {} 
\end{enumerate}
\begin{equation*}
\begin{split}\frac{p}{q} \cdot \frac{r}{s}=\frac{pr}{qs}\end{split}
\end{equation*}\begin{enumerate}
\sphinxsetlistlabels{\arabic}{enumi}{enumii}{}{.}%
\setcounter{enumi}{3}
\item {} 
\end{enumerate}
\begin{equation*}
\begin{split}\frac{p/q}{r/s} =\frac{ps}{qr}\end{split}
\end{equation*}
\begin{sphinxadmonition}{warning}{Aðvörun:}
Athugið að það má alls ekki stytta út lið! Dæmi:
\begin{equation*}
\begin{split}\frac{x+3}{3} \neq x\end{split}
\end{equation*}\end{sphinxadmonition}

\begin{sphinxadmonition}{tip}{Dæmi:}
\sphinxstylestrong{1.} Leggjum saman brotin \(\frac{7}{11}\) og \(\frac{10}{13}\).
\begin{quote}

Gerum brotin fyrst samnefnd með því að lengja fyrra brotið með nefnara seinna brotsins og seinna brotið með nefnara fyrra brotsins.
Að því loknu er lítið mál að leggja brotin saman með því að leggja saman teljarana
\begin{quote}
\begin{equation*}
\begin{split}\begin{aligned}
  \frac{7}{11}+\frac{10}{13}&=\frac{7 \cdot 13}{11 \cdot 13}+ \frac{11 \cdot 10}{13 \cdot 11}\\
   &= \frac{91}{143}+\frac{110}{143}\\
   &=\frac{91+110}{143}\\
   &=\frac{201}{143}
\end{aligned}\end{split}
\end{equation*}\end{quote}
\end{quote}

\sphinxstylestrong{2.} Leggjum saman brotin \(\frac{2}{7}\) og \(\frac{3}{2}\).
\begin{quote}

Eins og í dæmi \sphinxstylestrong{1} gerum við brotin fyrst samnefnd með því að lengja fyrra brotið með nefnara seinna brotsins og seinna brotið með nefnara fyrra brotsins.
Teljarar brotanna eru síðan lagðir saman
\begin{quote}
\begin{equation*}
\begin{split}\begin{aligned}
\frac{2}{7}+\frac{3}{2}&=\frac{2 \cdot 2}{7 \cdot 2} + \frac{7 \cdot 3}{7 \cdot 2}\\
&=\frac{4}{14}+\frac{21}{14}\\
&=\frac{4+21}{14}\\
&=\frac{25}{14}
\end{aligned}\end{split}
\end{equation*}\end{quote}
\end{quote}

\sphinxstylestrong{3.} Margföldum saman brotin \(\frac{11}{9}\) og \(\frac{7}{5}\).
\begin{quote}

Þegar brot eru margfölduð saman eru teljararnir margfaldaðir saman og nefnararnir margfaldaðir saman.
\begin{quote}
\begin{equation*}
\begin{split}\frac{11}{9} \cdot \frac{7}{5}=\frac{11 \cdot 7}{9 \cdot 5}=\frac{77}{45}\end{split}
\end{equation*}\end{quote}
\end{quote}

\sphinxstylestrong{4.} Deilum brotinu \(\frac{11}{45}\) með brotinu \(\frac{1}{2}\).
\begin{quote}
\begin{equation*}
\begin{split}\frac{11/45}{1/2}=\frac{11 \cdot 2}{45 \cdot 1}=\frac{22}{45}\end{split}
\end{equation*}
\begin{sphinxadmonition}{warning}{Aðvörun:}
Nefnari í nefnara verður teljari!
\end{sphinxadmonition}
\end{quote}
\end{sphinxadmonition}

\begin{sphinxadmonition}{tip}{Dæmi:}\begin{quote}

Reiknum úr þessu broti og fullstyttum það síðan:
\begin{equation*}
\begin{split}\frac{\left(\frac{1}{4}+\frac{2}{3} \right)\cdot\frac{3}{2}}{5/2}\end{split}
\end{equation*}\end{quote}

\sphinxstyleemphasis{Lausn:}

Byrjum á að leggja saman brotin í sviganum:
\begin{equation*}
\begin{split}\begin{aligned}
 \frac{\left(\frac{1}{4}+\frac{2}{3} \right)\cdot\frac{3}{2}}{5/2} &= \frac{\frac{1\cdot 3+4\cdot 2}{4\cdot 3}\cdot\frac{3}{2}}{5/2} \\
 &= \frac{\frac{11}{12}\cdot\frac{3}{2}}{5/2} \\
\end{aligned}\end{split}
\end{equation*}
Margföldum síðan saman brotin í teljaranum:
\begin{equation*}
\begin{split}\begin{aligned}
             \frac{\frac{11}{12}\cdot\frac{3}{2}}{5/2}&= \frac{(11\cdot 3)/(12\cdot 2)}{5/2} \\
             &= \frac{33/24}{5/2} \\
\end{aligned}\end{split}
\end{equation*}
Deilum nú brotinu í teljaranum með brotinu í nefnaranum:
\begin{equation*}
\begin{split}\begin{aligned}
\frac{33/24}{5/2} &= \frac{33\cdot 2}{24\cdot 5} \\
&= \frac{66}{120}
\end{aligned}\end{split}
\end{equation*}
Fullstyttum svo brotið:
\begin{quote}
\begin{equation*}
\begin{split}\begin{aligned}
\frac{66}{120} &= \frac{2\cdot 3\cdot 11}{2\cdot 2 \cdot 2 \cdot 3\cdot 5} \\
&= \frac{11}{2\cdot 2\cdot 5} \\
&= \frac{11}{20}
\end{aligned}\end{split}
\end{equation*}\end{quote}
\end{sphinxadmonition}


\section{Deiling með afgangi}
\label{\detokenize{Kafli01:deiling-me-afgangi}}
Látum \(a\) og \(b\) vera \textit{heiltölur}.
Það er ekki alltaf hægt að deila heilli tölu \(a\) með heilli tölu \(b\) og fá út aðra heila tölu.
Við getum hins vegar notað \sphinxstyleemphasis{deilingu með} \textit{afgangi}.
Hún gengur út á að finna tvær heiltölur \(x\) og \(y\) þannig að
\begin{equation*}
\begin{split}a=bx+y\end{split}
\end{equation*}
þar sem \(y\) er jákvæð tala og \(y<b\).

Við segjum að \(b\) gangi \(x\) sinnum upp í \(a\) með afgang \(y\).

Ef afgangurinn er \(0\) þá segjum við að \(b\) gangi upp í \(a\).

\begin{sphinxadmonition}{tip}{Dæmi:}
\sphinxstylestrong{1.} Deilum tölunni \(a=81\) með tölunni \(b=8\) með afgangi.
\begin{quote}

Athugum að \(8 \cdot 10=80\) en \(8 \cdot 11=88\). Við leitum að stærstu tölu sem er margfeldi af \(8\) og er minni en eða jöfn 81, þess vegna notum við \(10\) en ekki \(11\). Afgangurinn er síðan \(1\), þ.e.a.s. við getum ritað
\begin{equation*}
\begin{split}81=8 \cdot 10 + 1\end{split}
\end{equation*}
Því er \(x=10\) og \(y=1\). Við getum nú sagt að \(8\) gangi \(10\) sinnum upp í \(81\) með afgang \(1\).
\end{quote}

\sphinxstylestrong{2.} Deilum tölunni \(a=79\) með tölunni \(b=8\) með afgangi.
\begin{quote}

Athugum að \(8 \cdot 9=72\) en \(8 \cdot 10=80\). Afgangurinn er nú \(79-72=7\). Því getum við ritað
\begin{equation*}
\begin{split}79=8 \cdot 9 + 7\end{split}
\end{equation*}
en hér er \(x=9\) og \(y=7\). Við getum nú sagt að \(8\) gangi \(9\) sinnum upp í \(79\) með afgang \(7\).
\end{quote}
\end{sphinxadmonition}

\begin{sphinxadmonition}{note}{Athugasemd:}
Afgangurinn er alltaf minni en \(b\). Ef afgangurinn er stærri en (eða jafn) \(b\) þá getum við notað stærra \(x\).
\end{sphinxadmonition}

Stundum, til dæmis í forritun, þurfum við að reikna með afgangi.
Þá tölum við um \textit{módulus} eða leifastofn („\(a\) módulus \(b\) er \(y\)“ ) og táknum það með
\begin{equation*}
\begin{split}a\mod b = y.\end{split}
\end{equation*}
Þegar talan \(b\) gengur upp í \(a\) þá er módulus þeirra núll.

\begin{sphinxadmonition}{tip}{Dæmi:}\begin{equation*}
\begin{split}\begin{aligned}
  6 \mod \; 3 &=0 \quad & \text{ því } \quad 6 = 3\cdot 2 + 0\\
  5 \mod \; 3 &=2 \quad & \text{ því } \quad 5 = 3\cdot 1 + 2\\
  4 \mod \; 3 &=1 \quad & \text{ því } \quad 4 = 3\cdot 1 + 1 \\
  3 \mod \; 3 &=0 \quad & \text{ því } \quad 3 = 3\cdot 1 + 0 \\
\end{aligned}\end{split}
\end{equation*}\end{sphinxadmonition}


\section{Veldi og rætur}
\label{\detokenize{Kafli01:veldi-og-raetur}}

\subsection{Veldi}
\label{\detokenize{Kafli01:veldi}}

\bigskip\hrule\bigskip


\textit{Veldi} er þegar tala er margfölduð með sjálfri sér.
Látum \(a\) vera tölu og \(n>0\) vera heiltölu. Við skilgreinum
\begin{equation*}
\begin{split}\begin{aligned}
      a^0&=1 \\
      a^n&=a \cdot a \cdot \dots \cdot a \qquad (n\text{-sinnum}) \\
      a^{-n}&=\frac{1}{a^n}
\end{aligned}\end{split}
\end{equation*}
Talan \(a\) í rithættinum \(a^n\) nefnist \textit{veldisstofn} og talan \(n\) nefnist \textit{veldisvísir}.

\begin{sphinxadmonition}{note}{Athugasemd:}
Við segjum að \(a\) sé í \(n\)\sphinxhyphen{}ta veldi þegar \(a^n\).
\end{sphinxadmonition}

\begin{sphinxadmonition}{tip}{Dæmi:}
\sphinxstylestrong{1.}
\begin{equation*}
\begin{split}a \cdot a \cdot a \cdot a \cdot a=a^5\end{split}
\end{equation*}
\sphinxstylestrong{2.}
\begin{equation*}
\begin{split}4^3=4 \cdot 4 \cdot 4=64\end{split}
\end{equation*}
\sphinxstylestrong{3.}
\begin{equation*}
\begin{split}4^{-3}=\frac{1}{4^3}=\frac{1}{64}\end{split}
\end{equation*}\end{sphinxadmonition}


\subsection{Reiknireglur fyrir veldi}
\label{\detokenize{Kafli01:reiknireglur-fyrir-veldi}}
Höfum eftirfarandi reiknireglur fyrir veldi:
\begin{equation*}
\begin{split}\begin{aligned}
a^n\cdot a^m&=a^{n+m} &\\
\dfrac {a^n}{a^m}&=a^{n-m}&\\
a^n\cdot b^n&=(ab)^n&\\
(a^n)^m&=a^{nm} &\\
(-1)^n &= 1 &\quad \text{þegar } n \text{ er slétt tala} \\
(-1)^n &= -1& \quad \text{þegar } n \text{ er oddatala} \\
\end{aligned}\end{split}
\end{equation*}
\begin{sphinxadmonition}{tip}{Dæmi:}
\sphinxstylestrong{1.}
\begin{equation*}
\begin{split}2^2 \cdot 2^3=2^{2+3}=2^5\end{split}
\end{equation*}
\sphinxstylestrong{2.}
\begin{equation*}
\begin{split}\frac{2^{62}}{2^{60}}=2^{62-60}=2^{2}=4\end{split}
\end{equation*}
\sphinxstylestrong{3.}
\begin{equation*}
\begin{split}2^5 \cdot 36^5=(2 \cdot 36)^5=72^5\end{split}
\end{equation*}
\sphinxstylestrong{4.}
\begin{equation*}
\begin{split}(2^2)^3=2^{2 \cdot 3}=2^6=64\end{split}
\end{equation*}
\sphinxstylestrong{5.}
\begin{equation*}
\begin{split}2^2\cdot 2^3\cdot 5^5=2^{(2+3)}\cdot 5^5=2^5\cdot 5^5=(2\cdot 5)^5=10^5=100000.\end{split}
\end{equation*}
\sphinxstylestrong{6.} Einföldum
\begin{equation*}
\begin{split}(a^{x+y})^z (a^{x-z})^y.\end{split}
\end{equation*}
Við beitum veldareglunum og fáum:
\begin{equation*}
\begin{split}(a^{x+y})^z (a^{x-z})^y = a^{zx + zy}a^{yx - zy} = a^{zx + zy + yx - zy} = a^{zx + yx} = (a^{z+y})^x.\end{split}
\end{equation*}\end{sphinxadmonition}


\subsection{Rætur}
\label{\detokenize{Kafli01:raetur}}

\bigskip\hrule\bigskip


Látum \(q\) vera jákvæða heiltölu og \(a\) vera jákvæða tölu. Þá er til nákvæmlega ein tala \(x \geq 0\) þannig að \(x^q=a\). Þessi tala nefnist \(q\)\sphinxhyphen{}ta \textit{rótin} af \(a\) og er táknuð með
\begin{equation*}
\begin{split}\sqrt[q]{a}\end{split}
\end{equation*}
Við skrifum þó yfirleitt ekki \(\sqrt[2]{a}\) heldur \(\sqrt{a}\) og nefnum þessa stærð \textit{kvaðratrót}, oft kölluð ferningsrót.

\begin{sphinxadmonition}{tip}{Dæmi:}
\sphinxstylestrong{1.} \(\sqrt{4}=2 \quad\) því \(\quad 2^2=4\)

\sphinxstylestrong{2.} \(\sqrt{64}=8 \quad\) því \(\quad 8^2=64\)

\sphinxstylestrong{3.} \(\sqrt[3]{27}=3 \quad\) því \(\quad 3^3=27\)

\sphinxstylestrong{4.} \(\sqrt[4]{16}=2 \quad\) því \(\quad 2^4=16\).
\end{sphinxadmonition}


\subsection{Reiknireglur fyrir rætur}
\label{\detokenize{Kafli01:reiknireglur-fyrir-raetur}}
Höfum eftirfarandi reiknireglur fyrir rætur:
\begin{equation*}
\begin{split}      \begin{aligned}
        \sqrt[q]{ab} &=\sqrt[q]{a}\cdot \sqrt[q]{b} \\
  & \qquad \\
  \sqrt[q]{\dfrac ab}& =\dfrac{\sqrt[q]{a}}{\sqrt[q] {b}}\\
  & \qquad \\
  \sqrt[q]{a^p}& =(\sqrt[q]{a})^p\\
  & \qquad \\
  \sqrt[sq]{a^{sp}} &={\sqrt[q]{a^p}}\\
  & \qquad \\
  \sqrt[sq]{ a} &=\sqrt[s]{\sqrt[q]{a}}\\
\end{aligned}\end{split}
\end{equation*}
\begin{sphinxadmonition}{note}{Athugasemd:}
Rætur virða ekki samlagningu, þ.e.a.s. almennt er \(\sqrt[q]{a+b} \neq \sqrt[q]{a}+ \sqrt[q]{b}\).
\end{sphinxadmonition}

\begin{sphinxadmonition}{tip}{Dæmi:}
\sphinxstylestrong{1.} \(\sqrt[4]{2} \cdot \sqrt[4]{8}= \sqrt[4]{2 \cdot 8}= \sqrt[4]{16}=2\)

\sphinxstylestrong{2.} \(\frac{\sqrt[3]{135}}{\sqrt[3]{5}}=\sqrt[3]{\frac{135}{5}}=\sqrt[3]{27}=3\)

\sphinxstylestrong{3.} \(\sqrt[4]{256} = \sqrt[4]{16^2}=\left(\sqrt[4]16\right)^2=2^2=4\)
\end{sphinxadmonition}


\subsection{Brotin veldi}
\label{\detokenize{Kafli01:brotin-veldi}}
Þegar tala hefur ræða tölu í veldisvísi sem er ekki heiltala tölum við um brotið veldi. Látum \(r\) vera ræða tölu og skrifum hana \(r=\frac{p}{q}\) þar sem \(p\) og \(q\) eru heilar tölur, \(q>0\). Þá er
\begin{equation*}
\begin{split}a^r=a^{\frac{p}{q}}=\sqrt[q]{a^p}\end{split}
\end{equation*}
Það er, við skilgreinum \(a^{\frac{p}{q}}\) þannig að
\begin{equation*}
\begin{split}a^{\frac{p}{q}}=\sqrt[q]{a^p}\end{split}
\end{equation*}
Veldareglurnar gilda einnig fyrir ræða veldisvísa. Athugum að þriðja veldareglan segir að
\begin{equation*}
\begin{split}a^r=\sqrt[q]{a^p}=(\sqrt[q]{a})^p.\end{split}
\end{equation*}
Einnig er
\begin{equation*}
\begin{split}\sout{14}\end{split}
\end{equation*}\begin{equation*}
\begin{split}a^{\frac{1}{q}}=\sqrt[q]{a}\end{split}
\end{equation*}
fyrir allar heiltölur \(q\).

\begin{sphinxadmonition}{tip}{Dæmi:}
\sphinxstylestrong{1.} \(\sqrt{a}=a^{\frac12}\)

\sphinxstylestrong{2.} \(9^{\frac12}=\sqrt{9}=3\)

\sphinxstylestrong{3.} \(32^{\frac25}=\sqrt[5]{32^2}=\left(\sqrt[5]{32}\right)^2=2^2=4\)

\sphinxstylestrong{4.} \(8^{\frac{1}{3}}=\sqrt[3]8 = 2\)

\sphinxstylestrong{5.} \(16^{\frac{3}{4}}=\sqrt[4]{(16^3)}=\left(\sqrt[4]{16}\right)^3=2^3=8\)

\sphinxstylestrong{6.} Látum \(a,b > 0\). Einföldum
\begin{equation*}
\begin{split}\sqrt[3]{\left(\sqrt{a}\right)^3 \left(\sqrt[4]{b}\right)^6 }\end{split}
\end{equation*}
Hér beitum við reiknireglum fyrir rætur:
\begin{align*}\!\begin{aligned}
\begin{aligned}\\
\sqrt[3]{\left(\sqrt{a}\right)^3 \left(\sqrt[4]{b}\right)^6 }
&= \sqrt[3]{\left(\sqrt{a}\right)^3 } \sqrt[3]{ \left(\sqrt[4]{b}\right)^6 } \\
&= \sqrt{a} \left(\sqrt[4]{b}\right)^2 \\
&= \sqrt{a} \cdot b^{\frac24} \\
&=\sqrt{a} \cdot b^{\frac12} \\
&=\sqrt{a} \sqrt{b} = \sqrt{ab}\\\\
\end{aligned}\\
\end{aligned}\end{align*}\end{sphinxadmonition}


\chapter{Jöfnur og ójöfnur}
\label{\detokenize{Kafli02:jofnur-og-ojofnur}}\label{\detokenize{Kafli02::doc}}

\bigskip\hrule\bigskip



\section{Jöfnur}
\label{\detokenize{Kafli02:jofnur}}\phantomsection\label{\detokenize{Kafli02:s-jofnur}}
\textit{Jafna} lýsir sambandi milli stæða.
Jafnaðarmerkið \(=\) merkir að stæðurnar sem standa beggja vegna jafnaðarmerkisins séu sama talan.
Til dæmis, ef við höfum jöfnuna \(x+2=4\) sjáum við að jafnan gengur upp ef \(x=2\) .

Við megum beita reikniaðgerðum þegar við vinnum með jöfnur, svo lengi sem við \sphinxstylestrong{gerum það sama báðum megin jafnaðarmerkis}.

Byrjum á því að skoða jöfnur af einni breytistærð. \textit{Stig} stæðu er jafnt hæsta veldi \textit{breytistærðarinnar}. Þá er \(x+2=0\) fyrsta stigs jafna, \(x^2+2x+1=0\) er annars stigs jafna, \(x^3+2x^2+3x=0\) er þriðja stigs jafna og svo framvegis.


\subsection{Fyrsta stigs jöfnur}
\label{\detokenize{Kafli02:fyrsta-stigs-jofnur}}
\sphinxstyleemphasis{Fyrsta stigs jafna} hefur almennt form
\begin{equation*}
\begin{split}ax+b=0\end{split}
\end{equation*}
þar sem \(a \neq 0\) og \(b\) eru tölur. Jafnan hefur eina lausn, sem fæst með því að draga \(b\) frá báðum megin við jafnaðarmerkið og deila svo með \(a\). Við köllum þetta að \sphinxstyleemphasis{einangra} \(x\). Lausnin er því \(x=\frac{-b}{a}\).

\begin{sphinxadmonition}{note}{Athugasemd:}
Fyrsta stigs jafna er ekki alltaf á forminu \(ax+b=0\), en við getum alltaf komið henni yfir á það form með reikniaðgerðum. Til dæmis er \(5x=-2\) ekki á þessu formi en við getum lagt \(2\) við báðum megin og fengið \(5x+2=0\).
\end{sphinxadmonition}

\begin{sphinxadmonition}{tip}{Dæmi:}
\sphinxstylestrong{1.} Leysum \(2x+3=0\). Drögum \(3\) frá báðum megin og fáum \(2x=-3\). Deilum með \(2\) og fáum \(x=-\frac{3}{2}\).

\sphinxstylestrong{2.} Leysum \(17x=5\). Til þess að einangra \(x\) þurfum við nú bara að deila í gegn með \(17\). Fáum þá \(x=\frac{5}{17}\).
\end{sphinxadmonition}


\subsection{Annars stigs jöfnur}
\label{\detokenize{Kafli02:annars-stigs-jofnur}}\phantomsection\label{\detokenize{Kafli02:s-annars-stigs-jofnur}}
\textit{Annars stigs jafna} hefur almennt form
\begin{equation*}
\begin{split}ax^2+bx+c=0\end{split}
\end{equation*}
þar sem \(a\), \(b\) og \(c\) eru \textit{fastar}.

\begin{sphinxadmonition}{warning}{Aðvörun:}
Athugið b og c geta verið hvaða tölur sem er, en \(a \neq 0\) því annars væri jafnan fyrsta stigs jafna.
\end{sphinxadmonition}

Annars stigs jafna er ekki alltaf á almennu formi, en við getum komið henni á almennt form með reikniaðgerðum.

\begin{sphinxadmonition}{tip}{Dæmi:}
\sphinxstylestrong{1.} Jafnan \(x^2+3x=4x+1\) er ekki á almennu formi, en með því að beita reikniaðgerðum og ,,færa yfir jafnaðarmerkið‘‘ fæst \(x^2-x+1=0\).
Við sjáum því að þetta er annars stigs jafna með \(a=1\), \(b=-1\) og \(c=1\).

\sphinxstylestrong{2.} Jafnan \(2x(3-x)=2x\) er ekki á almennu formi. Ef við margföldum upp úr sviganum fæst \(6x-2x^2=2x\), sem er jafngilt \(4x-2x^2=0\).
Við getum víxlað röðinni og fengið \(-2x^2+4x=0\) og þá er auðvelt að sjá að þetta er annars stigs jafna með \(a=-2\), \(b=4\) og \(c=0\).

\sphinxstylestrong{3.} Jafnan \(x^2+3x=x^2-4\) er ekki á almennu formi. Með því að ,,færa yfir jafnaðarmerkið‘‘ fæst \(x^2+3x-x^2+4=0\) eða \(3x+4=0\).
Við sjáum því að þetta er ekki annars stigs jafna, heldur fyrsta stigs.
\end{sphinxadmonition}


\subsection{Lausnarformúla fyrir annars stigs jöfnur}
\label{\detokenize{Kafli02:lausnarformula-fyrir-annars-stigs-jofnur}}
Við höfum lausnarformúlu fyrir annars stigs jöfnur á almennu formi. Þær geta haft eina, tvær eða enga lausn.


\subsubsection{Setning}
\label{\detokenize{Kafli02:setning}}
Látum \(ax^2+bx+c=0\) vera annars stigs jöfnu. \textit{Aðgreinirinn} er \(d = b^2-4ac\).
\begin{enumerate}
\sphinxsetlistlabels{\arabic}{enumi}{enumii}{}{.}%
\item {} 
Ef \(d = b^2-4ac<0\) þá hefur jafnan enga rauntölulausn.

\item {} 
Ef \(d  = b^2-4ac=0\) þá hefur jafnan eina lausn:

\end{enumerate}
\begin{equation*}
\begin{split}x=\frac{-b}{2a}\end{split}
\end{equation*}\begin{enumerate}
\sphinxsetlistlabels{\arabic}{enumi}{enumii}{}{.}%
\setcounter{enumi}{2}
\item {} 
Ef \(d = b^2-4ac>0\) þá hefur jafnan tvær lausnir:

\end{enumerate}
\begin{equation*}
\begin{split}x_1=\frac{-b+\sqrt{b^2-4ac}}{2a} \qquad \text{og} \qquad x_2=\frac{-b-\sqrt{b^2-4ac}}{2a}\end{split}
\end{equation*}
\begin{sphinxadmonition}{note}{Athugasemd:}
Oft er almenna formúlan rituð
\begin{equation*}
\begin{split}x=\frac{-b\pm\sqrt{b^2-4ac}}{2a}\end{split}
\end{equation*}\end{sphinxadmonition}

\begin{sphinxadmonition}{warning}{Aðvörun:}
Áður en við notum þessa lausnarformúlu þurfum við að vera viss um að jafnan sé á almenna forminu \(ax^2+bx+c=0\).
\end{sphinxadmonition}

\begin{sphinxadmonition}{tip}{Dæmi:}
\sphinxstylestrong{1.} Leysum jöfnuna \(2x^2 + 3x - 5 = 2\).
\begin{quote}

Við byrjum á að koma jöfnunni á almennt form með því að draga 2 frá beggja vegna jafnaðarmerkis, fáum \(2x^2 + 3x - 7=0\).
Sjáum því að hér er \(a=2\), \(b=3\) og \(c=-7\).

Vitum að það eru tvær rauntölulausnir því \(d=(3)^2-4(2)(-7)=65 > 0\). Þá getum við notað lausnarformúlu annars stigs jöfnu.
\begin{equation*}
\begin{split}x = \frac{-3 \pm \sqrt{3^2 - 4 \cdot 2 \cdot (-7)}}{2 \cdot 2}
= \frac{-3 \pm \sqrt{65}}{4}.\end{split}
\end{equation*}
Lausnirnar eru \(x_1=\frac{-3 + \sqrt{65}}{4}\) og \(x_2=\frac{-3 - \sqrt{65} }{4}\).
\end{quote}

\sphinxstylestrong{2.} Leysum jöfnuna \(x^2-6(x-1)=-3\).
\begin{quote}

Byrjum á að koma jöfnunni á almennt form.
Margföldum upp úr sviganum og fáum \(x^2-6x+6=-3\) eða \(x^2-6x+9=0\).
Hér er því \(a=1\), \(b=-6\) og \(c=9\).
Vitum að það er bara ein lausn því \(d=(-6)^2-4(1)(9)=0\).
Notum lausnarformúluna:
\begin{equation*}
\begin{split}x=\frac{6 \pm \sqrt{(-6)^2-4 \cdot 1 \cdot 9}}{2}=\frac{6 \pm \sqrt{36-36}}{2}=\frac{6}{2}=3\end{split}
\end{equation*}
Jafnan hefur lausnina, \(x=3\).
\end{quote}

\sphinxstylestrong{3.} Leysum jöfnuna \(x^2+1=0\).
\begin{quote}

Hér er \(a=1\), \(b=0\) og \(c=1\).

Byrjum á að reikna út stærðina \(d=b^2-4ac\) og fáum \(0^2-4 \cdot 1 \cdot 1=-4\), svo \(b^2-4ac<0\) og jafnan hefur enga rauntölulausn.
\end{quote}
\end{sphinxadmonition}


\section{Liðun og þáttun}
\label{\detokenize{Kafli02:liun-og-attun}}

\subsection{Liðun}
\label{\detokenize{Kafli02:liun}}
\textit{Liðun} kallast það þegar stærðtákni sem samanstendur af einum lið er breytt í fleiri liði.
Við tölum oft um að „margfalda upp úr svigum“, til dæmis höfum við
\begin{equation*}
\begin{split}(a+b)(c+d)=a(c+d)+b(c+d)=ac+ad+bc+bd\end{split}
\end{equation*}
Í þessu tilviki þurfum við að margfalda báða liði fyrri svigans við báða liði seinni svigans, og við notum dreifireglu til þess. Höfum til dæmis
\begin{equation*}
\begin{split}\begin{aligned}
(x+2)(x+9)&=x \cdot x + x \cdot 9 + 2 \cdot x + 2 \cdot 9 \\
&= x^2 + 9x + 2x + 18\\
&= x^2 +11x +18
\end{aligned}\end{split}
\end{equation*}
Hér þarf að gæta þess að nota rétt formerki, höfum til dæmis
\begin{equation*}
\begin{split}\begin{aligned}
(x-2)(x+9)&=x \cdot x + x \cdot 9 + (-2) \cdot x + (-2) \cdot 9 \\
& = x^2 + 9x - 2x - 18 \\
& =x^2 +7x -18 \\
\end{aligned}\end{split}
\end{equation*}

\bigskip\hrule\bigskip


Eftirfarandi reglur eru mikilvægar í liðun:
\begin{equation*}
\begin{split}\begin{aligned}
& (a+b)^2=a^2+2ab+b^2 \qquad &\textit{(ferningsregla fyrir summu)} \\
& (a-b)^2=a^2-2ab+b^2 \qquad &\textit{(ferningsregla fyrir mismun)} \\
& (a+b)(a-b)=a^2-b^2 \qquad &\textit{(samokaregla)} \\
\end{aligned}\end{split}
\end{equation*}

\bigskip\hrule\bigskip


\begin{sphinxadmonition}{tip}{Dæmi:}
\sphinxstylestrong{1.} Liðum \((x+2)^2\).
\begin{quote}

Notum ferningsreglu fyrir summu. Þá fæst
\begin{equation*}
\begin{split}\begin{aligned}(x+2)^2&=x^2+2 \cdot 2 \cdot x + 2^2\\ &=x^2+4x+4 \end{aligned}\end{split}
\end{equation*}\end{quote}

\sphinxstylestrong{2.} Liðum \((x+6)(x-6)\).
\begin{quote}

Samokaregla gefur
\begin{equation*}
\begin{split}\begin{aligned} (x+6)(x-6)&=x^2-6^2\\ &=x^2-36 \end{aligned}\end{split}
\end{equation*}\end{quote}

\sphinxstylestrong{3.} Liðum \((x-1)(x+1)^2\).
\begin{quote}

Athugum að
\begin{equation*}
\begin{split}(x+1)^2=(x+1)(x+1)\end{split}
\end{equation*}
beitum samokareglu til að fá
\begin{quote}
\begin{equation*}
\begin{split}\begin{aligned} (x-1)(x+1)^2&=(x-1)(x+1)(x+1)\\ &=(x^2-1)(x+1)\\ \end{aligned}\end{split}
\end{equation*}\end{quote}

Nú þurfum við bara að margfalda úr svigunum og fáum þá
\begin{equation*}
\begin{split}\begin{aligned}
(x^2-1)(x+1)&=x^2 \cdot x + x^2 \cdot 1 - 1 \cdot x - 1 \cdot 1\\
&=x^3 + x^2 -x -1
\end{aligned}\end{split}
\end{equation*}\end{quote}

\sphinxstylestrong{4.} Liðum \((x+4)^2(x-4)^2\).
\begin{quote}

Höfum
\begin{quote}
\begin{equation*}
\begin{split}\begin{aligned} (x+4)^2(x-4)^2 &= ((x+4)(x-4))^2\\ & = (x^2 - 16)^2 \end{aligned}\end{split}
\end{equation*}\end{quote}

samkvæmt veldareglu og samokareglu. Notum nú ferningsreglu fyrir mismun og þá fæst
\begin{quote}
\begin{equation*}
\begin{split}\begin{aligned} (x^2 - 16)^2 &=(x^2)^2 - 2 \cdot 16x^2 + 16^2\\ &=x^4-32x^2+256\end{aligned}\end{split}
\end{equation*}\end{quote}
\end{quote}
\end{sphinxadmonition}


\subsection{Þáttun}
\label{\detokenize{Kafli02:attun}}
\textit{Þáttun} er \sphinxstylestrong{andhverf aðgerð} við liðun. Þá er stæðu með fleiri en einum lið breytt í jafngilda stæðu sem samanstendur aðeins af þáttum.
Til eru margar leiðir til að þátta. Við getum í sumum tilfellum notað ferningsreglu eða samokareglu, sem settar eru fram hér að ofan.

\begin{sphinxadmonition}{tip}{Dæmi:}
\sphinxstylestrong{1.} Þáttum \(x^2+4x+4\).
\begin{quote}

Notum ferningsreglu fyrir summu, það er,
\begin{equation*}
\begin{split}a^2+2ab+b^2=(a+b)^2\end{split}
\end{equation*}
Með samanburði sést að \(a=x\) og \(b=2\) gengur upp. Því er
\begin{equation*}
\begin{split}x^2+4x+4=(x+2)^2\end{split}
\end{equation*}\end{quote}

\sphinxstylestrong{2.} Þáttum \(x^2-1\).
\begin{quote}

Beitum samokareglu og fáum
\begin{equation*}
\begin{split}x^2-1 =(x+1)(x-1)\end{split}
\end{equation*}\end{quote}

\sphinxstylestrong{3.} Þáttum \(x^3-2x^2+x\).
\begin{quote}

Byrjum á því að athuga að \(x\) er sameiginlegur þáttur allra liðanna. Þáttum \(x\) út fyrir og fáum
\begin{equation*}
\begin{split}x(x^2-2x+1)\end{split}
\end{equation*}
Á seinni svigann getum við nú notað ferningsreglu fyrir mismun,
\begin{equation*}
\begin{split}a^2-2ab+b^2=(a-b)^2\end{split}
\end{equation*}
þar sem \(a=x\) og \(b=1\). Við höfum því
\begin{equation*}
\begin{split}x^3 - 2x^2 + x = x(x^2 - 2x + 1) = x(x-1)^2\end{split}
\end{equation*}\end{quote}
\end{sphinxadmonition}


\subsection{Þáttun með ágiskunaraðferð}
\label{\detokenize{Kafli02:attun-me-agiskunarafer}}
Ef við höfum stærð á forminu \(x^2+bx+c\) getum við beitt svokallaðari ágiskunaraðferð. Hún gengur út á að finna tvær tölur \(s\) og \(t\) þannig að summa þeirra sé jöfn \(b\) og margfeldi þeirra sé jafnt \(c\). Hún er kölluð ágiskunaraðferð þar sem við gætum þurft að prófa okkur áfram. Þess vegna getur verið gott að taka saman upplýsingar í töflu þannig að við byrjum á því að skrifa niður tvær tölur sem uppfylla það að margfeldi þeirra er jöfn síðustu tölunni. Þáttunin er síðan
\begin{equation*}
\begin{split}x^2+bx+c=(x+s)(x+t)\end{split}
\end{equation*}
þar sem \(s+t=b\) og \(s\cdot t =c\).
Þetta skýrist best með dæmum.

\begin{sphinxadmonition}{note}{Athugasemd:}
Tökum eftir að ef annars stigs jafnan er á forminu \(ax^2+bx+c\), þar sem \(a\neq 1\) þá tökum við \(a\) út fyrir eða deilum í gegn með \(a\) ef við erum að vinna með jöfnu (sjá dæmi \sphinxstylestrong{4} hér á eftir).
\end{sphinxadmonition}

\begin{sphinxadmonition}{tip}{Dæmi:}
\sphinxstylestrong{1.} Þáttum \(x^2+7x+12\).
\begin{quote}

Byrjum á að gera lista yfir tölur þannig að margfeldi þeirra sé \(12\). Fáum
\begin{equation*}
\begin{split}\begin{array}{c}
1 \text{ og } 12 \\
2 \text{ og } 6 \\
3 \text{ og } 4 \\
\end{array}\end{split}
\end{equation*}
Við getum séð að við erum komin með lausn með því að búa til dálk fyrir summu þessara tveggja talna
\begin{equation*}
\begin{split}\begin{array}{ | c | c | c | }
\hline
s & t & \text{Summa} \\
\hline
1 & 12 & 13\\
2 & 6 & 8\\
3 & 4  & 7 \\
\hline
\end{array}\end{split}
\end{equation*}
Við leitum að tveimur tölum sem hafa summuna \(7\). Við sjáum nú að tölurnar \(s = 3\) og \(t = 4\) uppfylla þetta skilyrði. Þáttunin er því
\begin{equation*}
\begin{split}x^2+7x+12=(x+3)(x+4)\end{split}
\end{equation*}\end{quote}

\sphinxstylestrong{2.} Skoðum næst dæmi þar sem við þurfum að huga að formerkjunum. Þáttum \(x^2-26x+25\).
\begin{quote}

Finnum tölur þannig að margfeldi þeirra sé \(25\). En við sjáum að summa þeirra er neikvæð og margfeldi þeirra er jákvæð. Því er ljóst að báðar tölurnar eru neikvæðar. Möguleikar eru
\begin{equation*}
\begin{split}\begin{array}{c}
-1 \text{ og } -25 \\
-5 \text{ og } -5 \\
\end{array}\end{split}
\end{equation*}
Leggjum tölurnar saman
\begin{equation*}
\begin{split}\begin{array}{ | c | c | c | }
\hline
s & t & \text{Summa} \\
\hline
-1 & -25 & -26\\
-5 & -5 & -10\\
\hline
\end{array}\end{split}
\end{equation*}
Sjáum að tölurnar \(s = -1\) og \(t= -25\) uppfylla skilyrðið svo þáttunin er
\begin{equation*}
\begin{split}x^2-26x+25=(x+(-1))(x+(-25))=(x-1)(x-25).\end{split}
\end{equation*}\end{quote}

\sphinxstylestrong{3.} Þáttum \(x^2-x-6\).
\begin{quote}

Þar sem síðasta talan er neikvæð leitum við að neikvæðri tölu og jákvæðri tölu. Möguleikar eru því
\begin{equation*}
\begin{split}\begin{array}{c}
-1 \text{ og } 6 \\
1 \text{ og } -6 \\
-2 \text{ og } 3 \\
2 \text{ og } -3 \\
\end{array}\end{split}
\end{equation*}
Leggjum tölurnar saman
\begin{equation*}
\begin{split}\begin{array}{ | c | c | c | }
\hline
s & t & \text{Summa} \\
\hline
-1 & 6 & 5\\
1 & -6 & -5 \\
-2 & 3 & 1\\
2 & -3 & -1\\
\hline
\end{array}\end{split}
\end{equation*}
Sjáum því að tölurnar \(s = 2\) og \(t = -3\) uppfylla skilyrðið og þáttunin er
\begin{equation*}
\begin{split}x^2-x-6=(x+2)(x-3).\end{split}
\end{equation*}\end{quote}

\sphinxstylestrong{4.} Þáttum \(2x^2+8x+6\).
\begin{quote}

Hér er \(a \neq 1\) svo við getum umritað \(2x^2+8x+6\) yfir í \(2(x^2+4x+3)\) og þáttað \(x^2+4x+3\). Sjáum að \(1\) og \(3\) uppfylla skilyrðin.
\begin{equation*}
\begin{split}\begin{array}{ | c | c | c | }
\hline
s & t & \text{Summa} \\
\hline
1 & 3 & 4\\
\hline
\end{array}\end{split}
\end{equation*}
Sjáum því að tölurnar \(s = 1\) og \(t = 3\) ganga en þá er þáttunin
\begin{quote}
\begin{equation*}
\begin{split}x^2+4x+3 = (x+1)(x+3)\end{split}
\end{equation*}\end{quote}

sem gefur okkur að \(2x^2+8x+6 = 2(x+1)(x+3)\).
\end{quote}
\end{sphinxadmonition}

\begin{sphinxadmonition}{note}{Athugasemd:}
Það er alls ekki nauðsynlegt að taka svona mörg skref við þáttun af þessu tagi.
Með æfingunni verður auðveldara að nota þessa aðferð í huganum og ekki er nauðsynlegt að sýna mörg skref.
Athugið að það getur komið í veg fyrir villur að sannreyna svarið, það er að liða stæðuna aftur.
\end{sphinxadmonition}


\subsection{Þáttun með lausnarformúlu}
\label{\detokenize{Kafli02:attun-me-lausnarformulu}}
Til þess að þátta stærð á forminu \(ax^2+bx+c\) getum við ávallt beitt lausnarformúlu fyrir annars stigs jöfnur. Munum að jafnan \(ax^2+bx+c=0\) hefur lausnir
\begin{equation*}
\begin{split}x_{1,2}=\frac{-b \pm \sqrt{b^2-4ac}}{2a}\end{split}
\end{equation*}
Þáttunin verður þá
\begin{equation*}
\begin{split}ax^2+bx+c=a(x-x_1)(x-x_2)\end{split}
\end{equation*}
\begin{sphinxadmonition}{tip}{Dæmi:}
\sphinxstylestrong{1.} Þáttum \(6x^2+5x-6\).
\begin{quote}

Finnum ræturnar. Hér er \(a=6\), \(b=5\) og \(c=-6\). Notum lausnarformúluna og fáum
\begin{equation*}
\begin{split}d = 5^2-4 \cdot 6 \cdot (-6) = 169 = 13^2\end{split}
\end{equation*}
og
\begin{equation*}
\begin{split}x_1 =\frac{-5+\sqrt{169}}{12}=\frac{-5+13}{12} =\frac{8}{12} =\frac{2}{3}\end{split}
\end{equation*}
og
\begin{equation*}
\begin{split}x_2=\frac{-5-13}{12}=-\frac{3}{2}\end{split}
\end{equation*}
Samkvæmt formúlu að ofan er þáttunin því \(6x^2+5x-6=a(x-x_1)(x-x_2)\), það er,
\begin{equation*}
\begin{split}\begin{aligned}
6x^2+5x-6&=6\left(x-\frac{2}{3}\right)\left(x-\left(-\frac{3}{2}\right)\right)\\
&=6\left(x-\frac{2}{3}\right)\left(x+\frac{3}{2}\right)
\end{aligned}\end{split}
\end{equation*}
Þetta svar er rétt, en við sjáum að við getum gert svarið snyrtilegra með því að athuga að \(6=2 \cdot 3\) og margfalda fyrri svigann með \(3\) og þann seinni með \(2\), það er,
\begin{equation*}
\begin{split}\begin{aligned}
6x^2+5x-6 &= 3 \left(x-\frac{2}{3}\right) \cdot 2\left(x+\frac{3}{2}\right)\\
&=(3x-2)(2x+3)
\end{aligned}\end{split}
\end{equation*}
svo þáttunin er
\begin{equation*}
\begin{split}6x^2+5x-6=(3x-2)(2x+3)\end{split}
\end{equation*}\end{quote}

\sphinxstylestrong{2.} Þáttum \(x^2+x-1\).
\begin{quote}

Hér er \(a=1\), \(b=1\) og \(c=-1\). Lausnarformúlan gefur
\begin{equation*}
\begin{split}x_1=\frac{-1+\sqrt{1^2-4 \cdot 1 \cdot (-1)}}{2 \cdot 1}=\frac{-1+\sqrt{5}}{2}\end{split}
\end{equation*}
og
\begin{equation*}
\begin{split}x_2=\frac{-1-\sqrt{5}}{2}\end{split}
\end{equation*}\end{quote}
\end{sphinxadmonition}

Fáum því að
\begin{quote}
\begin{equation*}
\begin{split}x^2+x-1=\left(x-\frac{-1+\sqrt{5}}{2}\right)\left(x-\frac{-1-\sqrt{5}}{2}\right)\end{split}
\end{equation*}\end{quote}


\section{Jöfnuhneppi}
\label{\detokenize{Kafli02:jofnuhneppi}}
Í sumum verkefnum koma fyrir fleiri en ein óþekkt stærð.
Til þess að leysa svoleiðis verkefni þurfum við að hafa jöfnur sem lýsa því hvernig stærðirnar tengjast.
Jöfnurnar þurfa að vera jafn margar og óþekktu stærðirnar.

Til eru tvær leiðir til þess að leysa svoleiðis verkefni, \textit{eyðingaraðferð} og \textit{innsetningaraðferð} .

Í \sphinxstylestrong{eyðingaraðferðinni} er reynt að eyða út einni óþekktri stærð úr jöfnuhneppi.
Til þess er önnur jafnan margfölduð með fasta svo stuðullinn fyrir framan stærðina sem á að eyða verði sá sami og stuðullinn í hinni jöfnunni.
Svo er önnur jafnan dregin frá hinni og útkoman er jafna án stærðarinnar sem eyða átti.

\begin{sphinxadmonition}{tip}{Dæmi:}
Leysum jöfnuhneppið
\begin{equation*}
\begin{split}\begin{aligned}
  x+2y&=3 \\
  2x-y &=11
\end{aligned}\end{split}
\end{equation*}
með eyðingaraðferð.

\sphinxstylestrong{Lausn:}

Stefnum á að eyða \(x\) . Það gerum við með því að marfalda efri jöfnuna með 2 (beggja vegna jafnaðarmerkis):
\begin{equation*}
\begin{split}\begin{aligned}
  2x+4y&=6 \\
  2x-y &=11
\end{aligned}\end{split}
\end{equation*}
Drögum nú neðri jöfnuna frá þeirri efri, hægri hlið efri frá hægri hlið neðri og vinstri hlið efri frá vinstri hlið neðri:
\begin{equation*}
\begin{split}\begin{aligned}
  (2x-y)-(2x+4y) &=11-6 \\
  -5y &= 5 \\
  y&=-1
\end{aligned}\end{split}
\end{equation*}
Stefnum nú á að eyða \(y\) úr upprunalega hneppinu:
\begin{equation*}
\begin{split}\begin{aligned}
  x+2y&=3 \\
  2x-y &=11
\end{aligned}\end{split}
\end{equation*}
Á sama hátt og áður margföldum við neðri jöfnuna með 2 og fáum:
\begin{equation*}
\begin{split}\begin{aligned}
  x+2y&=3 \\
  4x-2y &=22
\end{aligned}\end{split}
\end{equation*}
Leggjum nú jöfnurnar saman og fáum:
\begin{equation*}
\begin{split}\begin{aligned}
  (4x-2y )+ (x+2y) &=22 +3 \\
  5x &= 25 \\
  x&=5
\end{aligned}\end{split}
\end{equation*}
Lausn jöfnuhneppisins er því \(x=5\) og \(y=-1\) .
Myndrænt þýðir þetta að línurnar \(x+2y=3\) og \(2x-y =11\) skerist í punktinum \((5,-1)\) .
Farið er nánar í þetta í kaflanum um rúmfræði.
\end{sphinxadmonition}
\phantomsection\label{\detokenize{Kafli02:s-innsetning}}
\sphinxstylestrong{Innsetningaraðferðin} gengur út á að einangra eina breytistærðina út frá annarri jöfnunni og setja inn í hina.
Þá hefur sú jafna bara eina óþekkta stærð.

\begin{sphinxadmonition}{tip}{Dæmi:}
Leysum jöfnuhneppið
\begin{equation*}
\begin{split}\begin{aligned}
  2x+y&=2 \\
  3x+2y&=1 \\
\end{aligned}\end{split}
\end{equation*}
með innsetningaraðferð.

\sphinxstylestrong{Lausn:}

Einangrum \(y\) úr efri jöfnunni: \(y=2-2x\) .
Setjum þessa lýsingu á \(y\) inn í neðri jöfnuna:
\begin{equation*}
\begin{split}\begin{aligned}
  3x+2y&=1 \\
  3x + 2(2-2x) &=1 \\
  3x+4-4x &=1\\
  -x&=-3\\
  x&=3
\end{aligned}\end{split}
\end{equation*}
Nú vitum við rétt gildi á \(x\) en þá er hægt að finna gildi á \(y\) með því að setja gildið á \(x\) inn í aðra hvora jöfnuna.
Setjum því \(x=3\) inn í efri jöfnuna og fáum:
\begin{equation*}
\begin{split}\begin{aligned}
  2x+y&=2 \\
  2\cdot 3 + y &=2 \\
  6+y &=2 \\
  y&=-4
\end{aligned}\end{split}
\end{equation*}
Lausn jöfnuhneppisins
\begin{equation*}
\begin{split}\begin{aligned}
  2x+y&=2 \\
  3x+2y&=1 \\
\end{aligned}\end{split}
\end{equation*}
er því \(x=3\) og \(y=-4\) .
\end{sphinxadmonition}

\begin{sphinxadmonition}{note}{Athugasemd:}
Stundum getur verið þægilegt að finna fyrri breytistærðina með eyðingaraðferðinni og nota síðan innsetningu.
\end{sphinxadmonition}

\begin{sphinxadmonition}{tip}{Dæmi:}
Leysum jöfnuhneppið
\begin{equation*}
\begin{split}\begin{aligned}
3x+2y&=14 \\
x-y&=3
\end{aligned}\end{split}
\end{equation*}
\sphinxstylestrong{Lausn:}

Einangrum fyrst \(y\) með eyðingaraðferð, það er, eyðum \(x\) með því að margfalda neðri jöfnuna með þremur:
\begin{equation*}
\begin{split}\begin{aligned}
3x+2y&=14 \\
3x-3y&=9
\end{aligned}\end{split}
\end{equation*}
Drögum síðan neðri jöfnuna frá þeirri efri og fáum
\begin{equation*}
\begin{split}\begin{aligned}
2y-(-3y) &=14-9 \\
5y &= 5 \\
y&=1
\end{aligned}\end{split}
\end{equation*}
Þá getum við nýtt okkur að gildi \(y\) er þekkt og sett það inn í aðra hvora jöfnuna.
Setjum því \(y=1\) í neðri jöfnuna:
\begin{equation*}
\begin{split}\begin{aligned}
x-y&=3 \\
x-1&=3 \\
x&=4
\end{aligned}\end{split}
\end{equation*}
Lausn jöfnuhneppisins er því \(x=4\) og \(y=1\) .
\end{sphinxadmonition}


\section{Ójöfnur og tölugildi}
\label{\detokenize{Kafli02:ojofnur-og-tolugildi}}

\subsection{Ójöfnur}
\label{\detokenize{Kafli02:ojofnur}}
\textit{Ójöfnur} eru leystar á sambærilegan hátt og jöfnur. Eftirfarandi tákn eru notuð:
\begin{quote}

\(< \qquad\)  \textit{Minna}

\(> \qquad\) \textit{Stærra}

\(\leq \qquad\) \textit{Minna en eða jafnt}

\(\geq \qquad\) \textit{Stærra en eða jafnt}
\end{quote}

Þegar við leysum ójöfnur megum við beita sömu aðgerðum og þegar við leysum jöfnur, svo lengi sem við gerum það báðum megin. Það er bara ein undantekning: \sphinxstyleemphasis{margföldun/deiling með neikvæðri tölu}.

\begin{sphinxadmonition}{note}{Athugasemd:}
Tökum eftir að \(< \text{og} >\) gefa \textit{strangar ójöfnur}.
\end{sphinxadmonition}

\begin{sphinxadmonition}{tip}{Dæmi:}
\sphinxstylestrong{1.} Leysum ójöfnuna \(-2x+9 \leq 3\). Við megum draga frá \(9\) báðum megin, þá fæst \(-2x \leq -6\). Hér þurfum við þó að einangra \(x\) með því að deila í gegn með \(-2\), það er, margfalda með \(-\frac{1}{2}\). Þegar ójafna er margfölduð með neikvæðri tölu þarf að snúa ójöfnunni við vegna þess að \(a \leq b \implies -a \geq -b\). Við fáum því
\begin{equation*}
\begin{split}\left(-\frac12\right) \cdot (-2x) \geq \left(-\frac12\right) \cdot (-6) \quad \text{eða} \quad x \geq 3.\end{split}
\end{equation*}
Þetta þýðir því að \(x\) má vera hvaða tala sem er, svo lengi sem hún er stærri en eða jöfn \(3\). Með prófun er hægt að staðfesta að þetta gengur upp.

\sphinxstylestrong{2.} Leysum ójöfnuna \(2x+2 >30x\). Drögum frá \(2x\) báðum megin og fáum \(2>28x\). Deiling með \(28\) gefur \(\frac{1}{14}>x\). Hér þurftum við ekki að snúa ójafnaðarmerkinu við. Þó er skýrara að skila svarinu á forminu \(x<\frac{1}{14}\).
\end{sphinxadmonition}


\subsection{Tölugildi}
\label{\detokenize{Kafli02:tolugildi}}
Látum \(x\) vera rauntölu. Fjarlægð tölunnar \(x\) frá núllpunktinum á talnalínunni köllum við \textit{tölugildi} eða \textit{algildi} tölunnar \(x\). Við táknum það með \(|x|\). Athugum að fjarlægð getur ekki verið neikvæð svo að \(|x| \geq 0\) fyrir öll \(x\).

Ef \(x\) er jákvæð þá er \(|x|=x\). Ef \(x\) er neikvæð þá fæst tölugildi hennar með því að ,,taka mínusinn af henni“ . Í raun er jafngilt að margfalda hana með  \(-1\) því að þá ,,hverfur mínusinn af henni“. Með táknmáli er tölugildið skilgreint svona:
\begin{equation*}
\begin{split}|x|= \begin{cases} x & \text{ ef } x \geq 0 \\ -x  & \text{ ef } x < 0 \end{cases}\end{split}
\end{equation*}
Til dæmis er \(5 \geq 0\) svo \(|5|=5\) og \(-3 <0\) svo að \(|-3|=(-1) \cdot (-3) = 3\).


\subsection{Reiknireglur fyrir tölugildi}
\label{\detokenize{Kafli02:reiknireglur-fyrir-tolugildi}}
Látum \(a\) og \(b\) vera rauntölur. Þá gildir eftirfarandi
\begin{equation*}
\begin{split}\begin{aligned}
        a \leq |a|  \qquad & \text{(tölugildi getur aðeins stækkað tölu)}\\
        |a|=|-a|  \qquad & \text{(tölugildi eru óháð formerki)}\\
        |a|\cdot|b|=|ab| \qquad & \text{(tölugildi varðveitir margföldun)}\\
        |a|^2=a^2 \qquad  & \text{(önnur veldi eyða tölugildi)}\\
\end{aligned}\end{split}
\end{equation*}
\begin{sphinxadmonition}{note}{Athugasemd:}
Fyrir tölur \(a\) og \(b\) þá má túlka töluna \(|a-b|\) sem fjarlægð \(a\) frá \(b\) á talnalínunni. Til dæmis ef \(a=3\) og \(b=10\) þá er fjarlægðin á milli þessara talna á talnalínunni \(7\). Það er í samræmi við reikninga okkar, höfum \(|3-10|=|-7|=7\) og \(|10-3|=|7|=7\).
\end{sphinxadmonition}

\begin{sphinxadmonition}{tip}{Dæmi:}
Finnum öll \(x\) sem uppfylla \(|x+4|=10\).
Við komum með tvær lausnir:

\sphinxstylestrong{Lausn 1:} Jafnan \(|x+4|=|x-(-4)|=10\) segir okkur að fjarlægðin milli \(x\) og tölunnar \(-4\) er \(10\). Því er ljóst að \(x=-14\) eða \(x=6\).

\sphinxstylestrong{Lausn 2:} Jafnan \(|x+4|=10\) segir okkur að annað hvort er \(x+4=10\) eða \(x+4=-10\). Fyrri jafnan hefur lausnina \(x=6\) og sú seinni \(x=-14\).

\noindent{\hspace*{\fill}\sphinxincludegraphics[width=1.100\linewidth]{{mynd-algildi2}.svg}\hspace*{\fill}}

Sjáum að \(-14\) og \(6\) eru jafn langt frá \(-4\)
\end{sphinxadmonition}

\begin{sphinxadmonition}{tip}{Dæmi:}
Finnum öll \(x\) sem uppfylla \(|x-3|=|x+7|\).

Við komum með þrjár lausnir:

\sphinxstylestrong{Lausn 1:} Rúmfræðilega þýðir jafnan \(|x-3|=|x-(-7)|\) að fjarlægð tölunnar \(x\) frá \(3\) er jöfn fjarlægðar \(x\) frá \(-7\). Þá hlýtur \(x\) að vera talan sem er mitt á milli \(3\) og \(-7\) á talnalínunni. Með öðrum orðum er \(x\) meðaltal þessara talna:
\begin{quote}
\begin{equation*}
\begin{split}x=\frac{3+(-7)}{2}=-2\end{split}
\end{equation*}\end{quote}

\sphinxstylestrong{Lausn 2:} Skiptum í þrjú tilvik:
\begin{enumerate}
\sphinxsetlistlabels{\arabic}{enumi}{enumii}{}{.}%
\item {} 
Ef \(x<-7\) þá er \(x-3<0\) og \(x+7<0\).

\end{enumerate}
\begin{quote}

Samkvæmt skilgreiningu er því
\begin{equation*}
\begin{split}|x-3|=-(x-3)=-x+3\end{split}
\end{equation*}
og
\begin{equation*}
\begin{split}|x+7|=-(x+7)=-x-7.\end{split}
\end{equation*}
Eftir stendur jafnan \(-x+3=-x-7\) sem jafngildir \(3=-7\) sem er fráleitt.
Jafnan hefur því enga lausn \(x\) sem uppfyllir \(x<-7\)
\end{quote}
\begin{enumerate}
\sphinxsetlistlabels{\arabic}{enumi}{enumii}{}{.}%
\setcounter{enumi}{1}
\item {} 
Ef \(-7\leq x<3\) þá er \(x-3<0\) og \(x+7\geq 0\).

\end{enumerate}
\begin{quote}

Samkvæmt skilgreiningu er því
\begin{equation*}
\begin{split}|x-3|=-(x-3)=-x+3 \quad\end{split}
\end{equation*}
og
\begin{equation*}
\begin{split}|x+7|=x+7.\end{split}
\end{equation*}\end{quote}

Eftir stendur jafnan \(-x+3=x+7\) sem hefur lausnina \(x=-2\).
\begin{enumerate}
\sphinxsetlistlabels{\arabic}{enumi}{enumii}{}{.}%
\setcounter{enumi}{2}
\item {} 
Ef \(x\geq 3\) þá er \(x-3\geq 0\) og \(x+7>0\). Samkvæmt skilgreiningu er því

\end{enumerate}
\begin{quote}
\begin{equation*}
\begin{split}|x-3|=x-3 \quad\end{split}
\end{equation*}
og
\begin{equation*}
\begin{split}|x+7|=x+7.\end{split}
\end{equation*}
Eftir stendur jafnan \(x-3=x+7\) sem jafngildir \(-3=7\) sem er fráleitt.
Jafnan hefur því enga lausn sem uppfyllir \(x\geq 3\)
\end{quote}

\sphinxstylestrong{Lausn 3:} Setjum báðar hliðar jöfnunnar í annað veldi. Þá eyðast tölugildin skv. reiknireglu og eftir stendur: \((x-3)^2=(x+7)^2\) sem jafngildir
\begin{quote}
\begin{equation*}
\begin{split}x^2-6x+9=x^2+14x+49 \quad\end{split}
\end{equation*}
eða
\begin{equation*}
\begin{split}-20x=40 \quad \text{þ.e.} \quad x=-2.\end{split}
\end{equation*}
Ef þessari lausn er stungið inní upphaflegu jöfnuna þá sést að þetta er lausn sem virkar.
\end{quote}
\end{sphinxadmonition}

\begin{sphinxadmonition}{note}{Athugasemd:}
Takið eftir að í lausn 3 í dæminu hér á undan þá endum við á því að prófa lausnina sem við fengum.
Það er af því að þegar jöfnur eru settar í annað veldi geta skapast ,,falskar lausnir“.
Því þarf alltaf að athuga hvort lausnirnar sem fengust séu raunverulegar lausnir.

Skoðum til dæmis jöfnuna \(4x=8\). Þessi jafna hefur augljóslega bara eina lausn \(x=2\). Ef þessi jafna er sett í annað veldi fæst jafnan \(16x^2=64\) sem jafngildir \(x^2=4\) eða \(x=\pm 2\). Hér varð til falska lausnin \(-2\) sem er ekki lausn á upprunalegu jöfnunni. Upphaflega lausnin er hins vegar ennþá til staðar.
\end{sphinxadmonition}

\begin{sphinxadmonition}{tip}{Dæmi:}
Leysum ójöfnuna \(|x+2|>9\). Skiptum í tvö tilvik samkvæmt skilgreiningu, ef \(x+2>0\), eða \(x+2<0\).

\sphinxstylestrong{Tilvik 1}:
Ef \(x+2 \geq 0\), það er, \(x \geq -2\), þá fellum við tölugildið niður og þá fæst ójafnan \(x+2>9\), það er, \(x>7\).

Við þurfum því að uppfylla bæði skilyrðin \(x \geq -2\) og \(x>7\) en ljóst er að seinna skilyrðið er sterkara (það má sjá á talnalínu). Ef \(x >7\) þá hefur það í för með sér að \(x \geq -2\). Því er lausnin í þessu tilviki, \(x >7\).

\sphinxstylestrong{Tilvik 2}:
Ef \(x+2 \leq 0\), það er, \(x \leq -2\), þá fæst ójafnan \(-(x+2)>9\), eða \(-x-2>9\), eða \(x<-11\).

Við þurfum að uppfylla bæði skilyrðin \(x \leq -2\) og \(x<-11\) en ljóst er að það seinna er sterkara. Lausnin er því \(x<-11\).

Við fáum því lokasvarið
\begin{equation*}
\begin{split}x >7 \qquad \text{eða} \qquad x<-11\end{split}
\end{equation*}
Við getum líka skrifað þetta sem bil, jafngilt svar er
\begin{equation*}
\begin{split}x \in ]- \infty , -11[ \, \cup \, ]7, \infty[\end{split}
\end{equation*}\end{sphinxadmonition}


\section{Summu\sphinxhyphen{} og margfeldistáknið}
\label{\detokenize{Kafli02:summu-og-margfeldistakni}}

\bigskip\hrule\bigskip


Stundum kemur fyrir að stærðfræðingar vilja leggja saman marga liði. Til dæmis
\begin{equation*}
\begin{split}1+2+3+4+...+100\end{split}
\end{equation*}
eða
\begin{equation*}
\begin{split}\frac{1}{4}+\frac{1}{5}+\frac{1}{6}+...+\frac{1}{87}\end{split}
\end{equation*}
Í báðum þessum dæmum ætti að vera augljóst hvað summan þýðir þó að svona þrípunktur sé ekki alvöru stærðfræðitákn.
Í fyrra dæminu er verið að leggja saman allar heiltölur frá einum upp í hundrað og í seinna dæminu er verið að leggja saman einn á móti sérhverri heiltölu frá fjórum upp í áttatíu\sphinxhyphen{}og\sphinxhyphen{}sjö.

Stundum vinnum við samt með flóknari gerðir af summum og þá dugar svona táknmál skammt. Þess vegna innleiðum við summutáknið \(\sum\).

Segjum að við viljum leggja saman allar tölur af gerðinni \(n(n+1)\) þar sem \(n\) gengur frá einum upp í hundrað. Einhverjum gæti dottið í hug að skrifa
\begin{equation*}
\begin{split}\begin{aligned}
& 1 \cdot (1+1)+2 \cdot (2+1)+3 \cdot (3+1)+4 \cdot (4+1)+...+100 \cdot (100+1)\\ &=2+6+12+20+...+10100 \end{aligned}\end{split}
\end{equation*}
en þetta þykir ekki snyrtileg leið að tákna summu. Þess vegna skrifum við frekar
\begin{equation*}
\begin{split}\sum_{n=1}^{100}n(n+1)\end{split}
\end{equation*}
Fyrir neðan summutáknið stendur \(n=1\) sem merkir að breytistærðin sem við vinnum í er \(n\) og að við byrjum í einum. Fyrir ofan summutáknið stendur hundrað sem þýðir að við endum summuna þegar \(n=100\). Hægra megin við summutáknið stendur síðan formúlan fyrir sérhvern lið í summunni. Athugum að \(\sum_{n=1}^{100}n(n+1) = \sum_{n=0}^{99}(n+1)(n+2)\).

Summurnar sem teknar voru fram í byrjun kaflans yrðu táknaðar með
\begin{equation*}
\begin{split}\sum_{n=1}^{100}n=1+2+3+...+100\end{split}
\end{equation*}
og
\begin{equation*}
\begin{split}\sum_{n=4}^{87}\frac{1}{n} = \frac{1}{4}+\frac{1}{5}+\frac{1}{6}+...+\frac{1}{87}\end{split}
\end{equation*}
Stundum lendum við í sömu vandræðum með löng margfeldi. Þess vegna innleiðum við margfeldistáknið \(\prod\) sem er notað á sama hátt og summutáknið en bara fyrir margfeldi.

Þannig er
\begin{equation*}
\begin{split}\prod_{n=1}^{100}n=1\cdot 2\cdot 3\cdot 4\cdots 100\qquad \prod_{n=4}^{87}\frac{1}{n}=\frac{1}{4}\cdot\frac{1}{5}\cdot\frac{1}{6}\cdots\frac{1}{100}\end{split}
\end{equation*}
og
\begin{equation*}
\begin{split}\prod_{n=1}^{100}n(n+1)=1(1+1)\cdot 2(2+1)\cdot 3(3+1)\cdots 100(100+1)=2\cdot 6\cdot 12 \cdots 10100\end{split}
\end{equation*}
\begin{sphinxadmonition}{note}{Athugasemd:}
Þessi tákn eru mikilvæg þegar unnið er með \textit{runur} og \textit{raðir}.
\sphinxstyleemphasis{Runa} er raðaður listi af tölum og \sphinxstyleemphasis{röð} er summa liðanna.
\end{sphinxadmonition}

\begin{sphinxadmonition}{tip}{Dæmi:}
\sphinxstylestrong{1.} Reiknum
\begin{equation*}
\begin{split}\sum_{n=-3}^{5}n^2\end{split}
\end{equation*}
Höfum
\begin{equation*}
\begin{split}\begin{aligned}
\sum_{n=-3}^{5}n^2 &=(-3)^2+(-2)^2+(-1)^2+0^2+1^2+2^2+3^2+4^2+5^2 \\ &=9+4+1+0+1+4+9+16+25=69
\end{aligned}\end{split}
\end{equation*}
\sphinxstylestrong{2.} Reiknum
\begin{equation*}
\begin{split}\prod_{n=1}^{4}(1+n)\end{split}
\end{equation*}
Höfum
\begin{equation*}
\begin{split}\prod_{n=1}^{4}(1+n)=(1+1)(1+2)(1+3)(1+4)=2\cdot 3\cdot 4\cdot 5=120\end{split}
\end{equation*}\end{sphinxadmonition}


\chapter{Rúmfræði}
\label{\detokenize{Kafli03:rumfraei}}\label{\detokenize{Kafli03::doc}}

\section{Hnitakerfi}
\label{\detokenize{Kafli03:hnitakerfi}}
Kartesíska \textit{hnitakerfið} er búið til með því að leggja tvær \textit{talnalínur} í kross þannig að þær standi \textit{hornréttar} hvor á aðra og skerist í núllpunktum sínum.
Lárétta talnalínan er oftast kölluð \(x\)\sphinxhyphen{}ásinn og sú lóðrétta \(y\)\sphinxhyphen{}ásinn.

Til að marka punkt \((a,b)\) inn á svona hnitakerfi þá teiknum við lóðrétta línu í gegnum punktinn \(a\) á \(x\)\sphinxhyphen{}ásnum og lárétta línu í gegnum punktinn \(b\) á \(y\)\sphinxhyphen{}ásnum.
Þar sem línurnar skerast mörkum við punktinn \((a,b)\).

Á myndinni má sjá punktinn \((4,2)\) markaðan í hnitakerfi.

\noindent{\hspace*{\fill}\sphinxincludegraphics[width=0.900\linewidth]{{hnitakerfi}.svg}\hspace*{\fill}}


\subsection{Fjarlægð milli punkta í kartesísku hnitakerfi}
\label{\detokenize{Kafli03:fjarlaeg-milli-punkta-i-kartesisku-hnitakerfi}}
Látum \(P_1=(x_1,y_1)\) og \(P_2=(x_2,y_2)\) vera tvo punkta í hnitakerfinu. Til þess að finna fjarlægðina á milli þeirra skulum við nota reglu \sphinxhref{http://www.xn--st-2ia.is/fletta/setning\_p\%C3\%BD\%C3\%BEag\%C3\%B3rasar}{Pýþagórasar}.

Við myndum rétthyrndan þríhyrning með því að bæta við punkti sem við skulum kalla \(P_3\). Þessi punktur hefur sama \(y\)\sphinxhyphen{} \textit{hnit} og punkturinn \(P_1\), og sama \(x\)\sphinxhyphen{} \textit{hnit} og punkturinn \(P_2\). Því er \(P_3=(x_2, y_1)\).

\noindent{\hspace*{\fill}\sphinxincludegraphics[width=0.700\linewidth]{{fjarl}.svg}\hspace*{\fill}}

Fjarlægðin milli punktanna \(P_1\) og \(P_3\) er \(|x_2-x_1|\).

Fjarlægðin milli punktanna \(P_2\) og \(P_3\) er \(|y_2-y_1|\).

Við vitum því skammhliðarnar í þessum þríhyrningi og notum reglu Pýþagórasar til að finna \textit{langhliðina}. Munum að regla Pýþagórasar segir að fyrir rétthyrndan þríhyrning gildir
\begin{equation*}
\begin{split}a^2+b^2=c^2\end{split}
\end{equation*}
þar sem \(a, b\) eru skammhliðarnar og \(c\) er langhliðin. Því vitum við að lengd langhliðarinnar er \(c=\sqrt{a^2+b^2}\).

Með því að nota þessa reglu getum við því fundið fjarlægðina á milli \(P_1\) og \(P_2\):


\subsection{Regla}
\label{\detokenize{Kafli03:regla}}
Fjarlægðin milli punktanna \(P_1=(x_1,y_1)\) og \(P_2=(x_2,y_2)\) í hnitakerfinu er
\begin{equation*}
\begin{split}\sqrt{(x_2-x_1)^2+(y_2-y_1)^2}\end{split}
\end{equation*}
\begin{sphinxadmonition}{note}{Athugasemd:}
Hér er \(a = |x_2-x_1|\) og \(b = |y_2-y_1|\) miðað við reglu Pýþagórasar.
\end{sphinxadmonition}

\begin{sphinxadmonition}{tip}{Dæmi:}
Finnum fjarlægðina milli punktanna \((1,2)\) og \((5,7)\) í hnitakerfinu.

Hér er \(x_1=1\), \(x_2=5\), \(y_1=2\) og \(y_2=7\). Setjum þetta inn í jöfnuna að ofan og fáum að fjarlægðin milli punktanna er
\begin{equation*}
\begin{split}\sqrt{(5-1)^2+(7-2)^2}=\sqrt{16+25}=\sqrt{41}\end{split}
\end{equation*}\end{sphinxadmonition}


\section{Jafna línu í hnitakerfinu}
\label{\detokenize{Kafli03:jafna-linu-i-hnitakerfinu}}
Til að tákna línur í hnitakerfinu er algengt að nota jöfnur.

Almennt form jöfnu línu er
\begin{equation*}
\begin{split}ax+by+c=0\end{split}
\end{equation*}
þar sem \(a,b\) og \(c\) eru rauntölur. Það er, fyrsta stigs jafna er jafna línu.

\begin{sphinxadmonition}{note}{Athugasemd:}
Oft er jafna línu rituð á eftirfarandi hátt:
\begin{equation*}
\begin{split}y=hx+s\end{split}
\end{equation*}
sjá nánar í kafla 3.3

eða
\begin{equation*}
\begin{split}y - y_1 = h(x - x_1)\end{split}
\end{equation*}
þar sem \((x_1,y_1)\) er punktur á línunni og \(h\) er hallatalan.
\end{sphinxadmonition}

\begin{sphinxadmonition}{tip}{Dæmi:}
Skoðum jöfnuna
\begin{equation*}
\begin{split}-\frac12 x +y +1=0\end{split}
\end{equation*}
Hér er \(a=-1/2\), \(b=1\) og \(c=1\) miðað við almennu framsetninguna.
Finnum nú nokkur gildi á \(x\) og \(y\) sem uppfylla þessa jöfnu og mörkum samsvarandi punkta \((x,y)\) í hnitakerfið.

Ef \(x=0\) og \(y=-1\) þá stenst jafnan. Það sést með prófun:
\begin{equation*}
\begin{split}-\frac{1}{2} (0) +(-1) +1= -1+1 = 0\end{split}
\end{equation*}
Við mörkum því punktinn \((0,-1)\) í hnitakerfið.

Ef \(x=2\) og \(y=0\) þá stenst jafnan. Við mörkum því punktinn \((2,0)\) í hnitakerfið.
Á sama hátt getum við markað punktana \((-6,-4)\), \((-4,-3)\), \((-2,-2)\), \((4,1)\) og \((6,2)\) í hnitakerfið því að ef þessum hnitum er stungið inn í jöfnuna þá stenst hún.
Það hefur verið gert á myndinni.

\noindent{\hspace*{\fill}\sphinxincludegraphics[width=0.600\linewidth]{{lina}.svg}\hspace*{\fill}}

Við sjáum að allar lausnirnar lenda á sömu línunni í hnitakerfinu. Auk þess eru allir punktarnir á línunni (líka þeir sem ekki eru merktir) lausn á jöfnunni.

Þess vegna segjum við að jafnan \(-\dfrac{1}{2}x+y+1=0\) sé jafna línunnar sem fram kemur á myndinni. Jafnan hefur óendanlega margar lausnir og línan teygir sig óendanlega langt í báðar áttir.
\end{sphinxadmonition}


\section{Hallatala og skurðpunktur við ása}
\label{\detokenize{Kafli03:hallatala-og-skurpunktur-vi-asa}}
Jafnan \(ax+by+c=0\) kallast almenn jafna línu. Þó getur verið hentugra að koma henni yfir á annað form.

Við getum einangrað \(y\) úr þessari jöfnu með reikniaðgerðum. Þá fæst jafna
\begin{equation*}
\begin{split}y=hx+s\end{split}
\end{equation*}
þar sem \(h = -\frac{a}{b}\) og \(s = -\frac{c}{b}\).
Fastinn \(h\) kallast \textit{hallatala} línunnar og fastinn \(s\) kallast \textit{skurðpunktur} línunnar við \(y\)\sphinxhyphen{}ás.


\bigskip\hrule\bigskip



\begin{center}
\includegraphics[width=8cm,keepaspectratio=true]{./myndir/ggbjafnalinu.png}
\end{center}



\bigskip\hrule\bigskip


Hallatala línunnar, \(h\), táknar hversu mikið línan hallar. Ef við vitum einn punkt á línunni þá getum við fundið annan með því að færa okkur fyrst einn til hægri eða vinstri í hnitakerfinu og svo \(h\) upp eða niður.

Skurðpunktur línu við \(y\)\sphinxhyphen{}ásinn er tala \(s\) sem segir okkur hvar línan sker \(y\)\sphinxhyphen{}ásinn. Línan mun skera ásinn í punktinum \((0,s)\).

Skurðpunktur línu við \(x\)\sphinxhyphen{}ásinn er skylt hugtak, en sú tala segir okkur hvar línan sker \(x\)\sphinxhyphen{}ásinn. Hann finnum við með því að setja \(y=0\) inn í jöfnu línunnar og leysa fyrir \(x\).

\begin{sphinxadmonition}{note}{Athugasemd:}
\(x\)\sphinxhyphen{}ásinn er línan þar sem \(y=0\) og \(y\)\sphinxhyphen{}ásinn er línan þar sem \(x=0\). Þess vegna getum við fundið skurðpunkt línu við \(x\)\sphinxhyphen{}ás með því að setja \(y=0\) inn í jöfnu línunnar, og sömuleiðis finnum við skurðpunkt við \(y\)\sphinxhyphen{}ás með því að setja \(x=0\) inn í jöfnu línunnar.
\end{sphinxadmonition}

\begin{sphinxadmonition}{tip}{Dæmi:}
Finnum hallatölu og skurðpunkta línunnar
\begin{equation*}
\begin{split}2x-y-3=0\end{split}
\end{equation*}
við \(x\)\sphinxhyphen{}ás og \(y\)\sphinxhyphen{}ás. Teiknum svo línuna inn í hnitakerfi.

Byrjum á að koma línunni yfir á formið \(y=hx+s\). Einangrum \(y\) og fáum
\begin{equation*}
\begin{split}y=2x-3\end{split}
\end{equation*}
Nú getum við fundið skurðpunkt við \(y\)\sphinxhyphen{}ás út frá jöfnu línunnar. Hér er \(s=-3\) svo skurðpunktur við \(y\)\sphinxhyphen{}ás hefur hnit \((0,-3)\).

Við sjáum líka út frá jöfnu línunnar að \(h=2\) svo hallatala línunnar er \(2\).

Við finnum skurðpunkt línunnar við \(x\)\sphinxhyphen{}ás með því að setja \(y=0\) í jöfnu hennar og leysa fyrir \(x\). Fáum \(2x-3=0\), það er, \(x=\frac{3}{2}\). Skurðpunkturinn við \(x\)\sphinxhyphen{}ás er því punkturinn \((\frac{3}{2},0)\).

Finnum tvo punkta á línunni í viðbót með því að nota hallatölu hennar. Hallatalan er \(2\).
Færum okkur því einn til hægri og tvo upp frá skurðpunkti línunnar við \(y\)\sphinxhyphen{}ás og fáum því að punkturinn \((1,-1)\) tilheyrir línunni.

Færum okkur líka einn til hægri og tvo upp frá skurðpunkti línunnar við \(x\)\sphinxhyphen{}ás og fáum að punkturinn \((\frac{5}{2}, 2)\) tilheyrir línunni.

Nú höfum við fundið fjóra punkta línunnar. Merkjum þá inn á hnitakerfið og drögum beina línu í gegnum þá. Þá fæst graf línunnar.

\noindent{\hspace*{\fill}\sphinxincludegraphics[width=0.750\linewidth]{{hallatala}.svg}\hspace*{\fill}}
\end{sphinxadmonition}

\begin{sphinxadmonition}{note}{Athugasemd:}
Athugum að til að teikna línu er nóg að finna tvo punkta sem liggja á henni og finna beina línu í gegnum þá, en í dæminu hér að ofan fundum við fjóra.
\end{sphinxadmonition}

\begin{sphinxadmonition}{tip}{Dæmi:}
Finnum hallatölu og skurðpunkta línunnar
\begin{equation*}
\begin{split}3x + 2y -2 = 0\end{split}
\end{equation*}
við \(x\)\sphinxhyphen{}ás og \(y\)\sphinxhyphen{}ás.

Komum jöfnunni yfir á formið \(y=hx+s\)
\begin{equation*}
\begin{split}y = -\frac{3}{2} x + 1\end{split}
\end{equation*}
Þá fáum við skurðpunktinn við \(y\)\sphinxhyphen{}ás \(s = 1\) og hallatölu \(h= -\frac{3}{2}\).

Skurðpunkturinn við \(x\) \sphinxhyphen{}ás fæst með að setja \(y=0\) í jöfnuna og leysa fyrir \(x\) . Fáum \(3x + 2(0) -2 = 0\) sem gefur okkur \(x = \frac{2}{3}\)

Þá eru skurðpunktarnir tveir \((0, 1)\) og \((\frac{2}{3}, 0)\)

Þá fæst graf línunnar:

\noindent{\hspace*{\fill}\sphinxincludegraphics[width=0.750\linewidth]{{neg-hallatala}.svg}\hspace*{\fill}}
\end{sphinxadmonition}


\section{Meira um jöfnu línu}
\label{\detokenize{Kafli03:meira-um-jofnu-linu}}
Stundum fáum við gefna tvo punkta \((x_1,y_1)\) og \((x_2,y_2)\) og þurfum að finna jöfnu línunnar sem gengur í gegnum þá. Til þess getum við notað eftirfarandi reglu:


\subsection{Regla}
\label{\detokenize{Kafli03:id1}}
Hallatala línunnar, \(h\), sem gengur í gegnum punkta \((x_1,y_1)\) og \((x_2,y_2)\) fæst með formúlunni
\begin{equation*}
\begin{split}h=\frac{y_2-y_1}{x_2-x_1}\end{split}
\end{equation*}
\noindent{\hspace*{\fill}\sphinxincludegraphics[width=0.500\linewidth]{{almenn-lina}.svg}\hspace*{\fill}}


\subsection{Að finna jöfnu línu}
\label{\detokenize{Kafli03:a-finna-jofnu-linu}}
Þegar við höfum fundið hallatölu línunnar þá þurfum við líka að reikna út fastann \(s\), skurðpunkt við \(y\)\sphinxhyphen{}ás. Við vitum að línan gengur í gegnum punktana \((x_1,y_1)\) og \((x_2,y_2)\). Því þarf jafnan \(y_1=hx_1+s\) að ganga upp (sömuleiðis jafnan \(y_2=hx_2+s\) en það er nóg að notast við aðra hvora af þessum jöfnum). Við einangrum þá \(s\) út úr þessari jöfnu. Þá getum við ritað jöfnu línunnar
\begin{equation*}
\begin{split}y=hx+s\end{split}
\end{equation*}
þar sem við þekkjum fastana \(h\) og \(s\).

\begin{sphinxadmonition}{tip}{Dæmi:}
Finnum jöfnu línunnar sem gengur í gegnum punktana \((1,2)\) og \((13,17)\).

Byrjum á að finna hallatölu hennar. Samkvæmt reglu að ofan fæst
\begin{equation*}
\begin{split}h=\frac{17-2}{13-1}=\frac{15}{12}=\frac54\end{split}
\end{equation*}
Jafna línunnar er því af gerðinni
\begin{equation*}
\begin{split}y=\frac54 x +s\end{split}
\end{equation*}
Til að finna skurðpunkt við \(y\)\sphinxhyphen{}ás athugum við að þar sem línan gengur í gegnum punktinn \((1,2)\) þá þarf jafnan að standast þegar þessum punkti er stungið inn í hana. Því fæst
\begin{equation*}
\begin{split}(2)=\frac54 \cdot (1) +s\end{split}
\end{equation*}
Einangrum nú \(s\) og fáum \(s=\frac34\).

Nú höfum við fundið \(h\) og \(s\) og jafna línunnar er því
\begin{equation*}
\begin{split}y=\frac54 x + \frac34\end{split}
\end{equation*}
\noindent{\hspace*{\fill}\sphinxincludegraphics[width=0.700\linewidth]{{mynd-linudaemi}.svg}\hspace*{\fill}}
\end{sphinxadmonition}


\subsection{Samsíða og þverstæðar línur}
\label{\detokenize{Kafli03:samsia-og-verstaear-linur}}

\subsubsection{Samsíða línur}
\label{\detokenize{Kafli03:samsia-linur}}\phantomsection\label{\detokenize{Kafli03:s-samsia}}
Ef línur \(m_1\) og \(m_2\) eru með hallatölu \(h_1\) og \(h_2\) þá gildir
\begin{equation*}
\begin{split}m_1 \parallel m_2 \; \Longleftrightarrow \; h_1 = h_2\end{split}
\end{equation*}
Það er, ef hallatalan er sú sama þá segjum við að línurnar séu \textit{samsíða}.
Hér er dæmi um tvær samsíða línur með hallatöluna \(h_1=h_2=1\)

\noindent{\hspace*{\fill}\sphinxincludegraphics[width=0.700\linewidth]{{samsid}.svg}\hspace*{\fill}}


\subsubsection{Þverstæðar línur}
\label{\detokenize{Kafli03:verstaear-linur}}\phantomsection\label{\detokenize{Kafli03:s-verstae}}
Ef línur \(m_1\) og \(m_2\) eru með hallatölu \(h_1\) og \(h_2\) og um þær gildir
\begin{equation*}
\begin{split}m_1 \perp m_2 \; \Longleftrightarrow \; h_1 = -\frac{1}{h_2}\end{split}
\end{equation*}
þá eru línurnar \textit{þverstæðar} eða hornréttar.
Hér er dæmi um tvær hornréttar línur, með hallatölurnar \(h_1=3\) og \(h_2=-\frac{1}{3}\)

\noindent{\hspace*{\fill}\sphinxincludegraphics[width=0.700\linewidth]{{prepd}.svg}\hspace*{\fill}}

\begin{sphinxadmonition}{tip}{Dæmi:}
Látum \(l\) vera línuna sem gengur í gegnum punktana \((-3,6)\) og \((1,-2)\).
Finnum jöfnu línu sem gengur í gegnum punktinn \((1,1)\) og er \textit{samsíða} línunni \(l\).

Köllum línuna sem við erum að leita að \(m\). Það að línurnar \(l\) og \(m\) séu samsíða þýðir að þær hafa sömu hallatölu. Til að finna hallatölu línunnar \(m\) nægir því að finna hallatölu línunnar \(l\).
Hallatalan er:
\begin{equation*}
\begin{split}h=\dfrac{(-2)-6}{1-(-3)}=\dfrac{-8}{4}=-2\end{split}
\end{equation*}
Jafna línunnar \(m\) er því af gerðinni
\begin{equation*}
\begin{split}y=-2x+s\end{split}
\end{equation*}
en við eigum eftir að finna \(s\).
Gefið er að punkturinn \((1,1)\) er á línunni svo jafnan þarf að standast þegar \((x,y)=(1,1)\) er stungið inn í hana. Með öðrum orðum er
\begin{equation*}
\begin{split}1=-2\cdot 1+s \quad \rightarrow \quad s=1+2=3\end{split}
\end{equation*}
Jafna línunnar \(m\) er því \(y=-2x+3\).

\noindent{\hspace*{\fill}\sphinxincludegraphics[width=0.400\linewidth]{{mynd-linudaemi2}.svg}\hspace*{\fill}}
\end{sphinxadmonition}


\subsection{Miðpunktsregla}
\label{\detokenize{Kafli03:mipunktsregla}}
Reikna má miðpunkt striksins á milli \(A(x_1, y_1)\) og \(B(x_2,y_2)\) með:
\begin{equation*}
\begin{split}M = \left( \frac{x_1+x_2}{2} , \frac{y_1+y_2}{2} \right)\end{split}
\end{equation*}
\noindent{\hspace*{\fill}\sphinxincludegraphics[width=0.500\linewidth]{{midpkt}.svg}\hspace*{\fill}}


\section{Keilusnið}
\label{\detokenize{Kafli03:keilusni}}
\textit{Keilusnið} er regnhlífarhugtak yfir ákveðin söfn punkta.
Nafnið kemur frá því að \textit{ferlarnir} (punktasöfnin) hafa sömu lögun og ef (óendanleg) tvöföld \textit{keila} er skorin með (óendanlegu) blaði.

\begin{figure}[htbp]
\centering

\noindent\sphinxincludegraphics[width=0.700\linewidth]{{keilusnið}.svg}
\end{figure}

Á myndinni má sjá nokkrar tegundir keilusniða. Ferlarnir eru grænir á ljósbláu blaðinu.
\begin{itemize}
\item {} 
Mynd 1 er af \textit{fleygboga}, sem kemur fram þegar blaðið er samsíða brún keilanna.

\item {} 
Mynd 2 er af \textit{sporöskju}, sem kemur fram þegar blaðið hallar minna en brún keilanna.

\item {} 
Á mynd 2 er líka \textit{hringur}, sem kemur fram þegar blaðið er lárétt.

\item {} 
Mynd 3 er af \textit{breiðboga}, sem kemur fram þegar blaðið hallar meira en brún keilanna.

\item {} 
Línur eru líka keilusnið, en þær koma fram þegar blaðið leggst einmitt á brún keilanna.

\item {} 
Stakur punktur kemur fram þegar blaðið leggst einmitt þar sem keilurnar mætast.

\end{itemize}


\subsection{Fleygbogar}
\label{\detokenize{Kafli03:fleygbogar}}
\textit{Fleygbogi} hefur jöfnu á forminu \(y=ax^2+bx+c\) ({\hyperref[\detokenize{Kafli06:s-annarsstigs}]{\sphinxcrossref{\DUrole{std,std-ref}{annars stigs margliða}}}}).

\sphinxstyleemphasis{Topppunktur} fleygboga er í punktinum:
\begin{equation*}
\begin{split}T=\left(\frac{-b}{2a},\frac{4ac-b^2}{4a} \right)\end{split}
\end{equation*}
Línan \(x=-\frac{b}{2a}\) kallast \textit{samhverfuás} fleygbogans.
Fleygboginn speglast um ásinn.

Fleygboginn sker \(y\) \sphinxhyphen{}ás hnitakerfisins í punkti \((0,c)\) .

Fleygboginn sker \(x\) \sphinxhyphen{}ás hnitakerfisins:
\begin{itemize}
\item {} 
Aldrei ef \(b^2-4ac<0\) .

\item {} 
Einu sinni í punkti \((-b/2a, 0)\) ef \(b^2-4ac=0\) .

\item {} 
Tvisvar í punktum \(x_0=\left(\frac{-b+\sqrt{b^2-4ac}}{2a},0 \right)\) og \(x_1= \left(\frac{-b-\sqrt{b^2-4ac}}{2a},0 \right)\) ef \(b^2-4ac>0\) .

\end{itemize}

Fleygbogar hafa mismunandi lögun eftir formerkjum:
\begin{quote}
\begin{itemize}
\item {} 
ef \(a>0\) þá er fleygboginn \textit{kúptur} :

\end{itemize}

\noindent{\hspace*{\fill}\sphinxincludegraphics[width=0.400\linewidth]{{parabolapos}.svg}\hspace*{\fill}}
\begin{itemize}
\item {} 
ef \(a<0\)  þá er fleygboginn \textit{hvelfdur} :

\end{itemize}

\noindent{\hspace*{\fill}\sphinxincludegraphics[width=0.400\linewidth]{{parabolaneg}.svg}\hspace*{\fill}}
\end{quote}

\begin{sphinxadmonition}{tip}{Dæmi:}
Skoðum fleygbogann \(x^2-x-1=0\) .
Tökum fyrst eftir að fleygboginn er kúptur því \(a=1>0\) .
Reiknum topppunktinn:
\begin{equation*}
\begin{split}\begin{aligned}
  T&=\left(\frac{-b}{2a},\frac{4ac-b^2}{4a} \right) \\
  &= \left(\frac{-1}{2\cdot 1},\frac{4\cdot 1\cdot (-1)-(-1)^2}{4\cdot 1} \right) \\
  &= \left(\frac{1}{2},-\frac{5}{4} \right)
\end{aligned}\end{split}
\end{equation*}
Samhverfuásinn liggur lóðrétt í gegnum topppunktinn.

Reiknum næst skurðpunkta fleygbogans við \(x\) \sphinxhyphen{}ás.
Sjáum að \(b^2-4ac=(-1)^2-4\cdot1\cdot(-1) = 5 >0\) og því eru skurðpunktarnir tveir:
\begin{equation*}
\begin{split}\begin{aligned}
            x_0 &= \frac{-b+ \sqrt{b^2-4ac}}{2a} \\
            &= \frac{-(-1)+ \sqrt{(-1)^2-4\cdot 1\cdot(-1)}}{2\cdot 1} \\
            &= \frac{1+\sqrt{5}}{2}\\
            \\
            x_1 &= \frac{-b- \sqrt{b^2-4ac}}{2a} \\
            &= \frac{-(-1)- \sqrt{(-1)^2-4\cdot 1\cdot(-1)}}{2\cdot 1} \\
            &= \frac{1-\sqrt{5}}{2} \\
\end{aligned}\end{split}
\end{equation*}
\begin{figure}[H]
\centering

\noindent\sphinxincludegraphics[width=0.600\linewidth]{{fleygb}.svg}
\end{figure}
\end{sphinxadmonition}

\begin{sphinxadmonition}{note}{Athugasemd:}
Til eru fleygbogar sem snúa á hlið.
Þeir hafa formið \(ay^2+by+c=x\) .
\end{sphinxadmonition}

Ljósgeisli sem lendir á speglandi yfirborði speglast af yfirborðinu til baka með sama horni og hann lenti með.
Með öðrum orðum er \sphinxstyleemphasis{innfallshornið} jafnt \sphinxstyleemphasis{útfallshorninu}, \(\theta_1=\theta_2\).

\begin{figure}[htbp]
\centering

\noindent\sphinxincludegraphics[width=0.500\linewidth]{{speglun}.svg}
\end{figure}

Ljósgeislar sem lenda á innra yfirborð fleygboga speglast líka, en þar sem yfirborðið er sveigt speglast þeir ekki allir á sama hátt.
Það sem er merkilegast við fleygboga er að ljósið sem endurkastast fer allt einmitt í gegnum sama punkt! Sá punktur er kallaður \textit{brennipunktur} .

\begin{figure}[htbp]
\centering

\noindent\sphinxincludegraphics[width=0.500\linewidth]{{parabolabrenni}.svg}
\end{figure}

\begin{figure}[htbp]
\centering

\noindent\sphinxincludegraphics[width=0.500\linewidth]{{Satellite_dish}.jpg}
\end{figure}

Þversnið þessa gervihnattardisks hefur lögun fleygboga.
Móttakarinn er settur á stöng svo hann liggi í brennipunkti fleygbogans og merki (t.d. útvarpsbylgjur) sem falla á yfirborð disksins endurkastast í móttakarann.
Merkið verður því skýrt og stöðugt.


\subsection{Sporöskjur}
\label{\detokenize{Kafli03:sporoskjur}}
\textit{Sporöskjur} eru myndir sem lýsa má með jöfnu á forminu:
\begin{equation*}
\begin{split}\frac{(x-x_0)^2}{a^2} + \frac{(y-y_0)^2}{b^2} =1\end{split}
\end{equation*}
Miðja sporöskju er í punktinum \((x_0, y_0)\) .
Tölurnar \(a\) og \(b\) lýsa lengstu fjarlægð ferilsins frá miðju í \(x\) \sphinxhyphen{} og \(y\) \sphinxhyphen{}stefnu.

Sporöskjur hafa tvo \textit{brennipunkta}.


\bigskip\hrule\bigskip


Ef \(a>b\) þá er brennipunktarnir á \(x\) \sphinxhyphen{}ás sporöskjunnar, í fjarlægð \(c=\sqrt{a^2-b^2}\) frá miðju.

\begin{figure}[htbp]
\centering

\noindent\sphinxincludegraphics[width=0.800\linewidth]{{sporaskja1}.svg}
\end{figure}


\bigskip\hrule\bigskip


Ef \(a<b\) þá er brennipunktarnir á \(y\) \sphinxhyphen{}ás sporöskjunnar, í fjarlægð \(c=\sqrt{b^2-a^2}\) frá miðju.

\begin{figure}[htbp]
\centering

\noindent\sphinxincludegraphics[width=0.700\linewidth]{{sporaskja2}.svg}
\end{figure}


\bigskip\hrule\bigskip


\begin{sphinxadmonition}{tip}{Dæmi:}
\sphinxstylestrong{1.} Skoðum sporöskju með  \(a=3\) og \(b=2\)  og miðju í punktinum \((2,2)\) og finnum graf hennar.
Þar sem \(a>b\) þá eru brennipunktarnir hægra og vinsta megin við miðjuna, í fjarlægðinni \(c=\sqrt{a^2-b^2}=\sqrt{3^2-2^2}=\sqrt{5}\) .

Formúla sporöskjunnar er:
\begin{equation*}
\begin{split}\frac{(x-2)^2}{(3)^2} + \frac{(y-2)^2}{(2)^2} =1\end{split}
\end{equation*}
Skoðum graf sporöskjunnar

\noindent{\hspace*{\fill}\sphinxincludegraphics[width=0.700\linewidth]{{sporaskja3}.svg}\hspace*{\fill}}

\sphinxstylestrong{2.}  Skoðum graf sporöskju og finnum formúlu hennar.

\noindent{\hspace*{\fill}\sphinxincludegraphics[width=0.700\linewidth]{{sporaskja4}.svg}\hspace*{\fill}}

Miðjan er í punktinum \((\frac{3}{2},0)\) .
Lengsta fjarlægð ferilsins frá miðjunni í \(x\) \sphinxhyphen{} stefnu er \(a=2\) .
Lengsta fjarlægð ferilsins frá miðjunni í \(y\) \sphinxhyphen{} stefnu er \(b=3\) .

Þar sem \(b>a\) eru brennipunktarnir ofan og neðan við miðjuna, í fjarlægðinni \(c=\sqrt{3^2-2^2}=\sqrt{5}\) .

Formúla sporöskjunnar er því:
\begin{quote}
\begin{equation*}
\begin{split}\frac{(x-\frac{3}{2})^2}{(2)^2} + \frac{(y)^2}{(2)^2} =1\end{split}
\end{equation*}\end{quote}
\end{sphinxadmonition}

Plánetur eru hnettir sem eru á fastri braut í kringum stjörnu (sól).
Brautir pláneta eru sporöskjur (sporbaugar) þar sem stjarnan er í öðrum brennipunkti.
Pláneturnar eru því ekki alltaf í fastri fjarlægð frá sólinni og fara hraðar ef þær eru nálægt henni.

\begin{figure}[htbp]
\centering

\noindent\sphinxincludegraphics[width=0.500\linewidth]{{Kepler-second-law}.gif}
\end{figure}

Þýski stærðfræðingurinn og stjörnufræðingurinn \sphinxhref{https://en.wikipedia.org/wiki/Johannes\_Kepler}{Johannes Kepler} fylgdist með hreyfingum hnattanna og setti þessa uppgötvun fram í því sem við köllum nú \sphinxstyleemphasis{fyrsta lögmál Keplers}, eitt af þremur um hreyfingu himinhnatta.

Brennipunktar sporaskja tengjast með þeim skemmtilega hætti að ljós sem kemur frá öðrum brennipunktinum og endurkastast af brún sporöskjunnar lendir alltaf í hinum brennipunktinum.

Auk þess er, fyrir sérhvern punkt á sporbaugnum, summa fjarlægðanna í hvorn brennipunktinn fasti.

\begin{figure}[htbp]
\centering

\noindent\sphinxincludegraphics[width=0.500\linewidth]{{sporbaugurbrenni}.svg}
\end{figure}


\subsection{Hringir}
\label{\detokenize{Kafli03:hringir}}
\textit{Hringir} eru sértilvik af sporöskjum þegar \(a=b\) , það er, brennipunktarnir lenda saman í miðjunni.
Algengur ritháttur fyrir jöfnu hrings er \((x-x_0)^2+(y-y_0)^2=r^2\) .
Þá er \sphinxstyleemphasis{miðja} hringsins í punktinum \((x_0,y_0)\) og allir punktarnir eru í fjarlægðinni \(r\) frá miðjunni.
Hringur með miðju í \((0,0)\) og radíus \(r=1\) kallast \textit{einingarhringur} .

\begin{figure}[htbp]
\centering

\noindent\sphinxincludegraphics[width=0.500\linewidth]{{hringur}.svg}
\end{figure}


\subsection{Breiðbogar}
\label{\detokenize{Kafli03:breibogar}}
\textit{Breiðbogar} eru myndir sem lýsa má með jöfnu á forminu
\begin{equation*}
\begin{split}\frac{(x-x_0)^2}{a^2} - \frac{(y-y_0)^2}{b^2} =\pm 1\end{split}
\end{equation*}
Ferillinn er tveir samhverfir bogar sem fara út í óendanleikann og fylgja að lokum aðfellunum \(y=\pm \frac{b}{a}(x-x_0)+y_0\) .

Breiðbogar skera annað hvort \(x\) eða \(y\) \sphinxhyphen{}ásinn, eftir því hvort það er plús eða mínus 1 í lok jöfnunnar.

Aðfellurnar skerast í \sphinxstyleemphasis{miðjunni} \((x_0,y_0)\) og tölurnar \(a\) og \(b\) lýsa hallatölum þeirra.

Ef það er plús í lokin á jöfnunni lýsir \(a\) skurðpunkti ferlanna við \(x\) \sphinxhyphen{} ás en ef það er mínus lýsir \(b\) skurðpunkti ferlanna við \(y\) \sphinxhyphen{} ás.

Breiðbogar hafa ekki brennipunkt eins og fleygbogar og sporöskjur.
Þeir koma engu að síður fram í kringum okkur, t.d. í \sphinxhref{https://en.wikipedia.org/wiki/Hyperbola\#Sundials}{sólarúrum} .


\section{Flatarmyndir}
\label{\detokenize{Kafli03:flatarmyndir}}
Flatarmál \textit{rétthyrnings} er \(F=a\cdot b\) og ummálið er \(U=2a+2b\), þar sem \(a, b\) eru hliðarlengdirnar.

\begin{figure}[htbp]
\centering

\noindent\sphinxincludegraphics[width=0.400\linewidth]{{fl_rett}.svg}
\end{figure}

Flatarmál \textit{samsíðungs} er \(F=g\cdot h\) þar sem \(g\) stendur fyrir grunnlínu og \(h\) fyrir hæð.

\begin{figure}[htbp]
\centering

\noindent\sphinxincludegraphics[width=0.400\linewidth]{{fl_sams}.svg}
\end{figure}

Flatarmál hrings er \(F=r^2\cdot\pi\) og \textit{ummálið} er \(U=2r\pi\) þar sem \(r\) er geislinn og \(\pi\) fastinn \(3,14159...\), skilgreindur sem hlutfallið milli ummáls og þvermáls hrings.

\begin{figure}[htbp]
\centering

\noindent\sphinxincludegraphics[width=0.400\linewidth]{{fl_hring}.svg}
\end{figure}

Flatarmál sporöskju er \(F=a\cdot b\cdot\pi\) þar sem \(a, b\) eru lengstu fjarlægðir ferilsins frá miðju í x\sphinxhyphen{} og y\sphinxhyphen{}stefnu.

\begin{figure}[htbp]
\centering

\noindent\sphinxincludegraphics[width=0.400\linewidth]{{fl_spor}.svg}
\end{figure}

Flatarmál \textit{þríhyrnings} er \(F=\frac{1}{2}g\cdot h\) .

\begin{figure}[htbp]
\centering

\noindent\sphinxincludegraphics[width=0.400\linewidth]{{fl_thri1}.svg}
\end{figure}


\bigskip\hrule\bigskip


Rúmmál \textit{kúlu} er \(R=\frac{4}{3}\pi r^3\) og yfirborðsflatarmál hennar er \(Y=4\pi r^2\).

\begin{figure}[htbp]
\centering

\noindent\sphinxincludegraphics[width=0.400\linewidth]{{fl_kula}.svg}
\end{figure}


\chapter{Mengi}
\label{\detokenize{Kafli04:mengi}}\label{\detokenize{Kafli04::doc}}\phantomsection\label{\detokenize{Kafli04:s-mengi}}

\bigskip\hrule\bigskip



\section{Grunnhugtök}
\label{\detokenize{Kafli04:grunnhugtok}}
\textit{Mengi} er safn aðgreindra hluta eða hugtaka sem saman mynda eina heild. Hlutirnir eða
hugtökin sem mynda mengið nefnast \sphinxstyleemphasis{stök} þess. Ef \(x\) er \textit{stak} í menginu \(A\), þá skrifum við
\(x \in A\). Ef \(x\) er ekki stak í menginu \(A\) þá skrifum við \(x \notin A\).

\noindent{\hspace*{\fill}\sphinxincludegraphics[width=0.600\linewidth]{{mengiA}.svg}\hspace*{\fill}}

Hér dæmi um mengi \(A\) þar sem \(x_1 \in A\) en \(x_2 \notin A\).

Oft eru mengi sett fram sem upptalning á stökum. Til dæmis er
\begin{itemize}
\item {} 
\(\{1,2,3,\dots\}\) mengi náttúrlegra talna,

\item {} 
\(\{2,4,6,8,10\}\) mengið sem samanstendur af fimm fyrstu jákvæðu sléttu tölunum og

\item {} 
\(\{2,3,5,7,11\}\) mengið sem samanstendur af fimm fyrstu frumtölunum.

\end{itemize}

\textit{Tómamengið} er mengi sem inniheldur ekkert stak. Það er táknað með \(\emptyset\).


\subsection{Umfangsfrumsenda}
\label{\detokenize{Kafli04:umfangsfrumsenda}}
Tvö mengi \(A\) og \(B\) eru sögð vera jöfn,  ef þau innihalda sömu stök og við
skrifum þá \(A=B\).


\subsection{Hlutmengi}
\label{\detokenize{Kafli04:hlutmengi}}\phantomsection\label{\detokenize{Kafli04:s-hlutmengi}}
Mengið \(B\) er sagt vera \textit{hlutmengi} í menginu \(A\) ef sérhvert stak í \(B\) er einnig stak í \(A\). Við skrifum þá \(B \subset A\).

\noindent{\hspace*{\fill}\sphinxincludegraphics[width=0.600\linewidth]{{hlutmengi}.svg}\hspace*{\fill}}

Hér er dæmi um mengi \(A\) sem inniheldur mengið \(B\), m.ö.o. \(B \subset A\)

\begin{sphinxadmonition}{tip}{Dæmi:}
Mengið \(B=\{ 2,4,6 \}\) er hlutmengi í menginu \(A=\{1,2,3,4,5,6\}\) því öll stökin í \(B\) má líka finna í \(A\) .
\end{sphinxadmonition}


\section{Yrðingar til að skilgreina mengi}
\label{\detokenize{Kafli04:yringar-til-a-skilgreina-mengi}}
Stundum getur verið gagnlegt að skilgreina mengi með stökum sem öll hafa einhverja
ákveðna eiginleika. Við þurfum að geta táknað þetta mengi á einfaldan hátt en stundum eru
stökin óendanlega mörg og því ómögulegt að beinlínis telja þau upp eins og í dæmunum að
ofan.

\textit{Yrðing} er staðhæfing sem er annaðhvort sönn eða ósönn.
Oft skilgreinum við mengi með því að skrifa yrðingar um stök mengisins.
Við segjum að stak \(x\) sé í menginu ef og aðeins ef allar yrðingarnar um það eru sannar.

Formlegri leið til að segja þetta er:


\subsection{Skilgreining}
\label{\detokenize{Kafli04:skilgreining}}
Hægt er að setja fram mengi með opinni yrðingu \(p(x)\), þannig að mengið samanstandi af öllum stökum  \(x\) þannig að \(p(x)\) sé sönn yrðing.
\begin{equation*}
\begin{split}A = \{x \in C \ | \ p(x)\}\end{split}
\end{equation*}
Sjáum að mengið \(A\) er hlutmengi í \(C\) .

Þetta verður best skýrt með dæmum.

\begin{sphinxadmonition}{tip}{Dæmi:}
\sphinxstylestrong{1.} Látum \(A = \{x \text{ er frumtala }| x \text{ hefur } 3 \text{ í einingasætinu }\}\).
\begin{quote}

Nú getum við sagt að t.d. \(3 \in A, 13 \in A, 103 \in A\) þar sem allar þessar tölur eru frumtölur með \(3\) í einingarsætinu.

\(33\) er ekki stak í \(A\) (ritað \(33 \notin A)\) því \(33\) er ekki frumtala.

\(51\) er heldur ekki stak í \(A\) því að hún hefur \(1\) í einingasætinu en ekki \(3\).
\end{quote}

\sphinxstylestrong{2.} Látum \(C = \{x \text{ er heiltala }| x \text{ er slétt tala }, x \text{ er oddatala}\}\).
\begin{quote}

Hér er \(C = \emptyset\) þar sem að engin tala getur verið bæði slétt tala og oddatala í einu.
\end{quote}
\end{sphinxadmonition}


\section{Aðgerðir á mengjum}
\label{\detokenize{Kafli04:agerir-a-mengjum}}
Ef \(A\) og \(B\) eru mengi þá táknum við mengi allra staka sem eru í \(A\) eða í \(B\) með
\(A\cup B\). Þetta mengi köllum við \textit{sammengi} \(A\) og \(B\).
Formlega skilgreiningin er:
\begin{equation*}
\begin{split}A\cup B = \{x| x \in A \text{ eða } x \in B\}.\end{split}
\end{equation*}
\begin{sphinxadmonition}{note}{Athugasemd:}
Í stærðfræðilegu samhengi hefur samtengingin „eða“ merkinguna „og/eða“.
\end{sphinxadmonition}

Mengi allra staka sem eru bæði í \(A\) og \(B\) er táknað með \(A \cap B\). Þetta mengi er kallað
\textit{sniðmengi} \(A\) og \(B\).
Formlega skilgreiningin er:
\begin{equation*}
\begin{split}A\cap B = \{x| x \in A \text{ og } x \in B\}.\end{split}
\end{equation*}
\begin{sphinxadmonition}{warning}{Aðvörun:}
Við segjum að \(A\) og \(B\) séu \textit{sundurlæg} ef sniðmengið er tómamengið, þ.e. ef mengin hafa ekkert sameiginlegt stak.
\end{sphinxadmonition}

Mengi allra staka sem eru í \(A\) en ekki í \(B\) er kallað mismunur (eða \textit{mengjamismunur}) \(A\) og
\(B\). Hann er táknaður með \(A\backslash B\).
Formlega skilgreiningin er:
\begin{equation*}
\begin{split}A\backslash B = \{x| x \in A \text{ og }x \notin B\}.\end{split}
\end{equation*}
\noindent{\hspace*{\fill}\sphinxincludegraphics[width=1.000\linewidth]{{snidmengi}.svg}\hspace*{\fill}}

Hér er dæmi um tvö mengi \(A\) og \(B\) sem hafa sniðmengi, m.ö.o eru ekki \textit{sundurlæg}. \(A \cap B\) er merkt þar sem hringirnir skarast og \(A\backslash B\) er merkt með þykkum ramma.

\begin{sphinxadmonition}{tip}{Dæmi:}
Látum \(A=\{x\in\mathbb{N}|x\text{ er slétt tala}\},B=\{x\in\mathbb{N}|x\text{ er oddatala}\}\) og \(C=\{2,3,5,6,8\}\)

Hér er \(A\cup B=\mathbb{N}\) því að allar náttúrulegar tölur eru annað hvort sléttar tölur eða oddatölur.

\(A\cap B=\emptyset\) því að engin tala er bæði slétt tala og oddatala.

\(A\setminus B=A\) því að ekkert stak í \(A\) er líka í \(B\) og því er ekkert dregið frá.

\(A\cap C=\{2,6,8\}\)

\(C\setminus B=\{2,6,8\}\)
\end{sphinxadmonition}


\subsection{Faldmengi}
\label{\detokenize{Kafli04:faldmengi}}
\textit{Faldmengi} eða margfeldismengi \(A\times B\) tveggja mengja \(A\) og \(B\) er skilgreint sem mengi allra
para \((a,b)\) af stökum þ.a. \(a \in A\) og \(b \in B\). Með yrðingum er þetta skrifað:
\begin{equation*}
\begin{split}A\times B = \{(a,b)| a \in A \text{ og } b \in B\}.\end{split}
\end{equation*}
\begin{sphinxadmonition}{tip}{Dæmi:}
Látum \(A=\{2,3,6\}\)

\(\left(2,\dfrac{5}{4}\right)\) er stak í \(\mathbb{N}\times\mathbb{Q}\). Það er  ritað \(\left(2,\dfrac{5}{4}\right)\in \mathbb{N}\times\mathbb{Q}\)

\(\left(2,\dfrac{5}{4}\right)\) er líka stak í \(\mathbb{Q}\times\mathbb{Q}\) því að \(2=\dfrac{2}{1}\) er í báðum mengjunum \(\mathbb{N}\) og \(\mathbb{Q}\)

\(\left(2,\dfrac{5}{4}\right)\) er líka stak í menginu \(A\times\mathbb{Q}\)

\(\left(2,\dfrac{5}{4}\right)\) er \sphinxstylestrong{ekki} stak í menginu \(\mathbb{N}\times\mathbb{N}\) því \(\dfrac{5}{4}\) er ekki í \(\mathbb{N}\)
\end{sphinxadmonition}


\subsection{Fyllimengi}
\label{\detokenize{Kafli04:fyllimengi}}
Þegar verið er að fjalla um hlutmengi \(A\) í ákveðnu mengi \(X\), þá er mengið \(X \backslash A\) oft nefnt \textit{fyllimengi} hlutmengisins \(A\), það er einnig táknað \(A^c\).
Í \(A^c\) eru því öll stök sem eru í \(X\) en ekki í \(A\) .

\noindent{\hspace*{\fill}\sphinxincludegraphics[width=0.450\linewidth]{{fyllimengi}.svg}\hspace*{\fill}}

Hér er bláa svæðið fyllimengi hlutmengisins  \(A\), \(A^c\).

Mengið X er kallað \textit{almengi} og inniheldur alla hlutina sem verið er að vinna með. Oftast er ljóst af samhenginu hvað
þetta almengi er. Í dæminu að ofan sjáum við að \(X\) er allt svæðið inní rétthyrningum.


\section{Meira um aðgerðir á mengjum}
\label{\detokenize{Kafli04:meira-um-agerir-a-mengjum}}
Auðvelt er að sannfæra sig um eftirfarandi reiknireglur á mengjum:
\begin{equation*}
\begin{split}\left(A\cup B\right)\cup C=A\cup\left(B\cup C\right)\end{split}
\end{equation*}\begin{equation*}
\begin{split}\left(A\cap B\right)\cap C=A\cap\left(B\cap C\right)\end{split}
\end{equation*}
Þessi regla segir að það skipti ekki máli í hvaða röð maður tekur sammengi og sniðmengi.
Því má skrifa \(A\cup B\cup C\) eða \(A\cap B\cap C\) og sleppa öllum svigum.

\begin{sphinxadmonition}{warning}{Aðvörun:}
Það þarf alls ekki að gilda að \(\left(A\cup B\right)\cap C=A\cup\left(B\cap C\right)\), til dæmis. Lesandi er hvattur til að ganga úr skugga um þetta sjálfur.

Það skiptir höfuðmáli hvaða aðgerð er gerð fyrst þegar sam\sphinxhyphen{} og sniðmengjum er blandað saman. Að nota sviga er nauðsynlegt;  skrifa \(A\cup B\cap C\) eða \(A\cap B\cup C\) er merkingarlaust.
\end{sphinxadmonition}

\begin{sphinxadmonition}{tip}{Dæmi:}
Gefin eru mengin  \(A:= \{ 1,2,3,4,5 \}, B := \{ 2,4,6,8,10\}\) og \(C := \{ 6,7,8,9,10\}\)

\sphinxstylestrong{1.} Finnið \((A \cup B) \cap C\).
\begin{quote}

Byrjum á að finna \(A \cup B\). Það er mengi allra staka sem eru stök í öðru hvoru mengjanna \(A\) eða \(B\), það er, \(A \cup B = \{1,2,3,4,5,6,8,10 \}\).

\((A \cup B) \cap C\) inniheldur síðan nákvæmlega þau stök sem eru bæði í \(A \cup B\) og \(C\).

\((A \cup B) \cap C = \{6,8,10 \}\).
\end{quote}

\sphinxstylestrong{2.}  Finnið \(A \cup (B \cap C)\).
\begin{quote}

Nú er \(B \cap C = \{6,8,10 \}\) og þá er \(A \cup (B \cap C) = \{1,2,3,4,5,6,8,10 \}\).

Tökum eftir að hér er dæmi þar sem að \((A \cup B) \cap C \neq A \cup (B  \cap C)\) gildir.
\end{quote}

\sphinxstylestrong{3.} Finnið \((A \cap B) \cap C\).
\begin{quote}

Nú er \(A \cap B = \{ 2,4 \}\) svo \((A \cap B) \cap C = \{2,4 \} \cap \{6,7,8,9,10 \} = \emptyset\) því \(2\) og \(4\) eru ekki í \(C\) .
\end{quote}
\end{sphinxadmonition}


\bigskip\hrule\bigskip


Nú skulum við skilgreina sam\sphinxhyphen{} og sniðmengi fleiri en tveggja mengja.
Látum \(n \in \mathbb{N_+}\) og \(A_1,A_2,\dots,A_n\) vera mengi. Látum \(I = \{1, \dots, n \}\). Skilgreinum:
\begin{equation*}
\begin{split}\begin{aligned}
& \bigcup_{i=1}^n A_i=\{x|x\in A_i \text{ fyrir eitthvað } i = 1, \dots, n \}, \\
\quad\\
& \bigcap_{i=1}^n A_i=\{x|x\in A_i \text{ fyrir öll }  i = 1, \dots, n\}.
\end{aligned}\end{split}
\end{equation*}
Í raun er \(\bigcup_{i=1}^n A_i\) bara önnur leið til að skrifa
\begin{equation*}
\begin{split}A_1\cup A_2\cup A_3\cup...\cup A_n\end{split}
\end{equation*}
og \(\bigcap_{i=1}^n A_i\) er bara önnur leið til að skrifa
\begin{equation*}
\begin{split}A_1\cap A_2\cap A_3\cap...\cap A_n\end{split}
\end{equation*}
\begin{sphinxadmonition}{note}{Athugasemd:}
Hér nýtum við okkur reikniregluna að \(\left(A\cup B\right)\cup C=A\cup\left(B\cup C\right)\), og hliðstæðu hennar fyrir sniðmengi, aftur og aftur.
\end{sphinxadmonition}

Inn á milli kemur fyrir að stærðfræðingur vilji taka sammengi óendanlegra margra mengja. Segjum að við höfum eitthvað safn af mengjum (eða mengi af mengjum) þannig að búið sé að merkja öll mengin með einhverjum \textit{vísi} úr einhverju \textit{vísamengi} \(I\). Það er, öll mengin í safninu má tákna með \(A_i\) með \(i\in I\), þar sem \(I \neq \emptyset\). Þá er sammengi allra þessara mengja táknað með \(\bigcup_{i\in I}A_i\).

Með yrðingum er þetta skilgreint:
\begin{equation*}
\begin{split}\bigcup_{i\in I}A_i=\{x|x\in A_i \text{ fyrir eitthvað } i\in I \}\end{split}
\end{equation*}
Eins eru sniðmengin skilgreind:
\begin{equation*}
\begin{split}\bigcap_{i\in I}A_i=\{x|x\in A_i \text{ fyrir öll } i\in I \}\end{split}
\end{equation*}
Tökum nokkur dæmi um þetta.

\begin{sphinxadmonition}{tip}{Dæmi:}
\sphinxstylestrong{1.} Látum \(\mathbb{P}\) tákna mengi allra frumtalna.
\begin{quote}

Fyrir sérhvert \(p\in\mathbb{P}\) skulum við láta \(A_p\) vera mengi allra náttúrulegra talna sem \(p\) gengur upp í. Með yrðingum skrifum við:
\begin{equation*}
\begin{split}A_p=\{n\in\mathbb{N}|p\text{ gengur upp í }n \}\end{split}
\end{equation*}
Hér er vísismengið \(\mathbb{P}\) og
\begin{equation*}
\begin{split}\bigcup_{p\in\mathbb{P}}A_p=\mathbb{N}\setminus\{1\}\end{split}
\end{equation*}
Það er af því að sérhver tala í \(\mathbb{N}\) sem er stærri en \(1\) er deilanleg með einhverri frumtölu, og því er til \(p\) þannig að talan sé í \(A_p\).
\end{quote}

\sphinxstylestrong{2.} Fyrir sérhvert \(n\in \mathbb{Z}\) skulum við láta \(B_n\) vera mengi allra almennra brota sem hafa \(n\) sem teljara þegar þau eru fullstytt. Með yrðingum skilgreinum við þetta mengi:
\begin{quote}
\begin{equation*}
\begin{split}B_n=\{r\in\mathbb{Q}|\,\, \text{Ef }r=\dfrac{a}{b}\text{ og } \dfrac{a}{b}\text{ er fullstytt brot þá er }a=n \}\end{split}
\end{equation*}
Hér er \(\mathbb{Z}\) vísismengið og:
\begin{equation*}
\begin{split}\bigcup_{n\in\mathbb{Z}}B_n=\mathbb{Q}\end{split}
\end{equation*}
Af því að sérhvert almennt brot er í einhverju af mengjunum \(A_n\).
\end{quote}

\sphinxstylestrong{3.} Látum \(T\) vera mengið sem hefur sem stök öll tré í heiminum. Ef \(t\in T\) er eitthvað tré látum við mengið \(L_t\) vera mengi allra laufblaða á trénu \(t\). Með yrðingum skrifum við:
\begin{quote}
\begin{equation*}
\begin{split}L_t=\{l\text{ er laufblað}|\, l\text{ er á trénu }t \}\end{split}
\end{equation*}
Hér er \(T\) vísismengið og
\begin{equation*}
\begin{split}\bigcup_{t\in T}L_t=\{l\text{ er laufblað}|\, l\text{ er á einhverju tréi } \}\end{split}
\end{equation*}\end{quote}
\end{sphinxadmonition}


\section{Rauntalnabil}
\label{\detokenize{Kafli04:rauntalnabil}}
Látum \(I\) vera hlutmengi í \(\mathbb{R}\).
Við köllum hlutmengið \(I\) \textit{bil} ef engin göt eru í \(I\) .
Með öðrum orðum, við segjum að mengið \(I\) sé bil ef við getum táknað það á talnalínunni með breiðu línustriki með engum götum.
Á hvorn endapunkt striksins setjum við annað hvort fylltan hring eða tóman, eftir því hvort sá \textit{endapunktur} sé með í bilinu eða ekki.
Ef punkturinn á að vera með setjum við fylltan hring, annars tóman.

\noindent{\hspace*{\fill}\sphinxincludegraphics[width=0.500\linewidth]{{bil}.svg}\hspace*{\fill}}

Formlega skilgreiningin á bili er svohljóðandi:


\subsection{Skilgreining}
\label{\detokenize{Kafli04:id1}}
Hlutmengi \(I\) í \(\mathbb{R}\) kallast \textit{bil} ef fyrir sérhvert \(a,b\in I\) og \(c\in\mathbb{R}\) þ.a. \(a<c<b\) þá gildir \(c\in I\) .


\subsection{Gerðir af bilum}
\label{\detokenize{Kafli04:gerir-af-bilum}}
Til að tákna bil í prenti þarf að nota tvær tölur, hornklofa og/eða sviga eftir aðstæðum og eina kommu. Hér verða nokkur bil útskýrð í töluðu máli:

Bilið \([a,b]\) er mengi allra rauntalna sem eru á milli \(a\) og \(b\), meðtaldar eru tölurnar \(a\) og \(b\).

Bilið \((a,b)\) er mengi allra rauntalna sem eru á milli \(a\) og \(b\) en hér eru \(a\) og \(b\) frátaldar.

Bilið \((a,\infty)\) er mengi allra rauntalna sem eru stærri en \(a\) en hér er \(a\) ekki tekið með.

Bilið \([a,b)\) er mengi allra rauntalna sem eru á milli \(a\) og \(b\) að stakinu \(a\) meðtöldu en án staksins \(b\).

\begin{sphinxadmonition}{note}{Athugasemd:}
Hér eru notaðir svigar fyrir opin bil, en í sumum bókum er opið bil táknað með því að snúa hornklofunum öfugt.
Því \(]a,b[\) táknar það sama og \((a,b)\) .
\end{sphinxadmonition}

Hér er tæmandi listi yfir allar gerðir af endanlegum bilum, skilgreindum með yrðingum:

Látum \(a\) og \(b\) vera rauntölur þannig að
\(a<b\). Skilgreinum
\begin{enumerate}
\sphinxsetlistlabels{\arabic}{enumi}{enumii}{}{.}%
\item {} 
\textit{opið bil} \((a,b)=\{x\in \mathbb{R}| a<x<b\}\)

\item {} 
\textit{lokað bil} \([a,b]=\{x\in \mathbb{R}| a\leq x\leq b\}\)

\item {} 
\textit{hálfopið bil} \([a,b)=\{x\in \mathbb{R}| a\leq x<b\}\)

\item {} 
\textit{hálfopið bil} \((a,b]=\{x\in \mathbb{R}| a< x\leq b\}\)

\end{enumerate}

\noindent{\hspace*{\fill}\sphinxincludegraphics[width=0.600\linewidth]{{endanlegbil}.svg}\hspace*{\fill}}


\bigskip\hrule\bigskip


Óendanlegu bilin eru þau sem halda áfram óendanlega langt í aðra hvora eða báðar áttir.
Látum \(a\) vera rauntölu. Skilgreinum
\begin{enumerate}
\sphinxsetlistlabels{\arabic}{enumi}{enumii}{}{.}%
\setcounter{enumi}{4}
\item {} 
\sphinxstyleemphasis{opið óendanlegt bil} \((a,\infty)=\{x\in \mathbb{R}| a<x\}\)

\item {} 
\sphinxstyleemphasis{opið óendanlegt bil} \((-\infty, a)=\{x\in \mathbb{R}; x<a\}\)

\item {} 
\sphinxstyleemphasis{lokað óendanlegt bil} \([a,\infty)=\{x\in \mathbb{R}; a\leq x\}\)

\item {} 
\sphinxstyleemphasis{lokað óendanlegt bil} \((-\infty, a]=\{x\in \mathbb{R}; x\leq a\}\)

\item {} 
\sphinxstyleemphasis{öll rauntalnalínan} \((-\infty, \infty)= \mathbb{R}\).

\end{enumerate}

\noindent{\hspace*{\fill}\sphinxincludegraphics[width=0.600\linewidth]{{oendanlegbil}.svg}\hspace*{\fill}}


\chapter{Föll}
\label{\detokenize{Kafli05:foll}}\label{\detokenize{Kafli05::doc}}

\section{Varpanir og föll}
\label{\detokenize{Kafli05:varpanir-og-foll}}
\textit{Vörpun} er eitt allra mikilvægasta hugtakið sem notað er í stærðfræði.
Mjög mikilvægt er að nemendur reyni að skilja þetta hugtak fullkomlega og geti tileinkað sér notkun þess.
Útskýringar á vörpunum eru því miður oftast frekar torskiljanlegar fyrir þá sem ekki eru vanir stærðfræði, en með sýnidæmum líkt og hér að neðan skýrast hlutirnir vonandi betur.

Látum \(X\) og \(Y\) vera mengi. \sphinxstyleemphasis{Vörpun} frá \(X\) yfir í \(Y\) er regla sem úthlutar sérhverju staki í \(X\) nákvæmlega einu staki í \(Y\). Hefðin er að tákna slíka vörpun (reglu) með bókstaf.
Ef \(f\) er vörpun frá \(X\) yfir í \(Y\) og \(x \in X\), þá táknum við með \(f(x)\) stakið í \(Y\) sem vörpunin \(f\) úthlutar stakinu \(x\).

Mengið \(X\) kallast \textit{skilgreiningarmengi} (eða \sphinxstyleemphasis{formengi}) vörpunnarinnar og mengið \(Y\) \textit{bakmengi} (eða \sphinxstyleemphasis{myndmengi} eða \sphinxstyleemphasis{varpmengi}) hennar.

Varpanir eru táknaðar með ýmsum hætti, til dæmis
\begin{itemize}
\item {} 
\(f: X\to Y\)

\item {} 
\(X \overset{f} \to Y\)

\item {} 
\(X\to Y, \ x\mapsto f(x)\)

\end{itemize}

en sú fyrsta er mest notuð innan Háskóla Íslands.

\noindent{\hspace*{\fill}\sphinxincludegraphics[width=0.500\linewidth]{{vorpun}.svg}\hspace*{\fill}}

\begin{sphinxadmonition}{note}{Athugasemd:}\begin{description}
\item[{Þegar bakmengið \(Y\) er hlutmengi í rauntölunum \(\mathbb{R}\) þá tölum við yfirleitt um \textit{fall} í stað vörpunar.}] \leavevmode
Í raun er fall bara ákveðin gerð af vörpun, þar sem rauntölum er varpað á rauntölur.
Í þessum kafla munum við aðallega fjalla um föll.

\end{description}
\end{sphinxadmonition}

\begin{sphinxadmonition}{tip}{Dæmi:}
\sphinxstylestrong{1.} Látum \(X=\mathbb{R}\) og \(Y=\mathbb{R}\). Skilgreinum nú fall sem segir að sérhverju staki \(x\in X\) verði úthlutað stakinu \(x^2\in Y\).
\begin{quote}

Köllum þetta fall \(f\).

Fallið tekur stakið \(2 \in X\) og úthlutar því stakinu \(2^2 = 4 \in Y\). Fallið tekur stakið \(9 \in X\) og úthlutar því stakinu \(9^2 = 81 \in Y\).
Með öðrum orðum þá varpar fallið stakinu \(2 \in X\) í stakið \(2^2 = 4 \in Y\), og varpar \(9 \in X\) í \(9^2 = 81 \in Y\).

Það er, \(f(2)=4\) og \(f(9)=81\).

Við viljum oftast skrifa þessa reglu með því að nota táknmál, því það er fljótlegra en að skrifa allan textann að ofan. Við getum skilgreint þetta fall með því að skrifa:
\end{quote}
\begin{equation*}
\begin{split}f: \mathbb{R} \to \mathbb{R}, \qquad f(x)=x^2\end{split}
\end{equation*}
\sphinxstylestrong{2.} Skilgreinum fall
\begin{quote}
\begin{equation*}
\begin{split}g: \mathbb{R} \to \mathbb{R}, \qquad g(x)=3x+2\end{split}
\end{equation*}
Þetta táknmál er nóg til að skilgreina fallið því við getum lesið úr því hvað fallið gerir. Það tekur hvert stak \(x\) í \(\mathbb{R}\) og úthlutar því staki með því að margfalda \(x\) fyrst með \(3\) og bæta svo \(2\) við.

Við getum t.d. reiknað út að
\begin{equation*}
\begin{split}\begin{aligned} f(3)&=3 \cdot 3 +2 = 11 \\ f\left(\frac{17}{18}\right)&=3 \cdot \frac{17}{18}+2=\frac{29}{6}\\ \end{aligned}\end{split}
\end{equation*}
Þetta getum við gert við hvert einasta stak í \(\mathbb{R}\), það er, hverja einustu rauntölu. Þessi regla varpar hverri rauntölu í einhverja aðra rauntölu.
\end{quote}

\sphinxstylestrong{3.} Við getum líka skilgreint fall frá rauntalnabili. Skoðum vörpunina í fyrri lið, en látum núna
\begin{quote}
\begin{equation*}
\begin{split}g: [1,7] \to \mathbb{R}, \qquad g(x)=3x+2\end{split}
\end{equation*}
Hér er skilgreiningarmengið bil. Fallið varpar nú einungis hverju staki á bilinu \([1,7]\) í eitthvert stak \(g(x)\), sem er rauntala. Til dæmis er \(g(1)=3 \cdot 1 + 2=5\), en til dæmis er talan \(8\) ekki í skilgreiningarmenginu svo hún fær ekkert gildi.
\end{quote}
\end{sphinxadmonition}


\subsection{Dæmi um vörpun sem er ekki fall}
\label{\detokenize{Kafli05:daemi-um-vorpun-sem-er-ekki-fall}}
Munum að fall er vörpun þar sem bakmengið er \(\mathbb{R}\), það er, fyrir fall \(f\) er \(f(x)\) alltaf rauntala. En varpanir geta átt við um eitthvað annað en tölur.

Látum \(A\) vera mengi allra íslenskra orða og \(B\) vera mengi allra íslenskra bókstafa. Skilgreinum nú vörpun á \(A\) sem úthlutar sérhverju orði í \(A\) fyrsta bókstafnum í því. Köllum þessa vörpun \(h\).

Þessi vörpun úthlutar orðinu ,,grís‘‘ bókstafnum ,,g‘‘ og þess vegna skrifum við \(h(\text{grís})=\text{g}\).
Þessi vörpun úthlutar orðinu ,,kirkja‘‘ bókstafnum ,,k‘‘ og þess vegna skrifum við \(h(\text{kirkja})=\text{k}\).

Í liðnum á undan náðum við að skilgreina föllin \(f\) og \(g\) með formúlu. Hér er engin formúla til og við verðum að láta okkur nægja að útskýra hana með orðum.


\section{Graf vörpunnar}
\label{\detokenize{Kafli05:graf-vorpunnar}}
\textit{Graf} vörpunnar er mengið
\begin{equation*}
\begin{split}\{(x,y) \in X \times Y;y=f(x)\}\end{split}
\end{equation*}
Það er mengi allra \textit{tvennda} \((x,y)\) þannig að \(x \in X\) og \(y \in Y\) uppfyllir jöfnuna \(y=f(x)\).

Það getur verið gagnlegt að teikna upp mynd af grafinu í hnitakerfi. Skoðum dæmi um það að teikna myndir af gröfum falla.

\begin{sphinxadmonition}{tip}{Dæmi:}
\sphinxstylestrong{1.} Skoðum aftur fallið
\begin{quote}
\begin{equation*}
\begin{split}g: \mathbb{R} \to \mathbb{R}, \qquad g(x)=3x+2\end{split}
\end{equation*}
Til að teikna graf þess getum við reiknað út nokkra punkta sem tilheyra því.

Ef \(x=0\) þá er \(y=g(0)=3 \cdot 0+2=2\). Við merkjum punktinn \((0,2)\) á myndina.

Ef \(x=2\) þá er \(y=g(2)=3 \cdot 2 +2=8\). Við merkjum punktinn \((2,8)\) á myndina.

Ef \(x=-3\) þá er \(y=g(-3)= 3 \cdot (-3)+2=-7\). Við merkjum punktinn \((-3,-7)\) á myndina.

Næst teiknum við feril sem fer í gegnum punktana, en þeir liggja allir á beinni línu. Við höfum áður séð að allir punktar í hnitakerfi sem uppfylla jöfnu á forminu \(y=hx+s\) liggja á línu í plani. Grafið er því \textit{lína} og teygir sig óendanlega langt í báðar áttir.

\noindent{\hspace*{\fill}\sphinxincludegraphics{{graf1}.svg}\hspace*{\fill}}
\end{quote}

\sphinxstylestrong{2.} Skoðum fallið
\begin{quote}
\begin{equation*}
\begin{split}f: \mathbb{R} \to \mathbb{R}, \qquad f(x)=x^2\end{split}
\end{equation*}
Reiknum út nokkra punkta sem tilheyra grafinu.

Ef \(x=0\) þá er \(y=f(0)=0^2=0\). Við merkjum punktinn \((0,0)\) á myndina.

Ef \(x=1\) þá er \(y=f(1)= 1^2=1\). Við merkjum punktinn \((1,1)\) á myndina.

Ef \(x=-1\) þá er \(y=f(-1)= (-1)^2=1\). Við merkjum punktinn \((-1,1)\) á myndina.

Ef \(x=2\) þá er \(y=f(2)= 2^2=4\). Við merkjum punktinn \((2,4)\) á myndina.

Á sama hátt fást punktarnir \((-2,4)\), \((3,9)\) og \((-3,9)\).

Teiknum nú feril sem fer í gegnum alla punktana. Þessi ferill kallast \textit{fleygbogi}.

\noindent{\hspace*{\fill}\sphinxincludegraphics{{graf2}.svg}\hspace*{\fill}}

Öll gildin sem eru á þessum ferli, það er allir punktar á svörtu línunni, uppfylla skilyrðið \(y=x^2\).
\end{quote}
\end{sphinxadmonition}


\section{Jafnstæð og oddstæð föll}
\label{\detokenize{Kafli05:jafnstae-og-oddstae-foll}}

\subsection{Skilgreining}
\label{\detokenize{Kafli05:skilgreining}}
Látum \(f: \mathbb{R} \to \mathbb{R}\) vera fall.

Við segjum að \(f\) sé \textit{jafnstætt} ef \(f(-x)=f(x)\) fyrir öll \(x \in \mathbb{R}\).

Við segjum að \(f\) sé \textit{oddstætt} ef \(f(-x)=-f(x)\) fyrir öll \(x \in \mathbb{R}\).


\subsection{Myndræn útskýring}
\label{\detokenize{Kafli05:myndraen-utskyring}}
Skilgreiningin segir að fall sé jafnstætt ef graf þess er eins ef því er speglað um \(y\)\sphinxhyphen{}ásinn.

Eins þá er fall oddstætt ef graf þess er eins ef því er speglað um \(y\)\sphinxhyphen{}ás og svo speglað um \(x\)\sphinxhyphen{}ásinn.

\noindent{\hspace*{\fill}\sphinxincludegraphics{{jafnst}.svg}\hspace*{\fill}}

Myndin að ofan sýnir jafnstætt fall.

\noindent{\hspace*{\fill}\sphinxincludegraphics{{oddst}.svg}\hspace*{\fill}}

Myndin að ofan sýnir oddstætt fall.

\begin{sphinxadmonition}{note}{Athugasemd:}
Ekki eru öll föll oddstæð eða jafnstæð. Föll geta verið hvorugt.
\end{sphinxadmonition}

\begin{sphinxadmonition}{tip}{Dæmi:}
\sphinxstylestrong{Skerum úr um hvort föllin séu jafnstæð, oddstæð, eða hvorugt.}

\sphinxstylestrong{1.} Fallið \(f:\;\mathbb{R} \to\mathbb{R}\) gefið með  \(f(x)=x^2\).
\begin{quote}

\(f\) er jafnstætt því að fyrir öll \(x\in\mathbb{R}\) gildir \(f(-x)=(-x)^2=x^2=f(x)\)
\end{quote}

\sphinxstylestrong{2.} Fallið \(g:\;\mathbb{R} \to\mathbb{R}\) gefið með  \(g(x)=x^3\).
\begin{quote}

\(g\) er oddstætt því að fyrir öll \(x\in\mathbb{R}\) gildir \(g(-x)=(-x)^3=-x^3=-g(x)\)
\end{quote}

\sphinxstylestrong{3.} Fallið \(h:\;\mathbb{R} \to\mathbb{R}\) gefið með  \(h(x)=x^2+x^3\).
\begin{quote}

Hér er \(h\) hvorki jafnstætt né oddstætt. Til að sýna það þurfum við einfaldlega að finna dæmi um \(x\) þannig að \(h(-x)\not=h(x)\) og \(h(-x)\not=-h(x)\).

Ef að við prófum \(x=2\) fáum við að \(h(2)=12\), \(h(-2)=-4\).
Við sjáum þá að \(h(-2)\not=h(2)\) og \(h(-2)\not=-h(2)\).
\end{quote}
\end{sphinxadmonition}


\section{Einhalla föll}
\label{\detokenize{Kafli05:einhalla-foll}}

\subsection{Skilgreining}
\label{\detokenize{Kafli05:id1}}
Látum \(X\) vera hlutmengi í rauntölunum og  \(\ f: X \to \mathbb{R}\) vera fall.

Ef um sérhver \(x_1,x_2 \in X\) sem eru þannig að \(x_1<x_2\) gildir
\begin{equation*}
\begin{split}f(x_1) \leq f(x_2)\end{split}
\end{equation*}
þá er \(f\) sagt vera \textit{vaxandi} fall.

Ef um sérhver \(x_1,x_2 \in X\) sem eru þannig að \(x_1<x_2\) gildir
\begin{equation*}
\begin{split}g(x_1) \geq g(x_2)\end{split}
\end{equation*}
þá er \(g\) sagt vera \textit{minnkandi} fall.

\noindent{\hspace*{\fill}\sphinxincludegraphics[width=0.450\linewidth]{{vaxmin}.svg}\hspace*{\fill}}

Hér sjáum við dæmi um fall \(f\) sem er vaxandi og fall \(g\) sem er minnkandi.

Ef ójöfnurnar fyrir fallgildin í skilgreiningunum væru strangar væri \(f\) sagt vera \textit{stranglega vaxandi} og \(g\) sagt vera \textit{stranglega minnkandi} fall.

Fall sem er annaðhvort vaxandi eða minnkandi er sagt vera \textit{einhalla}.

Fall sem er annaðhvort stranglega vaxandi eða stranglega minnkandi er sagt vera stranglega einhalla.

\begin{sphinxadmonition}{note}{Athugasemd:}
Stranglega vaxandi fall er sér í lagi vaxandi, en vaxandi fall er ekki endilega stranglega vaxandi. Eins eru stranglega minnkandi föll sér í lagi minnkandi, en ekki endilega öfugt.
\end{sphinxadmonition}

\begin{sphinxadmonition}{tip}{Dæmi:}
\sphinxstylestrong{1.} Byrjum á því að skoða línur í plani.
\begin{quote}

Fallið \(f: \mathbb{R} \to \mathbb{R}\) þannig að \(f(x)=x+2\) er stranglega vaxandi. Ljóst er að fyrir öll \(x_1,x_2\) þannig að \(x_1>x_2\) þá er \(f(x_1)>f(x_2)\). Við vitum að hallatalan er jákvæð og því er ljóst að ef við færum okkur til hægri eftir \(x\)\sphinxhyphen{}ásnum á mynd grafsins þá hækkar fallgildið.

Fallið \(g: \mathbb{R} \to \mathbb{R}\) þannig að \(g(x)=-2x+1\) er stranglega minnkandi, vegna þess að hallatalan er neikvæð.

Fallið \(h: \mathbb{R} \to \mathbb{R}\) \(h(x)=2\) er fastafall.
Það varpar hverri einustu rauntölu yfir í töluna \(2\).
Graf þess er bein lína með hallatölu \(0\).

\sphinxstyleemphasis{Fallið} \(h\) \sphinxstyleemphasis{er bæði vaxandi og minnkandi}.

Þetta er vegna þess að jafnaðarmerkið í skilgreiningunni gildir, þ.e. um öll stök \(x_1, x_2 \in \mathbb{R}\) gildir \(h(x_1)=h(x_2)\).
Samkvæmt skilgreiningunni er vaxandi fall þannig að \(h(x_1) \geq h(x_2)\) fyrir \(x_1>x_2\), og minnkandi fall er þannig að \(h(x_1) \leq h(x_2)\) fyrir \(x_1<x_2\).
Í þessu tilfelli gildir jafnaðarmerkið í öllum tilfellum og því getum við sagt að fallið sé bæði minnkandi og vaxandi.
\(h\) er hins vegar hvorki stranglega minnkandi eða stranglega vaxandi.

\noindent{\hspace*{\fill}\sphinxincludegraphics{{einhalla}.svg}\hspace*{\fill}}
\end{quote}

\sphinxstylestrong{2.} Föll geta verið vaxandi/minnkandi á bili. Skoðum fallið \(f: \mathbb{R} \to \mathbb{R}, f(x)=x^2\). Þetta fall er stranglega vaxandi á bilinu \((0, \infty)\), en stranglega minnkandi á bilinu \((-\infty, 0)\).
\begin{quote}

\noindent{\hspace*{\fill}\sphinxincludegraphics{{einhalla2}.svg}\hspace*{\fill}}
\end{quote}
\end{sphinxadmonition}

\begin{sphinxadmonition}{note}{Athugasemd:}
Ávallt ber að varast að ákvarða hvort fall sé einhalla út frá mynd. Hægt er að nota diffrun til að ákvarða nákvæmlega hvar fall er minnkandi eða vaxandi, en við förum ekki yfir það hér.
\end{sphinxadmonition}


\section{Gaffalforskrift}
\label{\detokenize{Kafli05:gaffalforskrift}}
Sumum föllum er ekki endilega hægt að lýsa með einni jöfnu, t.d. þegar lýsa þarf lotubundnum föllum. Þá er fallinu oft lýst með mismunandi formúlum á mismunandi bilum. Skoðum fallið
\begin{equation*}
\begin{split}f(x) =
\begin{cases}
x^2 \qquad x \geq 0\\
-2x+1 \quad x < 0
\end{cases}\end{split}
\end{equation*}
Þetta fall tekur því gildið \(-2x+1\) fyrir öll neikvæð gildi á \(x\) en jákvæð gildi eru sett í annað veldi.

\begin{figure}[htbp]
\centering

\noindent\sphinxincludegraphics[width=0.600\linewidth]{{gaffal}.svg}
\end{figure}


\section{Lotubundin föll}
\label{\detokenize{Kafli05:lotubundin-foll}}
Við segjum að fall sé \textit{lotubundið} ef það í vissum skilningi endurtekur sjálft sig aftur og aftur. Setjum fram formlega skilgreiningu:


\subsection{Skilgreining}
\label{\detokenize{Kafli05:id2}}
Fall \(f: \mathbb{R} \to \mathbb{R}\) er sagt vera lotubundið með lotu \(a\) ef \(a \in \mathbb{R}\) og \(f(x+a)=f(x)\) fyrir öll \(x \in \mathbb{R}\).

\begin{sphinxadmonition}{note}{Athugasemd:}
Óformlega þýðir þessi skilgreining að ef við færum okkur um fjarlægðina \(a\) á \(x\)\sphinxhyphen{}ásnum þá hefur fallið sama gildi þar, það er, það hefur sama gildi í punktinum \(x\) og punktinum \(x+a\), og hér má \(x\) vera hvaða tala sem er.
\end{sphinxadmonition}

\begin{sphinxadmonition}{tip}{Dæmi:}
Látum \(f: \mathbb{R} \to \mathbb{R}\) vera fallið sem er með lotu \(2\) og er skilgreint með:
\begin{equation*}
\begin{split}f(x)=|x| \qquad \text{ef} \qquad x \in [-1,1[\end{split}
\end{equation*}
Tökum eftir að formúlan er aðeins tekin fram fyrir bilið \([-1,1[\) en hún nægir samt til að skilgreina fallið á öllu \(\mathbb{R}\), því ef \(x\) er tala sem er ekki á þessu bili þá getum við fundið fallgildið með því að notfæra okkur lotu fallsins. Til dæmis ef við ætlum að reikna \(f(5)\) þá athugum við að
\begin{equation*}
\begin{split}f(5)=f(-1+3 \cdot 2)=f(-1)=|-1|=1\end{split}
\end{equation*}
þar sem lota fallsins er \(2\) fæst þetta út frá skilgreiningu.

Hér að neðan er mynd af fallinu. Upphaflega lotan sem gefin er með formúlu er mörkuð innan við punktalínur.

\begin{figure}[H]
\centering

\noindent\sphinxincludegraphics{{lotub}.svg}
\end{figure}
\end{sphinxadmonition}


\section{Andhverfur falla}
\label{\detokenize{Kafli05:andhverfur-falla}}

\subsection{Skilgreining}
\label{\detokenize{Kafli05:id3}}
Látum \(A\) og \(B\) vera mengi og \(f: A \to B\) vera \textit{vörpun}. Ef til er vörpun \(g: B \to A\) þannig að
\begin{equation*}
\begin{split}f(g(b))=b \qquad \text{fyrir öll } b \in B\end{split}
\end{equation*}
og
\begin{equation*}
\begin{split}g(f(a))=a \qquad \text{fyrir öll } a \in A\end{split}
\end{equation*}
þá kallast fallið \(g\) \textit{andhverfa vörpunarinnar} \(f\). Andhverfa vörpunarinnar \(f\) er oft táknuð með \(f^{-1}\).

Þá er \(f: A \to B\) og \(f^{-1}: B \to A\)
\begin{equation*}
\begin{split}f(f^{-1}(b))=b \qquad \text{fyrir öll } b \in B\end{split}
\end{equation*}
og
\begin{equation*}
\begin{split}f^{-1}(f(a))=a \qquad \text{fyrir öll } a \in A\end{split}
\end{equation*}
\noindent{\hspace*{\fill}\sphinxincludegraphics[width=0.700\linewidth]{{andhverfa}.svg}\hspace*{\fill}}

Hér sjáum við einfalt dæmi um andhverfa vörpun, þar sem \(f\) hefur \textit{skilgreiningarmengi} \(A\) og \textit{bakmengi} \(B\).

\begin{sphinxadmonition}{note}{Athugasemd:}
Vörpunin \(f^{-1}\) er sú regla sem úthlutar sérhverju staki \(f(a)\) í \(B\) stakinu \(a\) í \(A\). Það má orða það óformlega að andhverfa \(f\) sé vörpun sem gerir ,,akkúrat öfugt‘‘ við það sem vörpunin \(f\) gerir.
\end{sphinxadmonition}

\begin{sphinxadmonition}{tip}{Dæmi:}
\sphinxstylestrong{1.} Skilgreinum fall \(f:\; \mathbb{R}_+\to \mathbb{R}_+\) með formúlunni \(f(x)=x^2\). Finnum andhverfu fallsins \(f\).
\begin{quote}

Skilgreiningarmengið er hér jákvæðu rauntölurnar. Andhverfan er \(f^{-1}(x)=\sqrt{x}\). Staðfestum það:

Fyrir sérhvert \(x\in \mathbb{R}_+\) er \(f(f^{-1}(x))=(\sqrt{x})^2=x\).

Fyrir sérhvert \(x\in \mathbb{R}_+\) er \(f^{-1}(f(x))=\sqrt{x^2}=x\).

Andhverfan hefur verið staðfest.
\end{quote}

\sphinxstylestrong{2.} Skilgreinum fall \(f:\; \mathbb{R}_-\to \mathbb{R}_+\) með formúlunni \(f(x)=x^2\). Finnum andhverfu fallsins \(f\).
\begin{quote}

Tökum eftir að þetta er ekki alveg sama fallið og í dæminu á undan því að skilgreiningarmengið er annað, nú er skilgreiningarmengið neikvæðu rauntölurnar. Við sjáum að ef \(x\) er neikvæð rauntala þá er \(\sqrt{x^2}=-x\).

Til dæmis ef \(x=-3\) þá fæst \(\sqrt{x^2}=\sqrt{(-3)^2}=\sqrt{9}=3=-(-3)=-x\).

Þess vegna er andhverfufallið í þetta skiptið \(f^{-1}=-\sqrt{x}\).
Staðfestum það:

Fyrir sérhvert \(x\in\mathbb{R}_+\) þá er \(f(f^{-1}(x))=(-\sqrt{x})^2=(\sqrt{x})^2=x\)

Fyrir sérhvert \(x\in\mathbb{R}_-\) þá er \(f^{-1}(f(x))-\sqrt{x^2}=-(-x)=x\)

Þetta staðfestir andhverfuna.
\end{quote}
\end{sphinxadmonition}


\bigskip\hrule\bigskip


\begin{sphinxadmonition}{note}{Athugasemd:}
Ef við ætlum að finna andhverfu \(f : X \to Y\) þurfum við að umrita það með því að skipta á \(y\) í staðinn fyrir \(f(x)\) í formúlu fallsins og svo einangra \(x\)\sphinxhyphen{}ið.
Þá eru við komin með nýtt fall af \(y\) sem passar vegna þess að andhverfan á að vera \(f^{-1} : Y \to X\).
\end{sphinxadmonition}


\bigskip\hrule\bigskip


\begin{sphinxadmonition}{tip}{Dæmi:}
\sphinxstylestrong{1.} Látum \(f:\;\mathbb{R}\to\mathbb{R}\) vera fall gefið með formúlunni
\begin{quote}
\begin{equation*}
\begin{split}f(x)=x+4\end{split}
\end{equation*}
Finnum andhverfu fallsins. Við leitum að vörpun sem gerir ,,öfugt‘‘ við það sem \(f\) gerir.

Skrifum \(y\) í staðinn fyrir \(f(x)\) í formúlu fallsins.
\begin{equation*}
\begin{split}y=x+4\end{split}
\end{equation*}
Einangrum \(x\) úr þessari jöfnu
\begin{equation*}
\begin{split}x=y-4\end{split}
\end{equation*}
Þetta gefur okkur að andhverfa \(f\) er (skiptum um nafn á breytunni)
\begin{equation*}
\begin{split}f^{-1}(x)=x-4.\end{split}
\end{equation*}\end{quote}

\sphinxstylestrong{2.} Látum \(f\) vera fall gefið með formúlunni
\begin{quote}
\begin{equation*}
\begin{split}f(x)=\frac{x+5}{x-2}\end{split}
\end{equation*}
Finnum andhverfu fallsins.

Skrifum \(y\) í staðinn fyrir \(f(x)\) í formúlu fallsins
\begin{equation*}
\begin{split}y=\frac{x+5}{x-2}\end{split}
\end{equation*}
Einangrum nú \(x\) í þessari jöfnu:
Fáum
\begin{equation*}
\begin{split}y(x-2)=x+5\end{split}
\end{equation*}
Margföldum upp úr sviganum og færum yfir jafnaðarmerkið til að fá
\begin{equation*}
\begin{split}yx-x=5+2y\end{split}
\end{equation*}
Tökum \(x\) út fyrir sviga vinstra megin
\begin{equation*}
\begin{split}x(y-1)=5+2y\end{split}
\end{equation*}
Deilum í gegn með \((y-1)\) til að fá
\begin{equation*}
\begin{split}x=\frac{5+2y}{y-1}\end{split}
\end{equation*}
Nú höfum við einangrað \(x\) úr upphaflegu jöfnunni. Andhverfufallið okkar er þá (skiptum um breytu)
\begin{equation*}
\begin{split}f^{-1}(x)=\frac{5+2x}{x-1}\end{split}
\end{equation*}
\begin{sphinxadmonition}{note}{Athugasemd:}
Athugum að þegar skilgreiningarmengi falls er ekki tilgreint má gera ráð fyrir að það sé stærsta mögulega skilgreiningarmengið. Skilgreiningarmengi \(f\) yrði þess vegna hér \(\mathbb{R}\setminus\{2\}\). Tveir eru dregnir frá menginu því annars yrði deilt með núlli. Skilgreiningarmengi andhverfufallsins \(f^{-1}\) yrði \(\mathbb{R}\setminus\{1\}\) út af sömu ástæðu.
\end{sphinxadmonition}
\end{quote}
\end{sphinxadmonition}


\bigskip\hrule\bigskip


Ítarlegri umfjöllun um föll má finna {\hyperref[\detokenize{Kafli09:s-meiraumfoll}]{\sphinxcrossref{\DUrole{std,std-ref}{hér}}}} .


\chapter{Margliður}
\label{\detokenize{Kafli06:margliur}}\label{\detokenize{Kafli06::doc}}

\section{Margliður}
\label{\detokenize{Kafli06:id1}}
\textit{Margliða} er fall \(\mathbb{R} \to \mathbb{R}\) sem tákna má með formúlu af gerðinni
\begin{equation*}
\begin{split}p(x)=a_nx^n+a_{n-1}x^{n-1}+ \cdots + a_1x+a_0\end{split}
\end{equation*}
þar sem \(n\) er \textit{náttúruleg tala} og stuðlarnir \(a_j\) eru \textit{rauntölur} og \(a_n \neq 0\). Þá kallast talan \(n\) \sphinxstyleemphasis{stig} margliðunnar. Skoðum dæmi um margliður.

\begin{sphinxadmonition}{tip}{Dæmi:}
\sphinxstylestrong{1.} \(p(x)=8x^4+3x^2+2x+1\) er dæmi um margliðu.
\begin{quote}

Þessi margliða hefur stigið \(n=4\). Hér er \(a_0=1\), \(a_1=2\), \(a_2=3\), \(a_3=0\) og \(a_4=8\).
\end{quote}

\sphinxstylestrong{2.} \(p(x)=3x^5+\pi x^2-\dfrac{4}{5}\) er dæmi um margliðu.
\begin{quote}

Þessi margliða hefur stig \(n=5\). Hér er \(a_0=-\frac{4}{5}\), \(a_2=\pi\), \(a_5=3\) og \(a_1=a_3=a_4=0\).
\end{quote}

\sphinxstylestrong{3.} \(p(x)=3^x\) er ekki margliða.

\sphinxstylestrong{4.} \(p(x)=\sin(x)+4x^2\) er ekki margliða.
\end{sphinxadmonition}


\section{Núllstöðvar margliða}
\label{\detokenize{Kafli06:nullstovar-marglia}}
Ef \(p\) er margliða og \(x_0\) er tala þ.a. \(p(x_0)=0\) þá segjum við að \(x_0\) sé \textit{rót} eða \textit{núllstöð} margliðunnar \sphinxtitleref{p}.

Margliður geta haft margar núllstöðvar, en fjöldi þeirra er takmarkaður eins og fram kemur í eftirfarandi reglu:


\subsection{Regla}
\label{\detokenize{Kafli06:regla}}
Látum \(p\) vera margliðu af stigi \(n\). Fjöldi mismunandi núllstöðva margliðunnar \(p\) er þá í mesta lagi \(n\).

\begin{sphinxadmonition}{note}{Athugasemd:}\begin{description}
\item[{Þessi regla þýðir því að:}] \leavevmode\begin{itemize}
\item {} 
fyrsta stigs margliða hefur enga eða eina núllstöð,

\item {} 
annars stigs margliða hefur enga, eina, eða tvær núllstöðvar,

\item {} 
þriðja stigs margliða hefur enga, eina, tvær, eða þrjár núllstöðvar,

\item {} 
… og svo framvegis.

\end{itemize}

Sumar margliður hafa engar rauntölunúllstöðvar.
Til dæmis hefur margliðan \(p(x)=x^2+1\) engar rauntölunúllstöðvar því rauntalan \(x^2\) getur aldrei verið neikvæð.

\end{description}
\end{sphinxadmonition}

\begin{sphinxadmonition}{tip}{Dæmi:}
\sphinxstylestrong{1.} Margliðan \(x^2-1\) hefur tvær mismunandi núllstöðvar, það er, jafnan \(x^2-1=0\) hefur lausnirnar \(x=1\) og \(x=-1\).

\sphinxstylestrong{2.} Margliðan \((x-1)^2\) hefur bara eina núllstöð, það er, jafnan \((x-1)^2=0\) hefur bara lausnina \(x=1\).

\begin{sphinxadmonition}{warning}{Aðvörun:}
Í seinna tilvikinu tölum við oft um að margliðan hafi eina \textit{tvöfalda núllstöð}.
\end{sphinxadmonition}
\end{sphinxadmonition}


\section{Fyrsta og annars stigs margliður}
\label{\detokenize{Kafli06:fyrsta-og-annars-stigs-margliur}}
Rifjum upp kaflan um {\hyperref[\detokenize{Kafli02:s-jofnur}]{\sphinxcrossref{\DUrole{std,std-ref}{jöfnur}}}}.


\subsection{Fyrsta stigs margliður}
\label{\detokenize{Kafli06:fyrsta-stigs-margliur}}
Fyrsta stigs margliða er fall af gerðinni
\begin{equation*}
\begin{split}\begin{aligned}
&p(x)=ax+b \\
&\text{þar sem} \qquad a \neq 0 \quad \text{og} \quad b \in \mathbb{R}.
\end{aligned}\end{split}
\end{equation*}
Graf hennar er \textit{lína}. Þessi margliða hefur í mesta lagi eina núllstöð og hana er auðvelt að finna.
Við leysum einfaldlega \(x\) úr jöfnunni \(ax+b=0\). Við þekkjum lausn þessarar jöfnu, hún er \(x=-\frac{b}{a}\).


\subsection{Annars stigs margliður}
\label{\detokenize{Kafli06:annars-stigs-margliur}}\label{\detokenize{Kafli06:s-annarsstigs}}
\textit{Annars stigs margliða} er fall af gerðinni
\begin{equation*}
\begin{split}\begin{aligned}
&p(x)=ax^2+bx+c \\
&\text{þar sem} \qquad a \neq 0 \quad \text{og} \quad b,c \in \mathbb{R}
\end{aligned}\end{split}
\end{equation*}
Graf hennar er \textit{fleygbogi}. Til að finna núllstöðvar hennar þá leysum við jöfnuna \(ax^2+bx+c=0\). Rifjum aftur upp regluna til að leysa slíkar jöfnur, sem má finna í kaflanum um {\hyperref[\detokenize{Kafli02:s-annars-stigs-jofnur}]{\sphinxcrossref{\DUrole{std,std-ref}{annars stigs jöfnur}}}}.


\subsection{Regla}
\label{\detokenize{Kafli06:id2}}
Látum \(ax^2+bx+c=0\) vera annars stigs jöfnu.
\begin{enumerate}
\sphinxsetlistlabels{\arabic}{enumi}{enumii}{}{.}%
\item {} 
Ef \(b^2-4ac<0\) þá hefur jafnan enga rauntölulausn.

\item {} 
Ef \(b^2-4ac=0\) þá hefur jafnan eina lausn:

\end{enumerate}
\begin{equation*}
\begin{split}x=\frac{-b}{2a}.\end{split}
\end{equation*}\begin{enumerate}
\sphinxsetlistlabels{\arabic}{enumi}{enumii}{}{.}%
\setcounter{enumi}{2}
\item {} 
Ef \(b^2-4ac>0\) þá hefur jafnan tvær lausnir:

\end{enumerate}
\begin{equation*}
\begin{split}x_1=\frac{-b+\sqrt{b^2-4ac}}{2a} \qquad \text{og} \qquad x_2=\frac{-b-\sqrt{b^2-4ac}}{2a}.\end{split}
\end{equation*}
\begin{sphinxadmonition}{tip}{Dæmi:}
\sphinxstylestrong{1.} Finnum núllstöð margliðunnar \(p(x)=81x+121\).
\begin{quote}

Hún hefur eina núllstöð þar sem þetta er fyrsta stigs margliða. Leysum þá jöfnuna \(81x+121=0\). Fáum
\begin{equation*}
\begin{split}\begin{aligned}
81x &=-121 \\
x &=-121/81
\end{aligned}\end{split}
\end{equation*}
Því er núllstöðin \(x=-121/81\) .
\end{quote}

\sphinxstylestrong{2.} Finnum núllstöðvar margliðunnar \(p(x)=2x^2-21x+1\).
\begin{quote}

Leysum jöfnuna \(2x^2-21x+1=0\). Höfum
\begin{equation*}
\begin{split}b^2-4ac=(-21)^2-4 \cdot 2 \cdot 1=441-8=433 >0\end{split}
\end{equation*}
Núllstöðvar eru því tvær: \(x_1=\frac{21+\sqrt{443}}{4}\) og \(x_2=\frac{21-\sqrt{443}}{4}\).
\end{quote}
\end{sphinxadmonition}


\section{Deiling með afgangi \sphinxhyphen{} margliður}
\label{\detokenize{Kafli06:deiling-me-afgangi-margliur}}
Ef tvær margliður \(p\) og \(q\) eru lagðar saman eða önnur dregin frá hinni verður útkoman ný margliða.
Margfeldið \(p \cdot q\) verður einnig ný margliða, en það sama verður ekki sagt um deilingu.

Eins og á heiltölunum er deiling á margliðum ekki fullkomin í þeim skilningi að ef einni margliðu er deilt með annarri fæst ekki alltaf margliða út. Þegar tölu er deilt með annarri fæst ekki alltaf heiltala.
Við notum því deilingu með afgangi til að hjálpa okkur:

Látum \(p\) og \(q\) vera margliður.
Þá eru til margliður \(s\) og \(r\) þannig að \(p=qs+r\) og stig \(r\) er minna en stig \(q\).

Það að finna þessar margliður \(s\) og \(r\) kallast deiling með afgangi. Margliðan \(s\) kallast \textit{kvóti} og margliðan \(r\) kallast \textit{afgangur}.

Hægt er að nota aðferð sem er mjög lík löngudeilingu með heiltölur til að deila margliðum með afgangi. Best er að sjá þessa aðferð með dæmum:

\begin{sphinxadmonition}{tip}{Dæmi:}
\sphinxstylestrong{1.} Deilið með margliðunni \(q(x)=x+4\) í margliðuna \(p(x) =x^4 + 2x - 4\) með afgangi.
\begin{quote}

Notum löngudeilingu: byrjum á því að margfalda \(q(x)=x+4\) með \(s_1=x^3\) til þess að fremsti liður \(q(x)\) verði jafn fremsta lið \(p(x)\) .
Drögum \(x^3 \cdot q(x)=x^3\cdot(x+4) = x^4+4x^3 \quad\) frá \(\quad p(x) =x^4 + 2x - 4\) og fáum afganginn \(p_1(x)=-4x^3+2x-4\) .

\noindent{\hspace*{\fill}\sphinxincludegraphics[width=0.600\linewidth]{{mdeilingA}.svg}\hspace*{\fill}}

Endurtökum skrefin fyrir afganginn.
Margföldum \(q(x)=x+4\) með \(s_2=-4x^2\) til þess að fremsti liður \(q(x)\) verði jafn fremsta lið \(p_1(x)\) .
Drögum \(-4x^2 \cdot q(x)=-4x^2\cdot(x+4) = -4x^3-16x^3 \quad\)  frá  \(\quad p_1(x)=-4x^3+2x-4\) og fáum afganginn \(p_2(x)=16x^2+2x-4\) .

\noindent{\hspace*{\fill}\sphinxincludegraphics[width=0.600\linewidth]{{mdeilingB}.svg}\hspace*{\fill}}

Margföldum \(q(x)=x+4\) með \(s_3=16x\) til þess að fremsti liður \(q(x)\) verði jafn fremsta lið \(p_2(x)\) .
Drögum \(16x \cdot q(x)=16x\cdot(x+4) = 16x^2+64x \quad\) frá \(\quad p_2(x)=16x^2+2x-4\) og fáum afganginn \(p_3(x)=-62x-4\) .

\noindent{\hspace*{\fill}\sphinxincludegraphics[width=0.600\linewidth]{{mdeilingC}.svg}\hspace*{\fill}}

Margföldum \(q(x)=x+4\) með \(s_4=-62\) til þess að fremsti liður \(q(x)\) verði jafn fremsta lið \(p_3(x)\) .
Drögum \(-62 \cdot q(x)=-62\cdot(x+4) = -62x-248 \quad\) frá \(\quad p_3(x)=-62x-4\) og fáum afganginn \(p_4(x)=r=244\) .

\noindent{\hspace*{\fill}\sphinxincludegraphics[width=0.600\linewidth]{{mdeilingD}.svg}\hspace*{\fill}}

Þetta segir okkur að \(s(x) = s_1+s_2+s_3+s_4 = x^3 -4x^2 +16x -62\) og \(r(x) = 244\). Við getum nú skrifað
\begin{equation*}
\begin{split}x^4 +2x -4 = (x+4)(x^3 - 4x^2 + 16x - 62) + 244\end{split}
\end{equation*}\end{quote}

\sphinxstylestrong{2.} Deilið með margliðunni \(q(x)=x-3\) í margliðuna \(p(x) =x^3 + 6x^2 -2x - 8\) með afgangi.
\begin{quote}

Með löngudeilingu fæst eftirfarandi

\noindent{\hspace*{\fill}\sphinxincludegraphics[width=0.600\linewidth]{{mdeiling2}.svg}\hspace*{\fill}}

Þetta segir okkur að \(s(x) =x^2+ 9x +25\) og \(r(x) = 67\). Við getum nú skrifað
\begin{equation*}
\begin{split}x^4 +2x -4 = (x-3)(x^2 + 9x - 25) + 67\end{split}
\end{equation*}\end{quote}
\end{sphinxadmonition}


\section{Þáttun margliða}
\label{\detokenize{Kafli06:attun-marglia}}
Ef afgangurinn er \(r=0\) þá getum við notað löngudeilingu (margliðudeilingu) til þess að \textit{þátta} margliður.


\subsection{Skilgreining}
\label{\detokenize{Kafli06:skilgreining}}
Látum \(p\) og \(q\) vera margliður. Ef að til er margliða \(h\) þannig að \(p=h \cdot q\) þá segjum við að margliðan \(q\) gangi upp í margliðunni \(p\). Þá skrifum við líka \(\dfrac{p}{q}=h\).

Að skrifa margliðu \(q\) sem margfeldi margliða af lægra stigi kallast \textit{þáttun} margliðu.

Margliða \(q\) er sögð \textit{óþáttanleg} ef engin margliða af lægra stigi en \(q\) gengur upp í \(q\).

Margliða er sögð vera fullþáttuð ef að búið er að skrifa hana sem margfeldi af óþáttanlegum margliðum.

\begin{sphinxadmonition}{tip}{Dæmi:}
Þessa margliðu má þátta svona:
\begin{equation*}
\begin{split}x^3-6x^2-9x+14 = (x-1)(x+2)(x-7)\end{split}
\end{equation*}
og til dæmis má þátta þessa margliðu svona:
\begin{equation*}
\begin{split}x^3+4x^2-x-4 = (x-1)(x+1)(x+4)\end{split}
\end{equation*}
Sjáum nánar dæmi um hvernig þessi lausn fæst hér að {\hyperref[\detokenize{Kafli06:s-daemi}]{\sphinxcrossref{\DUrole{std,std-ref}{neðan}}}}.
\end{sphinxadmonition}


\subsection{Núllstöðvar margliða og þáttun}
\label{\detokenize{Kafli06:nullstovar-marglia-og-attun}}
Margliða kallast \textit{stöðluð} ef \(a_n=1\), það er, fremsti stuðullinn, eða stuðullinn við hæsta veldið, er \(1\). Fyrir staðlaðar margliður gildir eftirfarandi regla:


\subsection{Regla}
\label{\detokenize{Kafli06:id3}}
Ef \(p\) er stöðluð margliða af stigi \(n\) og hún hefur \(n\) ólíkar rætur, \(x_1, x_2, \dots, x_n\), þá má skrifa
\begin{equation*}
\begin{split}p(x)=(x-x_1)(x-x_2) \dots (x-x_n)\end{split}
\end{equation*}
Raunar fæst eftirfarandi niðurstaða:


\subsection{Regla}
\label{\detokenize{Kafli06:id4}}
Látum \(p\) vera margliðu. Þá gengur margliðan \(x-x_0\) upp í margliðunni \(p\) þá og því aðeins að \(x_0\) sé núllstöð margliðunnar \(p\).

Sannreynum að hægt sé að þátta annars stigs margliðu í rætur sínar, þ.e. sýnum að:
\begin{equation*}
\begin{split}ax^2+bx+c=a\left(x-\frac{-b+\sqrt{b^2-4ac}}{2a}\right)\left(x-\frac{-b-\sqrt{b^2-4ac}}{2a}\right)\end{split}
\end{equation*}
Margföldum saman svigana:
\begin{equation*}
\begin{split}\begin{aligned}
        &a\left(x-\frac{-b+\sqrt{b^2-4ac}}{2a}\right)\left(x-\frac{-b-\sqrt{b^2-4ac}}{2a}  \right)\\
        &= a\left(x+\frac{b-\sqrt{b^2-4ac}}{2a}\right)\left(x+\frac{b+\sqrt{b^2-4ac}}{2a} \right)\\
        &=ax^2+a\cdot\frac{b-\sqrt{b^2-4ac}}{2a}x+a\cdot\frac{b+\sqrt{b^2-4ac}}{2a}x\\
        &+a\cdot\left(\frac{b-\sqrt{b^2-4ac}}{2a}\right)\left(\frac{b+\sqrt{b^2-4ac}}{2a}\right)\\
        &=ax^2 + \frac{a\cdot x}{2a}\left(b-\sqrt{b^2-4ac}+b+\sqrt{b^2-4ac}\right)\\
        &+\frac{a}{4a^2}\left(b-\sqrt{b^2-4ac}\right)\left(b-\sqrt{b^2-4ac}\right)\\
        &=ax^2+\frac{x}{2}(2b)+\frac{1}{4a}(b^2-(b^2-4ac)) \\
        &=ax^2+bx+c
\end{aligned}\end{split}
\end{equation*}
\begin{sphinxadmonition}{tip}{Dæmi:}\begin{quote}

Til þess að þátta margliður byrjum við á að finna allar núllstöðvar hennar og skrifum margliðuna síðan sem margfeldi óþáttanlegra margliða.

Fullþáttum \(p(x)=x^2+2x-5\). Notum lausnarformúlu annars stigs jöfnu til að finna núllstöðvarnar. Hér er \(a=1\), \(b=2\) og \(c=-5\). Fáum því
\begin{equation*}
\begin{split}\begin{aligned}
x=\dfrac{-2\pm\sqrt{2^2-4\cdot 1\cdot (-5)}}{2}&=\dfrac{-2\pm\sqrt{24}}{2}\\
&=\dfrac{-2\pm 2\sqrt{6}}{2}\\
&=-1\pm\sqrt{6}
\end{aligned}\end{split}
\end{equation*}
þ.e. \(x_1=-1+\sqrt{6}\) og \(x_2=-1-\sqrt{6}\). Samkvæmt reglunni hér fyrir ofan fáum við þá þáttunina
\begin{equation*}
\begin{split}\begin{aligned}
p(x)&=(x-x_1)(x-x_2)\\
&=(x-(-1+\sqrt{6}))(x-(-1-\sqrt{6}))\\
&=(x+1-\sqrt{6})(x+1+\sqrt{6})
\end{aligned}\end{split}
\end{equation*}\end{quote}

það er,
\begin{quote}
\begin{equation*}
\begin{split}p(x)=(x+1-\sqrt{6})(x+1+\sqrt{6})\end{split}
\end{equation*}
\begin{sphinxadmonition}{note}{Athugasemd:}
Þetta segir okkur að margliðurnar \(x+(1-\sqrt{6})\) og \(x+(1+\sqrt{6})\) ganga báðar upp í margliðuna \(p(x)\).
\end{sphinxadmonition}
\end{quote}
\end{sphinxadmonition}
\phantomsection\label{\detokenize{Kafli06:s-daemi}}
\begin{sphinxadmonition}{tip}{Dæmi:}
Þáttum þriðja stigs margliðuna \(x^3+4x^2-x-4\) .

Við þurfum að byrja á því að finna núllstöðvar margliðunnar, það er, þau \(x\) þannig að \(x^3+4x^2-x-4=0\).
Þægilegt er að sjá að \(x=1\) er núllstöð:
\begin{equation*}
\begin{split}x^3+4x^2-x-4 |_{x=1} = 1^3+4\cdot 1^2 -1-4 = 0\end{split}
\end{equation*}
Því má skrifa margliðuna sem liðinn \((x-1)\) margfaldaðan við annars stigs margliðu.
Finnum þá margliðu með margliðudeilingu:

\begin{figure}[H]
\centering

\noindent\sphinxincludegraphics[width=0.500\linewidth]{{longud1}.svg}
\end{figure}

Höfum því \(x^3+4x^2-x-4 = (x-1)(x^2+5x+4)\) .
Þáttum nú \(x^2+5x+4\) en við sjáum að \(x=-1\) er núllstöð hennar:
\begin{equation*}
\begin{split}x^2+5x+4|_{x=-1} = (-1)^2+5\cdot(-1)+4 =0\end{split}
\end{equation*}
Því má skrifa \(x^2+5x+4\) sem \((x+1)\) margfaldað við aðra fyrsta stigs margliðu.
Hana má líka finna með margliðudeilingu:

\begin{figure}[H]
\centering

\noindent\sphinxincludegraphics[width=0.400\linewidth]{{longud2}.svg}
\end{figure}

Sjáum að \(x=-4\) er líka núllstöð.
Við höfum því fundið þrjár núllstöðvar fyrir þriðja stigs margliðu (en þær geta ekki verið fleiri) og því er fullþáttun margliðunnar
\begin{equation*}
\begin{split}x^3+4x^2-x-4 = (x-1)(x+1)(x+4)\end{split}
\end{equation*}\end{sphinxadmonition}


\section{p/q\sphinxhyphen{}aðferð}
\label{\detokenize{Kafli06:p-q-afer}}
Engin almenn leið er til sem að finnur núllstöðvar margliða af háum stigum. Eftirfarandi regla kemur þó stundum að gagni, ef til er ræð núllstöð:


\subsection{Regla}
\label{\detokenize{Kafli06:id5}}
Látum \(r(x)=a_nx^n+a_{n-1}x^{n-1}+ \dots + a_1x+a_0\) vera margliðu af stigi \(n\) þar sem stuðlarnir eru heilar tölur. Ef til er ræð tala \(p/q\) sem er núllstöð margliðunnar \(r\) þá gengur \(p\) upp í \(a_0\) og \(q\) gengur upp í \(a_n\).

\begin{sphinxadmonition}{note}{Athugasemd:}
Þessi regla segir okkur að ef við viljum finna einhverja núllstöð margliðu, þá er ráðlagt að ,,giska‘‘ fyrst á núllstöðvarnar af gerðinni \(\frac{p}{q}\) þar sem \(p\) gengur upp í \(a_0\) og \(q\) gengur upp í \(a_n\). Það getur verið sniðugt að byrja á því að athuga hvort \(1\) eða \(-1\) eru núllstöðvar því það er fljótgert.
\end{sphinxadmonition}

\begin{sphinxadmonition}{tip}{Dæmi:}
\sphinxstylestrong{1.} Finnum einhverja núllstöð \(h(x)=15x^4-3x^3-10x^2+x-3\).
\begin{quote}

Góð regla er að byrja á því að athuga hvort \(1\) eða \(-1\) eru núllstöðvar. Fáum
\begin{equation*}
\begin{split}\begin{aligned}
        h(1)&=15 \cdot 1-3 \cdot 1 -10 \cdot 1 + 1 \cdot 1 -3 \\
        &=15-3-10+1-3\\
        &=0
\end{aligned}\end{split}
\end{equation*}
svo \(x=1\) er núllstöð.
\end{quote}

\sphinxstylestrong{2.} Finnum einhverja núllstöð \(g(x)=10x^4+8x^3+8x^2+5x-5\).
\begin{quote}

Við sjáum auðveldlega að \(1\) er ekki núllstöð.

Munum að \((-1)^n=-1\) ef \(n\) er oddatala og \((-1)^n=1\) ef \(n\) er slétt tala.

Fáum nú
\begin{equation*}
\begin{split}\begin{aligned}
        g(-1) &= 10 \cdot (-1)^4 + 8 \cdot (-1)^3 + 8 \cdot (-1)^2+5 \cdot (-1)-5\\
        & =10-8+8-5-5\\
        &=0
\end{aligned}\end{split}
\end{equation*}
svo að \(x=-1\) er núllstöð.
\end{quote}

\sphinxstylestrong{3.} Finnum einhverja núllstöð á margliðunni \(r(x)=2x^4-5x^3-2x^2-9\).
\begin{quote}

Sjáum með prófun að hvorki \(1\) né \(-1\) eru núllstöðvar. Beitum þá \(p/q\)\sphinxhyphen{}aðferð.

Mengi allra talna sem gengur upp í tölunni \(2\) er \(A=\{1,-1,2,-2\}\).
Mengi allra talna sem gengur upp í tölunni \(9\) er \(B=\{1,-1,3,-3,9,-9 \}\).
\(\frac{p}{q}\)\sphinxhyphen{}aðferð segir okkur að við eigum að giska á núllstöð af gerðinni \(\frac{p}{q}\) þar sem \(p\in B\) og \(q\in A\).

Öll möguleg brot af slíkri gerð eru mjög mörg talsins, hins vegar má í raun sleppa öllum mínustölum í öðru hvoru menginu því annars tvíteljum við margar tölur. Sleppum mínustölunum í \(A\) og þá eru möguleikarnir:
\begin{equation*}
\begin{split}\dfrac{1}{1}, \;
\dfrac{1}{2}, \;
\dfrac{-1}{1}, \;
\dfrac{-1}{2}, \;
\dfrac{3}{1},\;
\dfrac{3}{2}, \;
\dfrac{-3}{1}, \;
\dfrac{-3}{2}, \;
\dfrac{9}{1}, \;
\dfrac{9}{2}, \;
\dfrac{-9}{1},\;
\dfrac{-9}{2}\;\end{split}
\end{equation*}
Stingum öllum þessum tölum inn í margliðuna \(r\) (við erum búin að prófa \(1\) og \(-1\)):
\begin{equation*}
\begin{split}\begin{aligned}
        r\left(\frac{1}{2}\right)& =-10 \\
        r\left(\frac{-1}{2}\right)&=-\frac{35}{4}\\
        r\left(\frac{3}{1}\right)&=r(3)=0\\
        r\left(\frac{3}{2}\right)&=-\frac{81}{4}\\
        r\left(\frac{-3}{1}\right)&=r(-3)=270\\
        r\left(\frac{-3}{2}\right)&=\frac{27}{2}\\
        r\left(\frac{9}{1}\right)&=r(9)=9306\\
        r\left(\frac{9}{2}\right)&=315\\
        r\left(\frac{-9}{1}\right)&=r(-9)=16596\\
        r\left(\frac{-9}{2}\right)&=\frac{4905}{4}
\end{aligned}\end{split}
\end{equation*}
Með þessari aðferð fundum við eina núllstöð, \(x=3\), því að \(r(3)=0\).
\end{quote}
\end{sphinxadmonition}


\section{Pascal}
\label{\detokenize{Kafli06:pascal}}
Rifjum upp nokkrar mikilvægar \textit{liðanir}:
\begin{equation*}
\begin{split}\begin{aligned}
& (a+b)^2=a^2+2ab+b^2 \qquad &\textit{(ferningsregla fyrir summu)} \\
& (a-b)^2=a^2-2ab+b^2 \qquad &\textit{(ferningsregla fyrir mismun)} \\
& (a+b)(a-b)=a^2-b^2 \qquad &\textit{(samokaregla)} \\
\end{aligned}\end{split}
\end{equation*}
Prófum að liða \((a+b)^3\):
\begin{equation*}
\begin{split}\begin{aligned}
(a+b)^3 &= (a+b)(a+b)(a+b) \\
&= (a+b)(a^2+2ab+b^2) \\
&= a^3 + 2a^2b +ab^2 +ba^2 +2ab^2 + b^3\\
&=a^3+3a^2b+3ab^2+b^3\\
\end{aligned}\end{split}
\end{equation*}
Skoðum fleiri liðanir á forminu \((a + b)^n\):
\begin{equation*}
\begin{split}(a + b)^0 = 1\end{split}
\end{equation*}\begin{equation*}
\begin{split}(a + b)^1 = a + b\end{split}
\end{equation*}\begin{equation*}
\begin{split}(a + b)^2 = a^2 + 2ab + b^2\end{split}
\end{equation*}\begin{equation*}
\begin{split}(a + b)^3 = a^3 + 3a^2 b + 3a b^2 + b^3\end{split}
\end{equation*}\begin{equation*}
\begin{split}(a + b)^4 = a^4 + 4 a^3 b + 6a^2 b^2 + 4ab^3 + b^4\end{split}
\end{equation*}\begin{equation*}
\begin{split}(a + b)^5 = a^5 + 5a^4 b + 10a^3b^2 + 10a^2 b^3 + 5 ab^4 +b^5\end{split}
\end{equation*}
Skoðum aðeins mynstrið sem er að verða til:
\begin{enumerate}
\sphinxsetlistlabels{\arabic}{enumi}{enumii}{}{.}%
\item {} 
það eru alltaf \(n+1\) liðir,

\item {} 
í hverjum lið er summa veldana jafn \(n\),

\item {} 
veldið á \(a\) lækkar frá \(n\) niður í \(0\) og veldið á b hækkar frá \(0\) upp í \(n\),

\item {} 
stuðlarnir fyrir framan liðina byrja á því að hækka og svo speglast þeir og lækka þegar komið er að miðju liðinum.

\end{enumerate}

Þessir stuðlar mynda mynstur sem getur borgað sig að hafa á hreinu.

Mynstrið kallast \sphinxstylestrong{Pascal þríhyrningurinn}:

\noindent{\hspace*{\fill}\sphinxincludegraphics[width=0.800\linewidth]{{pascal}.svg}\hspace*{\fill}}

Hér eru stuðlarnir fyrir \(n=0,1,2,3,...,8\).

Hvert stak í Pascal þríhyrningnum er summa stakanna sem eru fyrir ofan það og auk þess er ásum raðað á endana.

\noindent{\hspace*{\fill}\sphinxincludegraphics[width=0.800\linewidth]{{pascal2}.svg}\hspace*{\fill}}


\bigskip\hrule\bigskip


\begin{sphinxadmonition}{tip}{Dæmi:}
Hvernig er liðuninn á \((a+b)^9\)?

Hér er \(n=9\) svo veldið á \(a\) byrjar í 9 og lækkar síðan niður í núll.
Veldið á \(b\) byrjar í núll og hækkar upp í 9.

Stuðlarnir fyrir framan liðina koma úr Pascal\sphinxhyphen{}þríhyrningnum.
Á efri myndinni eru stuðlarnir fyrir \(n=8\) í neðstu línunni.
Reiknum næstu línu, stuðlana fyrir \(n=9\) með því að leggja saman tölurnar fyrir ofan og bæta við einum á hvorn endann.

Fáum t.d. \(70+56=126\) svo stuðlarnir eru:

\begin{figure}[H]
\centering

\noindent\sphinxincludegraphics[width=0.600\linewidth]{{pascald}.svg}
\end{figure}

Því er liðunin á \((a+b)^9\) :

\(a^9 + 9a^8b + 36a^7b^2 + 84a^6b^3 + 126a^5b^4 + 126a^4b^5 + 84a^3b^6 + 36a^2b^7 +  9ab^8 + b^9\).
\end{sphinxadmonition}

Það er hægt að reikna stuðlana án þess að teikna upp allan þríhyrninginn.
Þessir stuðlar eru kallaðir \textit{tvíliðustuðlar} og fást úr eftirfarandi formúlu:
\begin{equation*}
\begin{split}\begin{pmatrix} n \\ k \end{pmatrix} = \frac{n!}{k!(n-k)!}\end{split}
\end{equation*}
Hér er \(n\) veldið á \((a+b)^n\) eða númer raðar í þríhyrningnum og \(k\) er númer liðsins sem við erum að skoða.
Athugum að \(k\) tekur heiltölugildi frá núll upp í \(n\) (oft segjum við að \(k\) \sphinxstyleemphasis{hlaupi} frá núll upp í \(n\) ).

\begin{sphinxadmonition}{warning}{Aðvörun:}
Athugið að við byrjum að telja línurnar og stökin í núlli!
\end{sphinxadmonition}

Þegar tölur eru hrópmerktar með ! þá erum við að reikna \textit{aðfeldi} þeirra og þá gildir
\begin{equation*}
\begin{split}n! = \prod_{i=1}^{n}i = 1 \cdot 2 \cdot 3 \cdot ... \cdot (n-1) \cdot n\end{split}
\end{equation*}
\begin{sphinxadmonition}{warning}{Aðvörun:}
Núll hrópmerkt er skilgreint sem \(0!=1\)
\end{sphinxadmonition}

\begin{sphinxadmonition}{tip}{Dæmi:}
\(5! = 1\cdot 2\cdot 3\cdot 4\cdot 5 = 120\)
\end{sphinxadmonition}

Því getum við skrifað liðunina á margliðum á forminu \((a+b)^n\) sem
\begin{equation*}
\begin{split}\begin{aligned}
        (a+b)^n &= \begin{pmatrix} n \\ 0 \end{pmatrix} a^nb^0 + \begin{pmatrix} n \\ 1 \end{pmatrix} a^{n-1}b^1 + \begin{pmatrix} n \\ 2 \end{pmatrix} a^{n-2}b^2 + \\ &...+ \begin{pmatrix} n \\ n-2 \end{pmatrix} a^2b^{n-2} + \begin{pmatrix} n \\ n-1 \end{pmatrix}a^1b^{n-1} + \begin{pmatrix} n \\ n \end{pmatrix}a^0b^n
\end{aligned}\end{split}
\end{equation*}
\begin{sphinxadmonition}{tip}{Dæmi:}
Skoðum liðunina á  \((a+b)^4\) . Hér er \(n=4\) svo fyrsti stuðullin við \(a^4\cdot b^0 = a^4\) er \(\begin{pmatrix} 4 \\ 0 \end{pmatrix}\)
\begin{equation*}
\begin{split}\begin{aligned}
 \begin{pmatrix} 4 \\ 0 \end{pmatrix} &= \frac{4!}{0!(4-0)!} \\
 &= \frac{4!}{4!}\\
 &= \frac{1\cdot 2\cdot 3\cdot 4}{1\cdot 2\cdot 3\cdot 4} \\
 &= 1
 \end{aligned}\end{split}
\end{equation*}
Næsti er \(\begin{pmatrix} 4 \\ 1 \end{pmatrix} a^3 b\)
\begin{equation*}
\begin{split}\begin{aligned}
 \begin{pmatrix} 4 \\ 1 \end{pmatrix} &= \frac{4!}{1!(4-1)!} \\
 &= \frac{4!}{3!}\\
 &= \frac{1\cdot 2 \cdot 3\cdot 4}{1\cdot 2\cdot 3} \\
 &= \frac{24}{6} \\
 &= 4
 \end{aligned}\end{split}
\end{equation*}
Nú  \(\begin{pmatrix} 4 \\ 2 \end{pmatrix} a^3 b\)
\begin{equation*}
\begin{split}\begin{aligned}
\begin{pmatrix} 4 \\ 2 \end{pmatrix} &= \frac{4!}{2!(4-2)!} \\
&= \frac{4!}{2!(2)!}\\
&= \frac{1\cdot 2\cdot 3\cdot 4}{1\cdot 2 \cdot(1 \cdot 2)} \\
&= \frac{24}{4} \\
&= 6
\end{aligned}\end{split}
\end{equation*}
og svo framvegis.

Fáum þá
\begin{equation*}
\begin{split}\begin{aligned}
(a + b)^4 &= \begin{pmatrix} 4 \\ 0 \end{pmatrix} a^4  + \begin{pmatrix} 4 \\ 1 \end{pmatrix} a^3b  + \begin{pmatrix} 4 \\ 2 \end{pmatrix}a^2b^2  + \begin{pmatrix} 4 \\ 3 \end{pmatrix} ab^3  + \begin{pmatrix} 4 \\ 4 \end{pmatrix} b^4 \\
&= \quad a^4 \quad +\quad 4 a^3b \quad + \quad 6 a^2b^2 \quad + \quad 4 ab^3 \quad + \quad b^4
\end{aligned}\end{split}
\end{equation*}
Sjáum að stuðlarnir eru einmitt fjórða línan í Pascal þríhyrningnum.
\end{sphinxadmonition}

Hér sjáum við samantekt af tvíliðustuðlum upp í \(n=6\) :

\noindent{\hspace*{\fill}\sphinxincludegraphics[width=1.100\linewidth]{{pascalbin}.svg}\hspace*{\fill}}

\begin{sphinxadmonition}{note}{Athugasemd:}
Takið eftir að eftirfarandi gildir alltaf:
\begin{equation*}
\begin{split}\begin{pmatrix} n \\ 0 \end{pmatrix} = \begin{pmatrix} n \\ n \end{pmatrix} = 1\end{split}
\end{equation*}\end{sphinxadmonition}


\chapter{Hornaföll}
\label{\detokenize{Kafli07:hornafoll}}\label{\detokenize{Kafli07::doc}}

\section{Bogaeiningar}
\label{\detokenize{Kafli07:bogaeiningar}}
Í daglegu tali notum við yfirleitt mælieininguna \textit{gráður} til þess að mæla horn.
Samkvæmt skilgreiningu skiptir hún hringnum í þrjúhundruð\sphinxhyphen{}og\sphinxhyphen{}sextíu jafna hluta og einn slíkur hluti er kallaður ein gráða.
Stærðfræðilega séð þykir talan \(360\) ekkert merkilegri en aðrar tölur og því engin ástæða til að búa til mælieiningakerfi byggt á henni.
Í raun er til miklu náttúrulegri leið til að skilgreina nýja mælieiningu á horn.
Hana köllum við \textit{bogaeiningu} en hún er skilgreind á eftirfarandi máta:


\subsection{Skilgreining}
\label{\detokenize{Kafli07:skilgreining}}
Látum \(\alpha\) vera horn. Köllum oddpunkt hornsins \(O\).
Teiknum hring með \textit{geisla} \(1\) og miðju í punktinum \((0,0)\).
Armar hornsins skera hringinn í tveimur punktum \(A\) og \(B\).
Stærð hornsins \(\alpha\) er þá jafnt lengd \textit{bogans} á milli punktanna \(A\) og \(B\).

\noindent{\hspace*{\fill}\sphinxincludegraphics{{mynd1}.png}\hspace*{\fill}}

\begin{sphinxadmonition}{note}{Athugasemd:}
Samkvæmt þessari skilgreiningu þá er heill hringur \(2 \pi\) bogaeiningar því það er \textit{ummál} hrings með geisla \(1\).
\begin{equation*}
\begin{split}2\pi\text{ Rad} = 360°\end{split}
\end{equation*}\end{sphinxadmonition}

Bogaeiningar eru oft kallaðar radíanar og þær má tákna með \(\text{Rad}\), en það er yfirleitt ekki gert. Ef það er ekki merkt að horn sé mælt í gráðum, þá er það mælt í radíönum.


\bigskip\hrule\bigskip


Það er hægt að breyta á milli gráða og bogaeininga svona:
\begin{equation*}
\begin{split}x \quad \text{Rad} = \left(x \cdot \frac{360}{2 \pi}\right)° \qquad og \qquad  x°=\left( x \cdot \frac{2 \pi}{360}\right) \text{Rad}\end{split}
\end{equation*}
\begin{sphinxadmonition}{tip}{Dæmi:}
Skrifum \(\frac{\pi}{6}\) í gráðum.

\sphinxstylestrong{Lausn}

Til þess finnum við hversu stór hluti hornið \(\frac{\pi}{6}\) er úr hringnum, en heill hringur er \(2 \pi\).
Höfum \(\frac{\pi/6}{2 \pi}=\frac{1}{12}\) , svo \(\frac{\pi}{6}\) er \(\frac{1}{12}\) úr hring.
Nú þurfum við bara að margfalda þessa stærð með \(360^{\circ}\) og fáum \(\frac{1}{12}\cdot 360^{\circ} = 30^{\circ}\).

Það er líka hægt að nota formúlurnar í skilgreiningunni hér að ofan:
\begin{equation*}
\begin{split}\frac{\pi}{6} \text{Rad} = \left(\frac{\pi}{6} \cdot \frac{360}{2 \pi}\right)° = 30°\end{split}
\end{equation*}\end{sphinxadmonition}

\begin{sphinxadmonition}{tip}{Dæmi:}
Skrifum \(70^{\circ}\) í bogaeiningum.

\sphinxstylestrong{Lausn}

Finnum hversu stórt hlutfall \(70^{\circ}\) er úr heilum hring, \(70/360\), og margföldum með \(2 \pi\). Fáum:
\begin{equation*}
\begin{split}\frac{70}{360} \cdot 2 \pi=\frac{7}{18} \pi\end{split}
\end{equation*}
Lausnin er því að \(70^{\circ}=\frac{7}{18}\pi\). Það er líka hægt að nota formúlurnar í skilgreiningunni hér að ofan:
\begin{equation*}
\begin{split}70°=\left( 70 \cdot \frac{2 \pi}{360}\right) \text{Rad} = \frac{7\pi}{18}\end{split}
\end{equation*}\end{sphinxadmonition}


\section{Kósínus og sínus}
\label{\detokenize{Kafli07:kosinus-og-sinus}}

\subsection{Eingingahringurinn}
\label{\detokenize{Kafli07:eingingahringurinn}}
Hringurinn með miðju í punktinum \((0,0)\) í hnitakerfinu og radíus einn er kallaður \textit{einingarhringurinn}. Í þessum kafla munum við nota einingarhringinn og bogaeiningar til að skilgreina \textit{hornaföll}.


\subsection{Kósínus og sínus}
\label{\detokenize{Kafli07:id1}}
Nú er markmiðið að skýra stærðirnar \(\cos(\alpha)\) og \(\sin(\alpha)\).

Teiknum einingarhring í \textit{hnitakerfið}.
Setjum blýantinn okkar í punktinn \((1,0)\) og færum hann rangsælis eftir einingarhringnum þar til blýanturinn er búinn að færast um vegalengdina \(\alpha\). (Ef \(\alpha\) er neikvæð tala förum við réttsælis um vegalengdina \(\alpha\)). Hér er í lagi þó að \(\alpha\) sé stór tala og við förum marga hringi á einingarhringinn.

\noindent{\hspace*{\fill}\sphinxincludegraphics{{alpha}.svg}\hspace*{\fill}}

Munum að \(\alpha\) er horn í bogalengdum og er jafnt lengd bogans frá upphafspunktinum.

Þegar blýanturinn er búinn að ferðast um vegalengdina \(\alpha\) þá stoppum við og mörkum punktinn \(P\) inn á hnitakerfið þar sem stoppað var.
Kósínus af horninu \(\alpha\) er nú skilgreindur sem \(x\)\sphinxhyphen{}hnit punktsins \(P\), og sínus af horninu \(\alpha\) er skilgreindur sem \(y\)\sphinxhyphen{}hnit punktsins \(P\). Við táknum þessi föll með \(\cos(\alpha)\) og \(\sin(\alpha)\).

\noindent{\hspace*{\fill}\sphinxincludegraphics[width=0.700\linewidth]{{mynd2}.svg}\hspace*{\fill}}

\begin{sphinxadmonition}{note}{Athugasemd:}
Bæði kósínus og sínus eru \(2 \pi\)\sphinxhyphen{}\textit{lotubundin} föll. Ef við förum heilan hring, sem er \(2 \pi\), þá endum við í sama punkti og fáum því sama gildið.
\end{sphinxadmonition}

Hornafallið tangens, \(\tan\), er skilgreint sem hlutfallið á milli \(\sin\) og \(\cos\).
\begin{equation*}
\begin{split}\tan(\alpha) = \frac{\sin(\alpha)}{\cos(\alpha)}\end{split}
\end{equation*}
Þar sem \(\cos(\alpha) \neq 0\)

Hægt er að nota allar hliðar þríhyrningsins sem myndast til að finna gildin á \(\cos(\alpha), \sin(\alpha)\) og \(\tan(\alpha)\).

\noindent{\hspace*{\fill}\sphinxincludegraphics[width=0.500\linewidth]{{sohcahtoa}.svg}\hspace*{\fill}}

Hér er \(c\) kölluð \textit{langhliðin}, \(a\) kölluð \textit{aðlæg skammhlið} og \(b\) kölluð \textit{mótlæg skammhlið} miðað við hornið \(\alpha\).


\subsection{Amma illa}
\label{\detokenize{Kafli07:amma-illa}}
Sumum þykir þægilegt að nota eftirfarandi töflu til þess að muna hvaða hlutföll hliðanna gefur hvaða hornafall.
Hér stendur \(\text{a}\) fyrir \sphinxstyleemphasis{aðlæga} skammhlið, \(\text{m}\) fyrir \sphinxstyleemphasis{mótlæga} skammhlið og \(\text{l}\) fyrir \sphinxstyleemphasis{langhlið}.
\begin{equation*}
\begin{split}\begin{array}{| c | c | c | c | c |}
        \hline
        & \cos(\alpha) & \sin(\alpha) & \tan(\alpha) & \\
        \hline
        & \text{a} &    \text{m} & \text{m} & \text{(a)}\\
        \hline
        \text{(i)} &    \text{l} & \text{l} & \text{a} &  \\
        \hline
\end{array}\end{split}
\end{equation*}
\(\cos\) af horni í þríhyrningi er aðlæg deilt með langhlið (\(\text{a}/\text{l}\)).

\(\sin\) af horni er mótlæg deilt með langhlið (\(\text{m}/\text{l}\)).

\(\tan\) er mótlæg deilt með aðlægri skammhlið (\(\text{m}/\text{a}\)).


\section{Þekkt gildi á hornaföllum}
\label{\detokenize{Kafli07:ekkt-gildi-a-hornafollum}}
Skoðum nú nokkur gildi á \(\alpha\) í samhengi við útskýringuna á hornaföllunum hér að ofan.

Munið að við látum blýant byrja í punktinum \((1,0)\) og færum okkur eftir einingarhringnum eins langt og \(\alpha\) segir til um, og endum í punkti \(P\).
\begin{enumerate}
\sphinxsetlistlabels{\arabic}{enumi}{enumii}{}{.}%
\item {} 
Ef \(\alpha=0\) þá færum við okkur ekki neitt. Við endum í sama punkti og við byrjum í og þess vegna verður \(P=(1,0)\). Þess vegna er \(\cos(0)=1\) og \(\sin(0)=0\).

\item {} 
Ef \(\alpha=\pi/2\) þá færum við okkur rangsælis um fjórðung af hringnum (ummál hringsins er \(2\pi\)). Við endum semsagt í topppunkti hringsins sem hefur hnit \(P=(0,1)\) svo \(\cos(\pi/2)=0\) og \(\sin(\pi/2)=1\).

\item {} 
Ef \(\alpha=\pi\) þá færum við okkur rangsælis um hálfan hring. Þá erum við stödd í punktinum \(P=(-1,0)\) svo að \(\cos(\pi)=-1\) og \(\sin(\pi)=0\).

\end{enumerate}

Vel þekkt gildi á hornaföllunum má lesa úr myndinni að neðan.
Stærðir hornanna eru merktar utan á hringinn og \(x\) \sphinxhyphen{} og \(y\) \sphinxhyphen{} hnit þeirra eru merkt á ásana.
Mikilvægt er að þekkja einingarhringinn og geta notað hann.
Við lesum gildin á kósínus á \(x\) \sphinxhyphen{} ásnum og  gildin á sínus á \(y\) \sphinxhyphen{} ásnum.

Þannig sést til dæmis á myndinni að \(\cos(5\pi/6)=-\frac{\sqrt{3}}{2}\) (\(x\)\sphinxhyphen{}ásinn) og \(\sin(5\pi/6)=\frac12\) (\(y\)\sphinxhyphen{}ásinn). Einnig er il dæmis \(\cos(7\pi/4)=\frac{\sqrt{2}}{2}\) og \(\sin(7\pi/4)=-\frac{\sqrt{2}}{2}\) og svona gætum við haldið áfram.

\noindent{\hspace*{\fill}\sphinxincludegraphics{{einingarhringur}.svg}\hspace*{\fill}}

\begin{sphinxadmonition}{warning}{Aðvörun:}
Það getur borgað sig að hafa þessi gildi á hreinu!
\begin{equation*}
\begin{split}\begin{array}{| c | c | c | c |}
        \hline
        & \alpha = 30°  = \frac{\pi}{6} & \alpha = 60° = \frac{\pi}{3} & \alpha = 45° = \frac{\pi}{4} \\
        \hline
        \cos(\alpha) & \frac{\sqrt{3}}{2} &     \frac{1}{2} & \frac{\sqrt{2}}{2} \\
        \hline
        \sin(\alpha) &  \frac{1}{2} & \frac{\sqrt{3}}{2} & \frac{\sqrt{2}}{2} \\
        \hline
        \tan(\alpha) & \frac{\sqrt{3}}{3} & \sqrt{3} & 1 \\
        \hline
\end{array}\end{split}
\end{equation*}\end{sphinxadmonition}

Til þess að læra gildin getur reynst vel að skoða þríhyrningana sem myndast út frá einingarhringnum þegar \(\alpha\) tekur gildin \(\frac{\pi}{6}, \frac{\pi}{3} \text{ og } \frac{\pi}{4}\).

Hér er rétthyrndi þríhyrningurinn sem myndast þegar við erum í \(30°\) eða \(\frac{\pi}{6}\) stefnu:

\noindent{\hspace*{\fill}\sphinxincludegraphics[width=0.700\linewidth]{{triangle1}.svg}\hspace*{\fill}}

Hér er rétthyrndi þríhyrningurinn sem myndast þegar við erum í \(60°\) eða \(\frac{\pi}{3}\) stefnu:

\noindent{\hspace*{\fill}\sphinxincludegraphics[width=0.700\linewidth]{{triangle2}.svg}\hspace*{\fill}}

Hér er rétthyrndi þríhyrningurinn sem myndast þegar við erum í \(45°\) eða \(\frac{\pi}{4}\) stefnu:

\noindent{\hspace*{\fill}\sphinxincludegraphics[width=0.700\linewidth]{{triangle3}.svg}\hspace*{\fill}}


\section{Tangens og kótangens}
\label{\detokenize{Kafli07:tangens-og-kotangens}}
Við skilgreinum föllin tangens og kótangens þannig:
\begin{equation*}
\begin{split}\tan(\alpha)=\frac{\sin(\alpha)}{\cos(\alpha)}, \qquad (\cos(\alpha)\neq 0 )\end{split}
\end{equation*}\begin{equation*}
\begin{split}\cot(\alpha)=\frac{\cos(\alpha)}{\sin(\alpha)}, \qquad (\sin(\alpha)\neq 0)\end{split}
\end{equation*}

\section{Myndir af hornaföllum}
\label{\detokenize{Kafli07:myndir-af-hornafollum}}
Hér eru myndir af gröfum hornafallanna, þar sem hornið er eftir \(x\) \sphinxhyphen{} ásnum.
Takið eftir að öll föllin eru lotubundin með lotu \(2\pi\).

\noindent{\hspace*{\fill}\sphinxincludegraphics[width=1.200\linewidth]{{mynd3}.svg}\hspace*{\fill}}

\noindent{\hspace*{\fill}\sphinxincludegraphics[width=1.200\linewidth]{{mynd4}.svg}\hspace*{\fill}}


\bigskip\hrule\bigskip


Takið eftir að kósínusinn lítur næstum alveg eins út og sínusinn, eini munurinn á gröfunum er að búið er að hliðra öðru um \(\frac{\pi}{2}\) miðað við hitt.
\begin{equation*}
\begin{split}\cos(\alpha) = \sin\left(\frac{\pi}{2} - \alpha\right)\end{split}
\end{equation*}\begin{equation*}
\begin{split}\sin(\alpha) = \cos\left(\frac{\pi}{2} - \alpha\right)\end{split}
\end{equation*}
Sínusinn og kósínusinn eru takmörkuð föll, takmörkuð af einum að ofan og mínus einum að neðan.
Það þýðir að þau taki \sphinxstyleemphasis{aldrei} gildi sem eru stærri en 1 eða minni en \sphinxhyphen{}1.

\begin{sphinxadmonition}{note}{Athugasemd:}\begin{description}
\item[{Ein af mikilvægum eiginleikum \(\cos\) og \(\sin\) er að}] \leavevmode\begin{itemize}
\item {} \begin{description}
\item[{\(\cos\) er \sphinxstylestrong{jafnstætt} fall}] \leavevmode\begin{itemize}
\item {} 
\(\cos(-\alpha) = \cos(\alpha)\)

\end{itemize}

\end{description}

\item {} \begin{description}
\item[{\(\sin\) er \sphinxstylestrong{oddstætt} fall}] \leavevmode\begin{itemize}
\item {} 
\(\sin(-\alpha) = -\sin(\alpha)\)

\end{itemize}

\end{description}

\end{itemize}

\end{description}
\end{sphinxadmonition}


\bigskip\hrule\bigskip


\noindent{\hspace*{\fill}\sphinxincludegraphics[width=1.200\linewidth]{{mynd5}.svg}\hspace*{\fill}}


\bigskip\hrule\bigskip


Tangensinn er ekki takmarkaður heldur stefnir á plús eða mínus óendanlegt á sumum stöðum.
Þá hefur \(\tan(x)\) \textit{lóðfellur} þar sem \(\cos(x)=0\), því þá er \(\tan(x) = \frac{\sin(x)}{\cos(x)}\) ekki skilgreint.


\bigskip\hrule\bigskip


\noindent{\hspace*{\fill}\sphinxincludegraphics[width=1.200\linewidth]{{mynd6}.svg}\hspace*{\fill}}


\bigskip\hrule\bigskip


Á sama hátt er kótangensinn eru ekki takmarkaður heldur stefnir á plús eða mínus óendanlegt á sumum stöðum. Einnig hefur \(\cot(x)\) \textit{lóðfellur} þar sem \(\sin(x)=0\), því þá er \(\cot(x) = \frac{\cos(x)}{\sin(x)}\) ekki skilgreint.


\section{Hornafallareglur}
\label{\detokenize{Kafli07:hornafallareglur}}\label{\detokenize{Kafli07:s-hornafoll}}
Hornaföllin hafa marga nytsamlega eiginleika. Rökstyðjum hér nokkrar hornafallareglur:

\sphinxstylestrong{1.} Rökstyðjum að
\begin{equation*}
\begin{split}\begin{aligned}
\cos(-\alpha)&=\cos(\alpha) \\
&\text{og} \\
\sin(-\alpha)&=-\sin(\alpha)
\end{aligned}\end{split}
\end{equation*}
Byrjum í punktinum \((1,0)\) og færum okkur \sphinxstyleemphasis{rangsælis} eftir einingarhringnum um vegalengdina \(\alpha\) . Mörkum þar punktinn \(P_1\).
Færum okkur svo úr \((1,0)\) \sphinxstyleemphasis{réttsælis} um \(\alpha\) og mörkum þar inn \(P_2\).

\noindent{\hspace*{\fill}\sphinxincludegraphics[width=0.500\linewidth]{{mynd7}.svg}\hspace*{\fill}}

Auðvelt er að sjá að punktarnir hafa sömu \(x\)\sphinxhyphen{}hnit þannig að \(\cos(-\alpha)=\cos(\alpha)\) .
Hins vegar hafa \(y\)\sphinxhyphen{}hnitin öfug formerki miðað við hvort annað, svo \(\sin(-\alpha)=-\sin(\alpha)\).


\bigskip\hrule\bigskip


\sphinxstylestrong{2.} Rökstyðjum að
\begin{equation*}
\begin{split}\begin{aligned}
\cos(\pi-\alpha)&=-\cos(\alpha) \\
&\text{og} \\
\sin(\pi-\alpha)&=\sin(\alpha)
\end{aligned}\end{split}
\end{equation*}
Við mörkum aftur tvo punkta inn á hnitakerfið.

\(P_1\) mörkum við með því að færa okkur um hornið \(\pi-\alpha\), en það er gert með því að færa sig fyrst rangsælis um \(\pi\) en svo aftur til baka réttsælis um hornið \(\alpha\).
\(P_2\) mörkum við inn á hnitakerfið með því að færa okkur um hornið \(\alpha\) rangsælis.

\noindent{\hspace*{\fill}\sphinxincludegraphics[width=0.500\linewidth]{{mynd8}.svg}\hspace*{\fill}}

Þá er auðvelt að sjá að \(P_1\) og \(P_2\) hafa sömu \(y\)\sphinxhyphen{}hnit þannig að \(\sin(\pi-\alpha)=\sin(\alpha)\) .
Þá hafa \(x\)\sphinxhyphen{}hnit punktanna gagnstæð formerki, þannig að \(\cos(\pi-\alpha)=-\cos(\alpha)\). En það er einmitt það sem við erum að reyna að rökstyðja.


\bigskip\hrule\bigskip


Hægt er að rökstyðja fleiri reglur á svipaðan hátt, en það getur verið auðveldara að sjá þær myndrænt fyrir sér en að reyna að muna þær allar.

Setjum fram nokkrar slíkar reglur.
\begin{equation*}
\begin{split}\begin{aligned}
\cos(-\theta)&=\cos \theta\\
\sin(-\theta)&=-\sin\theta\\
& \\
\cos(\pi-\theta)&=-\cos \theta\\
\sin(\pi-\theta)&=\sin \theta\\
& \\
\cos(\theta+\pi)&=-\cos \theta\\
\sin(\theta+\pi)&=-\sin \theta\\
& \\
\cos\left(\frac{\pi}{2}-\theta\right)&=\sin\theta\\
\sin\left(\frac{\pi}{2}-\theta\right)&=\cos\theta
\end{aligned}\end{split}
\end{equation*}
Almennt eru gildi \(\cos(\alpha), \sin(\alpha)\) og \(\tan(\alpha)\) jákvæð í fyrsta fjórðungi, svo eru gildi \(\sin(\alpha)\) jákvæð í öðrum fjórðungi, \(\tan(\alpha)\) í þriðja, og \(\cos(\alpha)\) í fjórða. Sjáum á mynd hvaða hornaföll eru jákvæð hvar.

\noindent{\hspace*{\fill}\sphinxincludegraphics[width=0.500\linewidth]{{astc}.svg}\hspace*{\fill}}


\section{Tvöföld horn}
\label{\detokenize{Kafli07:tvofold-horn}}
Lítum á horn af gerðinni \(2x\) þar sem \(x\) er einhver tala. Við höfum eftirfarandi reglur um tvöföld horn:
\begin{equation*}
\begin{split}\begin{aligned}
\sin(2x)&=2 \cos(x) \sin(x) \\
\quad\\
\cos(2x)&= \cos^2(x)-\sin^2(x) \\
&= 2\cos^2(x)-1 \\
&= 1-2 \sin^2(x)
\end{aligned}\end{split}
\end{equation*}
Þessar reglur eru nytsamlegar í útreikningum.


\section{Andhverfur hornafallanna}
\label{\detokenize{Kafli07:andhverfur-hornafallanna}}
\textit{Andhverfur hornafallanna}, bogafall ― \(\arcsin, \arccos\) og \(\arctan\) ― eru \textit{andhverfur} fallana \(\sin, \cos\) og \(\tan\).

Skoðum aðeins jöfnuna
\begin{equation*}
\begin{split}\sin(x) = 0\end{split}
\end{equation*}
Hvað ef við viljum einangra \(x\) út úr þessari jöfnu?
Nú gæti einhver stungið upp á að \(x = 0\) sé lausnin því að \(\sin(0) = 0\).
Það svar er rétt, en þó aðeins að hluta til, því að þessi jafna hefur í raun óendanlega margar lausnir.
Tökum eftir að \(x = \pi\) er einnig lausn á þessari jöfnu sem og \(x = 2 \pi\).
Raunin er að \(n \cdot \pi\) er lausn á þessari jöfnu fyrir öll \(n \in \mathbb{Z}\).

Allar lausnirnar sem til eru á \(\sin(x) = 0\) eru á forminu \(x=n \cdot \pi, \quad (n \in \mathbb{Z})\).
Þess vegna skrifum við stundum
\begin{equation*}
\begin{split}\sin^{-1}(0) = \{n \pi ; \; n \in \mathbb{Z}\}\end{split}
\end{equation*}
þar sem veldið \(^{-1}\) táknar andhverft fall.
Þetta gildir auðvitað um fleiri tölur en \(0\).

Jafnan \(\sin(x) = a\) hefur á sama hátt óendanlega margar lausnir \(x\) fyrir öll \(a \in [−1, 1]\).
Hins vegar er auðvelt að sjá að \sphinxstylestrong{nákvæmlega ein} af þessum lausnum er á bilinu \([−\pi/2, \pi/2]\).
Við skilgreinum þess vegna nýtt fall \(\arcsin\) sem að er þannig að
\begin{equation*}
\begin{split}\arcsin(a) = x_0\end{split}
\end{equation*}
þá og því aðeins að \(x_0\) sé talan af bilinu \([−\pi/2, \pi/2]\) sem uppfyllir jöfnuna
\begin{equation*}
\begin{split}\sin(x_0) = a\end{split}
\end{equation*}
Því er \(arcsin\) hálfgerð andhverfa sínusfallsins vegna þess að
\begin{equation*}
\begin{split}\sin(\arcsin(x)) = x \qquad \text{fyrir öll } x \in [−1, 1]\end{split}
\end{equation*}
\sphinxstylestrong{Hún nær þó ekki að verða algjör andhverfa því að það öfuga gildir ekki}.
Það er, ekki er hægt að fullyrða að \(\arcsin(\sin(x))\) sé jafnt og x.
Til dæmis er \(\sin(2\pi) = 0\) og \(\arcsin(0) = 0\) og því fæst
\begin{equation*}
\begin{split}\arcsin(\sin(2\pi)) = \arcsin(\sin(0)) = 0\end{split}
\end{equation*}
Við skulum nú skilgreina andhverfur allra hornafallanna formlega:


\subsection{Skilgreining}
\label{\detokenize{Kafli07:id2}}

\subsubsection{Andhverfa sínusar}
\label{\detokenize{Kafli07:andhverfa-sinusar}}
\(\arcsin: \; [-1,1] \rightarrow [−\pi/2, \pi/2]\) er fallið sem uppfyllir
\begin{equation*}
\begin{split}\sin(\arcsin(x)) = x \qquad \text{fyrir öll  } x \in [−1, 1]\end{split}
\end{equation*}
\begin{sphinxadmonition}{warning}{Aðvörun:}
Athugum að \(\arcsin(x)\) er oft ritað \(\sin^{-1}(x)\)
\end{sphinxadmonition}

\noindent{\hspace*{\fill}\sphinxincludegraphics[width=0.500\linewidth]{{arcsin}.svg}\hspace*{\fill}}

Hér er graf \(\arcsin(x)\).

\begin{sphinxadmonition}{tip}{Dæmi:}
Hverjar eru lausnir \(\sin(v)=\frac12\), þ.e. hvað er \(\sin^{-1} \left(\frac12 \right)\) ?

Hér er gildið \(\frac12 >0\) og því leitum við að lausnum á fyrsta og öðrum fjórðungi einingahringsins, því þar er \(\sin(v)\geq 0\).

Skoðum einingarhringinn:

\begin{figure}[H]
\centering

\noindent\sphinxincludegraphics[width=1.000\linewidth]{{hringad1}.svg}
\end{figure}

Við sjáum að þegar \(v=\frac{\pi}{6}=30^\circ\) þá er \(\sin(v) = \frac12\). Það gildir líka þegar \(v=\frac{5\pi}{6} = 150^\circ\), því \(\sin(\pi-u) = \sin(u)\) fyrir öll \(u\) .

Því eru allar lausnir \(\sin(v)=\frac12\)
\begin{equation*}
\begin{split}v =
\begin{cases}
\frac{\pi}{6} + n\cdot 2\pi \\
\frac{5\pi}{6} + n \cdot 2\pi
\end{cases}\end{split}
\end{equation*}
fyrir öll \(n \in \mathbb{Z}\), eins og sjá má á mynd hér að neðan:

\noindent{\hspace*{\fill}\sphinxincludegraphics[width=1.000\linewidth]{{hringad1a}.svg}\hspace*{\fill}}
\end{sphinxadmonition}


\subsubsection{Andhverfa kósínusar}
\label{\detokenize{Kafli07:andhverfa-kosinusar}}
\(\arccos: \; [-1,1] \rightarrow [0, \pi]\) er fallið sem uppfyllir
\begin{equation*}
\begin{split}\cos(\arccos(x)) = x \qquad \text{fyrir öll  } x \in [−1, 1].\end{split}
\end{equation*}
\begin{sphinxadmonition}{warning}{Aðvörun:}
Athugum að \(\arccos(x)\) er oft ritað \(\cos^{-1}(x)\).
\end{sphinxadmonition}

\noindent{\hspace*{\fill}\sphinxincludegraphics[width=0.500\linewidth]{{arccos}.svg}\hspace*{\fill}}

Hér er graf \(\arccos(v)\).

\begin{sphinxadmonition}{tip}{Dæmi:}
Hverjar eru lausnir \(\cos(x)=\frac{\sqrt{3}}{2}\), þ.e. hvað er \(\cos^{-1}\left( \frac{\sqrt{3}}{2} \right)\)?

Hér er \(\frac{\sqrt{3}}{2} >0\) svo við skoðum lausnir á fyrsta og fjórða fjórðungi einingahringsins, því þar er \(\cos(u)>0\).

Skoðum einingarhringinn:

\begin{figure}[H]
\centering

\noindent\sphinxincludegraphics[width=1.000\linewidth]{{hringad2}.svg}
\end{figure}

Við sjáum að
\begin{equation*}
\begin{split}\cos\left(\frac{\pi}{4}\right)= \frac{\sqrt{3}}{2}\end{split}
\end{equation*}
Þá er líka
\begin{equation*}
\begin{split}\cos\left(\frac{-\pi}{4}\right) = \cos\left(\frac{7\pi}{4}\right) = \frac{\sqrt{3}}{2}\end{split}
\end{equation*}
því \(\cos(u) = \cos(-u)\) fyrir öll \(u\).

Þá eru allar lausnir \(\cos(v)=\frac{\sqrt{3}}{2}\)
\begin{equation*}
\begin{split}v =
\begin{cases}
\frac{\pi}{4} + n\cdot 2\pi \\
\frac{-\pi}{4} + n \cdot 2\pi
\end{cases}\end{split}
\end{equation*}
fyrir öll \(n \in \mathbb{Z}\), eins og sjá má á mynd hér að neðan:

\noindent{\hspace*{\fill}\sphinxincludegraphics[width=1.000\linewidth]{{hringad2a}.svg}\hspace*{\fill}}
\end{sphinxadmonition}


\subsubsection{Andhverfa tangens}
\label{\detokenize{Kafli07:andhverfa-tangens}}
\(\arctan: \; [-\infty,\infty] \rightarrow [−\pi/2, \pi/2]\) er fallið sem uppfyllir
\begin{equation*}
\begin{split}\tan(\arctan(x)) = x \qquad \text{fyrir öll  } x \in [−\infty, \infty]\end{split}
\end{equation*}
\begin{sphinxadmonition}{warning}{Aðvörun:}
Athugum að \(\arctan(x)\) er oft ritað \(\tan^{-1}(x)\).
\end{sphinxadmonition}

\noindent{\hspace*{\fill}\sphinxincludegraphics[width=0.500\linewidth]{{arctan}.svg}\hspace*{\fill}}

Hér er graf \(\arctan(v)\).

\begin{sphinxadmonition}{tip}{Dæmi:}
Hverjar eru lausnir \(\tan(v)=-\sqrt{3}\), þ.e. hvað er \(\tan^{-1} (-\sqrt{3})\) ?

Hér er \(-\sqrt{3} <0\) svo við skoðum lausnir á öðrum og fjórða fjórðungi einingahringsins því þar er \(\tan(u)<0\).

\(\tan(v)\) er hlutfallið á milli \(\sin(v)\) og \(\cos(v)\) og út frá einingarhringnum getum við fundið að þegar \(v=\frac{2\pi}{3}\) þá er \(\sin(v) = \frac{\sqrt{3}}{2}\) og \(\cos(v) = -\frac12\).
\begin{equation*}
\begin{split}\begin{aligned}
\tan(v) &= \frac{\sin(v)}{\cos(v)} \\
&= \frac{\sqrt{3}/2}{-1/2} \\
&= -\sqrt{3}
\end{aligned}\end{split}
\end{equation*}
Önnur lausn er \(v=\frac{5\pi}{3}\), því \(\tan(u) = \tan(u+\pi)\).
Við getum sannfært okkur um að það passi með því að reikna:
\begin{equation*}
\begin{split}\begin{aligned}
\tan\left(\frac{5\pi}{3}\right) &= \frac{\sin(5\pi/3)}{\cos(5\pi/3)} \\
&= \frac{-\sqrt{3}/2}{1/2} \\
&= - \sqrt{3}
\end{aligned}\end{split}
\end{equation*}
\begin{figure}[H]
\centering

\noindent\sphinxincludegraphics[width=1.000\linewidth]{{hringad3}.svg}
\end{figure}

Þá eru allar lausnir \(\tan(v)=-\sqrt{3}\)
\begin{equation*}
\begin{split}v=\frac{2\pi}{3} + n \cdot \pi\end{split}
\end{equation*}
fyrir öll \(n \in \mathbb{Z}\), eins og sjá má á mynd hér að neðan:

\noindent{\hspace*{\fill}\sphinxincludegraphics[width=1.000\linewidth]{{hringad3a}.svg}\hspace*{\fill}}
\end{sphinxadmonition}


\section{Tengsl í rúmfræði}
\label{\detokenize{Kafli07:tengsl-i-rumfraei}}

\subsection{Regla Pýþagórasar}
\label{\detokenize{Kafli07:regla-pyagorasar}}
Rifjum upp að fyrir \textit{rétthyrndan þríhyrning} gildir
\begin{equation*}
\begin{split}a^2+b^2=c^2\end{split}
\end{equation*}
þar sem \(c\) er langhliðin. Þessi regla nefnist regla Pýþagórasar.

Með því að horfa á einingarhringinn fáum við samband á milli kósínusar og sínusar, með hjálp reglu Pýþagórasar. Við skilgreindum kósínus sem \(x\)\sphinxhyphen{}hnit og sínus sem \(y\)\sphinxhyphen{}hnit. Við vitum að langhliðin hefur lengd \(1\) þar sem hringurinn hefur radíus \(1\). Við fáum því:
\begin{equation*}
\begin{split}\cos^2(\alpha)+\sin^2(\alpha)=1\end{split}
\end{equation*}
\noindent{\hspace*{\fill}\sphinxincludegraphics{{pythagoras}.svg}\hspace*{\fill}}


\subsection{Sínusreglan}
\label{\detokenize{Kafli07:sinusreglan}}
Í \(\triangle ABC\) gildir
\begin{equation*}
\begin{split}\frac{a}{\sin(A)} = \frac{b}{\sin(B)} = \frac{c}{\sin(C)}\end{split}
\end{equation*}
Þar sem \(A\), \(B\) og \(C\) eru horn þríhyrningsins og \(a\), \(b\) og \(c\) eru lengdir hliðanna


\bigskip\hrule\bigskip


\begin{figure}[htbp]
\centering

\noindent\sphinxincludegraphics[width=0.500\linewidth]{{thrihr}.svg}
\end{figure}


\subsection{Kósínusreglan}
\label{\detokenize{Kafli07:kosinusreglan}}
Í \(\triangle ABC\) gildir
\begin{equation*}
\begin{split}\begin{aligned}
a^2 &= b^2+c^2-2\cdot b \cdot c \cdot \cos(A) \\
b^2 &= a^2+c^2-2\cdot a \cdot c \cdot \cos(B) \\
c^2 &= b^2+a^2-2\cdot b \cdot a \cdot \cos(C) \\
\end{aligned}\end{split}
\end{equation*}

\section{Hornafallareglurnar}
\label{\detokenize{Kafli07:hornafallareglurnar}}
Athugið að horn eru yfirleitt táknuð með stöfum á borð við \(\theta\) , \(\alpha\) , \(u\) og \(v\).
Hornaföll eru jafnan táknuð sem föll af \(x\).

Hér á eftir koma reglur sem eru mikið notaðar.


\subsection{Grunnreglan}
\label{\detokenize{Kafli07:grunnreglan}}\begin{equation*}
\begin{split}\sin^2(\theta) + \cos^2(\theta) = 1\end{split}
\end{equation*}

\subsection{Hliðrunarreglur}
\label{\detokenize{Kafli07:hlirunarreglur}}\begin{equation*}
\begin{split}\begin{aligned}
1.& \qquad \cos(-\theta)=\cos \theta\\
2.& \qquad \sin(-\theta)=-\sin\theta\\
3.& \qquad \cos(\pi-\theta)=-\cos \theta\\
4.& \qquad \sin(\pi-\theta)=\sin \theta\\
5.& \qquad \cos(\theta+\pi)=-\cos \theta\\
6.& \qquad \sin(\theta+\pi)=-\sin \theta\\
7.& \qquad \cos\left(\frac{\pi}{2}-\theta\right)=\sin\theta\\
8.& \qquad \sin\left(\frac{\pi}{2}-\theta\right)=\cos\theta
\end{aligned}\end{split}
\end{equation*}

\subsection{Summuformúlur}
\label{\detokenize{Kafli07:summuformulur}}
\sphinxstylestrong{1.}
\begin{equation*}
\begin{split}\sin( u + v ) = \sin(u)  \cos(v) + \cos(u) \sin(v)\end{split}
\end{equation*}
\sphinxstylestrong{2.}
\begin{equation*}
\begin{split}\sin( u - v ) = \sin(u) \cos(v) - \cos(u) \sin(v)\end{split}
\end{equation*}
\sphinxstylestrong{3.}
\begin{equation*}
\begin{split}\cos( u + v ) = \cos(u)  \cos(v) - \sin(u)  \sin(v)\end{split}
\end{equation*}
\sphinxstylestrong{4.}
\begin{equation*}
\begin{split}\cos( u - v ) = \cos(u)  \cos(v) + \sin(u)  \sin(v)\end{split}
\end{equation*}
\sphinxstylestrong{5.}
\begin{equation*}
\begin{split}\tan(u-v) = \frac{\tan(u) - \tan(v)}{1 + \tan(u)  \tan(v)}\end{split}
\end{equation*}
\sphinxstylestrong{6.}
\begin{equation*}
\begin{split}\tan(u+v) = \frac{\tan(u) + \tan(v)}{1 - \tan(u)  \tan(v)}\end{split}
\end{equation*}

\subsection{Tvöföldunarformúlur}
\label{\detokenize{Kafli07:tvofoldunarformulur}}
\sphinxstylestrong{1.}
\begin{equation*}
\begin{split}\sin(2u) = 2\sin(u)\cos(u)\end{split}
\end{equation*}
\sphinxstylestrong{2.}
\begin{equation*}
\begin{split}\begin{aligned}
\cos(2u)&= \cos^2(u)-\sin^2(u) \\
&= 2\cos^2(u)-1 \\
&= 1-2 \sin^2(u)
\end{aligned}\end{split}
\end{equation*}
\sphinxstylestrong{3.}
\begin{equation*}
\begin{split}\tan(2u) = \frac{2\tan(u)}{1-\tan^2(u)}\end{split}
\end{equation*}

\subsection{Helmingunarformúlur}
\label{\detokenize{Kafli07:helmingunarformulur}}
\sphinxstylestrong{1.}
\begin{equation*}
\begin{split}\sin^2(u) = \frac{1- \cos(2u)}{2} \qquad \text{eða} \qquad \sin\left(\frac{u}{2}\right) = \pm \sqrt{\frac{1- \cos(u)}{2} }\end{split}
\end{equation*}
\sphinxstylestrong{2.}
\begin{equation*}
\begin{split}\cos^2(u) = \frac{1+ \cos(2u)}{2} \qquad \text{eða} \qquad \cos\left(\frac{u}{2}\right) = \pm \sqrt{\frac{1+ \cos(u)}{2} }\end{split}
\end{equation*}
\sphinxstylestrong{3.}
\begin{equation*}
\begin{split}\tan^2(u) = \frac{1- \cos(2u)}{1+\cos(2u)} \qquad \text{eða} \qquad \tan\left(\frac{u}{2}\right) = \pm \sqrt{\frac{1- \cos(u)}{1+\cos(u)} }\end{split}
\end{equation*}

\subsection{Summu\sphinxhyphen{} og margfeldisformúlur}
\label{\detokenize{Kafli07:summu-og-margfeldisformulur}}
\sphinxstylestrong{Margfeldisritháttur í summurithátt}
\begin{quote}

\sphinxstylestrong{1.}
\begin{equation*}
\begin{split}\sin(u)\sin(v) = \frac{1}{2}\left(\cos(u-v) - \cos(u+v)\right)\end{split}
\end{equation*}
\sphinxstylestrong{2.}
\begin{equation*}
\begin{split}\cos(u)\cos(v) = \frac{1}{2}\left(\cos(u-v) + \cos(u+v)\right)\end{split}
\end{equation*}
\sphinxstylestrong{3.}
\begin{equation*}
\begin{split}\sin(u)\cos(v) = \frac{1}{2}\left(\sin(u+v) + \sin(u-v)\right)\end{split}
\end{equation*}
\sphinxstylestrong{4.}
\begin{equation*}
\begin{split}\cos(u)\sin(v) = \frac{1}{2}\left(\sin(u+v) - \sin(u-v)\right)\end{split}
\end{equation*}\end{quote}

\sphinxstylestrong{Summuritháttur í margfeldisrithátt}
\begin{quote}

\sphinxstylestrong{1.}
\begin{equation*}
\begin{split}\sin(u) + \sin(v) = 2\sin\left(\frac{u+v}{2}\right)\cos\left(\frac{u-v}{2}\right)\end{split}
\end{equation*}
\sphinxstylestrong{2.}
\begin{equation*}
\begin{split}\sin(u) - \sin(v) = 2\cos\left(\frac{u+v}{2}\right)\sin\left(\frac{u-v}{2}\right)\end{split}
\end{equation*}
\sphinxstylestrong{3.}
\begin{equation*}
\begin{split}\cos(u) + \cos(v) = 2\cos\left(\frac{u+v}{2}\right)\cos\left(\frac{u-v}{2}\right)\end{split}
\end{equation*}
\sphinxstylestrong{4.}
\begin{equation*}
\begin{split}\cos(u) - \cos(v) = -2\sin\left(\frac{u+v}{2}\right)\sin\left(\frac{u-v}{2}\right)\end{split}
\end{equation*}\end{quote}


\chapter{Vigrar}
\label{\detokenize{Kafli08:vigrar}}\label{\detokenize{Kafli08::doc}}

\section{Skilgreining}
\label{\detokenize{Kafli08:skilgreining}}\label{\detokenize{Kafli08:s-vigrar}}
\textit{Vigrar} er stærðfræðilegt hugtak fyrir stærð sem hefur bæði lengd og stefnu.
Þeir eru jafnan teiknaðir sem örvar í hnitakerfi og lýst með hnitum.
Hnit vigurs eru venjulegar tölur, líka kallaðar \textit{skalarstærðir}, og lýsa staðsetningu endapunkts miðað við upphafspunkt.
Hvert hnit er tengt einum ás í hnitakerfinu sem notað er og stærð hnitsins (tölunnar) lýsir lengd vigursins í þá átt.

Vigrar eru oftast táknaðir með striki eða ör fyrir ofan bókstafinn, \(\overline{a}\) , \(\vec{a}\) , en sumir setja strikið undir, \(\underline{a}\) .
Í kennslubókum eru vigrar stundum feitletraðir: \(\boldsymbol{a}\).

Lengd vigra er táknuð með lóðréttum strikum, algildismerkjum, \(|\overline{a}|\) , eða einfaldlega bókstafnum án yfirstriksins, \(a\) .
Þegar vísað er í lengd vigurs eftir einhverjum ás, \textit{þátt} hans, er það sýnt með því að merkja með heiti ássins í \textit{lágvísi}; \(a_x\) er lengd vigursins \(\overline{a}\) í stefnu \(x\) \sphinxhyphen{} áss.
Þættir vigra eru ekki vigrar sjálfir, heldur tölur.

\begin{figure}[htbp]
\centering

\noindent\sphinxincludegraphics[width=0.600\linewidth]{{vigur}.svg}
\end{figure}

Yfirleitt er notað rétthyrnt hnitakerfi (einnig nefnt \textit{kartesískt}).
Vigurinn á myndinni hefur lengd 4 eftir x\sphinxhyphen{}ásnum og 3 eftir y\sphinxhyphen{}ásnum, svo hnit hans eru:
\begin{equation*}
\begin{split}\overline{a} = (a_x,a_y) = (4,3) = \begin{pmatrix} 4 \\ 3 \end{pmatrix}\end{split}
\end{equation*}
Lengd vigursins sjálfs er reiknuð með jöfnu Pýþagórasar, \(|\overline{a}| = a = \sqrt{a_x^2 + a_y^2}\) .
Vigurinn á myndinni hefur því lengdina \(a = \sqrt{4^2 + 3^2} = 5\) .

\begin{sphinxadmonition}{note}{Athugasemd:}
Stærð og stefna vigurs er óháð því hvar í hnitakerfinu hann er.
\end{sphinxadmonition}

Algengt er að láta vigra liggja frá upphafspunkti hnitakerfisins (stöðu\sphinxhyphen{} eða staðarvigur) en það er hægt að reikna vigra á milli gefinna upphafs\sphinxhyphen{} og endapunkta.
Vigurinn frá punktinum \(A=(x_1,y_1)\) til punktsins \(B=(x_2,y_2)\) er
\begin{equation*}
\begin{split}\overline{AB} = (x_2-x_1,y_2-y_1) = \begin{pmatrix} x_2-x_1 \\ y_2-y_1 \end{pmatrix}\end{split}
\end{equation*}
\begin{sphinxadmonition}{tip}{Dæmi:}
Reiknum vigurinn frá punktinum \(A=(-1,7)\) til punktsins \(B=(5,2)\) .
\begin{quote}
\begin{equation*}
\begin{split}\begin{aligned}
        \overline{AB} &= \begin{pmatrix} x_2-x_1 \\ y_2-y_1 \end{pmatrix}\\
        &= \begin{pmatrix} 5-(-1) \\ 2-7 \end{pmatrix} \\
        &= \begin{pmatrix} 6 \\ -5 \end{pmatrix}
\end{aligned}\end{split}
\end{equation*}
\begin{figure}[H]
\centering

\noindent\sphinxincludegraphics[width=0.600\linewidth]{{vigurtveirpkt}.svg}
\end{figure}
\end{quote}

Vigurinn \(\begin{pmatrix} 6 \\ -5 \end{pmatrix}\) er sá sami, hvort sem hann liggur á milli \(A\) og \(B\) eða frá upphafspunktinum til punktsins \((6,-5)\) .
\end{sphinxadmonition}

Stundum er talað um að vigur hafi \textit{hallatölu} : \(h=\frac{a_y}{a_x}\) , ef \(a_x\neq 0\) .
Tveir vigrar eru \textit{samsíða} ef þeir hafa sömu hallatölu.

\begin{sphinxadmonition}{tip}{Dæmi:}
Finnum vigur sem er samsíða \(\overline{a}=(-1,6)\) og hefur lengdina 9.

Hallatala \(\overline{a}\) er \(h_{\bar{a}}=\frac{a_y}{a_x}=\frac{6}{-1}=-6\) .
Þá vitum við að vigurinn sem við leitum að, \(b\) , uppfyllir það sama:
\begin{equation*}
\begin{split}h_{\bar{b}}=\frac{b_y}{b_x}=-6\end{split}
\end{equation*}
sem er jafngilt því að \(b_y=-6b_x\) .

Skilyrðið að \(\overline{b}\) þurfi að hafa lengdina 9 gefur að:
\begin{equation*}
\begin{split}|\overline{b}| = \sqrt{b_x^2+b_y^2} =9\end{split}
\end{equation*}
Setjum \(b_y=-6b_x\) inn og fáum:
\begin{equation*}
\begin{split}\begin{aligned}
  9 &= \sqrt{b_x^2+b_y^2}\\
  &=\sqrt{b_x^2+(-6b_x)^2} \\
  &= \sqrt{b_x^2+36b_x^2} \\
  &=\sqrt{37b_x^2} \\
  &=b_x\sqrt{37} \\
  b_x&=\frac{9}{\sqrt{37}} \approx 1.480\\
  b_y&= -6b_x = \frac{-54}{\sqrt{37}} \approx -8.878
\end{aligned}\end{split}
\end{equation*}
Vigur sem er samsíða \(\overline{a}=(-1,6)\) og hefur lengdina 9 er því
\begin{equation*}
\begin{split}\overline{b}= \begin{pmatrix} \frac{9}{\sqrt{37}} \\  \frac{-54}{\sqrt{37}} \end{pmatrix}\end{split}
\end{equation*}\end{sphinxadmonition}


\section{Að liða vigra}
\label{\detokenize{Kafli08:a-lia-vigra}}
Vigra er líka hægt að tákna með lengd og stefnuhorni.
Hornið \(\theta\) er skilgreint frá jákvæðum x\sphinxhyphen{}ás og að vigrinum.
Með þessum upplýsingum er hægt að liða vigurinn eftir x\sphinxhyphen{} og y\sphinxhyphen{}ás með því að nota hornaföll.

Þættir vigursins eru föll af stefnuhorninu sem er oft merkt \(\theta\) eða \(\phi\) :
\begin{equation*}
\begin{split}a_x = a\cos(\theta) \\
a_y = a\sin(\theta)\end{split}
\end{equation*}
þar sem \(a=|\overline{a}|\).

Stefnuhorn vigurs \(\overline{a} = (a_x,a_y)\) má því reikna:
\begin{equation*}
\begin{split}\frac{a_y}{a_x} = \frac{a\sin(\theta)}{a\cos(\theta)} = \tan(\theta)\end{split}
\end{equation*}
\begin{figure}[htbp]
\centering

\noindent\sphinxincludegraphics[width=0.600\linewidth]{{mynd-vigur}.svg}
\end{figure}


\section{Einingarvigrar}
\label{\detokenize{Kafli08:einingarvigrar}}
Einingarvigrar eru vigrar sem hafa lengdina 1.
Vigrarnir
\begin{equation*}
\begin{split}\begin{aligned}
\hat{\imath} &= \begin{pmatrix} 1 \\0 \\0 \end{pmatrix} \\
\hat{\jmath} &= \begin{pmatrix} 0 \\1 \\0 \end{pmatrix} \\
\hat{k} &= \begin{pmatrix} 0\\0 \\1 \end{pmatrix} \\
\end{aligned}\end{split}
\end{equation*}
liggja samsíða \(x\) \sphinxhyphen{} , \(y\) \sphinxhyphen{} og  \(z\) \sphinxhyphen{}  ásunum í rétthyrndu hnitakerfi .
Þeir eru líka táknaðir:
\begin{equation*}
\begin{split}\begin{gather}
\hat{e}_x, \quad \hat{e}_y, \quad \hat{e}_z
\end{gather}\end{split}
\end{equation*}
\begin{figure}[htbp]
\centering

\noindent\sphinxincludegraphics[width=0.600\linewidth]{{einingarvigrar}.svg}
\end{figure}

Einingarvigrarnir \(\hat{e}_x, \hat{e}_y\) og \(\hat{e}_z\) eru \textit{línulega óháðir}, sem þýðir að engan þeirra er hægt að mynda úr hinum tveimur með samlagningu þeirra eða margföldun með tölu.
Hvernig sem þú teygir á og raðar saman \(\hat{\imath}\) og \(\hat{\jmath}\) færðu aldrei út \(\hat{k}\) .
Þessi eiginleiki kemur til vegna þess að einingarvigrarnir eru allir hornréttir á hvorn annan.

Alla vigra má skrifa sem skalarstærðir margfaldaðar við einingarvigrana:
\begin{equation*}
\begin{split}\overline{a} = (a_x, \; a_y, \; a_z ) = a_x \hat{\imath} + a_y \hat{\jmath} + a_z \hat{k}\end{split}
\end{equation*}

\section{Samlagning vigra}
\label{\detokenize{Kafli08:samlagning-vigra}}
Þegar vigrar eru lagðir saman eru hnit eftir hverjum ás fyrir sig lögð saman.
Summa tveggja vigra \(\overline{a} = (a_x,a_y)\) og \(\overline{b} = (b_x,b_y)\) er:
\begin{equation*}
\begin{split}\overline{c} = \overline{a} + \overline{b} = (a_x + b_x, a_y +b_y) = \begin{pmatrix} a_x+b_x \\ a_y+b_y \end{pmatrix}\end{split}
\end{equation*}
\begin{sphinxadmonition}{tip}{Dæmi:}
Leggjum saman vigrana \(\overline{a}=(4,3)\) og \(\overline{b}=(1,3)\) :
\begin{equation*}
\begin{split}\overline{a}+\overline{b}=(4,3) + (1,3) = (4+1, 3+3) = (5,6)\end{split}
\end{equation*}\end{sphinxadmonition}

Myndrænt má ímynda sér að upphafspunktur seinni vigursins sé settur í endapunkt fyrri vigursins,
og summa þeirra er frá upphafspunkti fyrri vigursins til endapunkts þess seinni.

\begin{figure}[htbp]
\centering

\noindent\sphinxincludegraphics[width=0.600\linewidth]{{vigrasamlagning}.svg}
\end{figure}

\begin{sphinxadmonition}{note}{Athugasemd:}
Þó að \(\overline{c} = \overline{a} + \overline{b}\) þýðir það \sphinxstylestrong{ekki} að \(c = a + b\).

Í dæminu hér á undan er til dæmis
\begin{equation*}
\begin{split}c = |\overline{c}| = \sqrt{5^2+6^2} \approx 7,8 \\
a + b = \sqrt{4^2+3^2} + \sqrt{1^2+3^2} \approx 8,2\end{split}
\end{equation*}\end{sphinxadmonition}


\bigskip\hrule\bigskip


Um samlagningu vigra gilda eftirfarandi reglur:
\begin{equation*}
\begin{split}\begin{aligned}
  \overline{a} +\overline{b} &= \overline{b} + \overline{a} & \text{Víxlregla}\\
  (\overline{a}+\overline{b})+\overline{c} &= \overline{a} + (\overline{b}+\overline{c}) & \text{Tengiregla}
\end{aligned}\end{split}
\end{equation*}

\bigskip\hrule\bigskip


\begin{sphinxadmonition}{tip}{Dæmi:}
Höfum þrjá punkta:
\begin{equation*}
\begin{split}\begin{aligned}
A&=(x_1,y_1)=(1,2) \\
B&=(x_2,y_2)=(4,5) \\
C&=(x_3,y_3)=(3,-1)
\end{aligned}\end{split}
\end{equation*}
Reiknum vigrana \(\overline{AB}, \overline{AC} \text{ og } \overline{BC}\) :
\begin{equation*}
\begin{split}\begin{aligned}
  \overline{AB} &= \begin{pmatrix} x_2-x_1 \\ y_2-y_1 \end{pmatrix}\\
  &=\begin{pmatrix} 4-1 \\5-2\end{pmatrix} =\begin{pmatrix} 3 \\3\end{pmatrix} \\
  & \\
  \overline{AC} &= \begin{pmatrix}x_3-x_1\\ y_3-y_1 \end{pmatrix}\\
  &=\begin{pmatrix} 3-1 \\(-1)-2\end{pmatrix} =\begin{pmatrix} 2 \\-3\end{pmatrix} \\
  & \\
  \overline{BC} &= \begin{pmatrix}x_3-x_2\\ y_3-y_2 \end{pmatrix}\\
  &=\begin{pmatrix} 3-4 \\(-1)-5\end{pmatrix} =\begin{pmatrix} -1 \\-6\end{pmatrix} \\
\end{aligned}\end{split}
\end{equation*}
\begin{figure}[H]
\centering

\noindent\sphinxincludegraphics[width=0.500\linewidth]{{innskots}.svg}
\end{figure}

Hér eru punktarnir teiknaðir inn ásamt vigrunum \(\overline{AB}, \overline{AC} \text{ og } \overline{BC}\) .
\end{sphinxadmonition}

Af þessu dæmi má sjá \sphinxstyleemphasis{innskotsregluna} :
\begin{equation*}
\begin{split}\overline{AC} = \overline{AB} + \overline{BC}\end{split}
\end{equation*}

\section{Margföldun vigra}
\label{\detokenize{Kafli08:margfoldun-vigra}}
Þegar vigur \(\overline{v}\) er margfaldaður með tölu \(s\) er hver þáttur vigursins margfaldaður með tölunni:
\begin{equation*}
\begin{split}\begin{aligned}
    s\cdot\overline{v}  &= s\cdot(v_x, v_y, v_z)  \\
    &= (s \cdot v_x, s \cdot v_y, s \cdot v_z)
\end{aligned}\end{split}
\end{equation*}
Margfeldi vigra er tvenns konar, \textit{innfeldi (punktfeldi)}, og \textit{krossfeldi}.


\bigskip\hrule\bigskip


\sphinxstylestrong{Innfeldi} tveggja vigra er táknað með punkti og útkoman er \sphinxstyleemphasis{tala}: \(\overline{a} \cdot \overline{b}\) .
Ef þættir vigranna eru þekktir er innfeldið:
\begin{equation*}
\begin{split}\overline{a} \cdot \overline{b} = a_x b_x + a_y b_y\end{split}
\end{equation*}
\begin{sphinxadmonition}{tip}{Dæmi:}
Reiknum innfeldi vigranna \(\overline{a}=(7,8)\) og \(\overline{b}=(-1,3)\)

\sphinxstylestrong{Lausn}
\begin{equation*}
\begin{split}\overline{a} \cdot \overline{b} = a_x b_x + a_y b_y = 7\cdot (-1)+ 8\cdot 3 = -7+24 =17\end{split}
\end{equation*}\end{sphinxadmonition}

Ef vigrarnir eru gefnir með lengd og stefnuhorni er innfeldið:
\begin{equation*}
\begin{split}\overline{a} \cdot \overline{b} = a b \cos{\phi}\end{split}
\end{equation*}
þar sem \(\phi\) er hornið milli \(\overline{a}\) og \(\overline{b}\) þegar þeir hafa sama upphafspunkt.

\begin{sphinxadmonition}{warning}{Aðvörun:}
\sphinxstylestrong{Tveir vigrar eru hornréttir ef innfeldi þeirra er núll.}
\end{sphinxadmonition}

\begin{sphinxadmonition}{tip}{Dæmi:}
Reiknum hornið á milli vigranna \(\overline{a}=(2,4)\) og \(\overline{b}=(4,2)\) :

\begin{figure}[H]
\centering

\noindent\sphinxincludegraphics[width=0.600\linewidth]{{innfeldi}.svg}
\end{figure}

\sphinxstylestrong{Lausn}

Við vitum að \(\overline{a} \cdot \overline{b} = a b \cos{\phi}\) , þar sem \(a\) og \(b\) eru lengdir vigranna.
Lengdirnar eru:
\begin{equation*}
\begin{split}\begin{aligned}
  a = \sqrt{a_x^2 + a_y^2} = \sqrt{2^2 + 4^2} = \sqrt{20} \\
  b = \sqrt{b_x^2 + b_y^2} = \sqrt{4^2 + 2^2} = \sqrt{20}
\end{aligned}\end{split}
\end{equation*}
Reiknum innfeldi vigranna:
\begin{equation*}
\begin{split}\overline{a} \cdot \overline{b} = a_x b_x + a_y b_y = 2\cdot 4+ 4\cdot 2 = 16\end{split}
\end{equation*}
Því er
\begin{equation*}
\begin{split}\begin{aligned}
  \overline{a} \cdot \overline{b} &= a b \cos{\phi} \\
  \cos{\phi} &= \frac{\overline{a} \cdot \overline{b}}{a b} = \frac{16}{\sqrt{20} \cdot \sqrt{20}} = \frac{16}{20}\\
  \phi &= 36.8 ° = 0.644
\end{aligned}\end{split}
\end{equation*}\end{sphinxadmonition}


\bigskip\hrule\bigskip


\sphinxstylestrong{Krossfeldi} tveggja vigra er táknað með krossi og útkoman er nýr \sphinxstyleemphasis{vigur}: \(\overline{c} = \overline{a} \times \overline{b}\) .
\textit{Krossfeldi} eru reiknuð með þáttum vigranna, það er vigrum gefnum á forminu \(\overline{a} = a_x \hat{\imath} + a_y \hat{\jmath} + a_z \hat{k}\) .
\begin{equation*}
\begin{split}\begin{aligned}
\overline{a} \times \overline{b} &= (a_x \hat{\imath} + a_y \hat{\jmath} + a_z \hat{k}) \times (b_x \hat{\imath} + b_y \hat{\jmath} + b_z \hat{k}) \\
&= (a_y b_z - a_z b_y)\hat{\imath} + (a_z b_x - a_x b_z)\hat{\jmath} + (a_x b_y - a_y b_x)\hat{k} \\
\end{aligned}\end{split}
\end{equation*}
\begin{figure}[htbp]
\centering

\noindent\sphinxincludegraphics[width=0.600\linewidth]{{krossfeldi}.svg}
\end{figure}

Útkoma krossfeldisins er vigur sem er hornréttur á bæði \(a\) og \(b\).
Til að ákvarða sefnu vigursins getum við notað \sphinxstylestrong{hægri} handar regluna.

\begin{figure}[htbp]
\centering

\noindent\sphinxincludegraphics[width=0.900\linewidth]{{hhr}.svg}
\end{figure}

Lengd krossfeldis \(\overline{a} \text{ og } \overline{b}\) má reikna úr frá lengdum vigranna og horninu á milli þeirra.
\begin{equation*}
\begin{split}|\overline{a} \times \overline{b}| = |\overline{a}| |\overline{b}| \sin(\phi)\end{split}
\end{equation*}
\begin{sphinxadmonition}{note}{Athugasemd:}
Þegar krossfeldi er reiknað skiptir máli hvor vigurinn er á undan!
\begin{equation*}
\begin{split}\overline{a} \times \overline{b} = - \overline{b} \times \overline{a}\end{split}
\end{equation*}\end{sphinxadmonition}

\begin{sphinxadmonition}{tip}{Dæmi:}
Reiknum krossfeldi vigranna \(\overline{a}=(1,2,3)\) og \(\overline{b}=(4,5,6)\).

\sphinxstylestrong{Lausn}
\begin{equation*}
\begin{split}\begin{aligned}
  \overline{a} \times \overline{b} &= (a_y b_z - a_z b_y)\hat{\imath} + (a_z b_x - a_x b_z)\hat{\jmath} + (a_x b_y - a_y b_x)\hat{k} \\
  &= (2\cdot 6-3\cdot 5)\hat{\imath} + (3\cdot 4 - 1 \cdot 6) \hat{\jmath} + ( 1\cdot 5 - 2\cdot 4) \hat{k}\\
  &= -3 \hat{\imath} +6 \hat{\jmath} - 3\hat{k}\\
  &= (-3,6,-3)
\end{aligned}\end{split}
\end{equation*}\end{sphinxadmonition}


\section{Flatarmyndir}
\label{\detokenize{Kafli08:flatarmyndir}}
Ef hliðar þríhyrnings eru gefnar með vigrunum \(\overline{a}\) og \(\overline{b}\) er hægt að reikna hornið \(\theta\) .
Flatarmálið er þá \(F=\frac{1}{2}|\overline{a}| \cdot |\overline{b}| \cdot \sin(\theta)\) .

\begin{figure}[htbp]
\centering

\noindent\sphinxincludegraphics[width=0.400\linewidth]{{fl_thri2}.svg}
\end{figure}

Ef hliðar samsíðungs eru gefnar með vigrunum \(\overline{a}\) og \(\overline{b}\) er hægt að reikna hornið \(\theta\) .
Flatarmálið er þá \(F=|\overline{a}| \cdot |\overline{b}| \cdot \sin(\theta)\) og ummálið er \(U=2|\overline{a}|+2|\overline{b}|\) .

\begin{figure}[htbp]
\centering

\noindent\sphinxincludegraphics[width=0.400\linewidth]{{fl_sams2}.svg}
\end{figure}


\chapter{Meira um föll}
\label{\detokenize{Kafli09:meira-um-foll}}\label{\detokenize{Kafli09:s-meiraumfoll}}\label{\detokenize{Kafli09::doc}}

\section{Andhverfur}
\label{\detokenize{Kafli09:andhverfur}}
Þegar kemur að \textit{andhverfum} falla er gott að geta hugsað myndrænt.

Skoðum fyrst fallið \(f: \mathbb{R}_+ \to \mathbb{R}_+, \quad  f(x)=x^2\).

Við vitum að andhverfa \(f(x)=x^2\) er \(f^{-1}(x)=\sqrt{x}\) vegna þess að:
\begin{equation*}
\begin{split}\begin{aligned}
&f(f^{-1}(x)) = (\sqrt{x})^2=x \\
&\text{ og } \\
&f^{-1}(f(x))=\sqrt{x^2} = x
\end{aligned}\end{split}
\end{equation*}
Ef við teiknum gröf fallanna fáum við:

\noindent{\hspace*{\fill}\sphinxincludegraphics[width=0.600\linewidth]{{andhv_parabola}.svg}\hspace*{\fill}}

Hér sjáum við að rauða grafið, \(\sqrt{x}\), er spegilmynd svarta grafsins, \(x^2\), um punktalínuna.

\begin{sphinxadmonition}{note}{Athugasemd:}
Graf andhverfu falls er spegilmynd grafs fallsins um línuna \(y=x\).
\end{sphinxadmonition}

Skoðum næst \(y=2x+1\) eða \(f(x)=2x+1\).
Reiknum andhverfuna:
\begin{equation*}
\begin{split}\begin{aligned}
    y &= 2x+1 \\
    y-1&=2x\\
    \frac{1}{2}\left(y-1\right) &= x\\
    x &= \frac{1}{2} y - \frac{1}{2}
\end{aligned}\end{split}
\end{equation*}
þá er \(f^{-1}(x) = \frac{1}{2} x - \frac{1}{2}\).

Teiknum nú línurnar:

\noindent{\hspace*{\fill}\sphinxincludegraphics[width=0.700\linewidth]{{anhv_lina}.svg}\hspace*{\fill}}

Hér er rauða línan andhverfa \(f(x)\), við sjáum að línan speglast nákvæmlega um \(y=x\) eins og í fyrra dæminu \(x^2\).


\bigskip\hrule\bigskip


Hægt er að sannfæra sig á þessu með því að horfa á einfaldan feril sem fer í gengum 6 þekkta punkta,
\begin{equation*}
\begin{split}(0.5,1), (1.5,1), (2.5,2), (4,2.5), (3,2.5) \text{ og } (4,3.5)\end{split}
\end{equation*}
Séu þessir punktar tengdir saman með strikum fæst svarti ferilinn sem við sjáum hér að neðan.
Rauði ferillinn myndast þegar við speglum svarta yfir \(y=x\).

Speglum punktinum \((4, 2.5)\) yfir \(y=x\) , hann lendir í punktinum \((2.5,4)\) og strikið á milli þeirra er hornrétt á \(y=x\).

\noindent{\hspace*{\fill}\sphinxincludegraphics[width=0.500\linewidth]{{andhv3}.svg}\hspace*{\fill}}

Við getum því speglað punktunum um línuna með því að víxla á \(x\)\sphinxhyphen{} og \(y\)\sphinxhyphen{}hnitum punktanna.
Andhverfi ferillinn fer því á milli punktana \((1, 0.5), (1,1.5), (2,2.5), (2.5,4), (2.5,3)\) og \((3.5,4)\).


\bigskip\hrule\bigskip


Skoðum að lokum \(g(x) = \ln(x+2)\).
Reiknum andhverfuna:
\begin{equation*}
\begin{split}\begin{aligned}
    y &= \ln(x+2) \\
    e^y&=e^{\ln(x+2)} \\
    e^y &= x+2 \\
    x &= e^y -2\\
\end{aligned}\end{split}
\end{equation*}
Þá er andhverfa fallið \(g^{-1}(x) = e^x -2\).
Á þessari mynd má sjá gröf ferlanna, \(g(x)\) er svart en \(g^{-1}(x)\) rautt.

\noindent{\hspace*{\fill}\sphinxincludegraphics[width=0.700\linewidth]{{andhv2}.svg}\hspace*{\fill}}


\section{Eintæk og átæk föll}
\label{\detokenize{Kafli09:eintaek-og-ataek-foll}}
Skoðum föll \(f: X \to Y\).

Mengið \(X\) kallast \textit{formengi} (eða skilgreiningarmengi) fallsins og mengið \(Y\) \textit{bakmengi} (eða myndmengi eða varpmengi) þess.

Í formengi falls \(y=f(x)\) eru þær tölur sem við getum sett inn í fallið (\(x\) \sphinxhyphen{}in) en í bakmenginu eru þær tölur sem geta komið út (\(y\) \sphinxhyphen{}in).


\subsection{Átæk föll}
\label{\detokenize{Kafli09:ataek-foll}}
Látum \(f: X \to Y\) vera fall.

Látum \(y_0 \in Y\) vera stak í bakmenginu.
Oft þurfum við að vita hvort hægt sé að finna einhverja lausn á jöfnunni
\begin{equation*}
\begin{split}f(x)=y_0,\end{split}
\end{equation*}
það er að segja, hvort hægt sé að finna eitthvað \(x_0 \in X\) þannig að \(f(x_0)=y_0\).
Ef þessi jafna hefur lausn fyrir öll stökin í bakmenginu þá segjum við að fallið sé \textit{átækt}.


\subsection{Skilgreining}
\label{\detokenize{Kafli09:skilgreining}}
Fall \(f: X \to Y\) er sagt vera átækt ef fyrir sérhvert \(y \in Y\) er til \(x \in X\) þannig að \(f(x)=y\).


\bigskip\hrule\bigskip


Skoðum mengjamyndir til að sjá þessa tegund af vörpun fyrir okkur.
Hér er \(X\) skilgreiningarmengið og \(Y\) myndmengið.

\noindent{\hspace*{\fill}\sphinxincludegraphics[width=0.700\linewidth]{{ataek}.svg}\hspace*{\fill}}

Við sjáum að öll stökin í myndmenginu eru með í vörpuninni og hér er í lagi að fleiri en eitt stak í skilgreiningarmenginu varpast á sama stak í myndmenginu.

Hér er dæmi um vörpun sem er \sphinxstylestrong{ekki} átæk:

\noindent{\hspace*{\fill}\sphinxincludegraphics[width=0.700\linewidth]{{ekkiataek}.svg}\hspace*{\fill}}

Hún er ekki átæk því hér eru tvö stök í myndmenginu sem eru ekki með í vörpuninni.

\begin{sphinxadmonition}{tip}{Dæmi:}
\sphinxstylestrong{1.} Skoðum fallið \(f: \mathbb{R} \to \mathbb{R}\), \(f(x)=x^2\).
\begin{quote}

Tökum eftir að bakmengið er allt \(\mathbb{R}\), en \(x^2\) verður aldrei neikvæð tala. Til dæmis er \(-1\) stak í bakmenginu, en jafnan \(f(x)=-1\), eða \(x^2=-1\) hefur enga lausn í rauntölunum. Fallið er því \sphinxstylestrong{ekki átækt}.
\end{quote}

\sphinxstylestrong{2.} Skoðum fallið \(g: \mathbb{R} \to \mathbb{R}_+\), \(g(x)=x^2\).
\begin{quote}

Hér er bakmengið mengi allra jákvæðra rauntalna. Fyrir sérhverja jákvæða rauntölu \(a\) hefur jafnan \(x^2=a\) lausn. Hún fæst með kvaðratrót. Fallið er því \sphinxstylestrong{átækt}.
\end{quote}

\sphinxstylestrong{3.} Skoðum fallið \(h: \mathbb{R} \to \mathbb{R}\), \(h(x)=0\).
\begin{quote}

Sérhvert stak varpast í stakið \(0\) í bakmenginu. Til dæmis, fyrir stakið \(1\) í bakmenginu þá er ekki til nein lausn á jöfnunni \(h(x)=1\) þar sem það gefur \(0=1\) sem er fráleitt. Svo fallið er \sphinxstylestrong{ekki átækt}.
\end{quote}
\end{sphinxadmonition}


\subsection{Eintæk föll}
\label{\detokenize{Kafli09:eintaek-foll}}
Skoðum aftur jöfnuna
\begin{equation*}
\begin{split}f(x)=y_0\end{split}
\end{equation*}
Oft getur verið gagnlegt að vita hvort þessi jafna hafi margar lausnir.
Við segjum að fallið sé \textit{eintækt} ef þessi jafna hefur í mesta lagi eina lausn fyrir sérhvert stak í bakmenginu.
Fall er ekki eintækt ef fleiri en eitt stak í formenginu vísar á sama stakið í bakmenginu.


\subsection{Skilgreining}
\label{\detokenize{Kafli09:id1}}
Fall \(f: X \to Y\) er sagt vera \sphinxstyleemphasis{eintækt} ef fyrir sérhvert \(y \in Y\) er til í mesta lagi eitt \(x \in X\) þannig að \(f(x)=y\).

Það er, ef \(f(x_1)=f(x_2)\) þá er \(x_1=x_2\).


\bigskip\hrule\bigskip


Skoðum aftur mengjamyndir til að sjá þessa tegund af vörpun fyrir okkur.
Hér er \(X\) skilgreiningarmengið og \(Y\) myndmengið.

\noindent{\hspace*{\fill}\sphinxincludegraphics[width=0.700\linewidth]{{eintaek}.svg}\hspace*{\fill}}

Við sjáum að hvert stak í skilgreiningarmenginu á sér stak í myndmenginu og það er í lagi að sum stök í myndmenginu séu ekki með í vörpuninni.

Hér er dæmi um vörpun sem er \sphinxstylestrong{ekki} eintæk:

\noindent{\hspace*{\fill}\sphinxincludegraphics[width=0.700\linewidth]{{ekkieintaek}.svg}\hspace*{\fill}}

Hér varpast tvö stök í skilgreiningarmenginu á sama stak í myndmenginu.
\sphinxstyleemphasis{Takið eftir að þessi vörpun er hvorki eintæk né átæk.}

\begin{sphinxadmonition}{tip}{Dæmi:}
\sphinxstylestrong{1.} Skoðum fallið \(f: \mathbb{R} \to \mathbb{R}\), \(f(x)=x^2\).
\begin{quote}

Þetta fall er \sphinxstylestrong{ekki eintækt}. Til dæmis gildir \(f(-2)=f(2)=4\), það er jafnan \(f(x)=4\) hefur tvær lausnir.
\end{quote}

\sphinxstylestrong{2.} Skoðum fallið \(f: \mathbb{R}_+ \to \mathbb{R}\), \(f(x)=x^2\).
\begin{quote}

Þetta fall er \sphinxstylestrong{eintækt}, þar sem skilgreiningarmengið inniheldur bara jákvæðar tölur. Ef \(x_1\) og \(x_2\) eru ólíkar tölur í \(\mathbb{R}_+\), þá eru \(x_1^2\) og \(x_2^2\) ólíkar.
\end{quote}

\sphinxstylestrong{3.} Skoðum fallið \(g: \mathbb{R} \to \mathbb{R}\), \(g(x)=x\).
\begin{quote}

Þetta fall er \sphinxstylestrong{eintækt}. Ljóst er að ólík stök úr formenginu varpast í ólík stök í bakmenginu. Ef \(y_1 \not= y_2\) þá gildir að \(g(y_1) \not= g(y_2)\).
\end{quote}
\end{sphinxadmonition}


\subsection{Gagntæk föll}
\label{\detokenize{Kafli09:gagntaek-foll}}

\subsection{Skilgreining}
\label{\detokenize{Kafli09:id2}}
Fall \(f: X \to Y\) er sagt vera \textit{gagntækt} ef það er bæði eintækt og átækt.


\subsection{Regla}
\label{\detokenize{Kafli09:regla}}
Fall á sér andhverfu þá og því aðeins að það sé gagntækt.


\bigskip\hrule\bigskip


Skoðum mengjamynd til að sjá þessa tegund af vörpun fyrir okkur:

\noindent{\hspace*{\fill}\sphinxincludegraphics[width=0.700\linewidth]{{gagntaek}.svg}\hspace*{\fill}}

Við sjáum að öll stökin í myndmenginu eru með í vörpuninni þannig að vörpunin átæk.
Hvert stak í skilgreiningarmenginu varpast á nákvæmlega eitt stak í myndmenginu þannig að vörpunin er eintæk.
Fallið er því gagntækt þar sem það er bæði átækt og eintækt.

\begin{sphinxadmonition}{tip}{Dæmi:}
Við höfum séð að fallið \(f: \mathbb{R} \to \mathbb{R}\), \(f(x)=x^2\) er hvorki eintækt né átækt.
Það á sér því ekki andhverfu.

Skoðum til dæmis stakið \(9\) í bakmenginu.
Stökin í formenginu sem varpast í \(9\) eru tvö, það er \(f(3)=9\) og \(f(-3)=9\).
Til þess að ,,fara til baka‘‘ þá þyrftum við að úthluta \(9\) stökunum \(3\) og \(-3\), þ.e. stökunum \(\pm \sqrt{9}\).
Andhverfan getur því ekki verið fall, því samkvæmt skilgreiningu á falli fær hvert stak í formenginu úthlutað \sphinxstyleemphasis{nákvæmlega einu} staki í bakmenginu, en í þessu tilfelli eru þau tvö.
\end{sphinxadmonition}


\section{Samskeyting falla}
\label{\detokenize{Kafli09:samskeyting-falla}}

\subsection{Skilgreining}
\label{\detokenize{Kafli09:id3}}
Látum \(f: X \to Y\) og \(g: Y \to Z\) vera föll. Við skilgreinum þá vörpun \(g \circ f: X \to Z\) með:
\begin{equation*}
\begin{split}g \circ f(x)=g(f(x))\end{split}
\end{equation*}
fyrir öll \(x \in X\). Þetta kallast \textit{samskeytt fall}.

\noindent{\hspace*{\fill}\sphinxincludegraphics[width=1.000\linewidth]{{compvorpun}.svg}\hspace*{\fill}}

\begin{sphinxadmonition}{note}{Athugasemd:}
Bakmengi \(f\) og formengi \(g\) þarf að vera það sama. Annars gengur skilgreiningin ekki upp.
\end{sphinxadmonition}

\begin{sphinxadmonition}{tip}{Dæmi:}
Látum \(f:\mathbb{R} \to \mathbb{R}\) vera gefið með \(f(x)=x^2+x\)
og \(g:\mathbb{R} \to \mathbb{R}\) vera gefið með \(g(x)=x+3\)

Finnum \(f\circ g\) og \(g\circ f\). Höfum
\begin{equation*}
\begin{split}\begin{aligned}
f\circ g(x) &=f(g(x))\\
&=f(x+3)\\
&=(x+3)^2+(x+3)\\
&=x^2+6x+9+x+3\\
&=x^2+7x+12
\end{aligned}\end{split}
\end{equation*}
og
\begin{equation*}
\begin{split}\begin{aligned}
g\circ f(x)&=g(f(x))\\
&=g(x^2+x)\\
&=(x^2+x)+3\\
&=x^2+x+3
\end{aligned}\end{split}
\end{equation*}\end{sphinxadmonition}


\section{Nokkur mikilvæg föll}
\label{\detokenize{Kafli09:nokkur-mikilvaeg-foll}}

\subsection{Vísisföll}
\label{\detokenize{Kafli09:visisfoll}}
\textit{Vísisfall} er fall \(f : \mathbb{R} \to \mathbb{R}\) sem skrifa má með formúlu af gerðinni
\begin{equation*}
\begin{split}f(x)=a^x\end{split}
\end{equation*}
þar sem \(a \geq 0\) er rauntala.

Dæmi um vísisfall er \(f(x)=2^x\). Þá er \(f(1)=2\), \(f(2)=4\) og \(f(3)=8\) og \(f(4)=16\) o.s.fr.v.

Graf þess má sjá hér að neðan.

\noindent{\hspace*{\fill}\sphinxincludegraphics{{visis}.svg}\hspace*{\fill}}


\subsection{Lograr}
\label{\detokenize{Kafli09:lograr}}
Látum \(a\) vera jákvæða rauntölu og \(f: \mathbb{R} \to \mathbb{R}_+\) vera vísisfall gefið með
\begin{equation*}
\begin{split}f(x)=a^x.\end{split}
\end{equation*}
Þetta fall á sér andhverfu sem við köllum \(a\)\sphinxhyphen{} \textit{logrann} og er táknaður
\begin{equation*}
\begin{split}\log_a.\end{split}
\end{equation*}
Samkvæmt skilgreiningu á andhverfu er því \(a\)\sphinxhyphen{} \textit{logrinn}  fallið sem uppfyllir:
\begin{equation*}
\begin{split}\log_a(a^x)=x \qquad \text{fyrir öll } x \in \mathbb{R},\end{split}
\end{equation*}
og
\begin{equation*}
\begin{split}a^{\log_a(x)}=x \qquad \text{fyrir öll } x \in \mathbb{R}_+.\end{split}
\end{equation*}
\begin{sphinxadmonition}{note}{Athugasemd:}
Óformlega getum við hugsað um töluna \(\log_a(x)\) þannig: „Í hvaða veldi þarf að setja \(a\) svo að útkoman verði \(x\)?“
\end{sphinxadmonition}

\begin{sphinxadmonition}{tip}{Dæmi:}
\sphinxstylestrong{1.} Reiknum \(\log_2(8)\). Í töluðu máli er spurningin þessi:
\begin{quote}

„Í hvaða veldi þarf að setja tvo svo að útkoman verði átta?“

Auðvelt er að reikna að \(2^3=8\), svarið er því \(3\) og við skrifum
\(\log_2(8)=3\)
\end{quote}

\sphinxstylestrong{2.} Reiknum \(\log_3(81)\).
\begin{quote}

Auðvelt er að staðfesta að \(3^4=81\), svo \(\log_3(81)=4\)
\end{quote}

\sphinxstylestrong{3.} Reiknum \(\log_9(1)\).
\begin{quote}

Athugum að um sérhverja tölu \(a\) gildir \(a^0=1\), sér í lagi er \(9^0=1\) svo \(\log_9(1)=0\).
\end{quote}
\end{sphinxadmonition}

\begin{sphinxadmonition}{note}{Athugasemd:}
10 \sphinxhyphen{} logrinn er oft skrifaður \(\log(x)\) frekar en \(\log_{10}(x)\) .
Þessi logri er mikið notaður og yfirleitt er sérstakur takki á reiknivélum til þess að reikna hann.
\end{sphinxadmonition}


\subsubsection{Lograreglur}
\label{\detokenize{Kafli09:lograreglur}}
Fyrir \(a,b,x,y\in \mathbb{R}_+\) og \(r \in \mathbb{R}\) gildir:
\begin{enumerate}
\sphinxsetlistlabels{\arabic}{enumi}{enumii}{}{.}%
\item {} 
\(\qquad \log_a(1)=0\)

\item {} 
\(\qquad \log_a(1/x)=-\log_a(x)\)

\item {} 
\(\qquad \log_a(xy)=\log_a(x)+\log_a(y)\)

\item {} 
\(\qquad \log_a(x/y)=\log_a(x)-\log_a(y)\)

\item {} 
\(\qquad \log_a(x^r)=r\log_a(x)\)

\item {} 
\(\qquad \log_a(x)=\dfrac{\log_b(x)}{\log_b(a)}\).

\end{enumerate}

\begin{sphinxadmonition}{tip}{Dæmi:}\begin{quote}

\sphinxstylestrong{1.} Reiknum \(\log_5(50)+\log_5(\frac{1}{2})\).
\begin{quote}

Við notum reiknireglur tvö, þrjú, og fjögur:
\begin{equation*}
\begin{split}\begin{aligned}\log_5(50)+\log_5(\frac{1}{2})&=\log_5(5^2\cdot 2)-\log_5(2)\\&=\log_5(5^2)+\log_5(2)-\log_5(2)\\&=\log_5(5^2)=2\end{aligned}\end{split}
\end{equation*}\end{quote}

\sphinxstylestrong{2.} Reiknum \(\log_2(49)\cdot \log_7(2)\)
\begin{quote}

Notum reiknireglu sex:
\begin{equation*}
\begin{split}\begin{aligned}
               \log_2(49)\cdot \log_7(2)&=\dfrac{\log_7(49)}{\log_7(2)}\cdot \log_7(2)\\
               &=\log_7(49)\\
               &=\log_7(7^2)=2
       \end{aligned}\end{split}
\end{equation*}\end{quote}

\sphinxstylestrong{3.} Reiknum \((\log_{12}(1))^{12}\)
\begin{quote}

Notum reiknireglu eitt:
\begin{equation*}
\begin{split}(\log_{12}(1))^{12}=0^{12}=0\end{split}
\end{equation*}\end{quote}

\sphinxstylestrong{4.} Reiknum \(\log_7(22)\)
\begin{quote}

Notum reiknireglu sex og setjum \(b=10\), stingum stærðinni \(\log(22)/\log(7)\) inn í vasareikninn og fáum
\end{quote}
\end{quote}
\begin{equation*}
\begin{split}\log_7(22)=\frac{\log(22)}{\log(7)}\approx 0,629532003\end{split}
\end{equation*}\end{sphinxadmonition}


\subsection{Náttúrulega veldisvísisfallið og nátturulegi logrinn}
\label{\detokenize{Kafli09:natturulega-veldisvisisfalli-og-natturulegi-logrinn}}
Náttúrulega veldisvísisfallið er skilgreint sem
\begin{equation*}
\begin{split}f(x) = e^x,\end{split}
\end{equation*}
þar sem \(e \approx 2.71828182846...\) er óræð tala.

Skoðum graf fallsins

\noindent{\hspace*{\fill}\sphinxincludegraphics[width=0.700\linewidth]{{e}.svg}\hspace*{\fill}}

Þá er andhverfa \(f(x) = e^x\) skilgreind sem \(f^{-1}(x) = \log_e (x)\) og yfirleitt skrifað
\begin{equation*}
\begin{split}f^{-1}(x) = \ln(x).\end{split}
\end{equation*}
Fallið \(\ln(x)\) er kallað \textit{náttúrulegi logrinn} .
Skoðum graf hans:

\noindent{\hspace*{\fill}\sphinxincludegraphics[width=0.700\linewidth]{{ln}.svg}\hspace*{\fill}}

Sjáum hér að \(\ln(x)\) er \(e^x\) speglað um \(y=x\).

\noindent{\hspace*{\fill}\sphinxincludegraphics[width=0.700\linewidth]{{lnoge}.svg}\hspace*{\fill}}

Sömu reglur gilda um náttúrulega logrann og um aðra logra.


\subsubsection{Lograreglur}
\label{\detokenize{Kafli09:id4}}
Fyrir \(x,y\in \mathbb{R}_+\) og \(r \in \mathbb{R}\) gildir:
\begin{enumerate}
\sphinxsetlistlabels{\arabic}{enumi}{enumii}{}{.}%
\item {} 
\(\qquad \ln(1)=0\)

\item {} 
\(\qquad \ln(xy)=\ln(x)+\ln(y)\)

\item {} 
\(\qquad \ln(x/y)=\ln(x)-\ln(y)\)

\item {} 
\(\qquad \ln(x^r)=r\ln(x)\)

\end{enumerate}


\subsection{Ræð föll}
\label{\detokenize{Kafli09:rae-foll}}
Ef \(r\) er fall sem tákna má með formúlu af gerðinni
\begin{equation*}
\begin{split}r(x)=\dfrac{a_nx^n+a_{n-1}x^{n-1}+...+a_1x+a_0}{b_mx^m+b_{m-1}x^{n-1}+...+b_1x+b_0}\end{split}
\end{equation*}
þá segjum við að \(r\) sé \textit{rætt fall}.
Í þessari formúlu er \(n,m\in\mathbb{N}\), \textit{stuðlarnir} \(a_i\) og \(b_i\) eru rauntölur fyrir öll \(i\) og fremstu stuðlarnir mega ekki vera \(0\), það er \(a_n,b_m\not=0\).

Þetta er bara önnur leið til að segja að fallið \(r\) kallist rætt fall ef til eru margliður \(p\) og \(q\) þannig að \(r=\frac{p}{q}\).


\subsubsection{Myndrænt}
\label{\detokenize{Kafli09:myndraent}}
Skoðum einföld ræð föll á forminu:
\begin{equation*}
\begin{split}\frac{ax+b}{cx+d}\end{split}
\end{equation*}
Ef stuðlarnir \(a,b,c, \; \text{og} \; d\) eru þekktir er fljótlegt að finna \textit{aðfellur} fallsins til þess að teikna grafið.

\sphinxstylestrong{Lóðfellan} verður í gegnum punktinn á \(x\) \sphinxhyphen{} ásnum sem er ekki í skilgreiningarmenginu, það er að segja þar sem deilt væri með núlli.
Lóðfella ræðs falls á þessu formi er því línan
\begin{equation*}
\begin{split}x=\frac{-d}{c}\end{split}
\end{equation*}
\sphinxstylestrong{Láfellan} verður í gegnum punktinn á \(y\) \sphinxhyphen{} ásnum sem er ekki í myndmenginu, það er að segja gildið sem fallið getur aldrei tekið.
Láfella ræðs falls á þessu formi er því línan
\begin{equation*}
\begin{split}y=\frac{a}{c}\end{split}
\end{equation*}
\begin{sphinxadmonition}{tip}{Dæmi:}
Skoðum ræða fallið
\begin{equation*}
\begin{split}f(x) = \frac{x-2}{x+3}\end{split}
\end{equation*}
Hér er \(a= 1, \; b =-2, \; c = 1\) og \(d = 3\).

Þá eru aðfellurnar:
\begin{equation*}
\begin{split}\begin{aligned}
& x = \frac{-d}{c} \; = \; \frac{-3}{1} \; = \; -3 \\
& \quad \\
& y = \frac{a}{c} \; = \; \frac{1}{1} \; = \; 1 \\
\end{aligned}\end{split}
\end{equation*}
Nú er lítið mál að sjá fyrir sér fallið:

\noindent{\hspace*{\fill}\sphinxincludegraphics[width=1.000\linewidth]{{adfellur}.svg}\hspace*{\fill}}
\end{sphinxadmonition}


\subsection{Stofnbrotaliðun}
\label{\detokenize{Kafli09:stofnbrotaliun}}
Þegar við erum að vinna með ræð föll getur verið þægilegra að liða þau niður áður en unnið er með þau.
Þegar margliðan í teljaranum hefur stigið 1 og margliðan í nefnaranum hefur stigið 2 er hægt að gera það svona:
\begin{equation*}
\begin{split}\begin{aligned}
\frac{ax+b}{(x-\alpha)(x-\beta)} &= \frac{A}{(x-\alpha)}+ \frac{B}{(x-\beta)} \\
\quad \\
&\text{þar sem} \\ \quad \\
\alpha \neq \beta, \quad A= \frac{a\alpha + b}{\alpha - \beta} & \quad \text{og} \quad B= \frac{a\beta + b}{\beta - \alpha} \\
\end{aligned}\end{split}
\end{equation*}
\begin{sphinxadmonition}{tip}{Dæmi:}
Liðum
\begin{equation*}
\begin{split}\frac{3x+2}{x^2+3x-4}\end{split}
\end{equation*}
í stofnbrot.

\sphinxstyleemphasis{Lausn:}

Þáttum nefnarann \(x^2+3x-4\) og fáum \((x+4)(x-1)\).
Hér er \(a=3\), \(b=2\), \(\alpha = -4\)  og \(\beta=1\).

Reiknum fastana \(A\) og \(B\) :
\begin{equation*}
\begin{split}\begin{aligned}
  A& = \frac{a\alpha+b}{\alpha-\beta} \\
  &= \frac{3\cdot(-4)+2}{-4-1} =\frac{-12+2}{-5} \\
  A&=2 \\
  B&= \frac{a\beta +b}{\beta-\alpha} \\
  &= \frac{3+2}{1+4} =\frac{5}{5} \\
  B&=1
\end{aligned}\end{split}
\end{equation*}
Því er hægt að skrifa:
\begin{equation*}
\begin{split}\frac{3x+2}{x^2+3x-4} = \frac{2}{x+4} + \frac{1}{x-1}\end{split}
\end{equation*}
Athugum hvort þetta sé rétt með því að leggja brotin saman:
\begin{equation*}
\begin{split}\begin{aligned}
  \frac{2}{x+4} + \frac{1}{x-1} &= \frac{2(x-1)}{(x+4)(x-1)}+\frac{1(x+4)}{(x-1)(x+4)} \\
  &=\frac{2x-2+x+4}{(x+4)(x-1)} \\
  &=\frac{3x+2}{x^2+3x-4}
\end{aligned}\end{split}
\end{equation*}\end{sphinxadmonition}

Látum \(p\) og \(q\) vera margliður og látum \(r=\frac{p}{q}\) vera rætt fall. Ef margliðurnar \(p\) og \(q\) eru af háum stigum getur ræða fallið \(r\) oft verið erfitt viðureignar. Þá er gagnlegt að geta skrifað \(r\) sem summu af einfaldari ræðum föllum. Eftirfarandi regla getur þá stundum verið gagnleg:


\subsubsection{Regla}
\label{\detokenize{Kafli09:id5}}
Látum \(p\) og \(q\) vera margliður af stigi \(n\) og \(m\).

Gerum ráð fyrir að margliðan \(q\) hafi \(m\) ólíkar rætur \(a_1,a_2,...,a_m\).

Þá er til margliða \(s\) og fastar \(b_1,b_2,...,b_m\) þannig að
\begin{equation*}
\begin{split}\frac{p(x)}{q(x)}=s(x)+\frac{b_1}{x-a_1}+\frac{b_2}{x-a_2}+...+\frac{b_m}{x-a_m}.\end{split}
\end{equation*}
Þegar þessari reglu er beitt þá segjumst við vera að \sphinxstyleemphasis{stofnbrotaliða} ræða fallið \(\frac{p}{q}\).


\bigskip\hrule\bigskip


Stofnbrotaliðum ræða fallið \(\frac{p}{q}\) þar sem \(p\) og \(q\) eru margliður og
\begin{equation*}
\begin{split}q(x)=c_mx^m+...+c_1x+c_0\end{split}
\end{equation*}
er af stigi \(m\).
\begin{enumerate}
\sphinxsetlistlabels{\arabic}{enumi}{enumii}{}{.}%
\item {} 
Finnum allar núllstöðvar margliðunnar \(q\). Ef margliðan hefur færri en \(m\) núllstöðvar hættum við hér, því þá virkar þessi aðferð ekki. Ef \(m\) ólíkar núllstöðvar finnast köllum við þær \(a_1,a_2,...,a_m\).

\item {} 
Deilum margliðunni \(q\) upp í margliðuna \(p\) með afgangi til þess að finna margliður \(s\) og \(p_1\) sem eru þannig að stig \(p_1\) er minna en stig \(q\) og \(p=sq+p_1\). Þá má skrifa:

\end{enumerate}
\begin{equation*}
\begin{split}\frac{p(x)}{q(x)}=s(x)+\frac{p_1(x)}{q(x)}\end{split}
\end{equation*}\begin{enumerate}
\sphinxsetlistlabels{\arabic}{enumi}{enumii}{}{.}%
\setcounter{enumi}{2}
\item {} 
Skilgreinum nýja margliðu \(q'\) með því að setja

\end{enumerate}
\begin{equation*}
\begin{split}q'(x)=mc_mx^{m-1}+(m-1)c_{m-1}x^{m-2}+...+2\cdot c_2x+1\cdot c_1\end{split}
\end{equation*}\begin{enumerate}
\sphinxsetlistlabels{\arabic}{enumi}{enumii}{}{.}%
\setcounter{enumi}{3}
\item {} 
Reiknum út stuðlana \(b_1,b_2,...,b_m\) með formúlunni

\end{enumerate}
\begin{equation*}
\begin{split}b_i=\frac{p_1(a_i)}{q'(a_i)} \qquad \text{fyrir öll i}\end{split}
\end{equation*}\begin{enumerate}
\sphinxsetlistlabels{\arabic}{enumi}{enumii}{}{.}%
\setcounter{enumi}{4}
\item {} 
Nú má skrifa

\end{enumerate}
\begin{equation*}
\begin{split}\frac{p(x)}{q(x)}=s(x)+\frac{b_1}{x-a_1}+\frac{b_2}{x-a_2}+...+\frac{b_m}{x-a_m}\end{split}
\end{equation*}
\begin{sphinxadmonition}{note}{Athugasemd:}
Þeir sem eru komnir aðeins lengra í stærðfræði og þekkja diffrun munu taka eftir að í aðferðinni að ofan þá er nýja margliðan \(q'\) afleiðan af margliðunni \(q\).
\end{sphinxadmonition}

\begin{sphinxadmonition}{tip}{Dæmi:}
Stofnbrotaliðið ræða fallið
\begin{equation*}
\begin{split}\frac{x^4-2}{x^3+2x^2-x-2}.\end{split}
\end{equation*}
Hér er \(p(x)=x^4-2\) og \(q(x)=x^3+2x^2-x-2\).
\begin{enumerate}
\sphinxsetlistlabels{\arabic}{enumi}{enumii}{}{.}%
\item {} 
Finnum núllstöðvar \(q\). \(p/q\)\sphinxhyphen{}aðferðin sem lýst var í fyrri kafla segir okkur að við eigum að prófa hvort tölurnar \(-1,1,-2\) eða \(2\) séu núllstöðvar margliðunnar \(q\):

\end{enumerate}
\begin{quote}
\begin{equation*}
\begin{split}q(-1)=0, \qquad q(1)=0, \qquad q(-2)=0, \qquad q(2)=12.\end{split}
\end{equation*}
Hér fundum við þrjár mismunandi núllstöðvar, \(q\) hefur stig \(3\) svo við getum haldið áfram. Við setjum \(a_1=-1, \; a_2=1 \; \text{og} \; a_3=-2\).
\end{quote}
\begin{enumerate}
\sphinxsetlistlabels{\arabic}{enumi}{enumii}{}{.}%
\setcounter{enumi}{1}
\item {} 
Deilum \(q\) uppí \(p\) með afgangi:

\end{enumerate}
\begin{quote}

\noindent{\hspace*{\fill}\sphinxincludegraphics[width=0.800\linewidth]{{rflongdiv}.svg}\hspace*{\fill}}

Samkvæmt þessu getum við skrifað
\begin{equation*}
\begin{split}p(x)=(x-2)q(x)+(5x^2-6)\end{split}
\end{equation*}
Við setjum \(p_1(x)=5x^2-6\) og \(s(x)=x-2\).
\end{quote}
\begin{enumerate}
\sphinxsetlistlabels{\arabic}{enumi}{enumii}{}{.}%
\setcounter{enumi}{2}
\item {} 
Skilgreinum margliðuna

\end{enumerate}
\begin{quote}
\begin{equation*}
\begin{split}q'(x)=3x^{3-1}+2\cdot 2x^{2-1} - 1x^{1-1}=3x^2+4x-1\end{split}
\end{equation*}\end{quote}
\begin{enumerate}
\sphinxsetlistlabels{\arabic}{enumi}{enumii}{}{.}%
\setcounter{enumi}{3}
\item {} 
Reiknum út:

\end{enumerate}
\begin{quote}
\begin{equation*}
\begin{split}\begin{aligned}
b_1&=\frac{p_1(a_1)}{q'(a_1)}\\&=\frac{p_1(-1)}{q'(-1)}\\&=\frac{5\cdot(-1)^2-6}{3\cdot(-1)^2+4\cdot(-1)-1}\\&=\frac{-1}{-2}\\&=\frac{1}{2}\\b_2&=\frac{p_1(a_2)}{q'(a_2)}\\&=\frac{-1}{6}\\b_3&=\frac{p_1(a_3)}{q'(a_3)}\\&=\frac{14}{3}\end{aligned}\end{split}
\end{equation*}
Þá er \(b_1 =\frac{1}{2}, \; b_2 =\frac{-1}{6} \; \text{og} \; b_3=\frac{14}{3}\).
\end{quote}
\begin{enumerate}
\sphinxsetlistlabels{\arabic}{enumi}{enumii}{}{.}%
\setcounter{enumi}{4}
\item {} 
Lausnin okkar er þess vegna:

\end{enumerate}
\begin{quote}
\begin{equation*}
\begin{split}\begin{aligned}
\frac{x^4-2}{x^3+2x^2-x-2}&=x-2+\frac{1/2}{x+1}+\frac{-1/6}{x-1}+\frac{14/3}{x+2} \\
&=x-2+\frac{1}{2(x+1)}-\frac{1}{6(x-1)}+\frac{14}{3(x+2)}.\\
\end{aligned}\end{split}
\end{equation*}\end{quote}
\end{sphinxadmonition}


\section{Ummyndanir}
\label{\detokenize{Kafli09:ummyndanir}}
Það er mjög mikilvægt að geta teiknað föll og séð þau fyrir sér, meðal annars að geta séð fyrir sér \textit{ummyndanir}.


\subsection{Hliðrun}
\label{\detokenize{Kafli09:hlirun}}
Færsla punktsins \((x,y)\) yfir á punktinn \((x+a,y+b)\) kalllast \textit{hliðrun} um vigurinn \(\begin{pmatrix} a \\ b \end{pmatrix}\).

Hliðrunarvigurinn \(\begin{pmatrix} a \\ b \end{pmatrix}\) færir feril fallsins \(f(x)\) yfir í feril fallsins
\begin{equation*}
\begin{split}g(x) =f(x-a)+b.\end{split}
\end{equation*}
\begin{sphinxadmonition}{tip}{Dæmi:}
Hliðrum \(f(x) = x^2\) um \(\begin{pmatrix} 2 \\ 1 \end{pmatrix}\)

Þessi hliðrun þýðir að hver punktur ferilsins færist um \(2\) til hægri á \(x\)\sphinxhyphen{}ásnum og \(1\) upp á \(y\)\sphinxhyphen{}ásnum. Þá er nýja hliðraða fallið:
\begin{equation*}
\begin{split}\begin{aligned}
 g(x) &= f(x-a)+b, \\
 &= f(x-2)+1, \\
 &= (x-2)^2 +1.
\end{aligned}\end{split}
\end{equation*}
\noindent{\hspace*{\fill}\sphinxincludegraphics[width=0.700\linewidth]{{hlidrun1}.svg}\hspace*{\fill}}

Sjáum \(g(x)= (x-2)^2 +1\) er í rauðu og hefur hliðrast upp til hægri.
Punkturinn \((-2,4)\) færist í \((-2+2,4+1)=(0,5)\).
\end{sphinxadmonition}

\begin{sphinxadmonition}{tip}{Dæmi:}
Hliðrum \(f(x) = \sin(x)\) um \(\begin{pmatrix} -\frac{\pi}{2} \\ 0 \end{pmatrix}\). Fáum
\begin{equation*}
\begin{split}\begin{aligned}
 g(x) &= f(x-a)+b, \\
 &= f(x-(-\frac{\pi}{2}))+0, \\
 &= \sin(x+\frac{\pi}{2}), \\
 &= \cos(x).
\end{aligned}\end{split}
\end{equation*}
\noindent{\hspace*{\fill}\sphinxincludegraphics[width=1.000\linewidth]{{hlidrun2}.svg}\hspace*{\fill}}

Hér erum við búin að hliðra sínus um \(\frac{\pi}{2}\) eftir \(x\)\sphinxhyphen{}ás og þá fáum við kósínus! Sjá kafla 7 um {\hyperref[\detokenize{Kafli07:s-hornafoll}]{\sphinxcrossref{\DUrole{std,std-ref}{hornaföll}}}}.
\end{sphinxadmonition}


\subsection{Speglun}
\label{\detokenize{Kafli09:speglun}}
\sphinxstylestrong{Speglun um} \(x\) \sphinxstylestrong{\sphinxhyphen{}ás}

Þegar falli \(f(x)\) er speglað um \(x\)\sphinxhyphen{}ás fæst fallið \(g(x) = -f(x)\). Tökum sem dæmi \(f(x)=x^2\), þá er speglunin \(g(x) = -f(x) = -x^2\).

\noindent{\hspace*{\fill}\sphinxincludegraphics[width=0.500\linewidth]{{speglunx}.svg}\hspace*{\fill}}

\sphinxstylestrong{Speglun um} \(y\) \sphinxstylestrong{\sphinxhyphen{}ás}

Þegar falli \(f(x)\) er speglað um \(y\)\sphinxhyphen{}ás fæst fallið \(h(x) = f(-x)\).
Tökum sem dæmi \(f(x)=\frac{x-2}{x+3}+2\), þá er speglunin \(h(x) = f(-x) = \frac{(-x)-2}{(-x)+3}+2\).

\noindent{\hspace*{\fill}\sphinxincludegraphics[width=0.500\linewidth]{{spegluny}.svg}\hspace*{\fill}}

Hér sjáum við líka lóðfellurnar sem speglast um \(y\)\sphinxhyphen{}ás.


\subsection{Stríkkun}
\label{\detokenize{Kafli09:strikkun}}

\subsubsection{Lóðrétt}
\label{\detokenize{Kafli09:lorett}}
Við getum ummyndað fall \(f(x)\) með því að margfalda það með jákvæðum fasta og þá kallast það stríkkun. Tökum sem dæmi \(a \cdot f(x)\).
\begin{itemize}
\item {} 
Ef \(0<a<1\) þá köllum við stríkkuninna herpingu.

\item {} 
Ef \(1<a\) þá er það stríkkun.

\end{itemize}

Skoðum áhrifin á fleyboga:

\noindent{\hspace*{\fill}\sphinxincludegraphics[width=1.000\linewidth]{{strikkun}.svg}\hspace*{\fill}}


\subsubsection{Lárétt}
\label{\detokenize{Kafli09:larett}}
Við getum líka ummyndað fall \(f(x)\) lárétt með því að margfalda það með jákvæðum fasta \(f(a\cdot x)\).
\begin{itemize}
\item {} 
Ef \(0<a<1\) þá erum við að tala um herpingu.

\item {} 
Ef \(1<a\) þá er það stríkkun.

\end{itemize}

\begin{sphinxadmonition}{note}{Athugasemd:}
Takið eftir að ummyndanir eru varpanir af vörpunum, þ.e.a.s. \sphinxstylestrong{samskeyting falla}. Til dæmis ef við viljum hliðra fallinu \(f(x) = x^2\) upp um \(2\) og stríkkum um helming þá er
\end{sphinxadmonition}


\chapter{Markgildi}
\label{\detokenize{Kafli10:markgildi}}\label{\detokenize{Kafli10::doc}}

\section{Markgildi falls}
\label{\detokenize{Kafli10:markgildi-falls}}
Áður en að við skilgreinum \textit{markgildi} \textit{falls} formlega skulum við taka dæmi til að útskýra hver hugmyndin er.
Skilgreinum fall:
\begin{equation*}
\begin{split}g:\;\mathbb{R}\setminus\{0\}\to \mathbb{R}, \qquad g(x)=\frac{\sin(x)}{x}.\end{split}
\end{equation*}
Tökum eftir að við látum fallið \(g\) ekki vera skilgreint í punktinum \(x=0\)  til að komast hjá því að deila með núlli.

Það getur hins vegar verið áhugavert að skoða hvernig fallið hagar sér \sphinxstylestrong{nálægt} punktinum \(x=0\).
Teiknum mynd af fallinu \(g\):

\begin{figure}[htbp]
\centering

\noindent\sphinxincludegraphics[width=0.600\linewidth]{{sin}.svg}
\end{figure}

Við notum vasareikni til að reikna nokkur gildi á \(g\) og merkjum inn í \textit{hnitakerfið} punktana \((x,g(x))\).
Athugum að fallið er \textit{jafnstætt} svo \(g(-x)=g(x)\) fyrir öll \(x\).
\begin{equation*}
\begin{split}\begin{aligned}
    g(-1) &=g(1) \approx 0,841470984 \\
    g(-0,5) &=g(0,5)\approx 0,958851077\\
    g(-0,25) &=g(0,25)\approx 0,989615837 \\
    g(-0,1) &=g(0,1)\approx 0,998334166 \\
    g(-0,001) &= g(0,001) \approx 0,9999833
\end{aligned}\end{split}
\end{equation*}
Takið sérstaklega eftir því að fallið virðist ekki fara upp eða niður í óendanleikann þegar við nálgumst gildið \(x=0\).
Öllu heldur þá virðist fallið stefna á gildið \(1\)!

Við höfum séð að ef \(x_0\) er tala sem er mjög nálægt núlli þá verður fallgildið \(g(x_0)\) mjög nálægt því að verða \(1\).

Munum að fallið \(g\) er \sphinxstylestrong{ekki skilgreint í núlli}, hins vegar höfum við sér orðalag fyrir svona tilvik.
Við segjum að \textit{markgildi} fallsins \(g\) í núlli sé einn.
Einnig má segja að \(g(x)\) stefni á einn þegar \(x\) stefnir á núll.
Á táknmáli er skrifað
\begin{equation*}
\begin{split}\lim_{x\to 0}g(x)=1.\end{split}
\end{equation*}
Þetta er hægt að gera almennt.

Látum nú \(a\) vera stak í \textit{bilinu} \(I\) í \(\mathbb{R}\).
Látum \(f:\;I\setminus\{a\}\to \mathbb{R}\) vera eitthvað fall og \(b\in \mathbb{R}\) vera tölu.
Við segjum að markgildi fallsins \(f\) í punktinum \(x=a\) sé \(b\) ef fyrir allar tölur \(x_0\) sem eru nálægt tölunni \(a\) þá er talan \(f(x_0)\) nálægt tölunni \(b\).
Þá skrifum við
\begin{equation*}
\begin{split}\lim_{x\to a}f(x)=b.\end{split}
\end{equation*}
Setjum fram skilgreininguna á markgildi.


\subsection{Skilgreining}
\label{\detokenize{Kafli10:skilgreining}}
Gerum ráð fyrir að \(I\subset \mathbb{R}\) sé bil í \(\mathbb{R}\) og að \(a\in I\) sé punktur á bilinu sem er hvorugur endapunkta þess.

Gerum ráð fyrir að \(f:\; I\setminus\{a\}\to \mathbb{R}\) sé fall og að \(b\in \mathbb{R}\) sé tala.

Við segjum að markgildi fallsins \(f\) í punktinum \(a\) sé \(b\) og ritum
\begin{equation*}
\begin{split}\lim_{x\to a}f(x)=b\end{split}
\end{equation*}
ef að eftirfarandi gildir:

Fyrir sérhvert \(\epsilon>0\) er til \(\delta>0\) þannig að
ef
\begin{equation*}
\begin{split}|x-a|<\delta,\end{split}
\end{equation*}
þá er
\begin{equation*}
\begin{split}|f(x)-b|<\epsilon.\end{split}
\end{equation*}
\begin{sphinxadmonition}{note}{Athugasemd:}
Í þessari skilgreiningu má ímynda sér að \(\epsilon\) og \(\delta\) séu rosalega litlar tölur.

Ójafnan \(|x-a|<\delta\) þýðir þá að \(x\) sé rosalega nálægt því að vera \(a\) og ójafnan \(|f(x)-b|<\epsilon\) þýðir að \(f(x)\) er rosalega nálægt því að vera \(b\).

Athugum að \(a\) og \(b\) geta verið hvaða tölur sem er, jafnvel \(\pm \infty\).
\end{sphinxadmonition}

\begin{sphinxadmonition}{warning}{Aðvörun:}
Við segjum að markgildi fallsins \sphinxstylestrong{sé til} ef fallið stefnir á rauntölu.

Ef fall stefnir á \(+ \infty\) eða \(-\infty\) segjum við að markgildið \sphinxstylestrong{sé ekki til} .
\end{sphinxadmonition}


\bigskip\hrule\bigskip


Skoðum ræða fallið \(f(x) = \frac{x}{x-1}\) .

\begin{figure}[htbp]
\centering

\noindent\sphinxincludegraphics[width=0.700\linewidth]{{hv}.svg}
\end{figure}

Hvert er markgildi \(f\) þegar \(x\) stefnir á 1?

Þegar við skoðum bláa grafið vinstra megin við vandræðapunktinn tekur fallið snögga dýfu niður í \(-\infty\) þegar það nálgast 1.
Þetta köllum við að skoða \sphinxstyleemphasis{vinstra markgildi} og táknum með litlum mínus í hávísi.
Í þessu tilviki myndum við skrifa:
\begin{equation*}
\begin{split}\lim_{x \to 1^{-}} \frac{x}{x-1} = -\infty\end{split}
\end{equation*}
Hins vegar, þegar við eltum rauða ferilinn hægra megin við vandræðapunktinn stefnir fallið hratt upp í \(+\infty\) þegar það nálgast 1.
Þetta köllum við \sphinxstyleemphasis{hægra markgildi} og táknum með litlum plús í hávísi.
Í þessu tilviki myndum við skrifa:
\begin{equation*}
\begin{split}\lim_{x \to 1^{+}} \frac{x}{x-1} = \infty\end{split}
\end{equation*}
\begin{sphinxadmonition}{note}{Athugasemd:}
Þegar hægra markgildi og vinstra markigildi falls í punkti er ekki það sama þá er \sphinxstyleemphasis{markgildið ekki til}.

Markgildi \(f(x)\) í punktinum \(x=a\) er ekki til nema ef
\begin{equation*}
\begin{split}\lim_{x\to a^+} f(x) = \lim_{x\to a^-} f(x)\end{split}
\end{equation*}\end{sphinxadmonition}

\begin{sphinxadmonition}{warning}{Aðvörun:}
Skoðum myndræn dæmi þar sem markgildið er \sphinxstylestrong{ekki til}.

\begin{figure}[H]
\centering

\noindent\sphinxincludegraphics{{markg}.svg}
\end{figure}

Hér er fallið ekki að stefna á eitt gildi í punktinum \(a\), því það er ekki að stefna á sama gildi hægra megin og vinstra megin. Markgildið er því ekki til.

\begin{figure}[H]
\centering

\noindent\sphinxincludegraphics{{markg2}.svg}
\end{figure}

Hér stefnir fallið á \(\infty\) í punktinum \(a\) og því er markgildið ekki til.
\end{sphinxadmonition}

Ef við höfum markgildið \(\lim_{x \to c} f(x)\) þar sem fallið er rætt, það er að segja á forminu \(f(x) = \frac{p(x)}{q(x)}\), þá þarf að passa að \(q(c) \neq 0\). Þá verða markgildisreikningarnir einfaldir, sjá dæmi 1 og 2.

\begin{sphinxadmonition}{tip}{Dæmi:}\begin{quote}

Finnið markgildin:
\begin{description}
\item[{\sphinxstylestrong{1.}}] \leavevmode
\(\lim_{x\to 2}\frac{x^3+2x^2}{x^3-x^2+1}\).

Hér er \(q(x)=x^3-x^2+1\)
og \(q(2)=2^3-2^2+1=5\not=0\)

Hér er \(2\) í skilgreiningarmenginu svo til þess að reikna markgildið er gildinu einfaldlega stungið inn.
Markgildið er því:
\begin{equation*}
\begin{split}\lim_{x\to 2}\frac{x^3+2x^2}{x^3-x^2+1}=\frac{2^3+2\cdot 2^2}{2^3-2^2+1}=\frac{16}{5}\end{split}
\end{equation*}
\item[{\sphinxstylestrong{2.}}] \leavevmode
\(\lim_{x\to 4}\frac{x^3-4x^2-4x+16}{x^2-16}\) .

Hér er \(q(x)=x^2-16\) og \(p(x)=x^3-4x^2-4x+16\) .

Við tökum eftir að \(q(4)=0\) og \(p(4)=0\).

Prófum að stytta út þætti.
Við getum umritað \(p(x) = (x-2)(x+2)(x-4)\) og \(q(x)=(x-4)(x+4)\)
og sjáum að \((x-4)\) er sameiginlegur þáttur sem styttist út.

Fáum því markgildið:
\begin{equation*}
\begin{split}\begin{aligned}
 \lim_{x\to 4}\frac{x^3-4x^2-4x+16}{x^2-16} &= \lim_{x\to 4} \frac{(x-2)(x+2)(x-4)}{(x-4)(x+4)} \\
       &=\lim_{x\to 4}\frac{(x-2)(x+2)}{(x+4)}\\
 &=\frac{(4-2)(4+2)}{4+4}\\
 &=\frac{12}{8}\\
 &=\frac{3}{2}
\end{aligned}\end{split}
\end{equation*}
\item[{\sphinxstylestrong{3.}}] \leavevmode
Látum \(f: \mathbb{R} \setminus \{1\} \to \mathbb{R} \qquad f(x)=x\).

Sýnið að \(\lim_{x \to 1} f(x) = 1\) .

Þetta verður nánast augljóst ef grafið er teiknað.
Hér setjum við lítinn hring í punktinn \((1, 1)\) því fallið er ekki skilgreint þar.

\begin{figure}[H]
\centering

\noindent\sphinxincludegraphics[width=0.700\linewidth]{{mkgexmp1}.svg}
\end{figure}

Það er því ljóst að fallið stefnir á punktinn \((1, 1)\) frá báðum áttum.

\item[{\sphinxstylestrong{4.}}] \leavevmode
Látum \(f : \mathbb{R} \setminus \{4\} \to \mathbb{R} \qquad f(x) = \sqrt{x}\).
Sýnið að \(\lim_{x \to 4} = 2\).

Aftur er nánast augljóst hvert markgildið er.
Ef ferill fallsins \(f\) er teiknaður upp sést greinilega að hann stefnir á punktinn \((4, 2)\).
Það er að segja, það sést að markgildi fallsins er tveir.

\begin{figure}[H]
\centering

\noindent\sphinxincludegraphics[width=0.700\linewidth]{{mkgexmp}.svg}
\end{figure}

Við skulum sýna þetta formlega.
Látum \(\epsilon > 0\) vera einhverja gefna tölu.
Sýna þarf að til sé \(\delta>0\) þannig að \(|x−4|<\delta\) hafi í för með sér að \(| x−2|<\epsilon\). Látum \(\delta = 2 \cdot \epsilon\).
Þá fæst að ef \(|x − 4| < \delta\) þá er

\end{description}
\end{quote}
\begin{equation*}
\begin{split}\begin{aligned}
|f(x) - 2| &= |\sqrt{x} - 2|,\\
& = \frac{|\sqrt{x} - 2||\sqrt{x} + 2|}{|\sqrt{x} + 2|},\\
& = \frac{|x - 4|}{|\sqrt{x} + 2|},\\
& \leqq \frac{|x-4|}{|2|} \leqq \frac{\delta}{2} = \epsilon.
\end{aligned}\end{split}
\end{equation*}\end{sphinxadmonition}


\section{Gerðir markgilda}
\label{\detokenize{Kafli10:gerir-markgilda}}
Hér er samantekt af helstu gerðum markgilda:
\begin{enumerate}
\sphinxsetlistlabels{\arabic}{enumi}{enumii}{}{.}%
\item {} 
\(\quad \lim_{x \to a} f(x) \qquad \qquad \quad\) markgildið af \(f(x)\) þegar \(x\)  stefnir á  \(a\)

\item {} 
\(\quad \lim_{x \to a+} f(x) \qquad\qquad\;\) markgildið af \(f(x)\) þegar \(x\)  stefnir á \(a\) frá hægri

\item {} 
\(\quad \lim_{x \to a-} f(x) \qquad \qquad \;\) markgildið af \(f(x)\) þegar \(x\) stefnir á \(a\) frá vinstri

\item {} 
\(\quad \lim_{x\to +\infty} f(x) \qquad \qquad\) markgildið af \(f(x)\) þegar \(x\)  stefnir á plús óendanlegt

\item {} 
\(\quad \lim_{x\to -\infty} f(x) \qquad \qquad\) markgildið af \(f(x)\) þegar \(x\) stefnir á mínus óendanlegt

\end{enumerate}


\bigskip\hrule\bigskip


Skoðum markgildið \(\lim_{x \to \infty}\frac{1}{x^n}\) fyrir öll \(n \in \mathbb{N}_+\)

Byrjum á \(n=1\), þá er augljóst að
\begin{equation*}
\begin{split}\lim_{x \to \infty}\frac{1}{x} = 0\end{split}
\end{equation*}
því ef \(x\) er stórt þá er \(\frac{1}{x}\) lítið.
Því er \(\frac{1}{x}\) að minnka þegar \(x\) er að stækka.

Ef \(n=2\) þá er líka augljóst að
\begin{equation*}
\begin{split}\lim_{x \to \infty}\frac{1}{x^2} = \lim_{x \to \infty}\left(\frac{1}{x}\right)^2 = 0\end{split}
\end{equation*}
vegna þess að
\begin{equation*}
\begin{split}\begin{aligned}
x&\leq x^2 \quad & \forall x \geq 1\\
\frac{1}{x^2} &\leq \frac{1}{x} \quad &\forall x \geq 1 \end{aligned}\end{split}
\end{equation*}
Þegar við hækkum veldið á \(x\) í nefnaranum þá minnkar \(\frac{1}{x}\) hraðar, því \(x\) er að stækka hraðar.

Við sjáum því að \(\frac{1}{x^n}\) stefnir á núll þegar \(x\) stefnir á \(\infty\) fyrir allar jákvæðar heiltölur tölur \(n\).
\begin{equation*}
\begin{split}\lim_{x \to \infty} \frac{1}{x^n} = 0 \qquad \forall n \in \mathbb{Z}_+\end{split}
\end{equation*}
\begin{figure}[htbp]
\centering

\noindent\sphinxincludegraphics[width=0.400\linewidth]{{xminusn}.svg}
\end{figure}


\bigskip\hrule\bigskip


En hvað með markgildið \(\lim_{x\to 0^+} \frac{1}{x^n}\) ?

Þegar við nálgumst núll ofan frá erum við að deila með sífellt minni tölu. Þá fáum við sífellt stærri tölu út:
\begin{equation*}
\begin{split}\begin{aligned}
&\frac{1}{2} <\frac{1}{1}<\frac{1}{0.5} <\frac{1}{0.1} <\frac{1}{0.001}\\
&0.5<1<2<10 < 1000
\end{aligned}\end{split}
\end{equation*}
Því hlýtur
\begin{equation*}
\begin{split}\lim_{x\to 0^+} \frac{1}{x^n} = \infty\end{split}
\end{equation*}

\bigskip\hrule\bigskip


Skoðum nokkur dæmi:

\begin{sphinxadmonition}{tip}{Dæmi:}
Finnið eftirfarandi markgildi:

\sphinxstyleemphasis{Hér munum við láta rökstuðning duga í staðinn fyrir að nota formlegu skilgreiningarnar.}

\sphinxstylestrong{1.} \(\lim_{x\to 0}\frac{1}{(2^x-1)^2}\)
\begin{quote}

Þegar \(x\) stefnir á núll þá stefnir \((2^x-1)^2\) á núll (því \(2^0=1\)) svo að
\(\frac{1}{(2^x-1)^2}\) stefnir annað hvort á plús eða mínus óendanlegt. Þar sem  \(\frac{1}{(2^x-1)^2}=\left(\frac{1}{2^x-1}\right)^2\) er stæða í öðru veldi þá er hún alltaf jákvæð. Hún hlýtur því að stefna á plús óendanlegt, við skrifum
\begin{equation*}
\begin{split}\lim_{x\to 0}\frac{1}{(2^x-1)^2}=\infty.\end{split}
\end{equation*}
\begin{figure}[H]
\centering

\noindent\sphinxincludegraphics[width=0.400\linewidth]{{daemi2}.svg}
\end{figure}

Sjáum út frá mynd að fallið stefnir á plús óendalegu þegar \(x\) stefnir á \(0\) frá báðum áttum.
\end{quote}

\sphinxstylestrong{2.} \(\lim_{x\to \frac{\pi}{2}^-}\frac{1}{\cos(x)}\)
\begin{quote}

Þegar \(x\) stefnir á \(\pi/2\) þá stefnir \(\cos(x)\) á núll svo \(\frac{1}{\cos(x)}\) stefnir á plús eða mínus óendanlegt.
Þekkt er að \(\cos(x)\) er jákvætt ef \(0<x<\pi/2\). Stæðan \(\frac{1}{\cos(x)}\) er þess vegna jákvæð á sama bili og þess vegna stefnir hún á plús óendanlegt ef \(x\) nálgast \(\pi/2\) frá vinstri.
Við skrifum
\begin{equation*}
\begin{split}\lim_{x\to \frac{\pi}{2}^{-}}\frac{1}{\cos(x)}=\infty.\end{split}
\end{equation*}
\begin{figure}[H]
\centering

\noindent\sphinxincludegraphics[width=1.000\linewidth]{{daemi3}.svg}
\end{figure}

Hér er fallið með \textit{aðfellu} í \(\frac{\pi}{2}\) sem þýðir að fallið tekur ekki gildi í \(\frac{\pi}{2}\) en við sjáum að fallið er að stefna á plús óendanlegt frá vinstri.
\end{quote}
\end{sphinxadmonition}


\section{Reikniaðgerðir á markgildum}
\label{\detokenize{Kafli10:reikniagerir-a-markgildum}}
Þegar  markgildi eru reiknuð gilda reiknireglur sem ættu ekki að koma á óvart.

Gerum ráð fyrir að \(f\) og \(g\) séu föll og að \(c\in \mathbb{R} \cup\{-\infty,\infty\}\)
Gerum ráð fyrir að bæði markgildin
\begin{equation*}
\begin{split}\lim_{x\to c}f(x)\qquad \text{og}\qquad \lim_{x\to c}g(x)\end{split}
\end{equation*}
séu skilgreind og að hvorugt þeirra sé jafnt plús eða mínus óendanlegu.
Gerum ráð fyrir að \(k\in\mathbb{R}\) sé fasti.
Þá gildir:
\begin{equation*}
\begin{split}\begin{aligned}
1. & \qquad \lim_{x\to c}k=k \\
\quad\\
2. & \qquad \lim_{x\to c} \left(kf(x) \right)=k \cdot \left(\lim_{x\to c}f(x)\right) \\
\quad\\
3. & \qquad \lim_{x\to c} \left(f(x)+g(x)\right)=\lim_{x\to c}f(x)+\lim_{x\to c}g(x) \\
\quad\\
4. & \qquad \lim_{x\to c} \left(f(x)-g(x)\right)=\lim_{x\to c}f(x)-\lim_{x\to c}g(x) \\
\quad\\
5. & \qquad \lim_{x\to c} \left(f(x)\cdot g(x)\right)= \left( \lim_{x\to c}f(x) \right)\cdot \left(\lim_{x\to c}g(x) \right) \\
\quad\\
6. & \qquad \lim_{x\to c} \left( \frac{f(x)}{g(x)} \right)=\frac{\lim_{x\to c}f(x)}{\lim_{x\to c}g(x)} \qquad \text{ef} \qquad \lim_{x\to c}g(x)\not=0
\end{aligned}\end{split}
\end{equation*}
\begin{sphinxadmonition}{tip}{Dæmi:}
Finnið eftirfarandi markgildið ef það er til.
\begin{equation*}
\begin{split}\lim_{x \to \infty} \frac{x^2 + 1}{2x^2 + 5x +1}\end{split}
\end{equation*}
Við byrjum á því að deila með \(x^2\) fyrir ofan og neðan strik. Við notum svo reiknireglurnar fyrir afganginn og þá staðreynd að \(\frac{1}{x^n}\) stefnir á núll þegar \(x\) stefnir á \(\infty\) fyrir allar náttúrulegar tölur \(n\).
\begin{equation*}
\begin{split}\begin{aligned}
\lim_{x \to \infty} \frac{x^2 + 1}{2x^2 + 5x +1}
&= \lim_{x \to \infty} \frac{1 + \frac{1}{x^2}}{2 + \frac{5}{x} + \frac{1}{x^2}} \\
&= \frac{\lim_{x \to \infty}( 1 + \frac{1}{x^2})}{\lim_{x \to \infty}( 2 + \frac{5}{x} + \frac{1}{x^2})} \\
&=\frac{\lim_{x \to \infty}1+\lim_{x \to \infty} \frac{1}{x^2}}{\lim_{x \to \infty}2+\lim_{x \to \infty} \frac{5}{x} +\lim_{x \to \infty} \frac{1}{x^2}} \\
&=\frac{1+0}{2+0+0}\\
&=\frac{1}{2}
\end{aligned}\end{split}
\end{equation*}\end{sphinxadmonition}

\begin{sphinxadmonition}{tip}{Dæmi:}
Finnið markgildið:
\begin{quote}
\begin{equation*}
\begin{split}\lim_{x\to \infty}\frac{(\sin(x))^2}{x}.\end{split}
\end{equation*}
Auðvelt er að sjá að fallið \(\frac{1}{x}\) stefnir á núll þegar \(x\) stefnir á óendanlegt.
Þekkt er að \(|\sin(x)|\leq 1\) fyrir öll \(x\) og því fæst að \(0\leq (\sin(x))^2 \leq 1\) fyrir öll \(x\).
Þegar \(x\) er stórt er því \(\frac{1}{x}\) nálægt núlli og \((\sin(x))^2\) er á milli núll og einn.
En þá er auðvelt að sjá að
\begin{equation*}
\begin{split}\frac{(\sin(x))^2}{x}=(\sin(x))^2\frac{1}{x}\end{split}
\end{equation*}\end{quote}

er nálægt núlli svo að markgildið er núll. Hér notuðum við reiknireglu 5.

Við skrifum:
\begin{equation*}
\begin{split}\lim_{x\to\infty}\frac{(\sin(x))^2}{x}=0.\end{split}
\end{equation*}
\begin{figure}[H]
\centering

\noindent\sphinxincludegraphics[width=1.000\linewidth]{{daemi1}.svg}
\end{figure}

Sjáum á grafinu að þegar við förum lengra eftir \(x\)\sphinxhyphen{}ás nálgast fallið \(0\).
\end{sphinxadmonition}

\begin{sphinxadmonition}{tip}{Dæmi:}
Finnið markgildið:
\begin{equation*}
\begin{split}\lim_{x\to \infty} \frac{x}{2x+1}\end{split}
\end{equation*}
Þegar \(x\) stefnir á stærri og stærri tölur byrjar \(1\) að skipta minna máli því hann er mikið minni í samanburði við það sem \(x\) stefnir á. Þá erum við komin með einfaldara markgildi til að skoða:
\begin{equation*}
\begin{split}\lim_{x\to \infty} \frac{x}{2x}\end{split}
\end{equation*}
Hér getum við stytt út \(x\)\sphinxhyphen{}in og fáum að
\begin{equation*}
\begin{split}\lim_{x\to \infty} \frac{1}{2} = \frac{1}{2}\end{split}
\end{equation*}
\noindent{\hspace*{\fill}\sphinxincludegraphics{{tutor}.svg}\hspace*{\fill}}
\end{sphinxadmonition}


\section{Samfelld föll}
\label{\detokenize{Kafli10:samfelld-foll}}
Óformlega má segja að fall \(f\) sé \textit{samfellt} ef að hægt er að teikna feril þess á blað með blýanti án þess að þurfa að lyfta blýantinum. Ef að fall er ekki samfellt segjum við að það sé \textit{ósamfellt}.
Ef að fall \(f\) er ósamfellt þá segjum við að það sé ósamfellt í þeim punktum þar sem lyfta þarf blýantinum af blaðinu.

\begin{figure}[htbp]
\centering

\noindent\sphinxincludegraphics{{samfell}.svg}
\end{figure}

Fallið að ofan er \sphinxstylestrong{ósamfellt}.

\begin{figure}[htbp]
\centering

\noindent\sphinxincludegraphics{{samfell2}.svg}
\end{figure}

Fallið að ofan er \sphinxstylestrong{samfellt}.

Skilgreinum nú samfelldni:


\subsection{Skilgreining}
\label{\detokenize{Kafli10:id1}}
Látum \(f\) vera fall. Ef \(f\) hefur markgildi í \(a\) og \(\lim_{x \to a} f(x)=f(a)\) þá segjum við að \(f\) sé \sphinxstyleemphasis{samfellt} í punktinum \(a\).

Ef \(f\) er samfellt á öllu \textit{skilgreiningarmengi} sínu köllum við \(f\) \sphinxstyleemphasis{samfellt fall}.


\subsection{Reglur um samfelldni}
\label{\detokenize{Kafli10:reglur-um-samfelldni}}
Látum \(f\) og \(g\) vera raunföll á bili \(I\) og samfelld í punktinum \(a \in I\), þá gildir:
\begin{enumerate}
\sphinxsetlistlabels{\arabic}{enumi}{enumii}{}{.}%
\item {} 
Fallið \(f+g\) er samfellt í \(a\).

\item {} 
Fallið \(f \cdot g\) er samfellt í \(a\).

\item {} 
Fallið \(\frac{f}{g}\) er samfellt í \(a\) ef \(g(a) \neq 0\).

\item {} 
Gerum ráð fyrir að fallið \(g\) sé samfellt í \(f(a)\). Þá er \((g \circ f)(x)=g(f(x))\) samfellt í \(a\).

\end{enumerate}


\subsection{Nokkur þekkt samfelld föll}
\label{\detokenize{Kafli10:nokkur-ekkt-samfelld-foll}}
Áður en við skoðum dæmi þá skulum við telja upp nokkur samfelld föll sem við þekkjum:
\begin{enumerate}
\sphinxsetlistlabels{\arabic}{enumi}{enumii}{}{.}%
\item {} 
Fallið \(f(x)=x\) er samfellt.

\item {} 
Sérhver margliða er samfelld.

\item {} 
Sérhvert rætt fall er samfellt á skilgreiningarmengi sínu (þ.e. þar sem nefnarinn er ekki núll).

\item {} 
Fyrir allar náttúrlegar tölur \(n\) þá er fallið \(f(x)=\root n \of {x}\) samfellt.

\item {} 
Ef \(r \in \mathbb{R}\) er rauntala þá er fallið \(f(x)=x^r\) samfellt.

\item {} 
Vísisföll eru samfelld.

\item {} 
Lograr eru samfelldir.

\item {} 
Hornaföllin eru samfelld.

\item {} 
Fallið \(f(x)=|x|\) er samfellt.

\end{enumerate}

\begin{sphinxadmonition}{tip}{Dæmi:}
Segjum til um hvort föllin eru samfelld eða ekki.

\sphinxstylestrong{1.}
\begin{quote}
\begin{equation*}
\begin{split}f(x)=|x|+\cos(x^3)\end{split}
\end{equation*}
Fallið \(f\) er samsett úr föllunum \(|x|\), \(\cos(x)\) og \(x^3\) sem eru öll samfelld föll. Þess vegna er \(f\) samfellt.
\end{quote}

\sphinxstylestrong{2.}
\begin{quote}
\begin{equation*}
\begin{split}g(x) =
\begin{cases}
        \hfill -1    \hfill & \text{ ef } x\leq 0,\\
        \hfill  1 \hfill & \text{ ef } x>0.\\
\end{cases}\end{split}
\end{equation*}
Í punktinum \(x=0\) tekur fallið \(g\) stökk frá því að vera jafnt mínus einum í það að vera jafnt einum. Fallið er þess vegna ósamfellt í þeim punkti.
\end{quote}

\sphinxstylestrong{3.}
\begin{quote}
\begin{equation*}
\begin{split}h(x) =
\begin{cases}
        \hfill \sin(x)    \hfill & \text{ ef } x\leq 0,\\
        \hfill  x^2 \hfill & \text{ ef } x>0.\\
\end{cases}\end{split}
\end{equation*}
Föllin \(\sin(x)\) og \(x^2\) eru bæði samfelld. Fallið \(h\) er því samfellt í öllum punktum nema kannski núllpunktinum.
Nú er þekkt að \(0^2=0\) og \(\sin(0)=0\).
Fallið \(h\) stefnir þá á töluna núll í \(x=0\) hvort sem að við nálgumst punktinn hægra eða vinstra megin frá. Fallið \(h\) er þvi samfellt í núllpunktinum, og við höfum þá rökstutt að það er samfellt allstaðar.
\end{quote}
\end{sphinxadmonition}


\chapter{Diffrun}
\label{\detokenize{Kafli11:diffrun}}\label{\detokenize{Kafli11::doc}}

\section{Skýringardæmi}
\label{\detokenize{Kafli11:skyringardaemi}}
Áður en við skilgreinum \textit{afleiðu} falls skulum við taka dæmi til þess að reyna að útskýra hver hugmyndin er.

Við skulum hugsa okkur bíl í spyrnukeppni á 500 metra braut.
Áður en bíllinn fer af stað er sett í hann nákvæmt staðsetningartæki sem getur teiknað upp graf sem lýsir staðsetningu hans á brautinni miðað við tímann sem er liðinn frá upphafi spyrnunnar.

Bíllinn fer af stað og keyrir brautina á nákvæmlega tíu sekúndum.
Eftir keppnina er staðsetningartækið skoðað og það sýnir að staðsetningu bílsins sem fall af tíma megi lýsa með formúlunni:
\begin{equation*}
\begin{split}s(t) = 5t^2\end{split}
\end{equation*}
\noindent{\hspace*{\fill}\sphinxincludegraphics{{bill}.svg}\hspace*{\fill}}

þar sem \(t \in [0, 10]\) táknar tímann frá upphafi spyrnunnar mældan í sekúndum og \(s(t)\) er vegalengdin sem bíllinn var þá búinn að keyra mæld í metrum.

Þessi formúla segir okkur til dæmis að eftir fjórar sekúndur var bíllinn búinn að keyra 80 metra því \(s(4) = 5 \cdot 4^2 = 80\); og eftir átta sekúndur var bíllinn búinn að keyra \(320\) metra þar eð \(s(8) = 5\cdot 8^2 =320\).

Eins og við má búast er \(s(0) = 0\) sem þýðir að eftir núll sekúndur stóð bíllinn enn óhreyfður, og \(s(10) = 500\), í samræmi við að bíllinn hafi keyrt brautina sem var \(500\) metrar á tíu sekúndum.

Úr formúlunni er þó hægt að fá meiri upplýsingar heldur en bara staðsetningu bílsins.
Þessi formúla dugar líka til þess að reikna út \sphinxstylestrong{hraða bílsins á sérhverjum tímapunkti}.
Skoðum hvernig það er gert og prófum að reikna nákvæman hraða bílsins að fjórum sekúndum liðnum í spyrnunni.

Athugum að \sphinxstyleemphasis{meðalhraði} bílsins yfir ákveðið tímabil er skilgreindur sem vegalengdin sem bíllinn fer á því tímabili deilt með lengdinni á tímabilinu.
Þannig var til dæmis meðalhraði bílsins í allri spyrnunni \(50\) metrar á sekúndu því hann fór \(500\) metra á \(10\) sekúndum og \(\frac{500}{10} = 50\).

Áður en við getum reiknað út nákvæman hraða bílsins á tímanum \(t = 4\) skulum við skoða meðalhraða bílsins á nokkrum tímabilum.
\begin{itemize}
\item {} 
Skoðum fyrst meðalhraða bílsins á tímabilinu frá fjórum upp í átta sekúndur. Á sekúndu átta er bíllinn kominn \(5 \cdot 8^2 = 320\) metra.
Á sekúndu fjögur er bíllinn kominn \(5 \cdot 4^2 = 80\) metra.
Vegalengdin sem hann færist á tímabilinu frá fjórum til átta sekúndur er þá \(320 − 80 = 240\) metrar.
Lengdin á þessu tímabili er fjórar sekúndur. Meðalhraði bílsins á tímabilinu frá fjórum upp í átta sekúndur er þá:

\end{itemize}
\begin{quote}
\begin{equation*}
\begin{split}\frac{230}{4} = 57.5 \text{ m/s}\end{split}
\end{equation*}
\noindent{\hspace*{\fill}\sphinxincludegraphics{{bill1}.svg}\hspace*{\fill}}

Hér er hallatala línunar á milli \((4, s(4) = 80)\) og \((8, s(8) = 320)\) jöfn \(57.5\).
\end{quote}
\begin{itemize}
\item {} 
Skoðum næst meðalhraða bílsins frá fjórum upp í sex sekúndur.
Vegalengdin sem hann fer á því tímabili er \(5 \cdot 6^2 −5 \cdot 4^2 = 100\) metrar.
Lengdin á tímabilinu er \(6 − 4 = 2\) sekúndur. Meðalhraði bílsins er þá:

\end{itemize}
\begin{quote}
\begin{equation*}
\begin{split}\frac{100}{2} = 50 \text{ m/s}.\end{split}
\end{equation*}
\noindent{\hspace*{\fill}\sphinxincludegraphics{{bill2}.svg}\hspace*{\fill}}

Hér er hallatala línunnar á milli \((4, s(4) = 80)\) og \((6, s(6) = 180)\) jöfn \(50\).
\end{quote}
\begin{itemize}
\item {} 
Skoðum meðalhraða bílsins frá fjórum upp í fimm sekúndur. Hann er:

\end{itemize}
\begin{quote}
\begin{equation*}
\begin{split}\frac{5 \cdot 5^2 - 5 \cdot 4^2}{5 -4} = \frac{45}{1} = 45 \text{ m/s}.\end{split}
\end{equation*}
\noindent{\hspace*{\fill}\sphinxincludegraphics{{bill3}.svg}\hspace*{\fill}}

Hér er hallatala línunar á milli \((4, s(4) = 80)\) og \((5, s(5) = 125)\) jöfn \(45\).
\end{quote}
\begin{itemize}
\item {} 
Meðalhraði bílsins frá fjórum upp í fjórar og hálfa sekúndu er:

\end{itemize}
\begin{quote}
\begin{equation*}
\begin{split}\frac{5 \cdot4.5^2 - 5 \cdot 4^2}{ 4.5-4} = \frac{21.25}{0.5} = 42.5 \text{ m/s}.\end{split}
\end{equation*}\end{quote}
\begin{itemize}
\item {} 
Meðalhraði bílsins frá fjórum upp í \(4.25\) sekúndu er:

\end{itemize}
\begin{quote}
\begin{equation*}
\begin{split}\frac{5 \cdot4.25^2 - 5 \cdot 4^2}{ 4.25-4} = \frac{10.3125}{0.25} = 41.25 \text{ m/s}.\end{split}
\end{equation*}\end{quote}

Nú eruð þið kannski farin að átta ykkur á hvað er að fara að gerast.
Með því að stytta tímabilið meira og meira þá verður meðalhraðinn nær og nær raunverulega hraðanum í tímanum \(t = 4\).

Þetta er farið að líkjast því að taka \textit{markgildi}.
Almennt gildir að meðalhraði bílsins frá fjórum sekúndum upp í \(t\) sekúndur er:
\begin{equation*}
\begin{split}\frac{5t^2−5 \cdot 4^2}{ t−4}\end{split}
\end{equation*}
Þegar við styttum tímabilið meir og meir þá erum við í raun að láta töluna hér að ofan stefna á fjóra.

Þess vegna fáum við að nákvæmur hraði bílsins á tímanum \(t = 4\) er:
\begin{equation*}
\begin{split}\begin{aligned}
\lim_{t \to 4} \frac{5t^2−5 \cdot 4^2}{ t−4} & = \lim_{t \to 4} \frac{5(t^2−4^2)}{ t−4} \\
& = \lim_{t \to 4} \frac{5(t+4)(t−4)}{t−4} \\
& = \lim_{t \to 4} \frac{5(t+4)}{1} \\
& = 5(4+4) = 40
\end{aligned}\end{split}
\end{equation*}
Það er að segja, nákvæmur hraði bílsins þegar akkúrat fjórar sekúndur eru liðnar er \(40\) metrar á sekúndu.

Hér fór mikið púður í að reikna út hraða bílsins á einum tímapunkti.
Segjum að við viljum reikna út hraða bílsins í hundrað mismunandi tímapunktum, þá yrði þreytandi að þurfa að reikna út hundrað mismunandi markgildi.
Skynsamlegra væri að reyna að finna út nákvæma formúlu sem gefur upp hraða bílsins á sérhverjum tíma. Það er hægt!

Til þess að reikna út hraða bílsins á tímanum \(t=4\) þá þurftum við að reikna út markgildið:
\begin{equation*}
\begin{split}\lim_{t\to 4}\frac{5t^2-5\cdot 4^2}{t-4}\end{split}
\end{equation*}
Ef við viljum reikna út hraða bílsins á einhverjum öðrum tíma \(t=t_0\) þá reiknum við markgildið:
\begin{equation*}
\begin{split}\lim_{t\to t_0}\frac{5t^2-5\cdot t_0^2}{t-t_0}.\end{split}
\end{equation*}
En þetta er eitthvað sem að við getum reiknað út almennt fyrir hvert fall sem lýsir staðsetningu:
\begin{equation*}
\begin{split}\begin{aligned}
\lim_{t\to t_0}\frac{5t^2-5\cdot t_0^2}{t-t_0} &=\lim_{t\to t_0}\frac{5(t^2-t_0^2)}{t-t_0} \\
&=\lim_{t\to t_0}\frac{5(t+t_0)(t-t_0)}{t-t_0}\\
&=\lim_{t\to t_0}\frac{5(t+t_0)}{1} \\
&=\frac{5(t_0+t_0)}{1}=10 t_0. \\
\end{aligned}\end{split}
\end{equation*}
Við höfum séð að á tímanum \(t=t_0\) þá er hraði bílsins \(10t_0\) m/s
Með öðrum orðum þá má lýsa hraða bílsins af tíma með fallinu sem gefið er með formúlunni
\begin{equation*}
\begin{split}v(t)=10t.\end{split}
\end{equation*}
Af þessu má lesa að hraði bílsins á sekúndu fjögur er \(4\cdot 10=40\) m/s og hraði bílsins á sekúndu átta er \(8\cdot 10=80\) m/s.
Munum að bíllinn byrjar kyrrstæður svo það kemur ekki á óvart að \(v(0)=0\) sem má túlka sem svo að hraði bílsins á sekúndunni núll sé núll.
\begin{quote}

\noindent{\hspace*{\fill}\sphinxincludegraphics{{bill4}.svg}\hspace*{\fill}}
\end{quote}

Hér sjáum við línu sem er með hallatöluna \(40\) í punktinum \((4, s(4) = 80)\) sem gefur okkur línuna \(y = 40x -80\).

Aðferðin sem hér var notuð kallast  \textit{diffrun} (deildun).
Þessi aðferð er ekki bundin við þetta einstaka fall, heldur má gera þetta almennt. Segjum að hraði bílsins hafi verið gefinn með einhverju öðru falli \(k(t)\).
Hraði bílsins í tímapunktinum \(t=t_0\) verður þá fundinn með því að reikna markgildið
\begin{equation*}
\begin{split}\lim_{t\to t_0}\frac{k(t)-k(t_0)}{t-t_0}.\end{split}
\end{equation*}

\section{Skilgreining}
\label{\detokenize{Kafli11:skilgreining}}
Gerum ráð fyrir að \(f:\;I\to \mathbb{R}\) sé fall sem er skilgreint á bili \(I\).
Látum \(a\in I\).
Fallið \(f\) er sagt vera \textit{diffranlegt} (deildanlegt) í punktinum \(a\) ef að markgildið
\begin{equation*}
\begin{split}\lim_{x\to a}\frac{f(x)-f(a)}{x-a}\end{split}
\end{equation*}
er skilgreint og jafnt einhverri rauntölu (\sphinxstyleemphasis{ekki plús eða mínus óendanlegt}).
Þessi rauntala er táknuð með \(f'(a)\) og kallast afleiða fallsins \(f\) í punktinum \(a\).

Þegar afleiða fallsins \(f\) er reiknuð í ótilteknum punkti getur verið þægilegra að notast við umritaða skilgreiningu:
\begin{equation*}
\begin{split}f'(x)=\lim_{h\to 0}\frac{f(x+h)-f(x)}{h}\end{split}
\end{equation*}
Ef fallið \(f\) er deildanlegt í sérhverjum punkti bilsins \(I\) þá segjum við að \(f\) sé diffranlegt (deildanlegt) fall á \(I\) og þá er afleiðan \(f'\) fall á \(I\).
Aðgerðin að finna afleiðu falls kallast \textit{diffrun} (deildun) falls og yfirleitt er talað um sögnina að \textit{diffra} (deilda).

\begin{sphinxadmonition}{tip}{Dæmi:}
Notið skilgreininguna á afleiðu til að reikna afleiðu fallanna.
\begin{description}
\item[{\sphinxstylestrong{1.} \(f(x) = 2x^2-16x+5\)}] \leavevmode
Notum skilgreininguna á afleiðu \(f'(x) = \lim_{h\to 0}\frac{f(x+h)-f(a)}{h}\) sem gefur
\begin{equation*}
\begin{split}f'(x) = \lim_{h\to 0} \frac{\left(2(x+h)^2 - 16(x+h) +5\right)-\left(2x^2 - 16x + 5\right)}{h}\end{split}
\end{equation*}
Sjáum að við þurfum að umrita til að fá markgildi sem við getum reiknað, því við getum ekki sett \(h= 0\) strax.
\begin{equation*}
\begin{split}\begin{aligned}
f'(x) &= \lim_{h\to 0} \frac{\left(2(x+h)^2 - 16(x+h) +5\right)-\left(2x^2 - 16x + 5\right)}{h}\\
&=\lim_{h\to 0} \frac{2x^2+4xh + 2h^2 - 16x -16h +5 -2x^2 + 16x-5}{h}\\
&= \lim_{h\to 0} \frac{4xh + 2h^2 - 16h}{h}
\end{aligned}\end{split}
\end{equation*}
Hér getum við tekið \(h\) út fyrir sviga og stytt það út:
\begin{equation*}
\begin{split}\begin{aligned}
f'(x) &= \lim_{h\to 0} \frac{4xh + 2h^2 - 16h}{h}\\
&=\lim_{h\to 0} \frac{h(4x + 2h- 16)}{h}\\
&=\lim_{h\to 0} 4x +2h - 16\\
&= 4x + 2(0) -16\\
&= 4x - 16
\end{aligned}\end{split}
\end{equation*}
Þá er afleiðan \(f'(x) = 4x-16\) .

\item[{\sphinxstylestrong{2.} \(g(x) = \frac{x}{x+1}\)}] \leavevmode
Notum skilgreininguna á afleiðu \(g'(x) = \lim_{h\to 0}\frac{g(x+h)-g(a)}{h}\) sem gefur
\begin{equation*}
\begin{split}g'(x) = \lim_{h\to 0} \frac{1}{h} \left(\frac{x+h}{x+h+1} - \frac{x}{x+1}\right)\end{split}
\end{equation*}
Hér þurfum við líka að umrita til að geta sett \(h=0\)
\begin{equation*}
\begin{split}\begin{aligned}
g'(x) &= \lim_{h\to 0} \frac{1}{h} \cdot \left(\frac{x+h}{x+h+1} - \frac{x}{x+1}\right) \\
&= \lim_{h\to 0} \frac{1}{h} \cdot \left(\frac{(x+h)(x+1)- x(x+h+1)}{(x+h+1)(x+1)}\right) \\
&= \lim_{h\to 0} \frac{1}{h} \cdot \left(\frac{x^2 +x +xh+h -(x^2+xh+x)}{(x+h+1)(x+1)}\right) \\
&= \lim_{h\to 0} \frac{1}{h} \cdot \left(\frac{h}{(x+h+1)(x+1)}\right) \\
&= \lim_{h\to 0} \frac{1}{(x+h+1)(x+1)} \\
&= \frac{1}{(x+1)(x+1)}\\
&= \frac{1}{(x+1)^2}
\end{aligned}\end{split}
\end{equation*}
Þá er afleiðan \(g'(x)= \frac{1}{(x+1)^2}\).

\end{description}
\end{sphinxadmonition}

\begin{sphinxadmonition}{warning}{Aðvörun:}
\sphinxstylestrong{Ritháttur}:
Takið eftir að
\begin{quote}
\begin{equation*}
\begin{split}f'(x), \qquad f', \qquad \frac{df}{dx}, \qquad \frac{d}{dx} f(x), \text{ og } \qquad D_x f\end{split}
\end{equation*}\end{quote}

eru mismunandi rithættir fyrir „afleiða \(f(x)\) m.t.t. \(x\).“
\end{sphinxadmonition}


\section{Reiknireglur}
\label{\detokenize{Kafli11:reiknireglur}}
Gerum ráð fyrir að \(f,g\) séu deildanleg föll á \(\mathbb{R}\).

Látum \(a\in \mathbb{R}\) vera fasta.

Þá gildir:
\begin{equation*}
\begin{split}\begin{aligned}
1.& \quad (a\cdot f)'=af' \\
&\\
2.& \quad (f+g)'=f'+g' \\
&\\
3.& \quad (f-g)'=f'-g' \\
&\\
4.& \quad (f\cdot g)'=f'g+fg' \\
&\\
4.& \quad (f\circ g)'=(f'\circ g)\cdot g' \\
\end{aligned}\end{split}
\end{equation*}
Ef \(g(x)\) er ekki jafnt núlli fyrir öll \(x\in I\), þá gildir einnig:
\begin{equation*}
\begin{split}\begin{aligned}
6.& \quad \left(\frac{1}{g}\right)'=\frac{-g'}{g^2} \\
&\\
7.& \quad \left(\frac{f}{g}\right)'=\frac{f'g-fg'}{g^2} \\
\end{aligned}\end{split}
\end{equation*}
Ef \(f\) er andhverfanlegt og \(f(x_0)=y_0\) þá er
\begin{equation*}
\begin{split}8. (f^{-1})'\circ f=\frac{1}{f'}\end{split}
\end{equation*}

\section{Þekktar afleiður}
\label{\detokenize{Kafli11:ekktar-afleiur}}\begin{enumerate}
\sphinxsetlistlabels{\arabic}{enumi}{enumii}{}{.}%
\item {} 
Ef \(a\) er fasti og \(f(x)=a\) þá er

\end{enumerate}
\begin{equation*}
\begin{split}f'(x)=0\end{split}
\end{equation*}\begin{enumerate}
\sphinxsetlistlabels{\arabic}{enumi}{enumii}{}{.}%
\setcounter{enumi}{1}
\item {} 
Ef \(n\in \mathbb{N}\) og \(f(x)=x^n\) þá er

\end{enumerate}
\begin{equation*}
\begin{split}f'(x)=nx^{n-1}\end{split}
\end{equation*}\begin{enumerate}
\sphinxsetlistlabels{\arabic}{enumi}{enumii}{}{.}%
\setcounter{enumi}{2}
\item {} 
Ef \(n\in \mathbb{Z}\setminus\{0\}\) og \(f(x)=x^{n}\) þá er

\end{enumerate}
\begin{equation*}
\begin{split}f'(x)=nx^{n-1}\end{split}
\end{equation*}\begin{enumerate}
\sphinxsetlistlabels{\arabic}{enumi}{enumii}{}{.}%
\setcounter{enumi}{3}
\item {} 
Ef \(n\in \mathbb{Q}\setminus\{0\}\) og \(f(x)=x^n\) þá er

\end{enumerate}
\begin{equation*}
\begin{split}f'(x)=nx^{n-1}\end{split}
\end{equation*}\begin{enumerate}
\sphinxsetlistlabels{\arabic}{enumi}{enumii}{}{.}%
\setcounter{enumi}{4}
\item {} 
Ef \(n\in \mathbb{R}\setminus\{0\}\) og \(f(x)=x^n\) þá er

\end{enumerate}
\begin{equation*}
\begin{split}f'(x)=nx^{n-1}\end{split}
\end{equation*}\begin{enumerate}
\sphinxsetlistlabels{\arabic}{enumi}{enumii}{}{.}%
\setcounter{enumi}{5}
\item {} 
Ef \(a\in \mathbb{R}_+\) og \(f(x)=a^x\) þá er

\end{enumerate}
\begin{equation*}
\begin{split}f'(x)=\ln(a)a^x\end{split}
\end{equation*}\begin{enumerate}
\sphinxsetlistlabels{\arabic}{enumi}{enumii}{}{.}%
\setcounter{enumi}{6}
\item {} 
Ef \(a\in \mathbb{R}_+\) og \(f(x)=\log_a(x)\) þá er

\end{enumerate}
\begin{equation*}
\begin{split}f'(x)=\frac{1}{\ln(a)x}\end{split}
\end{equation*}\begin{enumerate}
\sphinxsetlistlabels{\arabic}{enumi}{enumii}{}{.}%
\setcounter{enumi}{7}
\item {} 
Ef \(f(x) = \ln(x)\) þá er

\end{enumerate}
\begin{equation*}
\begin{split}f'(x) = \frac{1}{x}\end{split}
\end{equation*}\begin{enumerate}
\sphinxsetlistlabels{\arabic}{enumi}{enumii}{}{.}%
\setcounter{enumi}{8}
\item {} 
Ef \(f(x) = e^x\) þá er

\end{enumerate}
\begin{equation*}
\begin{split}f'(x) = e^x\end{split}
\end{equation*}\begin{enumerate}
\sphinxsetlistlabels{\arabic}{enumi}{enumii}{}{.}%
\setcounter{enumi}{9}
\item {} 
Ef \(f(x)=\cos(x)\) þá er

\end{enumerate}
\begin{equation*}
\begin{split}f'(x)=-\sin(x)\end{split}
\end{equation*}\begin{enumerate}
\sphinxsetlistlabels{\arabic}{enumi}{enumii}{}{.}%
\setcounter{enumi}{10}
\item {} 
Ef \(f(x)=\sin(x)\) þá er

\end{enumerate}
\begin{equation*}
\begin{split}f'(x)=\cos(x)\end{split}
\end{equation*}\begin{enumerate}
\sphinxsetlistlabels{\arabic}{enumi}{enumii}{}{.}%
\setcounter{enumi}{11}
\item {} 
Ef \(f(x)=\tan(x)\) þá er

\end{enumerate}
\begin{equation*}
\begin{split}f'(x)=\frac{1}{\cos^2(x)}\end{split}
\end{equation*}\begin{enumerate}
\sphinxsetlistlabels{\arabic}{enumi}{enumii}{}{.}%
\setcounter{enumi}{12}
\item {} 
Ef \(f(x)=\cot(x)\) þá er

\end{enumerate}
\begin{equation*}
\begin{split}f'(x)=\frac{-1}{\sin^2(x)}\end{split}
\end{equation*}\begin{enumerate}
\sphinxsetlistlabels{\arabic}{enumi}{enumii}{}{.}%
\setcounter{enumi}{13}
\item {} 
Ef \(f(x)=\text{arcsin(x)}\) þá er

\end{enumerate}
\begin{equation*}
\begin{split}f'(x)=\frac{1}{\sqrt{1-x^2}}\end{split}
\end{equation*}\begin{enumerate}
\sphinxsetlistlabels{\arabic}{enumi}{enumii}{}{.}%
\setcounter{enumi}{14}
\item {} 
Ef \(f(x)=\text{arccos(x)}\) þá er

\end{enumerate}
\begin{equation*}
\begin{split}f'(x)=\frac{-1}{\sqrt{1-x^2}}\end{split}
\end{equation*}\begin{enumerate}
\sphinxsetlistlabels{\arabic}{enumi}{enumii}{}{.}%
\setcounter{enumi}{15}
\item {} 
Ef \(f(x)=\text{arctan(x)}\) þá er

\end{enumerate}
\begin{equation*}
\begin{split}f'(x)=\frac{1}{1+x^2}\end{split}
\end{equation*}\begin{enumerate}
\sphinxsetlistlabels{\arabic}{enumi}{enumii}{}{.}%
\setcounter{enumi}{16}
\item {} 
Ef \((x)=\text{arccot(x)}\) þá er

\end{enumerate}
\begin{equation*}
\begin{split}f'(x)=\frac{-1}{1+x^2}\end{split}
\end{equation*}
\begin{sphinxadmonition}{tip}{Dæmi:}
Notum okkur nú reiknireglurnar og þekktar afleiður til að reikna eftirfarandi afleiður.

\sphinxstylestrong{1.} \(f(x) = 2x^2-16x+5\)
\begin{quote}

Við vitum út frá reiknireglu 2, \((f+g)'=f'+g'\) sem þýðir að við getum horft á hvern lið sér og svo lagt þá saman að lokum.
Byrjum á að nota okkur að \(f(x)=x^n\) gefur \(f'(x)=nx^{n-1}\) og reiknireglu 1, \((a\cdot f)'=af'\)
\begin{itemize}
\item {} 
þá er afleiðan af \(2(x^2)\) jöfn \(2(2x^{2-1}) = 4x\)

\item {} 
þá er afleiðan af \(16x\) jöfn \(16(1x^{1-1} )= 16\)

\item {} 
munum líka að afleiða fastafalls er jafnt og núll, þ.e.a.s. afleiða \(5\) er \(0\)

\end{itemize}

Leggjum alla liðina saman og fáum \(f'(x) = 4x + 16 + 0\)

Þá er afleiðan
\begin{equation*}
\begin{split}f'(x) = 4x + 16\end{split}
\end{equation*}\end{quote}

\sphinxstylestrong{2.} \(h(x) = \frac{x}{x+1}\)
\begin{quote}

Hér þurfum við að nota reiknireglu 7. \(\left(\frac{f}{g}\right)'=\frac{f'g-fg'}{g^2}\)
\begin{itemize}
\item {} 
nefnarinn  gefur okkur \(1\) þar sem afleiða \(x\) er \(1\)

\item {} 
teljarinn gefur okkur líka \(1\) þar sem afleiða \(x\) er \(1\) og afleiða \(1\) er \(0\)

\end{itemize}

Setjum þessar niðurstöður inn í reikniregluna (í þessu tilfelli er \(f = x\), \(g = x+1\), \(f' = 1\) og \(g' = 1\))
\begin{equation*}
\begin{split}\begin{aligned}
   \left(\frac{f}{g}\right)' &=\frac{f'g-fg'}{g^2}\\
   (\frac{x}{x+1})' &= \frac{(1) \cdot (x+1) - (x) \cdot (1)}{(x+1)^2}\\
   &= \frac{(x+1)- x}{(x+1)^2} \\
   &= \frac{1}{(x+1)^2}
\end{aligned}\end{split}
\end{equation*}
Þá er
\begin{equation*}
\begin{split}h'(x)= \frac{1}{(x+1)^2}.\end{split}
\end{equation*}\end{quote}

\sphinxstylestrong{3.} \(g(x) = \cos(\ln(x))\)
\begin{quote}

Hér notum við reiknireglu 5, \((f\circ g)'=(f'\circ g)\cdot g'\).
Notum okkur líka þekktu afleiðurnar \(f(x)=\cos(x)\) þá er \(f'(x)=-\sin(x)\) og ef \(f(x) = \ln(x)\) þá er \(f'(x) = \frac{1}{x}\). Fáum
\begin{equation*}
\begin{split}g'(x) = -\sin(\ln(x)) \cdot \frac{1}{x} = \frac{-\sin(\ln(x))}{x}.\end{split}
\end{equation*}\end{quote}

\sphinxstylestrong{4.} \(k(x) = 3^x \cdot \sin(x^5)\)
\begin{quote}

Hér þurfum við að nota tvær reiknireglur 5, \((f\circ g)'=(f'\circ g)\cdot g'\) og 4, \((f\cdot g)'=f'g+fg'\). Notum okkur líka þekktu afleiðurnar \(f(x)=\sin(x)\) þá er \(f'(x)=\cos(x)\) og \(f(x)=a^x\) þá er \(f'(x)=\ln(a)a^x\). Fáum
\begin{equation*}
\begin{split}g'(x) = ln(3) \cdot 3^x \cdot \sin(x^5) + 3^x \cdot cos(x^5) \cdot 5x^4.\end{split}
\end{equation*}\end{quote}
\end{sphinxadmonition}


\chapter{Viðauki}
\label{\detokenize{Kafli12:viauki}}\label{\detokenize{Kafli12::doc}}

\section{Þrepun}
\label{\detokenize{Kafli12:repun}}
Þrepun er leið til þess að sanna setningar og formúlur.
Þá er fyrst sýnt fram á að setningin gildi um grunntilvik (upphafstilvik, núlltilvik).
Að því loknu er sýnt fram á að ef setningin gildir um eitthvert ótiltekið tilvik, þá gildir það um það næsta.

Þetta svipar til þess að ganga upp stiga: ef við getum komist upp fyrsta þrepið og frá hverju þrepi í það næsta, þá hljótum við að geta gengið upp allan stigann.

\begin{figure}[htbp]
\centering

\noindent\sphinxincludegraphics[width=0.700\linewidth]{{threpun}.svg}
\end{figure}

Hér er til dæmis reglan um að summa allra heiltalna frá einum upp í \(n\) er:
\begin{equation*}
\begin{split}\sum_{i=1}^n = \frac{n(n+1)}{2}\end{split}
\end{equation*}
Við ætlum að sýna fram á að þessi formúla gildi um allar jákvæðar heiltölur \(n=1,2,3, \dots\) .
Sjáum til dæmis að hún gildir fyrir \(n=4\) því
\begin{equation*}
\begin{split}\sum_{i=1}^4 = 1+2+3+4 = 10\end{split}
\end{equation*}
og
\begin{equation*}
\begin{split}\frac{4\cdot(4+1)}{2} = \frac{20}{2} = 10\end{split}
\end{equation*}
Sönnum setninguna með þrepun:

\sphinxstylestrong{Grunntilvik:}

Prófum að setja \(n=1\) í setninguna:
\begin{equation*}
\begin{split}\begin{aligned}
  \sum_{i=1}^n &= \frac{n(n+1)}{2}\\
  \sum_{i=1}^1 &= \frac{1(1+1)}{2}\\
  1&=\frac{1\cdot2}{2}=1
\end{aligned}\end{split}
\end{equation*}
Formúlan virkar því til þess að reikna summu allra talna frá einum upp í einn.

\sphinxstylestrong{Þrepun:}

Gerum núna ráð fyrir að formúlan virki fyrir eitthvert \(n\) , þ.e. að
\begin{equation*}
\begin{split}\sum_{i=1}^n = \frac{n(n+1)}{2}\end{split}
\end{equation*}
sé sönn staðhæfing.

Þá ætlum við að sýna að hún gildi líka fyrir næsta skref \(n+1\) .
\begin{equation*}
\begin{split}\begin{aligned}
  \sum_{i=1}^{n+1} &= \sum_{i=1}^n + (n+1) \\
  &= \frac{n(n+1)}{2} + (n+1) \\
  &= \frac{n(n+1)}{2} + \frac{2(n+1)}{2} \\
  &=\frac{n(n+1)+2(n+1)}{2}\\
  &= \frac{(n+1)(n+2)}{2} \\
\end{aligned}\end{split}
\end{equation*}
Í síðustu línunni birtist formúlan eins og hún væri ef \(n+1\) væri sett inn í formúluna og því gildir formúlan líka fyrir \(n+1\) .

Því getum við ályktað að setningin gildi um öll \(n\) .


\section{Ritháttur}
\label{\detokenize{Kafli12:rithattur}}\begin{itemize}
\item {} \begin{description}
\item[{\(\mathbb{N} \quad\) „Náttúrulegu tölurnar“}] \leavevmode\begin{itemize}
\item {} 
\(0,1,2,3, \dots\) köllum við náttúrulegu tölurnar

\end{itemize}

\end{description}

\item {} \begin{description}
\item[{\(\mathbb{Z} \quad\) „Heiltölur“}] \leavevmode\begin{itemize}
\item {} 
\(\dots ,-3, -2, -1, 0,1,2,3, \dots\) köllum við heiltölurnar

\end{itemize}

\end{description}

\item {} \begin{description}
\item[{\(\mathbb{Q} \quad\) „Ræðu tölurna“}] \leavevmode\begin{itemize}
\item {} 
\(\frac{p}{q}\) þar sem \(p\) og \(q\) eru heilar tölur og \(q \neq 0\)

\end{itemize}

\end{description}

\item {} \begin{description}
\item[{\(\mathbb{R} \quad\) „Rauntölurnar“}] \leavevmode\begin{itemize}
\item {} 
mengi allra ræðra talna, auk óræðra talna

\end{itemize}

\end{description}

\item {} \begin{description}
\item[{\(:= \; \text{eða} \; \equiv \quad\) „skilgreint sem“}] \leavevmode\begin{itemize}
\item {} 
notað til að skilgreina stærðir

\end{itemize}

\end{description}

\item {} \begin{description}
\item[{\(\in \quad\) „stak í“}] \leavevmode\begin{itemize}
\item {} 
oft erum við að tala um stak í menginu \(x \in A\) \(x\) er stak í menginu \(A\)

\end{itemize}

\end{description}

\item {} \begin{description}
\item[{\(\notin \quad\) „ekki stak í“}] \leavevmode\begin{itemize}
\item {} 
\(x \notin A\) \(x\) er ekki stak í menginu \(A\)

\end{itemize}

\end{description}

\item {} \begin{description}
\item[{\(\forall \quad\) „fyrir öll“}] \leavevmode\begin{itemize}
\item {} 
\(\forall n \in \mathbb{N}\) fyrir öll stök í mengi náttúrulegra talna

\end{itemize}

\end{description}

\item {} \begin{description}
\item[{\(\approx \quad\) „um það bil“}] \leavevmode\begin{itemize}
\item {} 
\(\pi \approx 3.14\)

\end{itemize}

\end{description}

\item {} \begin{description}
\item[{\(< \qquad\)  „minna en“}] \leavevmode\begin{itemize}
\item {} 
\(3<4\)

\end{itemize}

\end{description}

\item {} \begin{description}
\item[{\(> \qquad\) „stærra en“}] \leavevmode\begin{itemize}
\item {} 
\(6>2\)

\end{itemize}

\end{description}

\item {} \begin{description}
\item[{\(\leq \qquad\) „minna en eða jafnt og“}] \leavevmode\begin{itemize}
\item {} 
\(a \leq x \leq b\) lokað bil

\end{itemize}

\end{description}

\item {} \begin{description}
\item[{\(\geq \qquad\) „stærra en eða jafnt og“}] \leavevmode\begin{itemize}
\item {} 
\(a \geq x \geq b\) opið bil

\end{itemize}

\end{description}

\item {} \begin{description}
\item[{\(\sum_{n}^{m} \quad\) „summa frá n upp í m“}] \leavevmode\begin{itemize}
\item {} 
við getum haf endanlegar og óendanlegar summur, þ.e.a.s. \(m= \infty\) eða/og  \(n= -\infty\)

\end{itemize}

\end{description}

\item {} \begin{description}
\item[{\(\prod_{n}^{m} \quad\) „margfeldi frá n upp í m“}] \leavevmode\begin{itemize}
\item {} 
við getum haft endanleg og óendanleg margfeldi, þ.e.a.s. \(m= \infty\) eða/og  \(n= -\infty\)

\end{itemize}

\end{description}

\item {} \begin{description}
\item[{\(\parallel \quad\) „samsíða“}] \leavevmode\begin{itemize}
\item {} 
sjá {\hyperref[\detokenize{Kafli03:s-samsia}]{\sphinxcrossref{\DUrole{std,std-ref}{kafla um samsíða}}}} línur

\end{itemize}

\end{description}

\item {} \begin{description}
\item[{\(\perp  \quad\) „hornrétt“ eða „þverstætt“}] \leavevmode\begin{itemize}
\item {} 
sjá {\hyperref[\detokenize{Kafli03:s-verstae}]{\sphinxcrossref{\DUrole{std,std-ref}{kafla um þverstæð}}}} línur

\end{itemize}

\end{description}

\item {} \begin{description}
\item[{\(\subset \quad\) „hlutmengi“}] \leavevmode\begin{itemize}
\item {} 
t.d. \(A \subset B\)  \(B\) er hlutmengi í \(A\)

\end{itemize}

\end{description}

\item {} \begin{description}
\item[{\(| \quad \text{eða} \quad ; \quad\) „þar sem“}] \leavevmode\begin{itemize}
\item {} 
oft þegar við erum að skilgreina mengi þá erum við með skilyrði \(p(x)\), t.d. \(A = \{x \in C \ | \ p(x)\}\). Þetta þýðir að í \(A\) eru öll stök í \(C\) \sphinxstylestrong{þar sem} \(p(x)\) gildir

\end{itemize}

\end{description}

\item {} \begin{description}
\item[{\(\cup \quad\) „sammengi“}] \leavevmode\begin{itemize}
\item {} 
\(A \cup B\) „sammengi \(A\) og \(B\)“

\end{itemize}

\end{description}

\item {} \begin{description}
\item[{\(\cap \quad\) „sniðmengi“}] \leavevmode\begin{itemize}
\item {} 
\(A \cap B\) „sniðmengi \(A\) og \(B\)“

\end{itemize}

\end{description}

\item {} \begin{description}
\item[{\(\setminus \quad\) „mengjamismunur“}] \leavevmode\begin{itemize}
\item {} 
\(A\setminus B\) „\(A\) án \(B\)“

\end{itemize}

\end{description}

\item {} \begin{description}
\item[{\(A^c \quad\) „fyllimengi \(A\)“}] \leavevmode\begin{itemize}
\item {} 
\(A^c \quad\) „allt sem er ekki í \(A\)“

\end{itemize}

\end{description}

\item {} \begin{description}
\item[{\((a,b) \quad \text{eða} \quad ]a,b[ \quad\) „opið bil“}] \leavevmode\begin{itemize}
\item {} 
\((a,b)=\{x\in \mathbb{R}; a<x<b\}\)

\end{itemize}

\end{description}

\item {} \begin{description}
\item[{\([a,b] \quad\) „lokað bil“}] \leavevmode\begin{itemize}
\item {} 
\([a,b]=\{x\in \mathbb{R}; a\leq x\leq b\}\)

\end{itemize}

\end{description}

\item {} \begin{description}
\item[{\(f(x)|_{x=a} \quad\) „stingum in \(x=a\)“}] \leavevmode\begin{itemize}
\item {} 
\(x^2+3x-1|_{x=3} = (3)^2+3(3)-1 = 17\)

\end{itemize}

\end{description}

\item {} \begin{description}
\item[{\(! \qquad\) „aðfeldi“ eða „hrópmerkt“}] \leavevmode\begin{itemize}
\item {} 
\(6! = 1 \cdot 2\cdot 3\cdot 4 \cdot 5 \cdot 6 =720\)

\end{itemize}

\end{description}

\item {} \begin{description}
\item[{\(°  \qquad\)“gráður“}] \leavevmode\begin{itemize}
\item {} 
\(30°\)

\end{itemize}

\end{description}

\item {} \begin{description}
\item[{\(Rad \quad\) „radian“ eða „bogaeining“}] \leavevmode\begin{itemize}
\item {} 
\(\frac{\pi}{6} Rad\)

\end{itemize}

\end{description}

\item {} \begin{description}
\item[{\(\boldsymbol{a} \quad \vec{a} \quad \bar{a}  \quad\) „vigurinn \(a\)“}] \leavevmode\begin{itemize}
\item {} 
sjá  {\hyperref[\detokenize{Kafli08:s-vigrar}]{\sphinxcrossref{\DUrole{std,std-ref}{Vigrar}}}}

\end{itemize}

\end{description}

\item {} \begin{description}
\item[{\(f'(x) \quad f' \quad \frac{df}{dx} \quad \frac{d}{dx} f(x) \quad D_x f \qquad\) „afleiða \(f(x)\)“}] \leavevmode\begin{itemize}
\item {} 
með tilliti til \(x\)

\end{itemize}

\end{description}

\end{itemize}


\section{Formúlublað}
\label{\detokenize{Kafli12:formulubla}}
Samantekt af hentugum reglum og formúlum úr köflunum.


\subsection{Algebra}
\label{\detokenize{Kafli12:algebra}}

\subsubsection{Einfaldar reiknireglur}
\label{\detokenize{Kafli12:einfaldar-reiknireglur}}\begin{equation*}
\begin{split}\begin{aligned}
    &(a+b)+c=a+(b+c)   \qquad &\textit{ (tengiregla samlagningar)}\\
\qquad \\
    &(ab)c=a(bc)  \qquad &\textit{ (tengiregla margföldunar)}\\
\qquad \\
    &a+b=b+a  \qquad &\textit{ (víxlregla samlagningar)} \\
\qquad \\
    &ab=ba  \qquad &\textit{ (víxlregla margföldunar)}\\
\qquad \\
    &a(b+c)=ab+ac  \qquad &\textit{ (dreifiregla)}\\
\qquad \\
    &1 \cdot a=a  \qquad &\textit{ (1 er margföldunarhlutleysa)}\\
\qquad \\
    &a+0=a  \qquad &\textit{ (0 er samlagningarhlutleysa)}\\
\qquad \\
    &0 \cdot a=0   \qquad &\textit{ (margföldun með núlli gefur núll)}\\
    \end{aligned}\end{split}
\end{equation*}

\subsubsection{Brotareiknireglur}
\label{\detokenize{Kafli12:brotareiknireglur}}\begin{equation*}
\begin{split}\begin{aligned}
& \frac{p}{q}+\frac{r}{s}=\frac{ps}{qs}+\frac{qr}{qs}=\frac{ps+qr}{qs} \\
\qquad \\
&\frac{p}{q}-\frac{r}{s}=\frac{ps-qr}{qs} \\
\qquad \\
& \frac{p}{q} \cdot \frac{r}{s}=\frac{pr}{qs} \\
\qquad \\
&\frac{p/q}{r/s} =\frac{ps}{qr} \\
\end{aligned}\end{split}
\end{equation*}

\subsubsection{Veldareiknireglur}
\label{\detokenize{Kafli12:veldareiknireglur}}\begin{equation*}
\begin{split}    \begin{aligned}
&a^0=1\\
\qquad \\
&a^n=a \cdot a \cdot \dots \cdot a \\
\qquad \\
    &a^{-n}=\frac{1}{a^n} \\
\qquad \\
&a^n\cdot a^m=a^{n+m}\\
\qquad \\
    &\dfrac {a^n}{a^m}=a^{n-m}\\
\qquad \\
    &a^n\cdot b^n=(ab)^n\\
\qquad \\
    &(a^n)^m=a^{nm}
\qquad \\
(-1)^n &= 1 \qquad \text{þegar } n \text{ er slétt tala} \\
\qquad \\
    (-1)^n &= -1 \qquad \text{þegar } n \text{ er oddatala} \\
    \end{aligned}\end{split}
\end{equation*}

\subsubsection{Reiknireglur fyrir rætur}
\label{\detokenize{Kafli12:reiknireglur-fyrir-raetur}}\begin{equation*}
\begin{split}      \begin{aligned}
        \sqrt[q]{ab} &=\sqrt[q]{a}\cdot \sqrt[q]{b} \\
  & \qquad \\
  \sqrt[q]{\dfrac ab}& =\dfrac{\sqrt[q]{a}}{\sqrt[q] {b}}\\
  & \qquad \\
  \sqrt[q]{a^p}& =(\sqrt[q]{a})^p\\
  & \qquad \\
  \sqrt[sq]{a^{sp}} &={\sqrt[q]{a^p}}\\
  & \qquad \\
  \sqrt[sq]{ a} &=\sqrt[s]{\sqrt[q]{a}}\\
\end{aligned}\end{split}
\end{equation*}

\subsection{Jöfnur og ójöfnur}
\label{\detokenize{Kafli12:jofnur-og-ojofnur}}

\subsubsection{Lausnarformúla annars stigs jöfnu}
\label{\detokenize{Kafli12:lausnarformula-annars-stigs-jofnu}}
Látum \(ax^2+bx+c=0\) vera annars stigs jöfnu.
\begin{enumerate}
\sphinxsetlistlabels{\arabic}{enumi}{enumii}{}{.}%
\item {} 
Ef \(d = b^2-4ac<0\) þá hefur jafnan enga rauntölulausn.

\item {} 
Ef \(d  = b^2-4ac=0\) þá hefur jafnan eina lausn:

\end{enumerate}
\begin{equation*}
\begin{split}x=\frac{-b}{2a}\end{split}
\end{equation*}\begin{enumerate}
\sphinxsetlistlabels{\arabic}{enumi}{enumii}{}{.}%
\setcounter{enumi}{2}
\item {} 
Ef \(d = b^2-4ac>0\) þá hefur jafnan tvær lausnir:

\end{enumerate}
\begin{equation*}
\begin{split}x_1=\frac{-b+\sqrt{b^2-4ac}}{2a} \qquad \text{og} \qquad x_2=\frac{-b-\sqrt{b^2-4ac}}{2a}\end{split}
\end{equation*}
eða
\begin{equation*}
\begin{split}x=\frac{-b\pm\sqrt{b^2-4ac}}{2a}\end{split}
\end{equation*}

\subsubsection{Reiknireglur fyrir tölugildi}
\label{\detokenize{Kafli12:reiknireglur-fyrir-tolugildi}}
Látum \(a\) og \(b\) vera rauntölur. Þá gildir eftirfarandi:
\begin{equation*}
\begin{split}\begin{aligned}
    & a \leq |a|  \qquad  &\text{(tölugildi getur aðeins stækkað tölu)}\\
\qquad \\
    & |a|=|-a|  \qquad  &\text{(tölugildi eru óháð formerki)}\\
\qquad \\
    & |a|\cdot|b|=|ab| \qquad  &\text{(tölugildi varðveitir margföldun)}\\
\qquad \\
    & |a|^2=a^2 \qquad   &\text{(önnur veldi eyða tölugildi)}\\
    \end{aligned}\end{split}
\end{equation*}

\subsubsection{Regla Pýþagórasar}
\label{\detokenize{Kafli12:regla-pyagorasar}}\begin{equation*}
\begin{split}a^2+b^2=c^2\end{split}
\end{equation*}

\subsubsection{Fjarlægð milli punkta}
\label{\detokenize{Kafli12:fjarlaeg-milli-punkta}}
Fjarlægðin milli punktanna \(P_1=(x_1,y_1)\) og \(P_2=(x_2,y_2)\) í hnitakerfinu er
\begin{equation*}
\begin{split}\sqrt{(x_2-x_1)^2+(y_2-y_1)^2}\end{split}
\end{equation*}

\subsubsection{Hallatala línu}
\label{\detokenize{Kafli12:hallatala-linu}}
Ef við höfum tvo punkta \((x_1,y_1)\) og \((x_2,y_2)\) fæst með formúlunni
\begin{equation*}
\begin{split}h=\frac{y_2-y_1}{x_2-x_1}\end{split}
\end{equation*}

\subsubsection{Miðpunktsregla}
\label{\detokenize{Kafli12:mipunktsregla}}
Reikna má miðpunkt striksins á milli \(A(x_1, x_2)\) og \(B(x_2,y_2)\) með:
\begin{equation*}
\begin{split}M = \left( \frac{x_1+x_2}{2} , \frac{y_1+y_2}{2} \right)\end{split}
\end{equation*}

\subsubsection{Flatarmál}
\label{\detokenize{Kafli12:flatarmal}}
Flatarmál \textit{rétthyrnings} er \(F=a\cdot b\) og ummálið er \(U=2a+2b\) .

\begin{figure}[htbp]
\centering

\noindent\sphinxincludegraphics[width=0.400\linewidth]{{fl_rett}.svg}
\end{figure}

Flatarmál \textit{samsíðungs} er \(F=g\cdot h\).

\begin{figure}[htbp]
\centering

\noindent\sphinxincludegraphics[width=0.400\linewidth]{{fl_sams}.svg}
\end{figure}

Flatarmálið er þá \(F=|\bar{a}| \cdot |\bar{b}| \cdot \sin(\theta)\) og ummálið er \(U=2|\bar{a}|+2|\bar{b}|\) .

\begin{figure}[htbp]
\centering

\noindent\sphinxincludegraphics[width=0.400\linewidth]{{fl_sams2}.svg}
\end{figure}

Flatarmál hrings er \(F=r^2\cdot\pi\) og ummálið er \(U=2r\pi\) .

\begin{figure}[htbp]
\centering

\noindent\sphinxincludegraphics[width=0.400\linewidth]{{fl_hring}.svg}
\end{figure}

Flatarmál sporöskju er \(F=a\cdot b\cdot\pi\) .

\begin{figure}[htbp]
\centering

\noindent\sphinxincludegraphics[width=0.400\linewidth]{{fl_spor}.svg}
\end{figure}

Flatarmál \textit{þríhyrnings} er \(F=\frac{1}{2}g\cdot h\) .

\begin{figure}[htbp]
\centering

\noindent\sphinxincludegraphics[width=0.400\linewidth]{{fl_thri1}.svg}
\end{figure}

Flatarmálið er þá \(F=\frac{1}{2}|\bar{a}| \cdot |\bar{b}| \cdot \sin(\theta)\) .

\begin{figure}[htbp]
\centering

\noindent\sphinxincludegraphics[width=0.400\linewidth]{{fl_thri2}.svg}
\end{figure}


\subsection{Föll}
\label{\detokenize{Kafli12:foll}}

\subsubsection{Oddstætt og jafnstætt}
\label{\detokenize{Kafli12:oddstaett-og-jafnstaett}}
Látum \(f: \mathbb{R} \to \mathbb{R}\) vera fall.

\textit{Jafnstætt} ef \(f(-x)=f(x)\) fyrir öll \(x \in \mathbb{R}\).

\textit{Oddstætt} ef \(f(-x)=-f(x)\) fyrir öll \(x \in \mathbb{R}\).


\subsubsection{Lograreglur}
\label{\detokenize{Kafli12:lograreglur}}
Fyrir \(a,b,x,y\in \mathbb{R}_+\) og \(r \in \mathbb{R}\) gildir:
\begin{enumerate}
\sphinxsetlistlabels{\arabic}{enumi}{enumii}{}{.}%
\item {} 
\(\qquad \log_a(1)=0\)

\item {} 
\(\qquad \log_a(1/x)=-\log_a(x)\)

\item {} 
\(\qquad \log_a(xy)=\log_a(x)+\log_a(y)\)

\item {} 
\(\qquad \log_a(x/y)=\log_a(x)-\log_a(y)\)

\item {} 
\(\qquad \log_a(x^r)=r\log_a(x)\)

\item {} 
\(\qquad \log_a(x)=\dfrac{\log_b(x)}{\log_b(a)}\).

\end{enumerate}


\subsubsection{Stofnbrotaliðun}
\label{\detokenize{Kafli12:stofnbrotaliun}}\begin{equation*}
\begin{split}\begin{aligned}
&\frac{ax+b}{(x-\alpha)(x-\beta)} = \frac{A}{(x-\alpha)}+ \frac{B}{(x-\beta)} \\
&\qquad \text{þar sem} \\
& \alpha \neq \beta, \quad A= \frac{a\alpha + b}{\alpha - \beta} \quad \text{og} \quad B= \frac{a\beta + b}{\beta - \alpha}
\end{aligned}\end{split}
\end{equation*}

\subsection{Margliður}
\label{\detokenize{Kafli12:margliur}}

\subsubsection{Nokkrar liðanir}
\label{\detokenize{Kafli12:nokkrar-lianir}}\begin{equation*}
\begin{split}(a + b)^0 = 1\end{split}
\end{equation*}\begin{equation*}
\begin{split}(a + b)^1 = a + b\end{split}
\end{equation*}\begin{equation*}
\begin{split}(a + b)^2 = a^2 + 2ab + b^2\end{split}
\end{equation*}\begin{equation*}
\begin{split}(a + b)^3 = a^3 + 3a^2 b + 3a b^2 + b^3\end{split}
\end{equation*}\begin{equation*}
\begin{split}(a + b)^4 = a^4 + 4 a^3 b + 6a^2 b^2 + 4ab^3 + b^4\end{split}
\end{equation*}\begin{equation*}
\begin{split}(a + b)^5 = a^5 + 5a^4 b + 10a^3b^2 + 10a^2 b^3 + 5 ab^4 +b^5\end{split}
\end{equation*}

\subsection{Hornaföll}
\label{\detokenize{Kafli12:hornafoll}}

\subsubsection{Gráður og bogaeiningar}
\label{\detokenize{Kafli12:graur-og-bogaeiningar}}\begin{equation*}
\begin{split}x \text{Rad} = \left(x \cdot \frac{360}{2 \pi}\right)° \qquad og \qquad  x°=\left( x \cdot \frac{2 \pi}{360}\right) \text{Rad}\end{split}
\end{equation*}

\subsubsection{Hliðar þríhyrnings}
\label{\detokenize{Kafli12:hliar-rihyrnings}}\begin{equation*}
\begin{split}\begin{aligned}
\sin(\alpha) = \frac{b}{c} \\
\qquad \\
\cos(\alpha) = \frac{a}{c} \\
\qquad \\
\tan(\alpha) = \frac{b}{a} \\
\qquad \\
\end{aligned}\end{split}
\end{equation*}
\noindent{\hspace*{\fill}\sphinxincludegraphics[width=0.400\linewidth]{{sohcahtoa2}.svg}\hspace*{\fill}}


\subsubsection{Grunnreglan}
\label{\detokenize{Kafli12:grunnreglan}}\begin{equation*}
\begin{split}\sin^2(x) + \cos^2(x) = 1\end{split}
\end{equation*}

\subsubsection{Hliðrunarreglur}
\label{\detokenize{Kafli12:hlirunarreglur}}\begin{equation*}
\begin{split}\begin{aligned}
    & \qquad \cos(-\theta)=\cos \theta\\
    \qquad \\
& \qquad \sin(-\theta)=-\sin\theta\\
    \qquad \\
& \qquad \cos(\pi-\theta)=-\cos \theta\\
    \qquad \\
& \qquad \sin(\pi-\theta)=\sin \theta\\
    \qquad \\
& \qquad \cos(\theta+\pi)=-\cos \theta\\
    \qquad \\
    & \qquad \sin(\theta+\pi)=-\sin \theta\\
    \qquad \\
    & \qquad \cos\left(\frac{\pi}{2}-\theta\right)=\sin\theta\\
    \qquad \\
    & \qquad \sin\left(\frac{\pi}{2}-\theta\right)=\cos\theta
    \end{aligned}\end{split}
\end{equation*}

\subsubsection{Summuformúlur}
\label{\detokenize{Kafli12:summuformulur}}
\sphinxstylestrong{1.}
\begin{equation*}
\begin{split}\sin( u + v ) = \sin(u)  \cos(v) + \cos(u) \sin(v)\end{split}
\end{equation*}
\sphinxstylestrong{2.}
\begin{equation*}
\begin{split}\sin( u - v ) = \sin(u) \cos(v) - \cos(u) \sin(v)\end{split}
\end{equation*}
\sphinxstylestrong{3.}
\begin{equation*}
\begin{split}\cos( u + v ) = \cos(u)  \cos(v) - \sin(u)  \sin(v)\end{split}
\end{equation*}
\sphinxstylestrong{4.}
\begin{equation*}
\begin{split}\cos( u - v ) = \cos(u)  \cos(v) + \sin(u)  \sin(v)\end{split}
\end{equation*}
\sphinxstylestrong{5.}
\begin{equation*}
\begin{split}\tan(u-v) = \frac{\tan(u) - \tan(v)}{1 + \tan(u)  \tan(v)}\end{split}
\end{equation*}
\sphinxstylestrong{6.}
\begin{equation*}
\begin{split}\tan(u+v) = \frac{\tan(u) + \tan(v)}{1 - \tan(u)  \tan(v)}\end{split}
\end{equation*}

\subsubsection{Tvöföldunarformúlur}
\label{\detokenize{Kafli12:tvofoldunarformulur}}
\sphinxstylestrong{1.}
\begin{equation*}
\begin{split}\sin(2u) = 2\sin(u)\cos(u)\end{split}
\end{equation*}
\sphinxstylestrong{2.}
\begin{equation*}
\begin{split}\begin{aligned}
\cos(2x)&= \cos^2(x)-\sin^2(x) \\
&= 2\cos^2(x)-1 \\
&= 1-2 \sin^2(x)
\end{aligned}\end{split}
\end{equation*}
\sphinxstylestrong{3.}
\begin{equation*}
\begin{split}\tan(2u) = \frac{2\tan(u)}{1-\tan^2(u)}\end{split}
\end{equation*}

\subsubsection{Helmingunarformúlur}
\label{\detokenize{Kafli12:helmingunarformulur}}
\sphinxstylestrong{1.}
\begin{equation*}
\begin{split}\sin^2(u) = \frac{1- \cos(2u)}{2} \qquad \text{eða} \qquad \sin\left(\frac{u}{2}\right) = \pm \sqrt{\frac{1- \cos(u)}{2} }\end{split}
\end{equation*}
\sphinxstylestrong{2.}
\begin{equation*}
\begin{split}\cos^2(u) = \frac{1+ \cos(2u)}{2} \qquad \text{eða} \qquad \cos\left(\frac{u}{2}\right) = \pm \sqrt{\frac{1+ \cos(u)}{2} }\end{split}
\end{equation*}
\sphinxstylestrong{3.}
\begin{equation*}
\begin{split}\tan^2(u) = \frac{1- \cos(2u)}{1+\cos(2u)} \qquad \text{eða} \qquad \tan\left(\frac{u}{2}\right) = \pm \sqrt{\frac{1- \cos(u)}{1+\cos(u)} }\end{split}
\end{equation*}

\subsubsection{Summu\sphinxhyphen{} og margfeldisformúlur}
\label{\detokenize{Kafli12:summu-og-margfeldisformulur}}
\sphinxstylestrong{Margfeldisritháttur í summurithátt}
\begin{quote}

\sphinxstylestrong{1.}
\begin{equation*}
\begin{split}\sin(u)\sin(v) = \frac{1}{2}\left(\cos(u-v) - \cos(u+v)\right)\end{split}
\end{equation*}
\sphinxstylestrong{2.}
\begin{equation*}
\begin{split}\cos(u)\cos(v) = \frac{1}{2}\left(\cos(u-v) + \cos(u+v)\right)\end{split}
\end{equation*}
\sphinxstylestrong{3.}
\begin{equation*}
\begin{split}\sin(u)\cos(v) = \frac{1}{2}\left(\sin(u+v) + \sin(u-v)\right)\end{split}
\end{equation*}
\sphinxstylestrong{4.}
\begin{equation*}
\begin{split}\cos(u)\sin(v) = \frac{1}{2}\left(\sin(u+v) - \sin(u-v)\right)\end{split}
\end{equation*}\end{quote}

\sphinxstylestrong{Summuritháttur í margfeldisrithátt}
\begin{quote}

\sphinxstylestrong{1.}
\begin{equation*}
\begin{split}\sin(u) + \sin(v) = 2\sin\left(\frac{u+v}{2}\right)\cos\left(\frac{u-v}{2}\right)\end{split}
\end{equation*}
\sphinxstylestrong{2.}
\begin{equation*}
\begin{split}\sin(u) - \sin(v) = 2\cos\left(\frac{u+v}{2}\right)\sin\left(\frac{u-v}{2}\right)\end{split}
\end{equation*}
\sphinxstylestrong{3.}
\begin{equation*}
\begin{split}\cos(u) + \cos(v) = 2\cos\left(\frac{u+v}{2}\right)\cos\left(\frac{u-v}{2}\right)\end{split}
\end{equation*}
\sphinxstylestrong{4.}
\begin{equation*}
\begin{split}\cos(u) - \cos(v) = -2\sin\left(\frac{u+v}{2}\right)\sin\left(\frac{u-v}{2}\right)\end{split}
\end{equation*}\end{quote}


\subsubsection{Kósínusreglan}
\label{\detokenize{Kafli12:kosinusreglan}}
Í \(\triangle ABC\) gildir
\begin{equation*}
\begin{split}\begin{aligned}
a^2 &= b^2+c^2-2\cdot b \cdot c \cdot \cos(A) \\
b^2 &= a^2+c^2-2\cdot a \cdot c \cdot \cos(B) \\
c^2 &= b^2+a^2-2\cdot b \cdot a \cdot \cos(C) \\
\end{aligned}\end{split}
\end{equation*}

\subsubsection{Sínusreglan}
\label{\detokenize{Kafli12:sinusreglan}}
Í \(\triangle ABC\) gildir
\begin{equation*}
\begin{split}\frac{a}{\sin(A)} = \frac{b}{\sin(B)} = \frac{c}{\sin(C)}\end{split}
\end{equation*}
Þar sem \(A\), \(B\) og \(C\) eru horn þríhyrningsins og \(a\), \(b\) og \(c\) eru lengdir hliðanna


\subsection{Vigrar}
\label{\detokenize{Kafli12:vigrar}}

\subsubsection{Reiknireglur}
\label{\detokenize{Kafli12:reiknireglur}}\begin{equation*}
\begin{split}\begin{aligned}
& |\bar{a}| = a = \sqrt{a_x^2 + a_y^2 + a_z^2} \\
\qquad \\
& \bar{a} + \bar{b} = (a_x+b_x, a_y+b_y, a_z+b_z) \\
\qquad \\
& c \cdot \bar{v} = (c \cdot v_x, c \cdot v_y, c \cdot v_z) \\
\qquad \\
& \bar{a} \cdot \bar{b} = a b \cos{\phi} \\
\qquad \\
& \bar{a} \times \bar{b} = (a_y b_z - a_z b_y)\hat{\imath} + (a_z b_x - a_x b_z)\hat{\jmath} + (a_x b_y - a_y b_x)\hat{k} \\
\qquad \\
& \bar{a} \times \bar{b} = - \bar{b} \times \bar{a}
\end{aligned}\end{split}
\end{equation*}

\subsection{Markgildi}
\label{\detokenize{Kafli12:markgildi}}

\subsubsection{Reiknireglur fyrir markgildi}
\label{\detokenize{Kafli12:reiknireglur-fyrir-markgildi}}
Gerum ráð fyrir að \(f\) og \(g\) séu föll og að \(c\in \mathbb{R} \cup\{-\infty,\infty\}\)
Gerum ráð fyrir að bæði markgildin
\begin{equation*}
\begin{split}\lim_{x\to c}f(x)\qquad \text{og}\qquad \lim_{x\to c}g(x)\end{split}
\end{equation*}
séu skilgreind og að hvorugt þeirra sé jafnt plús eða mínus óendanlegu.
Gerum ráð fyrir að \(k\in\mathbb{R}\) sé fasti.
Þá gildir:
\begin{align*}\!\begin{aligned}
1. & \qquad \lim_{x\to c}k=k\\
2. & \qquad \lim_{x\to c} \left(kf(x) \right)=k \cdot \left(\lim_{x\to c}f(x)\right)\\
3. & \qquad \lim_{x\to c} \left(f(x)+g(x)\right)=\lim_{x\to c}f(x)+\lim_{x\to c}g(x)\\
4. & \qquad \lim_{x\to c} \left(f(x)-g(x)\right)=\lim_{x\to c}f(x)-\lim_{x\to c}g(x)\\
5. & \qquad \lim_{x\to c} \left(f(x)\cdot g(x)\right)= \left( \lim_{x\to c}f(x) \right)\cdot \left(\lim_{x\to c}g(x) \right)\\
6. & \qquad \lim_{x\to c} \left( \frac{f(x)}{g(x)} \right)=\frac{\lim_{x\to c}f(x)}{\lim_{x\to c}g(x)} \qquad \text{ef} \qquad \lim_{x\to c}g(x)\not=0\\
\end{aligned}\end{align*}

\subsection{Diffrun}
\label{\detokenize{Kafli12:diffrun}}

\subsubsection{Reiknireglur}
\label{\detokenize{Kafli12:id1}}
Gerum ráð fyrir að \(f,g\) séu deildanleg föll á \(\mathbb{R}\).

Látum \(a\in \mathbb{R}\) vera fasta.

Þá gildir:
\begin{enumerate}
\sphinxsetlistlabels{\arabic}{enumi}{enumii}{}{.}%
\item {} 
\((a\cdot f)'=af'\)

\item {} 
\((f+g)'=f'+g'\)

\item {} 
\((f-g)'=f'-g'\)

\item {} 
\((f\cdot g)'=f'g+fg'\)

\item {} 
\((f\circ g)'=(f'\circ g)\cdot g'\)

\end{enumerate}

Ef \(g(x)\) er ekki jafnt núlli fyrir öll \(x\in I\), þá gildir einnig:
\begin{enumerate}
\sphinxsetlistlabels{\arabic}{enumi}{enumii}{}{.}%
\setcounter{enumi}{5}
\item {} 
\(\left(\frac{1}{g}\right)'=\frac{-g'}{g^2}\)

\item {} 
\(\left(\frac{f}{g}\right)'=\frac{f'g-fg'}{g^2}\)

\end{enumerate}

Ef \(f\) er andhverfanlegt gildir einnig:
\begin{enumerate}
\sphinxsetlistlabels{\arabic}{enumi}{enumii}{}{.}%
\setcounter{enumi}{7}
\item {} 
Ef \(f(x_0)=y_0\) þá er \((f^{-1})'\circ f=\frac{1}{f'}\)

\end{enumerate}


\subsubsection{Þekktar afleiður}
\label{\detokenize{Kafli12:ekktar-afleiur}}\begin{enumerate}
\sphinxsetlistlabels{\arabic}{enumi}{enumii}{}{.}%
\item {} 
Ef \(a\) er fasti og \(f(x)=a\) þá er

\end{enumerate}
\begin{equation*}
\begin{split}f'(x)=0\end{split}
\end{equation*}\begin{enumerate}
\sphinxsetlistlabels{\arabic}{enumi}{enumii}{}{.}%
\setcounter{enumi}{1}
\item {} 
Ef \(n\in \mathbb{R}\setminus\{0\}\) og \(f(x)=x^n\) þá er

\end{enumerate}
\begin{equation*}
\begin{split}f'(x)=nx^{n-1}\end{split}
\end{equation*}\begin{enumerate}
\sphinxsetlistlabels{\arabic}{enumi}{enumii}{}{.}%
\setcounter{enumi}{2}
\item {} 
Ef \(a\in \mathbb{R}_+\) og \(f(x)=a^x\) þá er

\end{enumerate}
\begin{equation*}
\begin{split}f'(x)=\ln(a)a^x\end{split}
\end{equation*}\begin{enumerate}
\sphinxsetlistlabels{\arabic}{enumi}{enumii}{}{.}%
\setcounter{enumi}{3}
\item {} 
Ef \(a\in \mathbb{R}_+\) og \(f(x)=\log_a(x)\) þá er

\end{enumerate}
\begin{equation*}
\begin{split}f'(x)=\frac{1}{\ln(a)x}\end{split}
\end{equation*}\begin{enumerate}
\sphinxsetlistlabels{\arabic}{enumi}{enumii}{}{.}%
\setcounter{enumi}{4}
\item {} 
Ef \(f(x) = \ln(x)\) þá er

\end{enumerate}
\begin{equation*}
\begin{split}f'(x) = \frac{1}{x}\end{split}
\end{equation*}\begin{enumerate}
\sphinxsetlistlabels{\arabic}{enumi}{enumii}{}{.}%
\setcounter{enumi}{5}
\item {} 
Ef \(f(x) = e^x\) þá er

\end{enumerate}
\begin{equation*}
\begin{split}f'(x) = e^x\end{split}
\end{equation*}\begin{enumerate}
\sphinxsetlistlabels{\arabic}{enumi}{enumii}{}{.}%
\setcounter{enumi}{6}
\item {} 
Ef \(f(x)=\cos(x)\) þá er

\end{enumerate}
\begin{equation*}
\begin{split}f'(x)=-\sin(x)\end{split}
\end{equation*}\begin{enumerate}
\sphinxsetlistlabels{\arabic}{enumi}{enumii}{}{.}%
\setcounter{enumi}{7}
\item {} 
Ef \(f(x)=\sin(x)\) þá er

\end{enumerate}
\begin{equation*}
\begin{split}f'(x)=\cos(x)\end{split}
\end{equation*}


\renewcommand{\indexname}{Atriðaskrá}
\printindex
\end{document}